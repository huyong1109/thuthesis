\chapter{相关工作}
\label{cha:related}
本章主要介绍本文的正压求解器、迭代算法预处理技术和气候模式正确性验证工具这三个工作目前的背景。第\ref{solver:Backgroud}节介绍了海洋模式正压模态求解器的优化,第\ref{related:precond}节介绍了迭代算法中主要的预处理技术,第\ref{related:verify}节介绍当前气候模式中常见的正确性验证方法。


\section{海洋模式正压求解技术}
\label{solver:Backgroud} 

过去的二十多年中,用来解决科学问题的超级计算机变得越来越强大。 
很多高性能计算领域的研究都在关注如何使得科学应用能够更加适应大规模的并行环境。 
没有可扩展的应用, 超级计算机的功能再强大也不能对很多像海洋模拟这样科学领域十分重要的问题起到促进作用。 
数值气候模式利用超级计算机的并行环境能够提高我们 模拟和理解海洋运动过程、监视和预报海洋状态的能力。 
目前的气候模式为了能够更加准确的模拟大气、海洋和海冰等运动过程, 都趋向于采用更细粒度的水平和垂直分辨率,导致气候模式的规模也变得越来越大。 
随着气候模式分辨率的提高, 利用气候模式来做模拟的计算需求也会变得越来越巨大,这也使得在大规模并行环境中对海洋模式进行优化变得极其重要。 



目前有很多研究工作都是在关注气候系统模式的性能,尤其高分辨率气候模式在大规模并行环境中的可扩展性。 
气候模式,尤其是气候系统模式,由于包含有很多分量模式,结构复杂,代码量巨大,导致气候模式的性能并不乐观。 
本文主要以目前使用最为广泛的美国国家大气研究中心主要研发的CESM为研究对象。
在很多真实模拟中,海洋模式POP都是CESM中计算开销最大的一个分量\cite{Worley:2011:PCE:2063384.2063457, dennis2012computational}。 
因此,本文中,我们专注于提高海洋模式POP的性能优化。 
POP是一个十分有影响力的海洋模式,它是由美国Los Alamos国家实验室研发,多家研究机构共同发展。
海洋模式POP被广泛的应用于涡分辨率的海洋模拟\cite{mcclean2002eulerian, stark2004towards},以及海洋和海冰或者大气和海洋相耦合的耦合模拟  \cite{May2002preliminary}。 
POP目前被著名的公共地球系统模式CESM采纳为其海洋模式分量。  
海洋模式POP采用经过静力平衡近似和  Boussinesq近似的三维原始方程。 
为了避免快波(如重力波等)对时间步长的苛刻的要求, 它将时间积分分成两个部分: 一个是求解三维动力学和热动力学过程的斜压模态,另一个则是求解求解二维的垂直积分后的动量方程和连续性方程得到的二维海表高度(SSH)变化的正压模态。\cite{smith2010parallel}.

已有的很多研究都明确的指出,海洋模式POP中,斜压模态的计算量占整个模式的绝大部分,但通信相对较少、可扩展性较好,其计算时间随着计算规模的增大而减少。
但是正压模态主要由边界更新和全局归约操作等通信开销组成,其计算时间开销占总计算时间开销的比例从几百核上的10\%增大到10,000多核上的50\%\cite{pop05,stone2011cgpop,Worley:2011:PCE:2063384.2063457, dennis2012computational}。
因此,海洋模式POP的性能主要受到正压模态中通信瓶颈的影响,提高海洋模式正压模态的可扩展性将有助于提高整个公共地球系统模式的性能。

在时间差分方案上,目前海洋模式中求解器正压模态既有显式方法,又有隐式方法\cite{griffies2000developments}。 
显式方法的优点是求解简单,容易并行,缺点是为了使得数值格式稳定需要很小的时间步长。
与之相反,隐式方法的求解复杂,但是计算效率更高。 
 

\subsection{正压模态通信瓶颈}
\label{related:bottleneck}

隐式自由面方法在求解海洋模式正压模态中是一个很常见的选择,因为它能够允许较大的时间步长来高效的计算速度很快(约200m/s)的重力波。
但是隐式自由面的求解需要隐式的求解一个椭圆方程,这正是正压模态的可扩展性较差主要原因。 
海洋模式正压模态中的椭圆方程通常会被近似为一个线性系统$Ax=b$。 
国际上很多流行的海洋模式都采用共轭梯度法及其变种求解海洋模式正压模态中的线性系统\citep{adcroft2014mitgcm,lai2010nonhydrostatic,madec1997ocean}。
但是,共轭梯度法自身有一个不适合大规模并行坏境的缺陷,那就是每一步迭代过程中都涉及到两次求內积操作。 
当采用成千上百个处理器核心时, 做內积所需要的全局通信和同步操作就会是一个主要的瓶颈。

目前有很多研究工作都是在关注海洋模式POP的性能,尤其是它的正压模态的比较差的可扩展性。 
Jones\cite{pop05}等人在向量架构和常规集群的并行环境中测试了海洋模式POP1.4.3版本的可移植性,并且发现POP中斜压模态主要是由计算组成, 但是正压模态主要由边界更新和全局归约操作等通信开销组成。
Stone  \cite{stone2011cgpop}等人发现,正压模态的时间开销占总计算时间开销的比例从几百核上的10\%增大到10,000多核上的50\%。 
他们甚至还为此而开发了一个POP的简化版本,称之为CGPOP,专门用来研究海洋模式POP中的正压模态的新算法、数据结构和编程模型等。 
Worley  \cite{Worley:2011:PCE:2063384.2063457} 和 \cite{dennis2012computational} 等人在近30,000 核上测试了公共地球系统模式CESM的海洋模式分量POP 2.0.1。 
他们发现,海洋模式POP在很多真实模拟中都是CESM中计算开销最大的一个分量,并且证明了在大规模并行时,海洋模式POP的性能主要受到正压模态中通信瓶颈的影响。 

\subsection{正压求解器的优化}
\label{related:improve}


正压求解器是高分辨率CESM版本中POP的主要性能瓶颈,因为当在大核数上运行时它的运行时间占了POP总运行时间的一大半。 
这是由于求解自由海表面的正压求解器中算法所固有的明显的全局通信操作在大核数下的可扩展性比较差所导致的。
共轭梯度法(CG)以及它的变换形式是比较常用的求解海洋模式中因隐式计算自由海表面而得到的椭圆方程组的方法。
比较著名的海洋模式如MITgcm\citep{adcroft2014mitgcm}, FVCOM\citep{lai2010nonhydrostatic}, MOM3\citep{pacanowsky1999mom3}, OPA \citep{madec1997ocean}
都采用了共轭梯度法或其变种来求解自由海表面方程。
然而,共轭梯度法在当前的海洋模式分量POP中造成了比较严重的通信瓶颈\citep{Worley:2011:PCE:2063384.2063457}。 

正压模态的执行时间占整个海洋模式POP的很大一部分,尤其是当它所使用大量的处理器核心时。
目前已经有很多的关于如何优化正压模态性能的工作,其中大部分都是通过减少进程间的通信量或者加速每个进程上的计算速度。 

全局归约操作的开销和所使用的进程数的多少成正相关的, 所以通信开销随着进程数的增加会逐渐变得不可接受。
O
很多研究工作都尝试去提高共轭梯度法的性能,减少其负面影响。 
一些方法尝试减少共轭梯度法的全局通信的开销\cite{dAzevedo1999lapack},以及将计算与通信相重叠\cite{beare1997optimisation}。 
另外一些方案在海洋模式中尝试使用陆地点移除和负载均衡\cite{dennis2007inverse, dennis2008scaling} 的方法来减少进程数,进而减少其相应的全局归约操作的开销。 
我们这里简单的从两个方面介绍一下相关工作: 一是提高通用的并行环境中共轭梯度法效率的工作,二是针对海洋模式中的共轭梯度法的效率的改进。

在第一个方面, 减少共轭梯度法的全局通信的开销在这个算法被并行化之后就一直是研究的热点。 
现在有很多的可选方案来缓解大规模并行环境中的预处理的共轭梯度法所固有的比较差的可扩展性。
这方面工作比较好的一个综述可以参见\cite{ghysels2014}。 
这个领域最早的成果是在标准的共轭梯度法的基础上做算法的修正从而减少全局归约操作的次数, 比如在海洋模式POP中被采用的ChronGear方法\cite{dAzevedo1999lapack} 。 这些方法至今仍然被广泛的使用着。 
早期还有s-步方法\cite{chron1989} 和最近的一些变种(比如论文\inlinecite{hoemmen2010})也能够减少全局通信, 但是这些方法很难结合稍微复杂一点的预处理方法使用。
他们中的大多数要么是减少处理器间的通信次数,要么是加快每个处理器上的计算速度。
其中最为著名的工作是,修正标准的共轭梯度法,从而减少全局通信的次数。 比如POP中采用的Chronopoulos-Gear (ChronGear, \cite{dAzevedo1999lapack})就是共轭梯度法的一个变种。
这些方法和最近的一些变形(比如\cite{hoemmen2010})都是尝试去减少全局通信,但是要么效果不太明显,要么变形后的算法很难与复杂一点的预处理子相结合\citep{ghysels2014}。
另外,还有一些研究是通过将全局通信与矩阵向量乘法利用流水线的方式来重叠,进而提高共轭梯度法的并行效率。关于如何减少共轭梯度法中全局通信的开销,文章\cite{ghysels2014}给出了一个非常详尽的总结。
另外,最近也有一些工作提出来将全局归约操作和矩阵向量乘积操作利用流水线的方式进行重叠\cite{ghysels2014},从而提高并行共轭梯度法的效率。  



针对海洋模式, 有很多工作都是关于如何减少求解器中的通信瓶颈的。
正压模态的执行时间占整个海洋模式POP的很大一部分,尤其是当它所使用大量的处理器核心时。
目前已经有很多的关于如何优化正压模态性能的工作,其中大部分都是通过减少进程间的通信量或者加速每个进程上的计算速度。 
全局归约操作的开销适合所使用的进程数的多少成正相关的, 所以通信开销随着进程数的增加会逐渐变得不可接受。
OpenMP并行和陆地点移除在海洋模式中是比较常见的减少进程数量及其相应的通信开销的常用策略。  
在论文\cite{Worley:2011:PCE:2063384.2063457}中, 他们在正压模态中采用OpenMP并行技术,解决斜压模态需要很多的处理器核心来进行计算而正压模态的进程数太多会造成通信瓶颈越大这一矛盾,进而提高了海洋模式在大核数上的性能。 
另一个比较常见的减少通信开销的策略是陆地点的移除\cite{dennis2007inverse,dennis2008scaling}。 
Dennis等人提出基于最新的空间填充曲线划分算法的负载均衡的策略,不仅能够改善负载均衡,还能够通过去掉全是陆地点的块来减少需要参与通信的处理器核心的数目,从而减少通信的开销。 
他们的实验表明,在接近30,000个处理器核心上,新的策略使得模式的模拟速度增加了近一倍。 
另外,减少通信的频率同样可以减轻正压模态中的性能瓶颈。 
 早在1997年,  Beare \cite{beare1997optimisation}等人就提出来,通过增大边界通信缓存区域的块的大小以及将通信和计算相重叠减少通信开销,进而提高并行大洋环流模式的性能。  
尽管这些方法都能够对性能有一定的提升, 但是他们并没有完全的消除掉全局归约操作的瓶颈。 

 


共轭梯度法的数据局部性比较差,而且很多操作只能串行执行,导致以上的改进方案的效果都十分有限。 
其他一些工作利用异构加速平台,比如说GPUs \cite{cuomo2012pcg} 和FPGAs \cite{Shida2007}来实现对预处理共轭梯度法的加速。 
Cuomo 等人\cite{cuomo2012pcg}在全球环流海洋数值模式中引入了稀疏近似逆预处方法,并且利用一个科学计算库将其实现在GPU上。 
Shida 等人\cite{Shida2007}将海洋模式中的正压模态移植到FPGAs上,并且发现发现当适当的使用块上内存和流式的直接内存访问,一块100MHz的FPGA卡的性能能与1GHz的CPU处理器相当。 
GPUs和FPGAs还能够减少正压模态中的全局通信开销。这些设备跟传统的CPU相比有更强的计算能力和更大的内存,因此我们只需要使用很少的几块加速卡就能够实现只有在大规模并行环境下才能计算的任务。 




\section{预处理技术}
\label{related:precond}

正压模态求解器的总时间开销等于求解器达到收敛所需要的迭代步数乘以每一次迭代的时间开销。
随着计算使用的处理器核心数的增大,每一步迭代中的计算所消耗的时间是逐渐减少的,
但是通信所需要的时间会逐渐增加。 
为了减少通信的开销,人们通常采用预处理这一技术来减少收敛所需要的迭代次数。
附加的前提条件是预处理过程的开销在合理的范围之内。 
目前海洋模式POP中采用的默认求解器ChronGear通过使用一个简单的对角预处理方法,性能就已经取得了很好的提升\cite{pini1990simple, reddy2013comparison}。 
如果能够进一步的提高求解器的收敛速度,求解过程中通信的开销将会大大的减少,从而进一步的提高求解器的可扩展性。
事实上,一个有效的预处理子不仅能够给新的P-CSI求解器带来性能提升,同时也能够改进原有的默认的ChronGear求解器。


共轭梯度法中的预处理方法自从上世纪90年代以来就一直备受重视。 
很多的线性系统在采用了适当的预处理子之后,预处理共轭梯度法只需要少数几步迭代就能够收敛。 
然是,大多数的最有效的预处理技术, 比如说不完全的Cholesky分解和不完全的LU分解,在海洋模式中并不是十分的有效。 
海洋模式中的椭圆方程组需要为之设计专门的可并行化的预处理子。 
在1985年,Concus 等人\cite{concus1985block}采用系数矩阵的逆的一个条带状的近似,来对椭圆微分方程上的共轭梯度法进行预处理,结果取得了比其他通用的预处理方法更高的效率。 
Smith等人\cite{smith1992parallel}采用多项式预处理的方法和一个局部近似逆的预处理方法来加速并行大洋环流模式中的共轭梯度法的收敛。 
Adamidis等人 \cite{adamidis2011high} 在Max Planck研究中心的全球海洋海冰模式MPIOM中采用了一个不完全的Cholesky预处理子,在提高预处理共轭梯度法的性能的同时,也改善了它的可扩展性。 
Watanabe \cite{Watanabe2006pcg} 设计了一个与重叠区域分解法相结合的预处理共轭梯度法来加速收敛,同时减少了处理单元之间的通信开销。 



\subsection{传统预处理方法}
\label{related:classical}

由于海洋模式POP中的数据是分布在多个进程上的,而进程之间的通信的开销很大, 如果使用因式分解的预处理方法,通常只能接受比较浅层次的分解,比如说ILU(0)。 
ILU(0)分解中,L和U分别与原始矩阵A的上三角和下三角矩阵具有相同的非零元结构。
这导致L和U的乘积只是A的一个非常粗糙的近似,从而使得这种预处理的效果并不是很明显。 
更为精确的ILU因式分解方法并不太适用于海洋模式,因为它需要更多的额外的通信和计算。 
同时,计算这些高阶的预处理子的代价也会越高。 

\subsection{误差向量传播方法}
\label{related:evp}

\subsection{并行预处理方法}
\label{related:parall}

使用最广的串行预处理技术当属不完全的因式分解方法,比如不完全的LU分解(ILU)和它的变种\cite{benzi2002preconditioning}。 
但是,这些方法并不适合在并行的环境中使用,因为它们需要对整个矩阵做一个按顺序的逐次的计算。 
随着并行计算的兴起,一些可以并行化的预处理技术,比如说多项式预处理、近似逆预处理、多重网格预处理和块预处理,逐渐因此了人们的重视。 
高阶的多项式预处理能够像不完全的LU分解(ILU)以及它的变种在串行的环境中一样高效的减少迭代算法的迭代次数\cite{benzi2002preconditioning} 。 
但是多项式预处理子巨大的计算开销抵消了它与在减少迭代次数上的优势,因为一个$k$阶的多项式预处理子在每次一次迭代过程中需要进行$k$次矩阵向量乘操作。 
最终的实际运行中,它的时间开销有时候反而比不上简单的对角预处理\cite{meyer1989numerical,smith1992parallel}。
近似逆预处理,尽管可并行度很高,但是它需要求解一个比原始方程大很多倍的线性方程组\cite{smith1992parallel,bergamaschi2007numerical}, 这就使得它的魅力相比于简单的预处理方法大打折扣。

在1985, Concus等人 \cite{concus1985block} 使用原始矩阵的逆矩阵的条带状近似作为预处理矩阵,在椭圆微分方程问题的求解中取得了比其他预处理子更好的效果。 
Smith et al. \cite{smith1992parallel}在海洋模式POP 的最原始版本中采用多项式预处理的方法以及一个局部近似逆预处理的方法,将迭代步数缩短为原来的三分之到五分之一。
但是由于多项式预处理方法中,前期得到多项式预处理子的开销比较大,在后期的版本中这一方法并没有得到继承。  
 
 
多重网格是一种针对椭圆微分系统中的线性方程组的高可扩展和十分高效的预处理方法。 
最近的研究表明几何多重网格(GMG)在使用规则网格和简单地形的大气模式
 \cite{muller2014massively}和海洋模式中\cite{matsumura2008non,kanarska2007algorithm}中有很好的效果。 
但是,全球海洋模式通常使用比较复杂的地形(如群岛、海峡、通道和复杂的海岸等),并结合以非规则或者各向异性的网格, 结果导致简单的几何多重网格并不能取得较好的效果
\cite{matsumura2008non,fulton1986multigrid,tseng2003ghost,stuben2001review}。 
在公共地球系统模式CESM的海洋模式分量POP中,为了避开极点奇异性问题,采用的是极点偏移的广义正交网格。 
海洋模式POP将陆地点掩盖掉,从而只计算地球表面的海洋部分。
这些选择导致了模式中的椭圆方程所得到的系数是定义在一个不规则的网格上的不规则区域, 进而导致几何多重网格很难取得好的效果。 
地形的复杂性在高分辨率的海洋网格中会变得更加糟糕,因为成千上万的群岛和海峡通道(比如白令海峡)在稍微粗一点的网格上并不可见。
对于这些涉及复杂的地形的问题,代数多重网格通常比几何多重网格更加的实用。 
但是代数多重网格也有一个缺陷,那就是在某些例子里面,代数多重网格的启动开销比它的迭代求解过程还要耗时,
这使得它反而不如一些预处理得比较好的共轭梯度法
\cite{muller2014massively}, 尤其是在使用共轭梯度法所需要的迭代步数比较少的时候。 
地球系统模式CESM中的海洋模式分量POP中的正压模态求解问题就是这样一个例子(参见下一节的图
 \ref{fig:iter})。
更进一步,不管是高分辨率还是低分辨率,海洋模式POP的每一个模拟天内都需要对正压模态中的线性方程求解几十次甚至上百次,
而通常的一次模拟中往往需要模拟成百上千个模拟年,这就导致代数多重网格方法的启动开销变得无法接受。


最后,块预处理技术被证明是一个比较有效的并行预处理技术\cite{concus1985block, white2011block}。
这个技术对于海洋模式POP尤为有潜力,因为它能够充分地利用POP中由椭圆方程离散化后得到的矩阵的块状结构的性质。 

\section{气候模式的正确性验证}
\label{related:verify}

目前,使用计算机可执行程序来表示一个模拟模型的需求正在不断增加。 
这个软件工程这个领域中,有很多的软件验证继续能够用来对数值模拟程序进行验证。
但是,由于气候模式的不确定性,模式的验证要比气候模式的软件验证要艰难得多。
 

气候预测本质上有五个维度:三维空间,时间和概率。 
主流的气候模式,目前基本上都牺牲掉概率分辨率,而对其它四个分辨率不断地做优化。
这主要是由于气候本身具有很大的不确定性。
没有考虑概率维度的气候模式的模拟结果存在很大不确定性,要想对气候模式进行正确性验证也是十分困难。
正是这种困难,使得整个数值气候模拟社区中,都缺乏对于数值模拟模式正确性验证的重要性的理解和意识\cite{whitner1989guidelines}。 

 
要想实现对气候模式的正确性验证,必须考虑到气候模式的概率这一维度, 即考虑到气候模式的不确定性。 
迄今为止,已经有一些研究利用初始条件中的不确定性,以及改变边界条件所带来的影响等方法来研究气候模式的不确定性。
但是对于大部分的变量,几十年的气候模拟中主要的不确定性并不在初始条件或者外部驱动,而在于气候系统的响应。
这个问题在大气海洋环流模式(AOGCMs)中得到解决,主要是通过采用非直接指定的观测数据的集合模式的比较,来减少对可能性集合的限制\cite{allen2002towards} 。 
这种方法可能会造成预测的不确定性很小的误解。
在制定未来气候变化缓解和适应的对应策略时,必须要考虑到未来气候变化的不确定范围。
这就要求有几十年的集合模拟的结果来评估气候自身的混动性和模式相应的不确定性。 
基于近代气候变化的观测,对模式相应作出的统计学估计,认为气候的敏感性,也就是全球平均温度对于大气中二氧化碳浓度增加一倍的平衡相应,要大于5摄氏度\cite{stainforth2005uncertainty}。
通过利用常规的环流模式来进行几千次模拟而得到的集合模拟。 发现这些版本跟其他的前沿的模式的模拟结果一样真实,但是气候的敏感性从2K到11K不等。
这个几段范围的敏感性对于气候系统对于大气中温室气体的升高的相应的研究,以及评估将特定目标稳定在某个层次所带来的风险是非常有意义的。


 
文章\cite{reynolds1994random}通过对误差的来源的研究,给出了美国国家气象中心的中范围预测模型的随机增长误差的三维结构,进而为研究其不确定提供了定量手段。
他们认为气候模式中的随机误差的增长分为两类: 一类是外部误差增长,主要是由于模式的缺陷导致的;一类是内部误差增长,主要是由于初始条件中的误差的自增长 。 
通过对预测和验证分析的相关性的参数化,可以确定内外部增长的速率。
这种参数化假设线性的随机增长主要是由于模式的缺陷引起的。 
他们的结果表明,在热带地区,结合模型和分析的结果,可以显著的改善模拟结果。
而在中纬度地区,模式的改进很难改善预测的结果,所以减少分析的误差变得尤为重要。 
参数化能够得到物理上有意义,并且和之前的预测结果相一致的结果。
他们的这种方法为评估预测不确定性在空间和时间上的分布提供量化手段。


\section{本章小结}
\label{related:Conclude}

随着气候模拟的需求日益增加,以及观测资料和计算资源等辅助条件的日益成熟,气候模式不断地提高模拟精度已经成为一个大的趋势。
很多高性能计算领域的研究都在关注如何使得科学应用能够更加适应大规模的并行环境。 
在将地球系统模式移植到大规模并行环境上的过程中,海洋模式中的正压模态的差的可扩展性逐渐的暴露出来。 
海洋模式POP中,斜压模态的计算量占整个模式的绝大部分,但通信相对较少、可扩展性较好,其计算时间随着计算规模的增大而减少。
目前有很多研究工作指出海洋模式POP的正压模态是它的可扩展性的瓶颈。


改进海洋模式正压模态的方法可以概括为以下三个方面。第一,减少正压求解器中的全局规约操作的次数。
第二,改进海洋模式中正压模态的并行效率。 
 

高效的预处理方法是提高海洋模式的性能的另一种有效方法。在并行环境中,传统的高效的串行与预处理方法有着先天的缺陷。

由于模式的不确定性,模式中的任何改变都可能得到二进制不一致的结果。
为了确定这种改变是否会对模式的结果造成统计学上显著的改变,就必须考虑到模式自身的不确定性。
集合模拟是用来对模式不确定性进行量化的常用方法。 




