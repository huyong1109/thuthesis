\chapter{海洋模式正压模态}
\label{cha:related}
\section{并行海洋模式}
\label{solver:Backgroud} 

过去的二十多年中,用来解决科学问题的超级计算机变得越来越强大。 
很多高性能计算领域的研究都在关注如何使得科学应用能够更加适应大规模的并行环境。 
没有可扩展的应用, 超级计算机的功能再强大也不能对很多像海洋模拟这样科学领域十分重要的问题起到促进作用。 
数值海洋模式利用超级计算机的并行环境能够提高我们 模拟和理解海洋运动过程、监视和预报海洋状态的能力。 
目前的海洋模式为了能够更加准确的模拟海洋过程, 都趋向于采用更细粒度的水平和垂直分辨率,导致海洋模式的规模也变得越来越大。 
随着海洋模式分辨率的提高, 利用海洋模式来做模拟的计算需求也会变得越来越巨大,这也使得在大规模并行环境中对海洋模式进行优化变得极其重要。 


本章中,我们专注于提高海洋模式POP的性能优化。 
POP是一个十分有影响力的海洋模式,它是由美国Los Alamos国家实验室研发,多家研究机构共同发展。
海洋模式POP被广泛的应用于涡分辨率的海洋模拟\cite{mcclean2002eulerian, stark2004towards},以及海洋和海冰或者大气和海洋相耦合的耦合模拟  \cite{May2002preliminary}。 
POP目前被著名的公共地球系统模式(CESM)采纳为其海洋模式分量。  
海洋模式POP采用经过静力平衡近似和  Boussinesq近似的三维原始方程。 
为了避免快波(如重力波等)对时间步长的苛刻的要求, 它将时间积分分成两个部分: 一个是求解三维动力学和热动力学过程的斜压模态,另一个则是求解二维海表高度(SSH)的变化。\cite{smith2010parallel}.

目前有很多研究工作都是在关注海洋模式POP的性能,尤其是它的正压模态的比较差的可扩展性。 
Jones\cite{pop05}等人在向量架构和常规集群的并行环境中测试了海洋模式POP1.4.3版本的可移植性,并且发现POP中斜压模态主要是由计算组成, 但是正压模态主要由边界更新和全局归约操作等通信开销组成。
Stone  \cite{stone2011cgpop}等人发现,正压模态的时间开销占总计算时间开销的比例从几百核上的10\%增大到10,000多核上的50\%。 
他们甚至还为此而开发了一个POP的简化版本,称之为CGPOP,专门用来研究海洋模式POP中的正压模态的新算法、数据结构和编程模型等。 
Worley  \cite{Worley:2011:PCE:2063384.2063457} 和 \cite{dennis2012computational} 等人在近30,000 核上测试了公共地球系统模式CESM的海洋模式分量POP 2.0.1。 
他们发现,海洋模式POP在很多真实模拟中都是CESM中计算开销最大的一个分量,并且证明了在大规模并行时,海洋模式POP的性能主要受到正压模态中通信瓶颈的影响。 
 

 
正压模态的可扩展性较差主要原因是由于需要隐式的求解一个椭圆方程。 
海洋模式POP中的正压模态可以近似为$Ax=b$,它采用常规的预处理共轭梯度法来求解这个线性系统。 
预处理共轭梯度法每一步迭代过程中都涉及到两次求內积操作。 
当采用成千上百个处理器核心时, 做內积所需要的全局通信和同步操作就会是一个主要的瓶颈。 
目前有很多减少预处理共轭梯度法的负面影响的方法。 
一些方法尝试减少全局通信的开销\cite{dAzevedo1999lapack},以及将计算与通信相重叠\cite{beare1997optimisation}。 
另外一些方案尝试使用陆地点移除和负载均衡\cite{dennis2007inverse, dennis2008scaling} 的方法来减少进程数,进而减少其相应的全局归约操作的开销。 


这些工作会有一定的效果。 
但是他们并没有从根源上解决这个问题, 也就是没有消除掉全局归约操作的开销。 
本章中,我们首先给出预处理共轭梯度法的复杂度模型,进而定量的分析正压模态的可扩展性。 
通过这个模型,我们确信随着进程数的增加而逐渐增大的全局通信开销是正压模态可扩展性的瓶颈。 
另外,  我们设计一个新的基于传统Stiefel迭代(CSI)的新的可扩展的求解器,以此解决可扩展性瓶颈。 
CSI方法的迭代参数是用系数矩阵$A$的特征值谱计算得到,而不需要用到迭代过程中通信密集型的残差的內积计算。
这个不需要全局归约操作的特点, 我们的CSI求解器相比于原始的预处理共轭梯度法在大规模并行环境中有更好的可扩展性。  
我们利用Lanczos方法来估计系数矩阵$A$的特征值。 
Lanczos方法可以构造出一个规模小得多的三对角矩阵$T$ , 这个矩阵的特征值能够逐渐的逼近原系数矩阵 $A$的特征值,从而解决了直接求解系数矩阵特征值的难题。 
CSI中估计特征值的额外开销比正压模态执行一步的开销还要小。 
实验表明, 预处理共轭梯度法在小于1,000核上市可扩展性比较好,但是当使用超过5,000核时,预处理共轭梯度法的执行时间反而增加了。 
与之形成鲜明对比的是,CSI方法在超过10,000核上仍然保持较好的可扩展性, 它使得正压模态在15,000核的执行时间从原来的41.96秒 下降到6.67秒。 


为了完整的推导我们的新的P-CSI求解器,这里我们简要的描述一下海洋模式POP中的控制方程。
海洋模式中的原始动量和连续性方程可以表达如下:
\begin{align}
&\frac{\partial }{\partial t} \textbf{u} +\mathcal{L}(\textbf{u}) + f\times \textbf{u} = - \frac{1}{\rho_0}\nabla p +F_H(\textbf{u}) +F_V(\textbf{u}) \label{eq:momen}\\
&\mathcal{L}(1) = 0 \label{eq:continuous}
\end{align}
这里$\mathcal{L}(\alpha ) = \frac{\partial }{\partial x} (u\alpha)  +\frac{\partial }{\partial y} (v\alpha) +\frac{\partial }{\partial z} (w\alpha)$, 它在$\alpha =1$时与平流算子等价, $x,y,z$ 分别是水平和垂直坐标变量, $\textbf{u} = [u,v]^T$ 水平速度向量, $w$ 垂直速度, $f$ 表示科氏力系数,  $p$ 和 $\rho_0$ 分别表示压力和密度, $F_H$ 和 $F_V$分别表示水平和垂直耗散项 \cite{smith2010parallel}。  
为了求解这个三维的原始方程,海洋模式分量将时间积分分解为两个模态。 一个是求解三维动力学和热动力学过程的斜压模态,另一个则是求解二维海表高度(SSH)的变化。

%----------------------------------------------------------------------------
\section{正压模态} \label{solver:mode}


  
在静力平衡近似条件下,POP中海底深度为$z$的位置上的压力可以分解为两个部分:  
\begin{align}
\displaystyle p = p_h + p_s = \int^0_z g\rho dz +p_s
\end{align}
这里$p_h$表示静力平衡压力,$p_s$表示由于海表自由面波动而引起的海表压力。 
正压模态的控制方程是由原始动量方程和连续性方程(\ref{eq:momen},\ref{eq:continuous}) 从海洋底部到海洋表面垂直积分而得到的:
\begin{align}
&\displaystyle \frac{\partial \textbf{U} }{\partial t}  = -g \nabla \eta + F  \label{eq:baro_mon}\\
&\displaystyle \frac{\partial \eta }{\partial t} = - \nabla \cdot H\textbf{U} + q_w  \label{eq:baro_con}
\end{align}
这里  $\textbf{U} =  \frac{1}{H+\eta}\int_{-H}^\eta dz \textbf{u}(z) \approx \frac{1}{H }\int_{-H}^0 dz \textbf{u}(z)$是正压速度的垂直积分,
$\eta = p_s/{\rho_0g}$是海表高度,$H$是海洋底部的深度,$q_w$单位海表面积的活水通量,$g$表示由于地球动力引起的加速度而$F$表示对海洋动量方程中除了与时间趋势和海表压力梯度相关的所有项做垂直积分所得到的量(参见方程\ref{eq:momen})。
 
垂直方向上的边界条件是
\begin{align}
\label{eq:bound_w}
w = \left\{ \begin{array}{ll}
\frac{\partial}{\partial t} \eta  +\textbf{u}\cdot\nabla \eta - q_w, & z = \eta  \\
0, & z = -H
\end{array} \right.
\end{align}
 
值得一提的是,为了简化求解过程,正压连续性方程 (\ref{eq:baro_con}) 中忽略掉了边界条件中的小项 $\nabla \eta$\cite{smith2010parallel}。
为了能够使用更长的时间步长,海洋模式分量的正压模态中采用了隐式格式,并且将椭圆方程简化成一个线性系统。 通过时间差分,方程  (\ref{eq:baro_mon})和(\ref{eq:baro_con}) 可化为
\begin{align}
&\displaystyle \frac{ \textbf{U}^{n+1} - \textbf{U}^{n-1}}{\tau}  = -g \nabla \eta + F \label{eq:udt} \\
&\displaystyle \frac{\eta^{n+1} - \eta^n }{\tau}  = - \nabla \cdot H\textbf{U} + q_w \label{eq:etadt}
\end{align}
这里 $\tau$ 与差分格式相关的时间步长。 
用方程(\ref{eq:udt})中下一时刻的正压速度来替换方程(\ref{eq:etadt})中的正压速度, 就得到了一个关于海表高度$\eta$的椭圆系统
\begin{equation}
\label{eq:sshdiscret}
     [-\nabla\cdot H \nabla + \frac{1}{g  \tau^2}]\eta^{n+1}
           = -\nabla\cdot H[\frac{\textbf{U}^{n-1}}{g \tau} + \frac{F}{g}] + \frac{\eta^n}{g\tau^2} +\frac{q_w}{g\tau}
\end{equation}
 
为了简单,我们将椭圆方程(\ref{eq:sshdiscret})重新标记为
\begin{equation}
\label{eq:ssh}
[-\nabla \cdot H\nabla +\frac{1}{g  \tau^2}]\eta^{n+1} = \psi(\eta^n,\eta^{n-1},\tau)
\end{equation}
这里 $\psi$表示一个关于当前时刻和上一时刻$\eta$状态的函数。
 

\begin{figure}%[!htbp]
\centering
\includegraphics[width=12cm, height=6.5cm]{grid_domain.pdf}
\caption[] {海洋模式分量中网格划分\label{fig:grid1}}
\end{figure}

\subsection{正压模态的求解}
\label{related:barosolver}
 

\section{正压线性方程组}
\label{related:baroproperty}

如图\ref{fig:grid1}所示, POP采用在水平方向上采用Arakawa B网格\cite{smith2010parallel},并且采用九点差分格式对方程(\ref{eq:ssh})进行离散。得到如下离散格式:
\begin{align}
    & \nabla\cdot H \nabla \eta  =\frac{1}{\Delta y}\delta_x \overline{[\Delta y H  \delta_x\overline{\eta}^y]}^y +\frac{1}{\Delta x}\delta_y \overline{[\Delta x H  \delta_y\overline{\eta}^x]}^x \label{eq:nabla2}
  \end{align}

这里 $\Delta_\xi$和$\delta_\xi$  ($\xi \in \{x, y\} $)分别为有限差分和它们相应的偏导,  $\delta_\xi (\cdot) $和$\overline{(\cdot)}^\xi $ 分别表示有限差分算子和平均算子。  
\begin{align}
&\delta_\xi \psi = [\psi (\xi+\Delta_\xi/2) -\psi(\xi-\Delta_\xi/2)]/\Delta_\xi \\
&\overline{\psi}^\xi  =[\psi (\xi+\Delta_\xi/2) +\psi(\xi-\Delta_\xi/2)]/2
\end{align}
%----------------------------------------------------------------------------


为了避免极点问题,POP采用偏移的或者三极点的广义正交网格。这使得方程的系数非常的繁杂。
为了能够简单而不失偏颇,我们将给出系数矩阵在均匀的网格和给定的不变的海底深度$H$的情况下的显式表达式。
 

表达式\ref{eq:nabla2}此时为
\begin{align}
 [\nabla\cdot H \nabla \eta]_{i,j}&= -\frac{H}{S_{i,j}}(A_{i,j}^O\eta_{i,j}+A_{i,j}^{NW}\eta_{i-1,j+1}+A_{i,j}^N\eta_{i,j+1} \nonumber\\
 &+A_{i,j}^{NE}\eta_{i+1,j+1}+A_{i,j}^W\eta_{i-1,j} +A_{i,j}^E\eta_{i+1,j} \nonumber\\
& +A_{i,j}^{SW}\eta_{i-1,j-1} +A_{i,j}^S\eta_{i,j-1}+ A_{i,j}^{SE}\eta_{i+1,j-1})
\end{align}
这里$S_{i,j}  = \Delta x\Delta y$, $A_{i,j}^{\chi } ( \chi \in \mathcal{Q} = \{NW,NE, SW, SE, W, E, N, S\})$ 分别表示网格点 $(i,j)$和它自己、以及它的邻居点之间的在 九点差分格式中的系数(\ref{eq:nabla2})。 这些系数由$\Delta x$, $\Delta y$, $\tau$和$H$决定。
\begin{equation} \label{defineA}
\begin{aligned}
&\alpha  = \frac{ \Delta y}{ \Delta x }, \quad \beta  = 1/\alpha \\
&A_{i,j}^{NW} = A_{i,j}^{NE} =A_{i,j}^{SW} = A_{i,j}^{SE} = - (\alpha  +  \beta  )/4 \\
&A_{i,j}^{W} = A_{i,j}^{E} = (  \beta  -\alpha  )/2 \\
&A_{i,j}^{N} = A_{i,j}^{S} = (\alpha  -\beta )/2 \\
&A_{i,j}^{O} =   \alpha   +\beta  \\
\end{aligned}
\end{equation}


%&\alpha_1 = \frac{2\Delta y}{2\Delta x_j +\Delta x_{j+1}2}\\
%& \alpha_2= \frac{2\Delta y}{ \Delta x_j +\Delta x_{j-1}  } \\
%&\alpha_3 = \frac{\Delta y(\Delta x_{j-1}+2\Delta x_j +\Delta x_{j+1})}{2 (\Delta x_j +\Delta x_{j-1})(\Delta x_j +\Delta x_{j+1})} \\
%& \beta_1 = \frac{(\Delta x_j +\Delta x_{j+1})}{8 \Delta y} \\
%& \beta_2 = \frac{(\Delta x_j +\Delta x_{j-1})}{8 \Delta y} \\
%& \beta_3 = \frac{(\Delta x_{j-1}+2\Delta x_j +\Delta x_{j+1})}{8 \Delta y} \\
%&A_{i,j}^{NW} = A_{i,j}^{NE} = \alpha_1 + \beta_1 \\
%&A_{i,j}^{SW} = A_{i,j}^{SE} = \alpha_2 +  \beta_2 \\
%&A_{i,j}^{W} = A_{i,j}^{E} = \alpha_3 - 2 \beta_3 \\
%&A_{i,j}^{N} = -2\alpha_1  + 2 \beta_1 \\
%&A_{i,j}^{S} = -2\alpha_2  + 2 \beta_2  \\
%&A_{i,j}^{O} = -2\alpha_3  -4 \beta_3  \\
%Then, the coefficients between the given point $(i,j)$和its other neighbors can be computed from $A^n$, $A^e$和$A^{ne}$ on its neighbors。
方程\ref{eq:ssh}在给定点$(i,j)$上的离散格式则可以表示为  
\begin{align}
\label{eq:sten}
&(A_{i,j}^O+\phi ) \eta_{i,j}+A_{i,j}^{NW}\eta_{i-1,j+1}+A_{i,j}^N\eta_{i,j+1} +A_{i,j}^{NE}\eta_{i+1,j+1}+A_{i,j}^W\eta_{i-1,j}  \nonumber\\
& +A_{i,j}^E\eta_{i+1,j} +A_{i,j}^{SW}\eta_{i-1,j-1} +A_{i,j}^S\eta_{i,j-1}+ A_{i,j }^{SE}\eta_{i+1,j-1}= \frac{S_{i,j}}{H}\psi_{i,j}
\end{align}
这里$\phi = \frac{S_{i,j}}{g  \tau^2H}$ 是与单位网格大小、海底深度和时间步长有关的一个变量。

\begin{figure}
\centering
\includegraphics[height=8cm]{SparsePatternSample}
\caption[] {大小为$30\times 15$的网格上所得到的系数矩阵的稀疏模式。 \label{fig:spy}}
\end{figure}
因此椭圆方程(\ref{eq:sshdiscret})就变成了关于$\eta$的一个线性方程组,也就是 $Ax= b$。 其中$A$表示由系数 $A^*$构成的快对角矩阵。
方程 \ref{defineA}和 \ref{eq:sten}表明$A$是有水平网格大小、海底深度和时间步长决定。 
他们之间具体的关系将在第\textbf{ref}节中深入讨论。
%In the POP, only the nonzero elements are stored。
方程(\ref{eq:sten})同时还表明 $A$ 每一行只有九个非零元素, 也就是说$A$是一个系数矩阵。 图\ref{fig:spy}展示了$A$的稀疏模式。

\subsubsection{特征值谱和条件数}
\label{solver:Algorithm:condition}

POP中的系数矩阵 $A$ 是正定对称的\cite{smith2010parallel},因此它的特征是都是正实数\cite{stewart1976positive}。
假设系数矩阵的特征值谱\cite{golub2012matrix} 是 $\mathcal{S} = \{\lambda_1, \lambda_2, \cdots, \lambda_N\}$,这里 $\lambda_{min} = \lambda_1 \le \lambda_i \le \lambda_\mathcal{N} = \lambda_{max}$( $1<i <\mathcal{N}$, $\aleph$ 为 $A$的大小 )表示$A$的特征值.

利用 Gershgorin圆盘定理\cite{bell1965gershgorin}可知,对于任意的 $\lambda \in \mathcal{S}$,都存在一个数对 $(i,j)$ 满足
\begin{align}
&|\lambda -  (A_{i,j}^O + \phi ) | \le \sum_{\chi \in \{NW,NE,SW,SE,W,E,N,S\}}|A_{i,j}^\chi|
\end{align}
由方程\ref{defineA}给出的系数的定义,我们可以得到如下结论 
\begin{align} \label{eigsGersh}
&\lambda_{max} \le  \max (  5\alpha - \frac{1}{\alpha}, \frac{5}{\alpha}- \alpha) +\phi   \\
&\lambda_{min} \ge 2\min (  \alpha - \frac{1}{\alpha},\frac{1} {\alpha} -  \alpha) + \phi
\end{align}\\

 
这个结论说明,当网格横纵比例越接近于1时,最大特诊值的上界会随之减少,而最小特征值的下界会随之变大。
也就是说,系数矩阵特征值谱的半径($[\lambda_{min}, \lambda_{max}]$)随着网格横纵比向着1的接近而变小。 
当水平网格的横纵比等于1时,即$ \alpha = \frac{ \Delta y}{ \Delta x} = 1$,我们可以得到$\lambda_{max} \le  4 +\phi$,$\lambda_{min} \ge   \phi$。
这时,系数矩阵的条件数(即 $\kappa=  \lambda_{max}/\lambda_{min}$),由于是由特征值谱的半径的所决定的,也会随着网格横纵比向1的靠近而变小。 

\begin{figure}[ht]
\centering
\includegraphics[height=8cm]{conditionNumberAspectRatio}
\caption[] {网格横纵比例和系数矩阵条件数之间的关系。 \label{fig:conditionNumberRatio}}
\end{figure}




图\ref{fig:conditionNumberRatio}所给出的结果验证了我们的这个结论。图\ref{fig:conditionNumberRatio}中给出的是固定的网格点数$\mathcal{N} = 20\times 20$和固定的$\phi = 0$, 在区间$(1, +\infty)$上,系数矩阵的条件数随着网格横纵比例的增加而增加,而在区间$(0,1)$上,系数矩阵的条件数随着横纵比例的增大而变小。
特征值在当网格横纵比例等于1时取到最小值。
 
\begin{figure}[ht]
\centering
\includegraphics[height=8cm]{conditionNumberTimestep}
\caption[] {时间步长和系数矩阵条件数之间的关系。 \label{fig:conditionNumberDt}}
\end{figure}



特征值的下界是$\phi=\frac{S }{g \tau^2 H}$,这个变量是由时间步长以及水平网格的面积和海底深度的比例确定的。
它表明,特征值的下界会随着时间步长的增大而减小。 
相应的,系数举证的 条件数也就随之增大。
图\ref{fig:conditionNumberDt}中证明了我们的这个推论。
这个图中的实验采用的是固定的网格点数$\mathcal{N} = 20\times 20$,和固定的网格横纵比$\Delta x /{\Delta y} = 1$。 

\begin{figure}[ht]
\centering
\includegraphics[height=8cm]{conditionNumberGridSize}
\caption[] {网格点的数目和系数矩阵条件数之间的关系。 \label{fig:conditionNumbGrid}}
\end{figure}
 
不考虑时间步长的因素(假设$\phi=0$)时,上面的分析表明,当网格的横纵比等于1时,不论网格点的个数是多少, 系数矩阵的特征值谱半径限制在 $(0,4)$ 区间内。
但是,系数矩阵的条件数的变化却可能会很大,因为当网格点的数量$\mathcal{N}$ 不断增加时,最小的特征值会组件的逼近零。 
图\ref{fig:conditionNumbGrid} 中解释了条件数和网格点数目之间的关系。 



\subsection{正压模态通信瓶颈}
\label{related:bottleneck}

我们这里简单的从两个方面介绍一下相关工作: 一是提高通用的并行环境中共轭梯度法效率的工作,二是针对海洋模式中的共轭梯度法的效率的改进。
在第一个方面, 减少共轭梯度法的全局通信的开销在这个算法被并行化之后就一直是研究的热点。 
这方面工作比较好的一个综述可以参见\cite{ghysels2014}。 
这个领域最早的成果是在标准的共轭梯度法的基础上做算法的修正从而减少全局归约操作的次数, 比如在海洋模式POP中被采用的ChronGear方法\cite{dAzevedo1999lapack} 。 这些方法至今仍然被广泛的使用着。 
早期还有s-步方法\cite{chron1989} 和最近的一些变种(比如论文\inlinecite{hoemmen2010})也能够减少全局通信, 但是这些方法很难结合稍微复杂一点的预处理方法使用。、
另外,最近也有一些工作提出来将全局归约操作和矩阵向量乘积操作利用流水线的方式进行重叠\cite{ghysels2014},从而提高并行共轭梯度法的效率。  
我们采用与论文\inlinecite{gutknecht2002chebyshev}中不一样的思路。
我们完全的摒弃了共轭梯度法,取而代之的是一个更为简单的不包含全局归约操作的迭代算法。 



针对海洋模式, 有很多工作都是关于如何减少求解器中的通信瓶颈的。 
在论文\cite{Worley:2011:PCE:2063384.2063457}中, 他们在正压模态中采用OpenMP并行技术,从而提高了海洋模式在大核数上的性能。 
另一个比较常见的减少通信开销的策略是陆地点的移除\cite{dennis2007inverse,dennis2008scaling}。 
使用空间填充曲线来划分网格,不仅能够改善负载均衡,还能够通过去掉全是陆地点的块来减少需要参与通信的处理器核心的数目,从而减少通信的开销。 
这一方法使得在30,000处理器核心上的模拟速度提升了近一倍。 
另外,早起还有一些工作尝试将并行大洋环流模式中的计算和通信相重叠 \cite{beare1997optimisation},以及通过减少边界缓存区域的大小来减少通信开销。 
尽管这些方法都能够对性能有一定的提升, 但是他们并没有完全的消除掉全局归约操作的瓶颈。 
事实上, 我们的早起论文\inlinecite{hu2013scalable}中给出了一个比较有潜力的工作,通过用Chebyshev迭代方法来替换掉共轭梯度法,
在独立版本的海洋模式POP中取得了很好的效果。 

 
最后,我们放弃了串行效率很高的共轭梯度法,使得我们不得不去寻找正压模态中一个可用的更加高效的预处理子。 
这里,我们提一下为了减少高分辨率海洋模式中正压模态求解的几个预处理技术。 
多项式预处理技术和局部逆预处理技术被证明是加速并行大洋环流模式中的共轭梯度法的收敛的比较好的技术\cite{smith1992parallel}。 
最近,Max Planck研究中心的海洋模式MPIOM采用了一种基于不完全Cholesky分解的预处理子。它能够改善共轭梯度法在大核数上的运行效率 \cite{adamidis2011high}。


  
共轭梯度法中的预处理方法自从上世纪90年代以来就一直备受重视。 
很多的线性系统在采用了适当的预处理子之后,预处理共轭梯度法只需要少数几步迭代就能够收敛。 
然是,大多数的最有效的预处理技术, 比如说不完全的Cholesky分解和不完全的LU分解,在海洋模式中并不是十分的有效。 
海洋模式中的椭圆方程组需要为之设计专门的可并行化的预处理子。 
在1985年,Concus 等人\cite{concus1985block}采用系数矩阵的逆的一个条带状的近似,来对椭圆微分方程上的共轭梯度法进行预处理,结果取得了比其他通用的预处理方法更高的效率。 
Smith等人\cite{smith1992parallel}采用多项式预处理的方法和一个局部近似逆的预处理方法来加速并行大洋环流模式中的共轭梯度法的收敛。 
Adamidis等人 \cite{adamidis2011high} 在全球海洋海冰模式MPIOM中采用了一个不完全的Cholesky预处理子,在提高预处理共轭梯度法的性能的同时,也改善了它的可扩展性。 
Watanabe \cite{Watanabe2006pcg} 设计了一个与重叠区域分解法相结合的预处理共轭梯度法来加速收敛,同时减少了处理单元之间的通信开销。 



\subsection{正压线性方程组的优化}
\label{related:improve}

    
目前已经有很多的研究工作专门研究如何提高海洋模式中隐式自由海表面问题中的椭圆方程的求解效率。 
现在有很多的可选方案来缓解大规模并行环境中的预处理的共轭梯度法所固有的比较差的可扩展性。 
 减少通信的频率同样可以减轻正压模态中的性能瓶颈。 
 早在1997年,  Beare \cite{beare1997optimisation}等人就提出来,通过增大边界通信缓存区域的块的大小以及将重新和计算相重叠可以提高并行大洋环流模式的性能。 
  

这篇文章所给出的方案结合了以上的两条策略,来提高正压求解器的性能。 
P-CSI求解器能够去掉共轭梯度法及其变化形式中所需要的全局通信。 
同时,它支持EVP预处理子。这个EVP预处理子加速求解器的收敛的效果与海洋模式中其他的预处理方法相当。 



预处理共轭梯度法的数据局部性比较差,而且很多操作只能串行执行,导致以上的改进方案的效果都十分有限。 
其他一些工作利用异构加速平台,比如说GPUs \cite{cuomo2012pcg} 和FPGAs \cite{Shida2007}来实现对预处理共轭梯度法的加速。 
Cuomo 等人\cite{cuomo2012pcg}在全球环流海洋数值模式中引入了稀疏近似逆预处方法,并且利用一个科学计算库将其实现在GPU上。 
Shida 等人\cite{Shida2007}将海洋模式中的正压模态移植到FPGAs上,并且发现发现当适当的使用块上内存和流式的直接内存访问,一块100MHz的FPGA卡的性能能与1GHz的处理相当。 
GPUs和FPGAs还能够减少正压模态中的全局通信开销。这些设备跟传统的CPU相比有更强的计算能力和更大的内存,因此我们只需要使用很少的几块加速卡就能够实现只有在大规模并行环境下才能计算的任务。


正压模态的执行时间占整个海洋模式POP的很大一部分,尤其是当它所使用大量的处理器核心时。
目前已经有很多的关于如何优化正压模态性能的工作,其中大部分都是通过减少进程间的通信量或者加速每个进程上的计算速度。 
全局归约操作的开销适合所使用的进程数的多少成正相关的, 所以通信开销随着进程数的增加会逐渐变得不可接受。
OpenMP并行和陆地点移除在海洋模式中是比较常见的减少进程数量及其相应的通信开销的常用策略。 
Worley et al. \cite{Worley:2011:PCE:2063384.2063457} 强烈推荐使用OpenMP的策略来解决斜压模态需要很多的处理器核心来进行计算而正压模态的进程数太多会造成通信瓶颈越大这一矛盾。 
Dennis \cite{dennis2007inverse,dennis2008scaling}提出一个基于最新的空间填充曲线划分算法的负载均衡的策略。 
这个策略不仅能够方便的去全是陆地点的块,同时还能减少通信开销。 因为通过去掉陆地块,模式运行所需要的进程数就随之减少,相应的通信开销也随之减少。 
他们的实验表明,在接近30,000个处理器核心上,新的策略使得模式的模拟速度增加了近一倍。 
减少通信的频率也是一种减少正压模态中通信开销的方法。 
早在1997年, Beare \cite{beare1997optimisation}等人就提出了通过增大边界缓存区域的大小以及重叠计算和通信部分的手段来提高并行大洋环流模式的性能。 



Adamidis等人\cite{adamidis2011high} 在全球海洋海冰耦合模式MPIOM中实现了一个基于不完全 Cholesky分解的预处理子,来提高预处理共轭梯度法的性能。 
Watanabe \cite{Watanabe2006pcg} 等人通过预处理共轭梯度法和重叠的区域分解法相结合,提高了迭代方法的收敛速度,减少了处理器间的通信开销。 

 

Cuomo  \cite{cuomo2012pcg} 等人在数值全球大洋环流模式中引入了一个系数近似逆预处理方法,并且将其是利用一个科学计算代码库将其实现在GPU上。
Shida  \cite{Shida2007}等人将正压模态移植到了 FPGAs, 并且发现100MHz FPGAs 和1GHz CPU处理器有着相同的性能。 
GPUs 和 FPGAs 对于减少全局通信的开销是非常有帮助的。
这些加速设备相比于传统的CPU,有着更强的计算能力更大的内存,因此,计算相同的科学任务,只需要更少的加速设备即可。 进而,全局通信的开销也相应的减少。 

\section{本章小结}
\label{related:Conclude}