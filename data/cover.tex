\thusetup{
  %******************************
  % 注意:
  %   1. 配置里面不要出现空行
  %   2. 不需要的配置信息可以删除
  %******************************
  %
  %=====
  % 秘级
  %=====
  %secretlevel={绝密},
  %secretyear={2100},
  %
  %=========
  % 中文信息
  %=========
  ctitle={高分辨率海洋模式可扩展的\\正压求解方法研究},
  cdegree={工学博士},
  cdepartment={计算机科学与技术系},
  cmajor={计算机科学与技术},
  cauthor={胡勇},
  csupervisor={杨广文教授},
  %cassosupervisor={黄小猛副教授}, % 副指导老师
  %ccosupervisor={某某某教授}, % 联合指导老师
  % 日期自动使用当前时间,若需指定按如下方式修改:
  % cdate={超新星纪元},
  %
  % 博士后专有部分
  cfirstdiscipline={计算机科学与技术},
  cseconddiscipline={系统结构},
  postdoctordate={2009年7月——2011年7月},
  id={编号}, % 可以留空: id={},
  udc={UDC}, % 可以留空
  catalognumber={分类号}, % 可以留空
  %
  %=========
  % 英文信息
  %=========
  etitle={Research on Scalable Barotropic Solver \\ in High Resolution Ocean Models},
  % 这块比较复杂,需要分情况讨论:
  % 1. 学术型硕士
  %    edegree:必须为Master of Arts或Master of Science(注意大小写)
  %             “哲学、文学、历史学、法学、教育学、艺术学门类,公共管理学科
  %              填写Master of Arts,其它填写Master of Science”
  %    emajor:“获得一级学科授权的学科填写一级学科名称,其它填写二级学科名称”
  % 2. 专业型硕士
  %    edegree:“填写专业学位英文名称全称”
  %    emajor:“工程硕士填写工程领域,其它专业学位不填写此项”
  % 3. 学术型博士
  %    edegree:Doctor of Philosophy(注意大小写)
  %    emajor:“获得一级学科授权的学科填写一级学科名称,其它填写二级学科名称”
  % 4. 专业型博士
  %    edegree:“填写专业学位英文名称全称”
  %    emajor:不填写此项
  edegree={Doctor of Philosophy},
  emajor={Computer Science and Technology},
  eauthor={Hu Yong},
  esupervisor={Professor Yang Guangwen},
  %eassosupervisor={Huang Xiaomeng},
  % 日期自动生成,若需指定按如下方式修改:
  % edate={December, 2005}
  %
  % 关键词用“英文逗号”分割
  ckeywords={海洋模式;并行数值算法;可扩展性;预处理;正确性验证},
  ekeywords={Ocean models; parallel numerical method; scalability; preconditioning; verification}
}

% 定义中英文摘要和关键字
\begin{cabstract}
  高分辨率气候模式的模拟所消耗的计算资源与日俱增。
  在众多分量模式中,海洋模式在高分辨网格的配置下计算开销最为突出。
  以被广泛使用的公共地球系统模式(CESM)为例, 在很多真实运行的例子中,并行海洋模式(POP)都是CESM中计算资源消耗最大的一个分量模式。 
  由于正压模态中求解线性方程组所用的共轭梯度法(CG)在大核数并行时可扩展很差,正压模态在大规模并行时成为高分辨率POP的性能瓶颈。
  为了提高POP和CESM的可扩展性,本文提出了一个高可扩展的正压求解器和一个高效的并行预处理子,并且利用统计方法验证了新的求解器和预处理子的正确性。

  首先,通过一个性能评估模型,CG方法每步迭代中计算残差所需要的全局通信被确定为正压求解器的性能瓶颈。 
  为了消除全局通信,本文提出了一个基于Chebyshev迭代方法的求解器(记为CSI)。
  CSI方法不需要利用每一步的残差来确定迭代方向,而是利用系数矩阵的最大最小特征值,
  从而去掉了时间开销很大的全局通信操作,极大的改善了正压模态的可扩展性。

  其次,为了进一步提高POP正压模态的性能,本文设计了一个基于误差向量传播方法(EVP)的并行块预处理子,使得CSI收敛更快。 
  EVP是求解由椭圆偏微分方程离散得到的线性方程组的一个十分高效的方法。
  新的预处理方法利用海洋模式并行划分的特点,在每一个进程的数据块上分别使用EVP方法求解,最终使得正压求解器的迭代步数缩短到了原来的三分之一左右。 
  结合CSI方法和EVP并行块预处理子,正压求解器的性能得到极大的提高。 在16,875核上,新的正压求解器在高分辨POP中取得了5.2倍的加速,并且使得整个POP的模拟速率提高了1.7倍。 
  

  最后,本文实现了一个基于集合模拟的海洋模式一致性检测工具,用来评估模式的改动是否会造成模拟结果与原始结果不一致。
  由于模式本身的不确定性,现存的验证工具很难对模拟结果做出正确的评估。
  本文利用集合模拟得到的数据集合作为评估新的模拟结果的依据,解决了模式结果的验证难题。   
  基于集合模拟的一致性检测工具在多组标准测试中表现良好,并且证明了新的求解器不会导致不一致的模拟结果。
  基于以上结果,本文提出的新的求解器最终被CESM采纳为默认的海洋模式正压求解器。

\end{cabstract}

% 如果习惯关键字跟在摘要文字后面,可以用直接命令来设置,如下:
% \ckeywords{\TeX, \LaTeX, CJK, 模板, 论文}

\begin{eabstract}
  High-resolution climate simulations are increasingly in demand and
  require tremendous computing resources. 
  In the Community Earth System Model (CESM), the Parallel Ocean Model (POP) is
  computationally expensive for high-resolution grids (e.g., 0.1 degree) and is frequently the least scalable component of CESM for certain
  production simulations. 
  In particular, the Conjugate Gradient (CG), used to solve the elliptic system of
  equations in the barotropic mode of POP, scales poorly at the high core counts, which is problematic for high-resolution simulations. 
  To improve the scaling of POP, and, therefore CESM, we focus on optimizing the barotropic solver by eliminating global reductions and developing a more effective preconditioner. 


  Firstly, with a theoretical evaluation model, the communication costs in the
  barotropic solver is demonstrated to be the bottleneck. 
  To mitigate this problem, we implement a Chebyshev-type iterative
  method in POP (called CSI), which requires far fewer global
  reductions than CG, thus breaking the scaling bottleneck in the barotropic solver. 


  Secondly, to further improve the performance, an effective block preconditioner based on the Error Vector Propagation Method (EVP) is developed, which improves solver convergence in the POP barotropic mode.
  The P-CSI and EVP preconditioning results in a 5.2x speedup of
  the barotropic mode in high-resolution POP on 16,875 cores, which
  yields a 1.7x speedup of the overall POP simulation.  

  Thirdly, it is ensured that the new solver produces an ocean climate consistent with the original one via an ensemble-based statistical method. 
  The chaotic nature of the ocean dynamics makes it hard to verify even tiny modifications in the model with exsiting methods.
  Instead, we quantify the variability of the ocean model by the statistical distributions of an ensemble and verify new results based on these distributions. 
  The resulting verification tool represents good performance in dectcting inconsisitency, and verifies the implementation of our new solver into CESM. 



\end{eabstract}

% \ekeywords{\TeX, \LaTeX, CJK, template, thesis}
