\thusetup{
  %******************************
  % 注意:
  %   1. 配置里面不要出现空行
  %   2. 不需要的配置信息可以删除
  %******************************
  %
  %=====
  % 秘级
  %=====
  %secretlevel={绝密},
  %secretyear={2100},
  %
  %=========
  % 中文信息
  %=========
  ctitle={可扩展的高分辨率海洋求解器},
  cdegree={工学博士},
  cdepartment={计算机科学与技术系},
  cmajor={计算机科学与技术},
  cauthor={胡勇},
  csupervisor={杨广文教授},
  %cassosupervisor={黄小猛副教授}, % 副指导老师
  %ccosupervisor={某某某教授}, % 联合指导老师
  % 日期自动使用当前时间,若需指定按如下方式修改:
  % cdate={超新星纪元},
  %
  % 博士后专有部分
  cfirstdiscipline={计算机科学与技术},
  cseconddiscipline={系统结构},
  postdoctordate={2009年7月——2011年7月},
  id={编号}, % 可以留空: id={},
  udc={UDC}, % 可以留空
  catalognumber={分类号}, % 可以留空
  %
  %=========
  % 英文信息
  %=========
  etitle={Scalable Barotropic Solver in High Resolution Ocean Models},
  % 这块比较复杂,需要分情况讨论:
  % 1. 学术型硕士
  %    edegree:必须为Master of Arts或Master of Science(注意大小写)
  %             “哲学、文学、历史学、法学、教育学、艺术学门类,公共管理学科
  %              填写Master of Arts,其它填写Master of Science”
  %    emajor:“获得一级学科授权的学科填写一级学科名称,其它填写二级学科名称”
  % 2. 专业型硕士
  %    edegree:“填写专业学位英文名称全称”
  %    emajor:“工程硕士填写工程领域,其它专业学位不填写此项”
  % 3. 学术型博士
  %    edegree:Doctor of Philosophy(注意大小写)
  %    emajor:“获得一级学科授权的学科填写一级学科名称,其它填写二级学科名称”
  % 4. 专业型博士
  %    edegree:“填写专业学位英文名称全称”
  %    emajor:不填写此项
  edegree={Doctor of Engineering},
  emajor={Computer Science and Technology},
  eauthor={Hu Yong},
  esupervisor={Professor Yang Guangwen},
  eassosupervisor={Huang Xiaomeng},
  % 日期自动生成,若需指定按如下方式修改:
  % edate={December, 2005}
  %
  % 关键词用“英文逗号”分割
  ckeywords={海洋模式, 并行数值求解器, 高分辨率, 可扩展性},
  ekeywords={Ocean models, parallel linear solvers, high resolution, scalable}
}

% 定义中英文摘要和关键字
\begin{cabstract}
高分辨率气候模拟的需求量与日剧增,它所消耗的计算资源也是极其巨大。 
  在公共地球系统模式(CESM)中, 并行海洋模式(POP)在高分辨网格(比如0.1度)的配置下的计算量极其巨大,在很多真实运行的实例中都是公共地球系统模式CESM中可扩展性最差的一个分量模式。 
  尤其是在正压模态中求解椭圆方程所用到的改进后的预处理共轭梯度法(PCG), 在大核数并行使可扩展很差, 从而成为了高分辨率的模拟中的一个主要瓶颈。 
  此论文中,我们证明了正压求解器中的通信开销在整个海洋模式POP的运行时间中所占的比随着并行核数的增加而增加。 
  为了解决这一问题, 我们在海洋模式POP中实现了一个预处理的Chebyshev迭代方法(称之为P-CSI方法), 它所需要的全局归约操作要比预处理共轭梯度法少很多。 
  我们还研发了一个高效的基于误差向量传播方法的块预处子,它是的P-CSI方法能在比较少的迭代步内收敛。 
  我们还证明了P-CSI由于可扩展性得到改善, 在16,875核上对高分辨海洋模式POP的正压模态加速了5.2倍,并且是的整个POP的模拟速率提高了1.7倍。 
  最后,我们还通过一个基于集合模拟的统计学方法来证明使用了我们的新的求解器后产生的模拟结果是与原始结果相容的。
本论文主要研究提高海洋模式中并行求解正压模态的可扩展性,解决高分辨率下可扩展性差的问题。 
  本文的创新点主要有:
  \begin{itemize}
    \item 提出了给予Chebyshev的海洋模式正压求解器;
    \item 基于误差向量传播算法的预处理子;
    \item 采用了基于统计的模式评估方法。 
  \end{itemize}

\end{cabstract}

% 如果习惯关键字跟在摘要文字后面,可以用直接命令来设置,如下:
% \ckeywords{\TeX, \LaTeX, CJK, 模板, 论文}

\begin{eabstract}
  High-resolution climate simulations are increasingly in demand and
  require tremendous computing resources. In the Community
  Earth System Model (CESM), the Parallel Ocean Model (POP) is
  computationally expensive for high-resolution grids (e.g.,
  0.1 degree) and is frequently the least scalable component of CESM for certain
  production simulations. In particular, the modified Preconditioned
  Conjugate Gradient (PCG), used to solve the elliptic system of
  equations in the barotropic mode, scales poorly at the high core
  counts, which is problematic for high-resolution simulations. In
  this work, we demonstrate that the communication costs in the
  barotropic solver occupy an increasing portion of the total POP
  execution time as core counts are increased. To mitigate this
  problem, we implement a preconditioned Chebyshev-type iterative
  method in POP (called P-CSI), which requires far fewer global
  reductions than PCG.  We also develop an effective block
  preconditioner based on the Error Vector Propagation Method to
  attain a competitive convergence rate for P-CSI.  We demonstrate
  that the improved scalability of P-CSI results in a 5.2x speedup of
  the barotropic mode in high-resolution POP on 16,875 cores, which
  yields a 1.7x speedup of the overall POP simulation.  Further,
  we ensure that the new solver produces an ocean climate consistent with the original
  one via an ensemble-based statistical method.
\end{eabstract}

% \ekeywords{\TeX, \LaTeX, CJK, template, thesis}
