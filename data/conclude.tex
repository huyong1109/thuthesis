\chapter{总结与展望}
\label{cha:conclusion}

\section{总结}
\label{sec:conclude}


这些工作会有一定的效果。 
但是他们并没有从根源上解决这个问题, 也就是没有消除掉全局归约操作的开销。 
本章中,我们首先给出预处理共轭梯度法的复杂度模型,进而定量的分析正压模态的可扩展性。 
通过这个模型,我们确信随着进程数的增加而逐渐增大的全局通信开销是正压模态可扩展性的瓶颈。 
另外,  我们设计一个新的基于传统Stiefel迭代(CSI)的新的可扩展的求解器,以此解决可扩展性瓶颈。 
CSI方法的迭代参数是用系数矩阵$A$的特征值谱计算得到,而不需要用到迭代过程中通信密集型的残差的內积计算。
这个不需要全局归约操作的特点, 我们的CSI求解器相比于原始的预处理共轭梯度法在大规模并行环境中有更好的可扩展性。  
我们利用Lanczos方法来估计系数矩阵$A$的特征值。 
Lanczos方法可以构造出一个规模小得多的三对角矩阵$T$ , 这个矩阵的特征值能够逐渐的逼近原系数矩阵 $A$的特征值,从而解决了直接求解系数矩阵特征值的难题。 
CSI中估计特征值的额外开销比正压模态执行一步的开销还要小。 
实验表明, 预处理共轭梯度法在小于1,000核上市可扩展性比较好,但是当使用超过5,000核时,预处理共轭梯度法的执行时间反而增加了。 
与之形成鲜明对比的是,CSI方法在超过10,000核上仍然保持较好的可扩展性, 它使得正压模态在15,000核的执行时间从原来的41.96秒 下降到6.67秒。    
目前已经有很多的研究工作专门研究如何提高海洋模式中隐式自由海表面问题中的椭圆方程的求解效率。 


由于海洋模式POP中的ChronGear正压求解在大核数上的不良表现, 高分辨的公共地球系统模式CESM的可扩展性也不太好。
这篇文章通过采用一个需要更少全局归约操作的新的求解器和一个专门为正压模态设计的基于误差向量传播方法的块预处理子,最终改进了求解器的性能。 
我们评估了这个最终的求解器,也就是采用EVP块预处理的P-CSI方法,在两个经常用来运行公共地球系统模式的超级计算机上的性能,并且发现新的求解器相对于原始的求解器有了高达5倍的性能提升。
同时,我们通过海洋模式POP中的一个基于集合模拟的一致性检验工具证明了我们的新的求解器不会影响海洋模式的模拟结果。 
这也使得我们的新的求解器最终可以出现在公共地球系统模式的下一个版本中。 
新的求解器将会有助于未来的低分辨率和高分辨率公共地球系统模式的模拟, 尤其是过去被海洋模式POP分量影响到可扩展性的全耦合模式。 



\section{未来工作展望}
\label{sec:futurework}

本文的主要贡献是提出了一个针对高分辨海洋模式的正压求解器,并专门为海洋模式在大规模并行环境下的运行设计了一个高效的预处理技术。
本文第\ref{solver:exp:ideal}节的实验部分证明了本文所提出的高可扩展的海洋模式是通用的。目前已经有相关计划将我们的求解器实现在其它海洋模式中,比如台湾的海洋模式TIMCOM\cite{tseng2011parallel}。 

本文所提出的基于误差向量传播方法的预处理方法,是针对高分辨率海洋模式在大规模并行环境下运行而设计的。它的特点是利用这种情况下进程所管辖的块之间的独立性。 为了更加通用,我们设想下一步能将基于误差向量传播方法的预处理和多重网格相结合。 
多重网格方法的特点是天然并行。 


最后,针对求解器和预处理技术所带来的改动,本文还设计了相应的正确性验证工具。目前,此工具已经可以应用于某些实际问题中。未来,我们希望将它发展成为更为通用的验证工具。 



