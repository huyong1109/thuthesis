\chapter{引言}
\label{cha:intro}

近年来,高分辨率全球气候模式已经成为了理解气候现象和预测未来气候变化的一个不可或缺的手段。地球系统
目前,以美国国家大气研究中心(the National Center for Atmospheric Research, NCAR) 主导开发的
公共地球系统模式(the Community Earth System Model, CESM) 是应用最为广泛的全球气候模式之一。
在政府间气候变化专门委员会(the
Intergovernmental Panel on Climate Change, IPCC) 第五次评估报告(the Fifth Assessment
Report, AR5)中,直接或间接使用公共地球系统模式的预测结果占了相当大的比例 \cite{stocker2013ipcc}。
 

公共地球系统模式是一个全耦合的气候系统模式,它包括了大气、海洋、海冰和路面等分量模式。
其中,海洋模式需要模拟的气候现象的时间尺度和空间尺度跨度都很大。 
海洋细粒度涡旋的空间尺度只有$\mathcal{O}$(10 - 100 km), 比与之在动力学角度上相对应的大气中的天气系统的尺度要小一到两个量级。
同时,深海环流的时间尺度长达数个世纪甚至几千年,比大气中相对应的时间尺度要长几个量级。
对更高分辨率和更长模拟时间的要求,使得全球涡分辨的海洋气候模式的计算开销要比大气模式的开销大出很多\cite{bryan2010frontal,mcclean2011prototype,graham2014importance}。


更为重要的是,气候模式通常需要模拟几十年甚至数百年。如此长期的模拟需求使得气候模式在采用高分辨率时的计算开销在目前的计算资源下完全无法承受。比如,在政府间气候变化专门委员会第五次评估报告中,大部分的公共地球系统模式的模拟都是采用的近似1度的海洋模式,组合以1度到2度的大气模式。
近几年超级计算机资源的增长以及高分辨卫星观测数据的出现促进了很多科研工作者研究如何通过提高高分辨率气候模式的并行计算效率,进而减少高分辨率模式的计算开销。


公共地球系统模式中的海洋分量模式,并行海洋模式POP(the Parallel Ocean Model) 求解的是采用静力平衡近似和布辛奈斯克(Boussinesq)近似的三维原始方程组。 
它将时间积分分成两个部分:斜压模态和正压模态。 
斜压模态描述的是三维的动力学和热动力学过程,而正压模态是求解二维的垂直积分后的动量方程和连续性方程。
隐士自由面方法在求解海洋模式正压模态中是一个很常见的选择,因为它能够允许较大的时间步长来高效的计算速度很快(约200m/s)的重力波。
然而,使用这个方法时需要求解大规模椭圆方程组,而这个方程的求解过程在并行海洋模式中可扩展性很差。
事实上,已有研究表明,并行海洋模式的正压求解的比较差的可扩展性主要是由于通信的开销导致的\cite{Worley:2011:PCE:2063384.2063457}。 
如果能优化并行海洋模式的正压求解,将会使得整个公共地球系统模式的性能有较大的提升\cite{dennis2012computational}。 


目前公共地球系统模式中并行海洋模式推荐的线性正压模态求解器是采用对角预处理的Chronopoulos-Gear (ChronGear) 方法
\cite{dAzevedo1999lapack}。 这是一种修正后的预处理共轭梯度法(Preconditioned Conjugate
Gradient method,PCG)。 
ChronGear继承了预处理共轭梯度法的缺陷--迭代过程中需要全局求和,从而使得它不能很好地扩展,并成为了高分辨模拟中的一个性能瓶颈。
为了提高并行海洋模式,同时也是公共地球系统模式的可扩展性,我们通过消除求解器迭代过程中的全局求和操作以及发展一个更高效的预处理子来提高正压求解器的并行计算效率。
我们的主要贡献如下: 


\begin{itemize}
\item We develop a new block parallel preconditioner based on the
Error Vector Propagation (EVP) method \cite{roache1995elliptic} designed to
improve solver convergence in the POP barotropic mode.
我们发展了一个新的基于误差向量传播方法的并行块预处理子\cite{roache1995elliptic},来提高并行海洋模式正压求解器的收敛。 
\item We add a preconditioning interface to the Classical Stiefel Iteration
(CSI) solver explored in \cite{hu2013scalable} and
implement the resulting preconditioned CSI (P-CSI) solver
and new EVP block preconditioner in CESM1.2.0 POP.
我们为传统Stiefel迭代算法(CSI)\cite{hu2013scalable}增加了一个预处理接口,最终在公共地球系统模式1.2.0版本中实现了一个采用了新的基于误差向量传播方法的并行块预处理子的CSI求解器(P-CSI)。
\item We demonstrate an improvement in convergence rate for both ChronGear and
P-CSI when using block EVP.
我们证明了使用块误差向量传播方法作为预处理子能够提高ChronGear方法和P-CSI方法的收敛性。 
\item We obtain a 5.2x speedup of
  the barotropic mode in high-resolution POP due to the improved scalability
of P-CSI with block EVP preconditioning, greatly improving the
scalability of POP (and ultimately CESM) at large core counts.
\item We develop and apply an ensemble-based statistical method to evaluate the impact
of changing the linear solver in POP and ensure that a consistent ocean climate is produced.
\end{itemize}

The remainder of this paper is organized as follows.
Section \ref{se:baro} discusses POP's barotropic solver and its scalability.
Sections \ref{se:psi} and \ref{se:evp} detail the design
of P-CSI for POP and the development of the block EVP preconditioner.
Section \ref{se:exp} compares the scalability of the ChronGear and
P-CSI solvers.  Section \ref{se:ver} verifies the new P-CSI solver
using the ensemble based statistical method.
Finally,  related work and conclusions are presented in Sections
\ref{se:rel} and \ref{se:conc}, respectively.

%{\textbf better to move related work in the next section.}

% You must have at least 2 lines in the paragraph with the drop letter
% (should never be an issue)

\cite{kocher99}

\section{海洋模式}
\section{海洋模式正压求解器}

