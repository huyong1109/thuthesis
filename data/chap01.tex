\chapter{引言}
\label{cha:intro}


High-resolution global climate models have become increasingly important
in recent years as a means for understanding climate variability
and projecting future climate change.  The Community Earth System
Model (CESM), whose development is centered at the National Center for
Atmospheric Research (NCAR), is one of the most widely used global
climate models, and its climate projections are a key component in the
Intergovernmental Panel on Climate Change (IPCC) Fifth Assessment
Report (AR5) \cite{stocker2013ipcc}.
近年来,高分辨率全球气候模式已经成为了理解气候现象和预测未来气候变化的一个不可或缺的手段。地球系统
目前,以美国国家大气研究中心(the National Center for Atmospheric Research, NCAR) 主导开发的
公共地球系统模式(the Community Earth System Model, CESM) 是应用最为广泛的全球气候模式之一。
在政府间气候变化专门委员会(the
Intergovernmental Panel on Climate Change, IPCC) 第五次评估报告(the Fifth Assessment
Report, AR5)中,直接或间接使用公共地球系统模式的预测结果占了相当大的比例 \cite{stocker2013ipcc}。
 

CESM is a fully-coupled climate system model, including atmosphere,
ocean, sea-ice and land components. In particular, the ocean component
is required to represent processes across a broad range of spatial and
temporal scales relevant to climate science. Ocean mesoscale eddies
have spatial scales of $\mathcal{O}$(10 - 100 km), one to two orders of magnitude
smaller than the dynamically analogous weather systems in the
atmosphere. The adjustment time scale for the deep ocean is many
centuries up to a few millennia, again orders of magnitude longer than
the corresponding timescales in the atmosphere. The computational
burden of a global eddy-resolving ocean climate model \cite{bryan2010frontal,mcclean2011prototype,graham2014importance} is
thus increased over that for an atmosphere model by the demand for
finer spatial resolution and longer integration times.
公共地球系统模式是一个全耦合的气候系统模式,它包括了大气、海洋、海冰和路面等分量模式。
其中,海洋模式需要模拟的气候现象的时间尺度和空间尺度跨度都很大。 
海洋细粒度涡旋的空间尺度只有$\mathcal{O}$(10 - 100 km), 比与之在动力学角度上相对应的大气中的天气系统的尺度要小一到两个量级。
同时,深海环流的时间尺度长达数个世纪甚至几千年,比大气中相对应的时间尺度要长几个量级。
对更高分辨率和更长模拟时间的要求,使得全球涡分辨的海洋气候模式的计算开销要比大气模式的开销大出很多\cite{bryan2010frontal,mcclean2011prototype,graham2014importance}。

Moreover, climate model simulations are often run for decades or
even centuries, but these long-term simulations are typically too
computationally expensive to run at high-resolution.  For example,
most CESM simulations in IPCC AR5 are carried out with a nominal
1\degree\space ocean and a 1\degree\space to 2\degree\space atmosphere model.
Recent increases in both supercomputing resources and
high-resolution satellite observations have motivated
efforts to improve the parallel performance of high-resolution climate
models so that they can be run more routinely (and for less cost).

%Recently, the increasing computational power of supercomputers and
%high-resolution satellite observations have inspired much research
%that focuses on adapting high-resolution climate models for massive
%parallelism.
%As noted, in most production simulations, the least scalable
%component of CESM is the ocean model Parallel Ocean Model (POP)
%\cite{dennis2012computational}.

The Parallel Ocean Model (POP) component of CESM solves the
three-dimensional primitive equations with hydrostatic and Boussinesq
approximations \cite{smith2010parallel} and divides the time
integration into two parts: the baroclinic and the
barotropic modes. The baroclinic mode describes the 
three-dimensional dynamic and thermodynamics processes, and the barotropic
mode solves the vertically-integrated momentum and continuity
equations in two dimensions. The implicit free-surface method is a common choice
in barotropic mode in ocean models because it allows a large time step to
efficiently compute the fast gravity mode.  However, this method
requires solving a large elliptic system of equations, which
scales poorly in POP.  In fact, the poor scaling performance of the
barotropic solver in POP, which is dominated by the communication
overhead \cite{Worley:2011:PCE:2063384.2063457}, is well known, and
its optimization will benefit the entire CESM model
\cite{dennis2012computational}.

The currently recommended linear solver for the barotropic mode in CESM
POP is the Chronopoulos-Gear (ChronGear) method
\cite{dAzevedo1999lapack}, a modified Preconditioned Conjugate
Gradient method (PCG), combined with a diagonal preconditioner.
The required global reduction in the ChronGear method does not scale well and
causes a bottleneck for high-resolution simulations.  To improve the scaling
of POP, and, therefore CESM, we focus on optimizing the barotropic
solver by eliminating global reductions and developing a more
effective preconditioner.  In particular, we make the following
contributions:


%MOVE TO RELATED WORK
%There are currently different alternatives to mitigate the poor behavior of the PCG type of solver in the massive parallelization.
%Some approaches attempt to overlap the communication with computation time\cite{beare1997optimisation}.
%Some use the land elimination and load-balance strategies \cite{dennis2007inverse, dennis2008scaling}
%to reduce the number of processes and the associated overhead of global reduction.
%Although these approaches may improve performance, they do not eliminate the major bottleneck of the global reduction.


\begin{itemize}
\item We develop a new block parallel preconditioner based on the
Error Vector Propagation (EVP) method \cite{roache1995elliptic} designed to
improve solver convergence in the POP barotropic mode.
\item We add a preconditioning interface to the Classical Stiefel Iteration
(CSI) solver explored in \cite{hu2013scalable} and
implement the resulting preconditioned CSI (P-CSI) solver
and new EVP block preconditioner in CESM1.2.0 POP.
\item We demonstrate an improvement in convergence rate for both ChronGear and
P-CSI when using block EVP.
\item We obtain a 5.2x speedup of
  the barotropic mode in high-resolution POP due to the improved scalability
of P-CSI with block EVP preconditioning, greatly improving the
scalability of POP (and ultimately CESM) at large core counts.
\item We develop and apply an ensemble-based statistical method to evaluate the impact
of changing the linear solver in POP and ensure that a consistent ocean climate is produced.
\end{itemize}

The remainder of this paper is organized as follows.
Section \ref{se:baro} discusses POP's barotropic solver and its scalability.
Sections \ref{se:psi} and \ref{se:evp} detail the design
of P-CSI for POP and the development of the block EVP preconditioner.
Section \ref{se:exp} compares the scalability of the ChronGear and
P-CSI solvers.  Section \ref{se:ver} verifies the new P-CSI solver
using the ensemble based statistical method.
Finally,  related work and conclusions are presented in Sections
\ref{se:rel} and \ref{se:conc}, respectively.

%{\textbf better to move related work in the next section.}

% You must have at least 2 lines in the paragraph with the drop letter
% (should never be an issue)

\cite{kocher99}

\section{海洋模式}
\section{海洋模式正压求解器}

