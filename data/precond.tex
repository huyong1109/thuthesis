\chapter{海洋模式正压预处理技术}
\label{cha:precond}

\section{本章概述}

第\ref{cha:precond}章中,我们在海洋模式中放弃了串行效率很高的共轭梯度法,而采用收敛速度不如共轭梯度法的CSI方法。
CSI方法由于没有时间开销很大的全局通信,使得最终的CSI求解器的可扩展性要比使用共轭梯度法的CG求解器要高很多。
但是,在当所使用的核心数达到10,000以后,CSI求解的时间开销仍然很难继续减少。 
为了进一步改进CSI求解器的性能,本章中,我们将探讨海洋模式正压模态中可用的高效的预处理子。

第\ref{cha:precond}章中,我们设计的CSI求解器极大的提高了海洋模式正压模态的可扩展性。
CSI方法的主要特点是每步迭代过程中没有了时间开销很大的全局通信,使得CSI每一步迭代的时间比原始的共轭梯度法的时间要短很多。 
这样,尽管达到相同的收敛条件,CSI方法需要的迭代步数通常比共轭梯度法要多,但是从总时间来看,CSI方法仍然要比共轭梯度法快。
如果能够进一步的减少CSI方法的迭代步数,CSI求解器的性能将会进一步的提高。

本章中,我们提出一种基于误差向量传播方法的并行块预处理方法来进一步提高正压求解器的性能。
然后,将新的预处理方法应用于已有的正压求解器中,评测预处理方法的效果。
本章中主要比较经过通信优化之后的预处理共轭梯度法(ChronGear)和加了预处理接口的CSI方法(记为P-CSI)。
这两个方法均为最新的公共地球系统模式的海洋模式分量POP采用正压求解器\cite{yong2015}。 


\section{EVP并行块预处理} 
\label{precond:EVP}

已有很多工作研究高分辨率海洋模式中正压模态求解的预处理技术。 
比如,有研究在并行大洋环流模式采用多项式预处理技术和局部逆预处理技术来加速的共轭梯度法的收敛\cite{smith1992parallel}。 
最近,海洋模式MPIOM采用了一种基于不完全Cholesky分解的预处理子,一定程度改善了共轭梯度法在并行环境中的运行效率\cite{adamidis2011high}。
但是这些方法,要么对收敛速度的加速效果不明显,要么前期计算预处理的开销过大。 

最近的一些工作表明,块预处理技术是一个针对海洋模式正压模态比较有效的并行预处理技术\cite{concus1985block, white2011block}。
块预处理技术,不同于前面介绍的预处理技术,并不考虑对整个系数矩阵进行预处理,而是在不同的分块内进行高效的预处理,然后利用这些预处理块来对整个系数矩阵进行预处理。 
这个技术能够充分地利用POP中由椭圆方程离散化后得到的矩阵的块状结构的性质。
在考虑块内预处理技术的时候,既要考虑到预处理的成本,又要考虑到预处理的效果。
预处理效果最好的是直接计算得到矩阵的逆矩阵,但是由于正压模态中系数矩阵是高度系数的,直接求逆会破坏矩阵的稀疏性。
比如在一个大小为$n\times n$的网格块内,系数矩阵的大小为$n^2\times n^2$。
由于稀疏性,一次矩阵向量乘法操作的复杂度为$\Theta(9n^2)$。 
直接计算出逆矩阵,预处理需要的开销为$\Theta(n^4)$,比原始系数矩阵乘的开销大两个量级,十分不适合作为预处理子。

误差向量传播方法(EVP)给我们提供了一种在块内预处理效果好,并且计算开销小的选。 




\subsection{并行块预处理} 
\label{precond:EVP:Parallel}
 
\begin{figure}
\centering
\includegraphics[height=8cm]{blockpreconditioning}
\caption[] {海洋模式POP中采用九点差分格式得到的系数矩阵的系数模式。这个例子中,总个区域被分成了$3\times3$个互不重叠的小块。
红色正方形中的元素表示块内的点之间的相关系数,而蓝色方块中的点表示第$i$个块中的点与其相邻块内的点之间的相关系数。 \label{fig:blockprecond}}
\end{figure}
 
为了能够更好的描述我们的新的EVP块预处理子,我们先以图\ref{fig:blockprecond}为例,简单的回顾一下常规的块预处理子。 
假使我们将一个在$\mathcal{N} \times \mathcal{N}$个点的网格上的线性系统以大小为$n\times n$的块为单位(图\ref{fig:blockprecond}中块的大小为$\mathcal{N}/3\times \mathcal{N}/3$)进行重排,这样得到的系数矩阵$A$就变成了一个就九对角块状矩阵。
这个块状矩阵每一排都包含九个子矩阵。 
$B_i$(图中的红色方块)表示由第$i$个块内的点之间的相关系数组成的块矩阵,它与$A$有相同的结构,只是大小要小一些 ($n^2\times n^2$)。 
$B_i^e$, $B_i^w$, $B_i^n$ 和$B_i^s$分别表示由第$i$个块内的点与东西南北四个方向上相邻的块内的点之间的关系系数组成的矩阵块,每个块只有分散在$n$行上的$3n$个非零元。
$B_i^{nw}$, $B_i^{ne}$, $B_i^{sw}$和 $B_i^{se}$都只包含一个非零元素, 分别表示第$i$个块内的角点与其西北边、东北边、西南边和东南边上的块内相邻的角点之间的相关系数。 
传统的块预处理方法是通过利用$B_i^{-1}$来做串行的因式分解从而构造出$A$的逆做近似,这种串行的因式分解并不适合与并行应用。
与之相对的,$A$的块对角矩阵是对$A$的一个很好的近似,而且它的逆是可以天然的并行求解得到。
这个块对角矩阵的逆可以表示为 
\begin{eqnarray*}
M^{-1}=    \left [
        \begin{array}{ccccccc}
        B_1^{-1} &   &  \\
         & \ddots&  \\
        &   &  B_{m^2}^{-1} \\
    \end{array}
    \right ]
\end{eqnarray*}
使用 $M$作为预处理子,每个块上的预处理过过程$\textbf{x}
= M^{-1}\textbf{y}$ 通常可以转化成求解稀疏矩阵线性方程组 $B_i \textbf{x}_i = \textbf{y}_i$,
(而不是显式的构造出矩阵$B_i^{-1}$)并且采用LU分解的方法来求解。  
在LU分解已经被初始过的前提下, 用LU分解的方法求解这些线性方程组的复杂度为 $\mathcal{O}(n^4)$,
 


 

\subsection{误差向量传播方法}
\label{precond:EVP:EVP}

相比于LU分解,使用误差向量传播方法(EVP)求解线性方程$B_i \textbf{x}_i =\textbf{y}_i$ 的计算复杂度是 $\mathcal{O}(n^2)$, 这里$B_i$ 的大小为$n^2\times n^2$。 
据我们所知,EVP方法的计算复杂度是串行求解椭圆方程的算法中最低的一个 \cite{roache1995elliptic}。 
EVP方法及其变化形式被用在很多个海洋模式中,如Sandia海洋模拟系统\cite{dietrich1987ocean}和加拿大版本的DIEcast模式\cite{tseng2011parallel}。 
同时,Tseng和Chien等人还将改进过的基于区域分解的并行EVP方法作为求解器应用在全球海洋模拟中。


\begin {figure}
\centering
\includegraphics[height=8cm]{evp9pmarch1.png}
\caption []{针对九点差分的EVP求解方法。 当$(i,j)$周边除了点 $(i+1,j+1)$之外的其他点的值都已知时, 点$(i+1,j+1)$上的值可以利用点$(i,j)$对应的方程求解。 \label {fig:evp9p}}
\end {figure}

EVP方法的计算流程如下. 我们将隐式求解海表高度的椭圆系统离散化之后所得到的线性方程 \ref{eq:ssh}变换成如下形式:
\begin{eqnarray}
\label{eq:evp9p}
&\eta_{i+1,j+1} = (1/A_{i,j}^{ne} )(\psi_{i,j} - A_{i,j}^0\eta_{i,j}-A_{i,j}^e\eta_{i+1,j} \nonumber\\
&-A_{i,j}^n\eta_{i,j+1}-A_{i-1,j}^{ne}\eta_{i-1,j+1} +A_{i-1,j}^e\eta_{i-1,j}\nonumber\\
&-A_{i-1,j-1}^{ne}\eta_{i-1,j-1}-A_{i,j-1}^n\eta_{i,j-1}- A_{i+1,j-1}^{ne}\eta_{i,j-1} )
\end{eqnarray}
利用这个形式我们就可以在其他周围点的解已知的情况下得到西北角点上的值。 


图 \ref{fig:evp9p} 所示的是在一个小区域上采用Dirichlet边界条件的椭圆方程
 $\mathcal{B}\textbf{x} = \psi$。  We
我们定义紧挨南面和西面的边界的那一层内部的点集为初始猜测层$\textbf{e}$ 
而紧挨着北面和东面的边界的那一层内部的点集定义为目标边界层$\textbf{f}$ (对应到图\ref{fig:evp9p}中有
$\textbf{e}= \{E_1, \dots, E_7\}$, $\textbf{f}= \{F_1, \dots, F_7\}$
)。 
假设我们知道了初始猜测层 $\textbf{e}$上的真解,  
那么总个区域上的真解也就可以通过等式\ref{eq:evp9p}来从西南角向东北角逐次的计算出来。 
这个过程我们称之为行进方法。
比较麻烦的是, 初始猜测层 $\textbf{e}$ 上的真解往往是直到整个椭圆方程都被解出来了以后才能知道。
但是,我们可以得到一组解
$\textbf{x}$在除了边界以外的整个区域上都满足椭圆方程的定义。 
具体做法是通过先在初始猜测层$\textbf{e}$上随机赋予初始值
$\textbf{x}|_\textbf{e}$ 然后通过行进方法得到整个区域上的值。 
这时$E=(\textbf{x} -\eta)|_\textbf{e}$
和$F=(\textbf{x} -\eta)|_\textbf{f}$ 分别表示在
初始猜测层$\textbf{e}$ 和目标边界层$\textbf{f}$上的误差向量。
误差向量$F$ 是可以直接由 目标边界层$\textbf{f}$ 上的边界条件计算得到(这里我们采用的是Dirichlet边界条件)。 
初始猜测层上的误差向量与目标边界层上的误差向量之间的关系可以表示为$F=W*E$.  
这里的关系矩阵$W$可以通过在初始猜测层上依次赋予不同的单位向量(初始猜测层上某一个点赋值为1,其它点赋值均为0)而区域内其它的点都赋予零值后使用行进方法得到。 
这里我们提供一个在零边界条件的椭圆方程中的EVP算法\ref{alg:evp}。 


 
EVP方法主要包括两个步骤: 前处理和求解过程。 
在前处理步中, 主要计算关系矩阵和它的逆。 其计算复杂度为$\mathcal{C}_{pre}=
(2n-5)* 9n^2 + (2n-5)^3 = \mathcal {O} (26n^3)$。 
求解过程的复杂度为$\mathcal{C}_{evp}= 2* 9n^2 + (2n-5)^2 = \mathcal{O} (22n^2)$。 
从这个复杂度的表示可以看出来,EVP方法的计算开销要比LU分解等其它的直接法低很多。 
从计算开销的角度来看, EVP方法能够被应用于实际应用中,因为虽然前处理过程的开销比较大,但是通常只需要在一开始的时候计算一次关系矩阵和它的逆, 往后就可以反复的使用了。

\begin{algorithm}[t!]
\caption{九点的误差向量传播法}
\label{alg:evp}
\begin{algorithmic}[1]
\REQUIRE 包含有$n\times n$个网格点的区域上的残差的$\psi$ , $k = size(\textbf{e})=2n-5$; \\
//\qquad \textit{前处理}
\STATE  $\textbf{x} = \textbf{0}$
\FOR {i = 1, k}
\STATE $\textbf{x}|_\textbf{e}(i) = 1$
\STATE $\textbf{x} = marching(\textbf{x},\textbf{0})$
\STATE $W(i,:) = \textbf{x}|_\textbf{f}$
\STATE $\textbf{x}|_\textbf{e}(i) = 0$
\ENDFOR
\STATE $R = inverse(W)$ \\
//\qquad \textit{求解 }
\STATE $\textbf{x}= marching(\textbf{x},\psi)$
\STATE $F = (\textbf{x} - \eta)|_\textbf{f}$
\STATE $\textbf{x}|_\textbf{e} =\textbf{x}|_\textbf{e} - R*F$
\STATE $\textbf{x} = marching(\textbf{x},\psi)$
\end{algorithmic}
\end{algorithm}

%----------------------------------------------------------------------------
\section{ChronGear求解器}

目前公共地球系统模式中并行海洋模式推荐的线性正压模态求解器是采用对角预处理的Chronopoulos-Gear (ChronGear) 方法
\cite{dAzevedo1999lapack}。 这是一种修正后的预处理共轭梯度法(Preconditioned Conjugate
Gradient method, PCG)。 
ChronGear继承了预处理共轭梯度法的缺陷--迭代过程中需要全局求和,从而使得它不能很好地扩展,并成为了高分辨模拟中的一个性能瓶颈。
为了提高并行海洋模式,同时也是公共地球系统模式的可扩展性,我们通过消除求解器迭代过程中的全局求和操作以及发展一个更高效的预处理子来提高正压求解器的并行计算效率。


\begin {figure}
\centering
\includegraphics[height=8cm]{newPOPStepComp}
\caption[] {0.1度POP采用默认的对角预处理的ChronGear求解器时个分量耗时的百分比。\label{fig:StepComp}}
\end{figure}

ChronGear \cite{dAzevedo1999lapack} 是一种修正过的共轭梯度法,它将每一步迭代过程中的两次全局通信组合成一次。
然而,如前面提到的,当成千上万个处理器核心被用来计算高分辨(比如0.1度)的POP时,ChronGear算法中计算內积所需要的全局通信操作仍然是一个严重的瓶颈。
这个瓶颈对0.1度POP的影响可以通过图\ref{fig:StepComp} 看出来。线性求解器的计算时间占总时间的比例随着所使用的处理器核心数的增大而增加。
当只使用470个处理器核心时,正压求解器的执行时间只占到整个POP的运行时间的5\% (这里不考虑初始化和 I/O时间),而此时斜压模态的运行时间却占总运行时间的90\%左右。
然而,当使用数千个处理器核心时,斜压模态所消耗的计算时间占总时间的比例会逐渐下降,而正压求解器所占的比例却在不断增加。
当使用1万6千多处理器核心时,正压求解的计算时间占中运行时间的比例高达50\%。 

\begin{algorithm}[!t]
\caption{Chronopoulos-Gear求解器}
\label{alg:cg}
\begin{algorithmic}[1]
\REQUIRE   与网格块$B_{i,j}$相应的系数矩阵$\textbf{B}$, 预处理子 $\textbf{M}$, 初始解 $\textbf{x}_0$ 和 $\textbf{b}$  \\
//\qquad    \textit{所有进程并行执行}
\STATE $\textbf{r}_0 = \textbf{b}-\textbf{B}\textbf{x}_0$, $\textbf{s}_0 =0$, $\textbf{p}_0 =0$;\quad $\rho_0=1$,$\sigma_0=0$, $k=0$;
\WHILE{$k \leq k_{max}$ }
\STATE $k=k+1$;
\STATE $\textbf{r}'_{k} =\textbf{M}^{-1}\textbf{r}_{k-1}$; \label{pcg_scale0} \COMMENT{预处理}
\STATE $\textbf{z}_k = \textbf{B}\textbf{r}'_{k}$; \label{pcg_mat}\COMMENT {矩阵向量乘}
\STATE $update\_halo(\textbf{z}_{k})$; \COMMENT{边界通信} \label{pcg_bc1}
\STATE $\tilde{\rho_k} = \textbf{r}_{k-1}^T\textbf{r}'_{k}$;\label{pcg_dot1}
\STATE $\tilde{\delta_k} = \textbf{z}_k^T\textbf{r}'_k$;\label{pcg_dot2}
\STATE $(\rho_k,\delta_k) = global\_sum(\tilde{\rho_k},\tilde{\delta_k})$;\label{pcg_global1} \COMMENT{全局规约操作}
\STATE $\beta_k = \rho_k / \rho_{k-1}$;\label{pcg_beta}
\STATE $\sigma_k = \delta_k - \beta_k^2\sigma_{k-1}$;\label{pcg_sigma}
\STATE $\alpha_k = \rho_k /\sigma_{k}$;\label{pcg_alpha}
\STATE $\textbf{s}_k = \textbf{r}'_{k} +\beta_k\textbf{s}_{k-1}$;\label{pcg_scale1}
\STATE $\textbf{p}_k = \textbf{z}_{k} +\beta_k\textbf{p}_{k-1}$;\label{pcg_scale2}
\STATE $\textbf{x}_k =\textbf{x}_{k-1} +\alpha_k \textbf{s}_k$;\label{pcg_scale3}
\STATE $\textbf{r}_k =\textbf{r}_{k-1} -\alpha_k\textbf{p}_k$;\label{pcg_scale4}
\IF{ $k \% n_{c} == 0$}
\STATE \textbf{if} $||\textbf{r}_k|| \le \epsilon$  \textbf{return} ;\COMMENT{每 $n_c$ 检查一次是否收敛}
\ENDIF
\ENDWHILE
\end{algorithmic}
\end{algorithm}

我们给出ChronGear方法的具体算法
\ref{alg:cg}, 以供参考.  ChronGear主要包括三个部分:计算部分,边界通信部分和全局通信部分。
计算部分主要包括矩阵向量乘操作,向量向量乘操作以及向量的伸缩操作(向量与常数相乘)等具有很好的可扩展性的操作。 
边界通信会在每次矩阵向量乘操作之后被调用,其开销随着计算使用的处理器核心数的增加和减少,但是有一个下界。 
但是,我们后面的章节将会证明,每一步迭代过程中的內积操作\ref{pcg_global1}之后需要调用的全局通信,当使用很大的核心数时,将会成为可扩展性的瓶颈。
 

%----------------------------------------------------------------------------
\subsection{通信瓶颈}
\label{precond:EVP:comm}
  
假设使用$p=m^2$个进程,每一个进程恰好分得一个网格块(这在高分辨率POP中是比较常见的配置)。 
这是正压模态的总运行时间就等于ChronGear求解器在每一个块上的运行时间。 
我们记求解器的每一次迭代过程中,计算、边界通信和全局通信操作的耗时分别为$\mathcal{T}_c$,$\mathcal{T}_b$ 和$\mathcal{T}_g$。 
%----------------------------------------------------------------------------


从算法\ref{alg:cg}中可以得出, 计算开销
$\mathcal{T}_c$主要涉及到四个向量伸缩操作 (算法中步骤\ref{pcg_scale1}, \ref{pcg_scale2} ,\ref{pcg_scale3}, 和
\ref{pcg_scale4}), 两个向量向量乘积操作或內积操作 (算法中步骤\ref{pcg_dot1} 和 \ref{pcg_dot2}), 以及一个矩阵向量乘积操作 (算法步骤\ref{pcg_mat}).  
因此,$\mathcal{T}_c= (4 n^2 +2n^2+ 9n^2)\theta + \mathcal{T}_{p}=15\frac{\mathcal{N}^2}{p}\theta+\mathcal{T}_{p}$,
这里$\theta$ 表示单位时间内浮点操作的次数,$\mathcal{T}_{p}$ 表示预处理的开销。
比如,当采用对角预处理时,$\mathcal{T}_{p} =\frac{\mathcal{N}^2}{p}\theta$。
上面的定义可以看出,当计算核心数增加时,$\mathcal{T}_c$减少,并且以零为下界。

 
当调用了像矩阵向量乘和除了对角预处理以外的预处理等需要一层或一层以上的边界区域的操作时,在每个进程的边界区域上需要进行边界更新操作。 
由于在海洋模式POP中,每个进程都维护着它自己的块以外的两个边界层,因此,即使使用了非对角的预处理,每次迭代过程中仍只需要进行一次边界通信。
边界通信的实际开销取决于网络延迟和边界区域的大小。 当边界区域的层数为2时(这也是POP中的默认配置), 边界区域的每个边的大小为$2n$ ,而且这个大小会随着进程数的增大而变小。 
 
因此,每一步迭代中,边界通信的时间复杂度为$\mathcal{T}_b =4\alpha +(4\times 2n)\beta=4\alpha +(\frac{8\mathcal{N}}{\sqrt{p}})\beta $这里$\alpha$ 表示单条点对点通信消息的延迟,而
$\beta$ 表示从网络中传输一比特需要的时间 (也就是网络带宽的倒数).
边界更新所需要的时间也是随着就算进程的增多而减少,但是有一个下界$4\alpha$。 


 
ChronGear求解器的每一步迭代中仅仅包含了一次全局归约操作。全局规约操作主要由一次MPI\_allreduce和一个去除路地网格点的掩盖操作,因此全局归约操作的开销满足$\mathcal{T}_g= 2\frac{\mathcal{N}^2}{p}\theta + \log p \alpha$ (假设网络采用二叉树结构)。 
路地点的掩盖操作的开销会随着进程数$p$的增大而减小,而MPI\_allreduce的开销将会一直随着进程数的增加而增大。 
注意到每次全局归约操作实际上近似于没有数据的交换,因为每个进程只传递了两个数值。
因此,全局通信的开销主要取决于网络的延迟而不是网络的传输速率。
结合以上三个部分,使用对角预处理的ChronGear求解器求解一次所需要的时间开销为:
\begin{eqnarray}
%\begin{tabular}{l}
\label{t_cg}
&\mathcal{T}_{cg}=\mathcal{K}_{cg} (\mathcal{T}_c + \mathcal{T}_b+\mathcal{T}_g )\nonumber \\
&=\mathcal{K}_{cg} [18 \frac{\mathcal{N}^2}{p}\theta + \frac{8\mathcal{N}}{\sqrt{p}}\beta +(4+\log p)\alpha]
%\end{tabular}
\end{eqnarray}
这里 $K_{cg}$ 表示算法达到指定收敛条件所需要的迭代次数。
这个迭代次数不会随着进程数的增加而改变 \cite{hu2013scalable}。 
方程\ref{t_cg})表明,计算和边界更新所需要的时间开销都会随着进程数的增加而减少。 
但是,全局归约操作的时间开销则会随着进程数的增多而增加。 
因此,我们可以预测到,当进程数操作某一个值时, ChronGear求解器的运行时将会增加。 
图 \ref{fig:ChronGearCOMP}给出了在美国国际大气研究中心的黄石超级计算(详情见节\ref{precond:exp})上,运行一个模拟天时,ChronGear求解器的全局通信和边界通信两个部分的耗时分析。 
这里我们可以看到,当使用数千个处理器核心时,全局归约操作的时间变成了整个求解器耗时的主要部分,并且这个时间还在继续增大。 
 
论文\inlinecite{dAzevedo1999lapack}证明了 ChronGear  和PCG在理论上收敛速度是相同的。 

\begin{figure}[!t]
\begin{center}
	\includegraphics[height=8cm]{newChronGear_comp.eps}
\caption[] {在美国黄石超级计算机上,0.1度POP的全局归约操作和边界更新操作占模拟一天总时间开销的比例。}
\label{fig:ChronGearCOMP}
\end{center}
\end{figure}
 


\section{P-CSI迭代法}
\label{precond:pcsi}

为了提高海洋模式POP的可扩展性,一个好的正压模态求解器应该具备的特点是需要尽可能少的全局归约操作。 
在文章\cite{hu2013scalable}中,胡勇等人提出了一个基于Stiefel的CSI方法的求解器。这个求解器被现在了单独版本的POP中。 
这里,我们将通过添加一个预处理的结构,进一步改进CSI方法。
这个改进后的方法我们称之为P-CSI方法。 我们将这个方法实现在地球系统模式CESM框架下的海洋模式分量POP中。 
早在1985年, Saad 等人\cite{saad1985solving}将传统的CSI的方法应用于向量机上,并且指出当方程中矩阵的特征值已知的条件下, 这种方法比共轭梯度法更适用于某些并行情况。 


P-CSI算法和它的性质与CSI方法相似,只是多加了一个预处理接口(将在下一章详细讨论)。 
值得一提的是,P-CSI方法与CSI方法一样,在每一步迭代过程中不需要做內积操作, 因此它即使在大核数上也能够保持较好的可扩展性。
为海洋模式POP设计的P-CSI求解器的伪码在算法\ref{alg:ppsi}中。
使用对角预处理的P-CSI方法的每一步迭代的计算时间为$T_c =\frac{12\mathcal{N}^2}{p}\theta+\mathcal{T}_p =\frac{13\mathcal{N}^2}{p}\theta$, 
而P-CSI求解器的总运行时间为如下
\begin{eqnarray}
\label{t_psi}
\mathcal{T}_{pcsi} = \mathcal{K}_{pcsi}(\mathcal{T}_c + \mathcal{T}_b ) \nonumber \\
= \mathcal{K}_{pcsi}[13\frac{\mathcal{N}^2}{p}\theta+ 4\alpha + \frac{8\mathcal{N}}{ \sqrt{p}}\beta]
\end{eqnarray}
这里$K_{pcsi}$ 表示P-CSI方法收敛到给定条件时所需要的迭代步数。 
\begin {figure}[!t]
\begin{center}
\includegraphics[height=8cm]{solver_iteration}
\caption []{1度POP中,Lanczos方法的迭代次数对P-CSI迭代步数的影响。 \label{fig:iter}}
\end{center}
\end {figure}

\begin{algorithm}[!t]
\caption{ 预处理的传统Stiefel迭代算法}
\label{alg:ppsi}
\begin{algorithmic}[1]
\REQUIRE 与网格块$B_{i,j}$相对应的系数矩阵 $\textbf{B}$, 预处理子$\textbf{M}$, 初始值$\textbf{x}_0$和方程右端向量$\textbf{b}$ ;预估的特征值区间$[\nu,\mu]$;  \\
 // \qquad    \textit{所有进程并行执行}
\STATE $\alpha =\frac{2}{\mu -\nu}$, $ \beta = \frac{\mu +\nu}{\mu -\nu}$, $\gamma = \frac{\beta}{\alpha}$, $\omega_0 =\frac{ 2}{\gamma}$;\quad $k = 0$;
\STATE $\textbf{r}_0 = \textbf{b}-\textbf{B}\textbf{x}_0$; $\Delta \textbf{x}_{0} = \gamma^{-1}\textbf{M}^{-1}\textbf{r}_0$; $\textbf{x}_1 =\textbf{x}_0 +\Delta \textbf{x}_{0}$; $\textbf{r}_1 =\textbf{b} -\textbf{B}\textbf{x}_1$;
\WHILE{$k \leq k_{max}$ }
\STATE $k=k+1$;
\STATE $\omega_k = 1/(\gamma - \frac{1}{4\alpha^2}\omega_{k-1})$; \COMMENT{迭代函数}
\STATE $\textbf{r}'_{k} =\textbf{M}^{-1}\textbf{r}_{k}$; \COMMENT{预处理} \
\STATE $\Delta \textbf{x}_{k} =\omega_k\textbf{r}'_{k}+(\gamma \omega_k-1)\Delta \textbf{x}_{k-1}$;
\STATE $\textbf{x}_{k+1} =\textbf{x}_{k}+\Delta \textbf{x}_{k}$;
\STATE $\textbf{r}_{k+1} =\textbf{b}- \textbf{B}\textbf{x}_{k+1}$; \COMMENT{矩阵向量乘}
\STATE $update\_halo(\textbf{r}_{k+1})$; \COMMENT{边界通信}
\IF{ $k \% n_{c} == 0$}
\STATE \textbf{if} $||\textbf{r}_{k+1}|| \le \epsilon$  \textbf{return} ;\COMMENT{每 $n_c$ 检查一次是否收敛}
\ENDIF
\ENDWHILE
\end{algorithmic}
\end{algorithm}

\subsection{特征值估计}
\label{precond:pcsi:eigs}
P-CSI需要对预处理后的系数矩阵$M^{-1}A$的最大特征值$\mu$和最小特征值$\nu$进行估计。 
海洋模式POP中的系数矩阵$A$ 和它的对角预处理矩阵$M = \Lambda(A)$都是实对称矩阵, 因此预处理后的矩阵的最小最大特征值并不难估计。 
论文\cite{hu2013scalable} 中给出$A$的特征值的估计,这里我们采用Lanczos方法\cite{Paige1980235}来对$M^{-1}A$的最大最小特征值进行估计。
实验中,我们发现将Lanczos方法的收敛条件因子$\varepsilon$ 设成$0.15$在不同的分辨率(1度或者0.1度)以及不同的预处理子(单位阵预处理、对角预处理和EVP预处理)的情况下都能达到较好的效果。 
图\ref{fig:iter} 可以看出只需要很少的Lanczos步骤就可以估计出比较恰当的$M^{-1}A$的最大最小特征值,从而使得P-CSI方法能够在较短的迭代步数内收敛。
在实际运行中,我们发现Lanczos方法的开销与调用数次ChronGear求解的开销相当。 

与ChronGear的迭代过程相比,P-CSI除了做收敛性检查之外不需要任何的全局归约操作。 
我们注意到,为了达到相同的收敛条件,P-CSI算法需要的迭代步骤比ChronGear的略多($K_{pcsi} > K_{cg}$)。 
以上的两点现象,我们可以解读出来,在当并行使用的处理器核心数比较小的时候,P-CSI应该会比ChronGear的开销更大一些。
因为处理器核心较少的时候,全局归约操作的开销并不算大。 
但是,当采用高分辨的网格时,需要用到更多的处理器核心,这是P-CSI的每一步迭代的开销应该会比ChronGear的要明显的快一些(参见公式\ref{t_cg})和
(\ref{t_psi})。
而这将进一步减少P-CSI算法到达收敛的时间。 

当满足条件$\nu = \lambda_{min}$和$\mu =\lambda_{max}$时,P-CSI的收敛速度将达到最优值。 
但是$\lambda_{min}$和$\lambda_{max}$是比较难以估计的。
更重要的是,我们不能对系数矩阵$A$随意的做变换,因为 $A$ 是分布在各个进程上的。
为了能够利用POP的并行特性,我们使用Lanczos方法来构造 出一系列三对角矩阵$T_m (m=1,2,...)$,这些矩阵的最大最小特征值逐渐的向$M^{-1}A$的最大最小特征值逼近。
为了利用海洋模式POP的并行性,我们基于Lanczos方法设计了一个并行的特征值估计方法。
实际运行中, 我们发现合适的选择Lanczos迭代的步数,我们能够得到对最大最小特诊一个比较好的估计,这个估计值能够使得P-CSI方法与预处理共轭梯度法有相近的收敛速度。  


\begin{algorithm}[!ht]
\caption{基于Lanczos方法的针对预处理矩阵的特征值估计方法}
\label{alg:lanczos_pre}
\begin{algorithmic}[1]
\REQUIRE 网格块$B_{i,j}$ 的系数矩阵$\textbf{B}$,预处理子$\textbf{M}$和随机向量$\textbf{r}_0$; \\
 //\qquad    \textit{所有进行并行执行}
\STATE $\textbf{s}_0=\textbf{M}^{-1}\textbf{r}_0$;\quad $\textbf{q}_1 = \textbf{r}_0/({\textbf{r}_0^T\textbf{s}_0})$;\quad $\textbf{q}_0=\textbf{0}$;
\STATE $T_0=\emptyset$;\quad $\beta_0 =0$;\quad  $\mu_0 =0$;\quad $j=1$;
\WHILE{$j<k_{max}$}
\STATE $\textbf{p}_j = \textbf{M}^{-1}\textbf{q}_j$; \quad $\textbf{r}_j=\textbf{B}\textbf{p}_j-\beta_{j-1}\textbf{q}_{j-1}$;
\STATE $update\_halo(\textbf{r}_j)$;
\STATE $\tilde{\alpha}_j =\textbf{p}_j^T\textbf{r}_j$; \quad $\alpha_j=global\_sum(\tilde{\alpha}_j)$;
\STATE $\textbf{r}_j=\textbf{r}_j-\alpha_{j}\textbf{q}_{j}$; \quad $\textbf{s}_j = \textbf{M}^{-1}\textbf{r}_j$;
\STATE $\tilde{\beta}_j = \textbf{r}_j^T\textbf{s}_j$; \quad $\beta_j=sqrt(global\_sum(\tilde{\beta}_j))$;
\STATE \textbf{if} $\beta_j == 0$ \textbf{then} \textbf{return}
\STATE $\mu_j = max(\mu_{j-1},\alpha_j+\beta_j+\beta_{j-1})$; \label{lan_gersh}\\
\STATE $T_j=tri\_diag(T_{j-1},\alpha_j,\beta_j)$; \COMMENT{三对角矩阵}\label{lan_tm}
\STATE $\nu_j = eigs(T_j,'smallest')$ ; \label{lan_nu}
\STATE \textbf{if} $|\frac{\mu_j}{\mu_{j-1}} -1 |< \varepsilon\quad\textbf{and}\quad|1- \frac{\nu_j}{\nu_{j-1}}|< \varepsilon$ \textbf{then} \textbf{return}; \label{lanczos_converge}
\STATE $\textbf{q}_{j+1}= \textbf{r}_j/\beta_j$;\quad $j=j+1$;
\ENDWHILE
\end{algorithmic}
\end{algorithm}

这个算法需要$\textbf{A}$ 是正定对称矩阵。海洋模式POP中直接得到的系数矩阵是负定对称矩阵,此时只需将方程两边同时乘以单位矩阵和-1,即可将系数矩阵转换成正定对称矩阵。 

\subsection{P-CSI方法计算复杂度}  \label{precond:PCSI:complex}

当使用了EVP预处理时,计算的复杂变成了$\mathcal{T}_c = 37 \frac{\mathcal{N}}{p}$。因此P-CSI方法在使用EVP方法时一次求解过程的复杂度为
\begin{eqnarray}
\label{t_pcsiEvp}
\mathcal{T}_{pcsi-evp}=\mathcal{O}(\mathcal{K}_{pcsi-evp} (37\frac{\mathcal{N}}{p} +8\sqrt{\frac{\mathcal{N} }{p}} + 4\alpha))
\end{eqnarray}

以上式子可以看出来,使用了EVP预处理后,每一步迭代过程中的计算开销翻了一倍。
但是,一次求解的总体时间是下降。 
采用EVP预处理后,迭代步数$\mathcal{K}_{pcsi-evp}$会下降到原来的一半(参见图\ref{fig:convergence_diag})。
其结果是,在大核上运行时,计算时间变得非常的少,而最耗时的通信部分的次数也随着迭代步数的减少而减少了一半左右。


以上分析对于采用了预处理子的情形也同样适用。
唯一的区别是上面分析中提到的系数矩阵是$M^{-1}A$ 而不是$A$。
值得一提的是,PCG,ChronGear和P-CSI算法中预处理后的矩阵实际上是$M^{-1/2}A(E^{-1/2})^T$,它是对称的,并且和$M^{-1}A$ 有着相同的特征值\cite{Shewchuk1994}。
因此,这个预处理后的矩阵的条件数为$\kappa =  \kappa_2(M^{-1/2}A(E^{-1/2})^T)$。这个条件数通常情况下比原始矩阵 $A$的条件数要小。 
 $M$ 与$A$越是相近,$M^{-1}A$的条件数就越小。当$M = A$,$M^{-1}A$的条件数为$\kappa_2(M^{-1 }A ) = 1$。 

 当使用少数几个处理器核心来运行海洋模式分量时,全局通信并不是瓶颈,导致P-CSI方法和ChronGear方法的时间开销比较接近。
但是当采用很多的处理器核心来计算高分辨的海洋模式分量的时候,P-CSI方法的每一个步迭代就应该会比ChronGear方法快很多,因为ChronGear方法中的全局归约操作就会变成一个明显的瓶颈。


\section{实验方法和结果分析}
\label{precond:exp}

\subsection{理想实验}\label{precond:exp:ideal}

为了验证第\ref{solver:CG:convergence_rate}节和第\ref{solver:CSI:convergence_rate}节中所给出的收敛的理论结果,我们在理想的试验设置中构造了一系列条件数不同的矩阵。
这里,我们不再采用全局固定的网格大小,而是采用均匀的经纬网格。
随着维度$\theta$的变化,沿着经度方向的网格大小为 $\Delta x_j  = \pi R \cos (\theta_j)$。
时间步长设置为$\tau = 10\frac{\Delta y}{c}$,这里$c = 200m/s$ 表示重力快波的速度\cite{smith2010parallel}。 
我们用PCG和P-CSI方法分别结合单位阵预处理、对角预处理和EVP预处理来求解得到的这些方程。
实验中,EVP预处的块的大小为$5\times5$,收敛条件设为$tol = 10^{-6}$。 
由于ChronGear 和PCG的收敛速度相同,因此这个图中PCG的收敛结果也是ChronGear的结果。
  

\begin{figure} 
\vspace{5pt}
\centering
\includegraphics[height=7cm]{iterationGridSize}
\caption[] {网格点数目和求解器的迭代步数之间的关系。\label{fig:iterationGridSizePrecond}}
\end{figure}

正如图\ref{fig:iterationGridSizePrecond}所示,随着问题规模的增加,系数矩阵变得越来越病态。
所有的求解器都需要更多的迭代步数才能得到相同的相对残差。
对于PCG和P-CSI,收敛速度会随着预处理方法的变化而变化。 
指定问题的规模,使用单位矩阵作为预处理的求解器需要的迭代步数最大,而使用EVP预处理的求解器所需要的迭代步数最小。 
这也说明,使用EVP预处理,矩阵的形态会比使用单位阵预处理或者对角预处理后的形态要好。 
图中还说明,在预处理方法相同的情况下,P-CSI方法的收敛速度要比PCG的慢一些。 




我们首先利用美国国家大气研究中心NCAR-Wyoming超级计算中心(NWSC)\cite{loft:2015}的黄石超级计算机来评测一下采用了我们的新的正压模态求解器的CESM1.2.0的性能。 
黄石超级计算机由 72,576个
2.6-Ghz英特尔型号为Xeon E5-2670 的“Sandy Bridge“ 处理器构成。 
这些处理器都是通过13.6 GBps InfiniBand网络连通。   
黄石超级计算机上50\%以上的计算资源都是在使用地球系统模式CESM来进行气候模拟, 
因此提高CESM在黄石超级计算机上的效率对于全球气候变化研究来讲十分的有意义\cite{wf2014}。 


由于我们只是改动了海洋模式分量POP中的求解器,所以实验中我们将着重测试POP的性能。 
我们采用的CESM的
”G\_NORMAL\_YEAR“分量配置集,这个配置中只有海洋和海冰分量模式是真实计算的,而大气等分量模式采用气候数据输入。
我们针对两种使用频率最高的海洋模式POP水平网格分辨率来进行实验:
1度 ($320\times 384$) 和0.1度 ($3600\times 2400$).
值得一提的是,CESM1.2.0的默认配置中将黄石超级计算机上的MPI的环境需求限制(MPIMP\_EAGER\_LIMIT)设置为0, 这个变量主要控制在网络通信约会协议被使用之前的最大的消息量。我们发现使用黄石超级计算机上的默认的环境需求限制配置,也就是MP\_EAGER\_LIMIT = 131072,能够显著的改善CESM的运行性能。 



\subsection{低分辨率模拟}
\label{precond:exp:low}
图\ref{fig:runtime1}给出了几个可用的求解器在黄石超级计算机上模拟1度海洋模式POP时正压模态的运行时间。  
采用默认的对角预处理子, 在各种并行度下, P-CSI要比ChronGear的性能好。
在最大的并行度下(768处理器核心),P-CSI将每个模拟天中求解器的的执行时间从0.58s减少到了0.41s,也就是1.4倍的加速比)。 
另外,使用新的EVP块预处理,默认的ChronGear和P-CSI求解器的收敛性都得到提升。 
在768核上, 使用EVP预处理的P-CSI的计算效率为0.37s每模拟天, 相对于默认的对角预处理的ChronGear求解器加速了1.6倍。 

\begin {figure}
\centering
\includegraphics[height = 8cm]{NEW1deg_solverruntime}
\caption []{1度POP每模拟天中正压模态的运行时间.\label {fig:runtime1}}
\end {figure}

\begin{table}[!h]
\begin{center}
\caption{在黄石超级计算机上,1度 POP总运行时间的优化比例。 \label{tab:improve_1}}
\begin{tabular}{|l||l|l|l|l|l|}
\hline
进程数 & 48  & 96  & 192 & 384 & 768\\\hline
\hline
ChronGear+EVP & -.5\% & 1.1\%  & 6.5\% & 10.8\%  & 12.1 \% \\\hline
P-CSI+Diagonal  & .7\% &3.9\% &9.3\%  &11.0\% & 12.6 \% \\\hline
P-CSI+EVP	      &-2.4\% & .4\%	& 7.4\%  & 14.4\% & 16.7\%\\\hline
\end{tabular}
\end{center}
\end{table}


正压模态的改进使得整个海洋模式POP的运行时间也随之下降。
表 \ref{tab:improve_1} 给出了三个新的求解器和预处理子的不同搭配相比于原始的对角预处理的ChronGear求解的所带来的性能提升百分比。 
这里给出的时间都是 取之于一个五天的模拟,同时模式的初始化和 I/O 操作都没有考虑在内。 
P-CSI 结合EVP预处理在768核上取得了  16.7\% 的加速效果。
虽然16.7\% 的提高看起来并不显著,但是由于 1度分辨率的海洋模式POP通常的模拟时间尺度都是几个世纪, 这样的一个性能的改进在如此长期的模拟中可以节省数百万个小时的CPU计算时间。 
另外,当海洋模式POP中开启生物化学的模块时,即使是1度的分辨率也需要使用相对较多的处理器核心,因为这个模块需要引进很多新的示踪物,进而导致计算量增加很多。 

\subsection{高分辨率模拟}
\label{precond:exp:high}

\begin {figure}
\centering
\includegraphics[height =8cm]{NEW01deg_solverruntime}
\caption []{ 在美国黄石超级计算机上0.1度 POP中正压模态模拟一天的的执行时间。\label {fig:runtime01_time}}
\end {figure}

\begin {figure}
\centering
\includegraphics[height =8cm]{NEW01deg_speedup_ys}
\caption []{ 在美国黄石超级计算机上0.1度 POP中正压模态模拟一天的的执行时间(左), 
在美国黄石超级计算机上0.1度 POP核心部分的模拟速率(右)。\label {fig:runtime01_rate}}
\end {figure}


现在我们测试在黄石超级计算机上,高分辨率0.1度的海洋模式POP的新的正压求解的可扩展性。 
在这个高分辨下, 海洋模式中网格块的大小和分布得选择,会影响到进程上任务的分发,进而对性能造成很大的影响。 
因此,为了不让这种选择影响到最终的可扩展性结果,我们非常仔细的将每个进程上的网格块的横纵比例统一地设置为3:2,而陆地的比例设置为0.25,同时还利用空间填充曲线的方法来划分网格。 
我们使用POP中默认的时间步长,也就是500个时间步每天(dt\_count = 500)。 
最后,为了统一,所有的求解器都是每十个迭代步做一次收敛判断。 
这里值得一提的是,由于P-CSI的迭代过程的开销相对比较小(相比于海洋模式POP中的收敛性判断), 因此,如果将收敛性判断的频率进一步降低的话, P-CSI的性能可能能够进一步的提高。 
 

如图\ref{fig:runtime01_time}所示,ChronGear的性能在所使用的处理器核心数大于2700时反而会有所下降,
而P-CSI的运行时间在这之后相对比较平缓。 
采用对角预处理的前提下, P-CSI方法在16,875核上将0.1度的海洋模式POP的正压模态的性能提高了4.3倍(从19.0秒下降到4.4秒每模拟天)。 
使用EVP预处理能够进一步的改进ChronGear和P-CSI的性能, 使得这两个算法分别相对于原始的正压模态求解器加速了1.4倍和5.2倍。
在\ref{precond:EVP}节中,我们证明了原始的正压模态求解器在海洋模式POP的总运行时间所占的比例会随着使用的核心数目的增加而增大。
尤其是在16,875个核上,采用对角预处理的ChronGear求解器占到总运行时间的 50\%以上。 
相对应的,图\ref{fig:StepComp_pcsi}表明,采用了具有 更高可扩展性的EVP预处理的P-CSI求解器后, 同样是在 16,875 核上运行,正压模态的执行时间仅仅占到总运行时间的16\%。 

 
正压求解器性能上的改进有助于提高海洋模式POP的总体性能, 尤其是在大规模并行环境中。
模拟速率(模拟年每天)是一个被广泛采用的评估模式性能的指标, 我们这里采用不考虑模式初始化和I/O开销的 核心模拟速率。这是由于模式初始化只是在模式运行开始的时候执行一次,而模式的I/O开销是随着用户设置的频率的变化而改变的。 
通常专家们认为,如果想要进行长期的气候模拟,那么模拟速度最小要达到五个模式年每天\cite{dennis2012computational}, 图\ref{fig:runtime01_rate} 可以看出,模式采用新的P-CSI求解器能够达到比采用ChronGear求解器更高的模拟速率。 
结合EVP预处理的P-CSI求解器在16,875核心上使得核心模拟速率提高了1.7倍,从原来的6.2模拟年年每天到10.5模拟年每天。 
 

\begin {figure}
\centering
\includegraphics[height=8cm]{NEWPOPStepComp_pcsi}
\caption[] { 0.1度海洋模式 POP采用EVP块预处理的P-CSI后的各部分的时间分析。 \label{fig:StepComp_pcsi}}
\end{figure}



为了找到性能提升的主要来源,\ref{fig:component}给出了正压求解器的具体的时间成分分析。 
图\ref{fig:component}中可以看出, P-CSI求解器比原始的ChronGear求解好的主要原因是因为全局归约操作的减少。 
全局归约操作的减少同时也显著的减少了海洋模式POP对操作系统噪声的敏感程度\cite{ferreira},因为更少的归约操作意味着相邻两次全局同步之间的时间变得更长了。 
除此之外,EVP块预处子还可以进一步减少边界通信,因为它是的达到收敛所需要的迭代次数变少了。 
在大规模并行环境下,正压求解器的计算开销相比于全局归约操作和边界通信的开销来说基本上可以忽略,因此,EVP预处理子引入的那多出来的一倍的计算量对总体的运行时间影响不大。
最后我们还要注意到,在处理器核心数小于1200时,全局归约操作的时间是随着进程数的增加而减少的,这与等式\ref{t_cg}所给出的理论结果是一致的。

 

\subsection{Edison超级计算机测试结果}
\label{precond:exp:Edison}

 
为了证明新的求解器不仅仅在黄石超级计算机上有性能提升,我们还在Edison超级计算机上测试0.1度海洋模式POP的模拟效果。Edison超级计算机是美国能源研究科学计算中心(NERSC)的最新的一台超级计算机,它是由 133,824 个2.4 GHz的英特尔的”Ivy Bridge“处理器核心组成,这些处理器核心是通过 8GBps 的Cray
Aries高速网络以蜻蜓拓扑的结构连接起来。  
\begin {figure} 
\begin{center}
\includegraphics[height =8cm]{01deg_comp_all_gs}
\hspace{10pt}
\includegraphics[height =8cm]{01deg_comp_all_halo}
\end{center}
\caption[] {黄石超级计算机上,0.1度海洋模式POP的正压求解器中主要的耗时成分的分析:全局归约(上)和边界通信(下)。}
\label{fig:component}
\end {figure}
\begin {figure}
\centering
\includegraphics[height=6.5cm]{01deg_solverruntime_edison}
\caption []{在美国Edison超级计算机上0.1度 POP中正压模态模拟一天的的执行时间。\label {fig:runtime01_edison_time}}
\end {figure}

\begin {figure}
\centering
\includegraphics[height=6.5cm]{01deg_speedup_edison}
\caption []{在美国黄石超级计算机上0.1度 POP核心部分的模拟速率。\label {fig:runtime01_edison_rate}}
\end {figure}

图\ref{fig:runtime01_edison_time}和图\ref{fig:runtime01_edison_rate}可以看出,四种配置的正压求解器在Edison超级计算机上的模拟性能与在黄石超级计算机上的模拟性能有相近的特点。 
我们注意到在Edison超级计算机上模拟时,全局通信的时间开销要比在黄石超级计算机上的开销更不稳定一些,主要可能是因为网络资源的竞争造成的\cite{wang2014}。 
这最终导致了ChronGear求解器(不管采用什么预处理子)的运行时间在多次模拟中相差很大,因此我们将这些实验中最好的三个结果取平均值作为它的运行时间。 
由于P-CSI几乎没有全局归约操作(只发生在检查收敛性的时候), 这种多次运行中的差别很小。 

 
在Edison超级计算机上, 采用对角预处理的P-CSI方法使得0.1度的海洋模式POP的正压模态在16,875核心上加速了3.7倍,从26.2秒每模拟天下降到7.0秒每模拟天。 
使用EVP预处理,ChronGear和P-CSI方法的性能都得到提升。 
采用EVP块预处理的P-CSI方法使得正压模态加速了5.6倍。 
 
 
\begin{table}
\begin{center}
\caption {在黄石超级计算机上,0.1度海洋模式POP的核心模拟速率\label{tab:improve_01}}
\begin{tabular}{|l||r|r|r|r|r|r|r|}
\hline
Solver & 470  & 1200   & 2700 & 4220 & 7500 & 10800 & 16875\\\hline
\hline
ChronGear     &0.7 &1.7&3.4  &4.6 &6.0 &6.0 &5.3\\\hline
ChronGear+Evp &0.7 &1.7&3.4  &4.9 &6.6 &6.9 &6.5\\\hline
P-CSI+Diag    &0.7 &1.7&3.5  &5.0 &7.0 &8.3 &8.9\\\hline
P-CSI+Evp     &0.7 &1.7&3.5  &5.0 &7.2 &8.6 &9.3\\
\hline
\end{tabular}
\end{center}
\end{table}

\section{本章小结}
\label{precond:Conclusion}

本章主要针对目前正压求解器在大规模并行环境中收敛速度较慢,而导致通信开销过大的情况,提出了专门的基于误差向量传播方法的预处理技术。
基于出差向量传播方法在直接求解椭圆方程时的效率很高。 
基于误差向量传播方法的并行预处理技术不仅能够应用在P-CSI方法中,对于原始的ChronGear方法也有类似的加速效果。

