\chapter{预处理}
\label{cha:precond}

\section{本章概述}

\section{背景和动机}
\label{sec:precondBackgroud}

\section{预处理方法比较}
\label{sec:precond1}
正压模态求解器的总时间开销等于求解器达到收敛所需要的迭代步数乘以每一次迭代的时间开销。
随着计算使用的处理器核心数的增大,每一步迭代中的计算所消耗的时间是逐渐减少的,
但是通信所需要的时间会逐渐增加。 
为了减少通信的开销,人们通常采用预处理这一技术来减少收敛所需要的迭代次数。
附加的前提条件是预处理过程的开销在合理的范围之内。 
目前海洋模式POP中采用的默认求解器ChronGear通过使用一个简单的对角预处理方法,性能就已经取得了很好的提升\cite{pini1990simple, reddy2013comparison}。 
如果能够进一步的提高求解器的收敛速度,求解过程中通信的开销将会大大的减少,从而进一步的提高求解器的可扩展性。
事实上,一个有效的预处理子不仅能够给新的P-CSI求解器带来性能提升,同时也能够改进原有的默认的ChronGear求解器。

由于海洋模式POP中的数据是分布在多个进程上的,而进程之间的通信的开销很大, 如果使用因式分解的预处理方法,通常只能接受比较浅层次的分解,比如说ILU(0)。 
ILU(0)分解中,L和U分别与原始矩阵A的上三角和下三角矩阵具有相同的非零元结构。
这导致L和U的乘积只是A的一个非常粗糙的近似,从而使得这种预处理的效果并不是很明显。 
更为精确的ILU因式分解方法并不太适用于海洋模式,因为它需要更多的额外的通信和计算。 
同时,计算这些高阶的预处理子的代价也会越高。 

在1985, Concus等人 \cite{concus1985block} 使用原始矩阵的逆矩阵的条带状近似作为预处理矩阵,在椭圆微分方程问题的求解中取得了比其他预处理子更好的效果。 
Smith et al. \cite{smith1992parallel}在海洋模式POP 的最原始版本中采用多项式预处理的方法以及一个局部近似逆预处理的方法,将迭代步数缩短为原来的三分之到五分之一。
但是由于多项式预处理方法中,前期得到多项式预处理子的开销比较大,在后期的版本中这一方法并没有得到继承。  

使用最广的串行预处理技术当属不完全的因式分解方法,比如不完全的LU分解(ILU)和它的变种\cite{benzi2002preconditioning}。 
但是,这些方法并不适合在并行的环境中使用,因为它们需要对整个矩阵做一个按顺序的逐次的计算。 
随着并行计算的兴起,一些可以并行化的预处理技术,比如说多项式预处理、近似逆预处理、多重网格预处理和块预处理,逐渐因此了人们的重视。 
高阶的多项式预处理能够像不完全的LU分解(ILU)以及它的变种在串行的环境中一样高效的减少迭代算法的迭代次数\cite{benzi2002preconditioning} 。 
但是多项式预处理子巨大的计算开销抵消了它与在减少迭代次数上的优势,因为一个$k$阶的多项式预处理子在每次一次迭代过程中需要进行$k$次矩阵向量乘操作。 
最终的实际运行中,它的时间开销有时候反而比不上简单的对角预处理\cite{meyer1989numerical,smith1992parallel}。
近似逆预处理,尽管可并行度很高,但是它需要求解一个比原始方程大很多倍的线性方程组\cite{smith1992parallel,bergamaschi2007numerical}, 这就使得它的魅力相比于简单的预处理方法大打折扣。

 
 
多重网格是一种针对椭圆微分系统中的线性方程组的高可扩展和十分高效的预处理方法。 
最近的研究表明几何多重网格(GMG)在使用规则网格和简单地形的大气模式
 \cite{muller2014massively}
 和海洋模式中\cite{matsumura2008non,kanarska2007algorithm}中有很好的效果。 
 但是,全球海洋模式通常使用比较复杂的地形(如群岛、海峡、通道和复杂的海岸等),并结合以非规则或者各向异性的网格, 结果导致简单的几何多重网格并不能取得较好的效果
\cite{matsumura2008non,fulton1986multigrid,tseng2003ghost,stuben2001review}。 
在公共地球系统模式CESM的海洋模式分量POP中,为了避开极点奇异性问题,采用的是极点偏移的广义正交网格。 
海洋模式POP将陆地点掩盖掉,从而只计算地球表面的海洋部分。
这些选择导致了模式中的椭圆方程所得到的系数是定义在一个不规则的网格上的不规则区域, 进而导致几何多重网格很难取得好的效果。 
地形的复杂性在高分辨率的海洋网格中会变得更加糟糕,因为成千上万的群岛和海峡通道(比如白令海峡)在稍微粗一点的网格上并不可见。
对于这些涉及复杂的地形的问题,代数多重网格通常比几何多重网格更加的实用。 
但是代数多重网格也有一个缺陷,那就是在某些例子里面,代数多重网格的启动开销比它的迭代求解过程还要耗时,
这使得它反而不如一些预处理得比较好的共轭梯度法
\cite{muller2014massively}, 尤其是在使用共轭梯度法所需要的迭代步数比较少的时候。 
地球系统模式CESM中的海洋模式分量POP中的正压模态求解问题就是这样一个例子(参见下一节的图
 \ref{fig:iteration})。
更进一步,不管是高分辨率还是低分辨率,海洋模式POP的每一个模拟天内都需要对正压模态中的线性方程求解几十次甚至上百次,
而通常的一次模拟中往往需要模拟成百上千个模拟年,这就导致代数多重网格方法的启动开销变得无法接受。


最后,块预处理技术被证明是一个比较有效的并行预处理技术\cite{concus1985block, white2011block}。
这个技术对于海洋模式POP尤为有潜力,因为它能够充分地利用POP中由椭圆方程离散化后得到的矩阵的块状结构的性质。 

\section{块预处理方法}
\label{sec:precond2}
\section{EVP块预处理子} \label{se:evp}
 

 

\subsection{块预处理方法}
 

\begin {figure}
\centering
\includegraphics[height=5.0cm]{blockpreconditioning.eps}
\caption[] {海洋模式POP中采用九点差分格式得到的系数矩阵的系数模式。这个例子中,总个区域被分成了$3\times3$个互不重叠的小块。
红色正方形中的元素表示块内的点之间的相关系数,而蓝色方块中的点表示第$i$个块中的点与其相邻块内的点之间的相关系数。 \label{fig:blockprecond}}
\end{figure}
 
为了能够更好的描述我们的新的EVP块预处理子,我们先以图\ref{fig:blockprecond}为例,简单的回顾一下常规的块预处理子。 
假使我们将一个在$\mathcal{N} \times \mathcal{N}$个点的网格上的线性系统以大小为$n\times n$的块为单位(图\ref{fig:blockprecond}中块的大小为$\mathcal{N}/3\times \mathcal{N}/3$)进行重排,这样得到的系数矩阵$A$就变成了一个就九对角块状矩阵。
这个块状矩阵每一排都包含九个子矩阵。 
$B_i$(图中的红色方块)表示由第$i$个块内的点之间的相关系数组成的块矩阵,它与$A$有相同的结构,只是大小要小一些 ($n^2\times n^2$)。 
$B_i^e$, $B_i^w$, $B_i^n$ 和$B_i^s$分别表示由第$i$个块内的点与东西南北四个方向上相邻的块内的点之间的关系系数组成的矩阵块,每个块只有分散在$n$行上的$3n$个非零元。
$B_i^{nw}$, $B_i^{ne}$, $B_i^{sw}$和 $B_i^{se}$都只包含一个非零元素, 分别表示第$i$个块内的角点与其西北边、东北边、西南边和东南边上的块内相邻的角点之间的相关系数。 
传统的块预处理方法是通过利用$B_i^{-1}$来做串行的因式分解从而构造出$A$的逆做近似,这种串行的因式分解并不适合与并行应用。
与之相对的,$A$的块对角矩阵是对$A$的一个很好的近似,而且它的逆是可以天然的并行求解得到。
这个块对角矩阵的逆可以表示为 
\begin{eqnarray*}
M^{-1}=    \left [
        \begin{array}{ccccccc}
        B_1^{-1} &   &  \\
         & \ddots&  \\
        &   &  B_{m^2}^{-1} \\
    \end{array}
    \right ]
\end{eqnarray*}
使用 $M$作为预处理子,每个块上的预处理过过程$\textbf{x}
= M^{-1}\textbf{y}$ 通常可以转化成求解稀疏矩阵线性方程组 $B_i \textbf{x}_i = \textbf{y}_i$,
(而不是显式的构造出矩阵$B_i^{-1}$)并且采用LU分解的方法来求解。  
在LU分解已经被初始过的前提下, 用LU分解的方法求解这些线性方程组的复杂度为 $\mathcal{O}(n^4)$,
 


 

\subsection{误差向量传播方法}
\label{sec:precondEVP}
相比于LU分解,使用误差向量传播方法(EVP)求解线性方程$B_i \textbf{x}_i =\textbf{y}_i$ 的计算复杂度是 $\mathcal{O}(n^2)$, 这里$B_i$ 的大小为$n^2\times n^2$。 
据我们所知,EVP方法的计算复杂度是串行求解椭圆方程的算法中最低的一个 \cite{roache1995elliptic}。 
EVP方法及其变化形式被用在很多个海洋模式中,如Sandia海洋模拟系统\cite{dietrich1987ocean}和加拿大版本的DIEcast模式\cite{tseng2011parallel}。 
同时,Tseng和Chien等人还将改进过的基于区域分解的并行EVP方法作为求解器应用在全球海洋模拟中。


\begin {figure}[!t]
\centering
\includegraphics[height=5.0cm]{evp9pmarch1.png}
\caption []{针对九点差分的EVP求解方法。 当$(i,j)$周边除了点 $(i+1,j+1)$之外的其他点的值都已知时, 点$(i+1,j+1)$上的值可以利用点$(i,j)$对应的方程求解。 \label {fig:evp9p}}
\end {figure}

EVP方法的计算流程如下. 我们将隐式求解海表高度的椭圆系统离散化之后所得到的线性方程 \ref{eq:ssh}变换成如下形式:
\begin{eqnarray}
\label{eq:evp9p}
&\eta_{i+1,j+1} = (1/A_{i,j}^{ne} )(\psi_{i,j} - A_{i,j}^0\eta_{i,j}-A_{i,j}^e\eta_{i+1,j} \nonumber\\
&-A_{i,j}^n\eta_{i,j+1}-A_{i-1,j}^{ne}\eta_{i-1,j+1} +A_{i-1,j}^e\eta_{i-1,j}\nonumber\\
&-A_{i-1,j-1}^{ne}\eta_{i-1,j-1}-A_{i,j-1}^n\eta_{i,j-1}- A_{i+1,j-1}^{ne}\eta_{i,j-1} )
\end{eqnarray}
利用这个形式我们就可以在其他周围点的解已知的情况下得到西北角点上的值。 



\section{试验方法和结果分析}
\label{sec:precondExp}

\section{本章小结}
\label{sec:precondConclusion}
