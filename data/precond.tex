\chapter{预处理}
\label{cha:precond}

\section{本章概述}

\section{背景和动机}
\label{sec:precondBackgroud}

\section{预处理方法比较}
\label{sec:precond1}
正压模态求解器的总时间开销等于求解器达到收敛所需要的迭代步数乘以每一次迭代的时间开销。
随着计算使用的处理器核心数的增大,每一步迭代中的计算所消耗的时间是逐渐减少的,
但是通信所需要的时间会逐渐增加。 
为了减少通信的开销,人们通常采用预处理这一技术来减少收敛所需要的迭代次数。
附加的前提条件是预处理过程的开销在合理的范围之内。 
目前海洋模式POP中采用的默认求解器ChronGear通过使用一个简单的对角预处理方法,性能就已经取得了很好的提升\cite{pini1990simple, reddy2013comparison}。 
如果能够进一步的提高求解器的收敛速度,求解过程中通信的开销将会大大的减少,从而进一步的提高求解器的可扩展性。
事实上,一个有效的预处理子不仅能够给新的P-CSI求解器带来性能提升,同时也能够改进原有的默认的ChronGear求解器。

由于海洋模式POP中的数据是分布在多个进程上的,而进程之间的通信的开销很大, 如果使用因式分解的预处理方法,通常只能接受比较浅层次的分解,比如说ILU(0)。 
ILU(0)分解中,L和U分别与原始矩阵A的上三角和下三角矩阵具有相同的非零元结构。
这导致L和U的乘积只是A的一个非常粗糙的近似,从而使得这种预处理的效果并不是很明显。 
更为精确的ILU因式分解方法并不太适用于海洋模式,因为它需要更多的额外的通信和计算。 
同时,计算这些高阶的预处理子的代价也会越高。 

在1985, Concus等人 \cite{concus1985block} 使用原始矩阵的逆矩阵的条带状近似作为预处理矩阵,在椭圆微分方程问题的求解中取得了比其他预处理子更好的效果。 
Smith et al. \cite{smith1992parallel}在海洋模式POP 的最原始版本中采用多项式预处理的方法以及一个局部近似逆预处理的方法,将迭代步数缩短为原来的三分之到五分之一。
但是由于多项式预处理方法中,前期得到多项式预处理子的开销比较大,在后期的版本中这一方法并没有得到继承。  

使用最广的串行预处理技术当属不完全的因式分解方法,比如不完全的LU分解(ILU)和它的变种\cite{benzi2002preconditioning}。 
但是,这些方法并不适合在并行的环境中使用,因为它们需要对整个矩阵做一个按顺序的逐次的计算。 
随着并行计算的兴起,一些可以并行化的预处理技术,比如说多项式预处理、近似逆预处理、多重网格预处理和块预处理,逐渐因此了人们的重视。 
高阶的多项式预处理能够像不完全的LU分解(ILU)以及它的变种在串行的环境中一样高效的减少迭代算法的迭代次数\cite{benzi2002preconditioning} 。 
但是多项式预处理子巨大的计算开销抵消了它与在减少迭代次数上的优势,因为一个$k$阶的多项式预处理子在每次一次迭代过程中需要进行$k$次矩阵向量乘操作。 
最终的实际运行中,它的时间开销有时候反而比不上简单的对角预处理\cite{meyer1989numerical,smith1992parallel}。
近似逆预处理,尽管可并行度很高,但是它需要求解一个比原始方程大很多倍的线性方程组\cite{smith1992parallel,bergamaschi2007numerical}, 这就使得它的魅力相比于简单的预处理方法大打折扣。

 
 
多重网格是一种针对椭圆微分系统中的线性方程组的高可扩展和十分高效的预处理方法。 
最近的研究表明几何多重网格(GMG)在使用规则网格和简单地形的大气模式
 \cite{muller2014massively}
 和海洋模式中\cite{matsumura2008non,kanarska2007algorithm}中有很好的效果。 
 但是,全球海洋模式通常使用比较复杂的地形(如群岛、海峡、通道和复杂的海岸等),并结合以非规则或者各向异性的网格, 结果导致简单的几何多重网格并不能取得较好的效果
\cite{matsumura2008non,fulton1986multigrid,tseng2003ghost,stuben2001review}。 
在公共地球系统模式CESM的海洋模式分量POP中,为了避开极点奇异性问题,采用的是极点偏移的广义正交网格。 
海洋模式POP将陆地点掩盖掉,从而只计算地球表面的海洋部分。
这些选择导致了模式中的椭圆方程所得到的系数是定义在一个不规则的网格上的不规则区域, 进而导致几何多重网格很难取得好的效果。 
地形的复杂性在高分辨率的海洋网格中会变得更加糟糕,因为成千上万的群岛和海峡通道(比如白令海峡)在稍微粗一点的网格上并不可见。
对于这些涉及复杂的地形的问题,代数多重网格通常比几何多重网格更加的实用。 
但是代数多重网格也有一个缺陷,那就是在某些例子里面,代数多重网格的启动开销比它的迭代求解过程还要耗时,
这使得它反而不如一些预处理得比较好的共轭梯度法
\cite{muller2014massively}, 尤其是在使用共轭梯度法所需要的迭代步数比较少的时候。 
地球系统模式CESM中的海洋模式分量POP中的正压模态求解问题就是这样一个例子(参见下一节的图
 \ref{fig:iteration})。
更进一步,不管是高分辨率还是低分辨率,海洋模式POP的每一个模拟天内都需要对正压模态中的线性方程求解几十次甚至上百次,
而通常的一次模拟中往往需要模拟成百上千个模拟年,这就导致代数多重网格方法的启动开销变得无法接受。


最后,块预处理技术被证明是一个比较有效的并行预处理技术\cite{concus1985block, white2011block}。
这个技术对于海洋模式POP尤为有潜力,因为它能够充分地利用POP中由椭圆方程离散化后得到的矩阵的块状结构的性质。 

\section{块预处理方法}
\label{sec:precond2}
\section{EVP块预处理子} \label{se:evp}
 

 

\subsection{块预处理方法}
 

\begin {figure}
\centering
\includegraphics[height=5.0cm]{blockpreconditioning.eps}
\caption[] {海洋模式POP中采用九点差分格式得到的系数矩阵的系数模式。这个例子中,总个区域被分成了$3\times3$个互不重叠的小块。
红色正方形中的元素表示块内的点之间的相关系数,而蓝色方块中的点表示第$i$个块中的点与其相邻块内的点之间的相关系数。 \label{fig:blockprecond}}
\end{figure}
 
为了能够更好的描述我们的新的EVP块预处理子,我们先以图\ref{fig:blockprecond}为例,简单的回顾一下常规的块预处理子。 
假使我们将一个在$\mathcal{N} \times \mathcal{N}$个点的网格上的线性系统以大小为$n\times n$的块为单位(图\ref{fig:blockprecond}中块的大小为$\mathcal{N}/3\times \mathcal{N}/3$)进行重排,这样得到的系数矩阵$A$就变成了一个就九对角块状矩阵。
这个块状矩阵每一排都包含九个子矩阵。 
$B_i$(图中的红色方块)表示由第$i$个块内的点之间的相关系数组成的块矩阵,它与$A$有相同的结构,只是大小要小一些 ($n^2\times n^2$)。 
$B_i^e$, $B_i^w$, $B_i^n$ 和$B_i^s$分别表示由第$i$个块内的点与东西南北四个方向上相邻的块内的点之间的关系系数组成的矩阵块,每个块只有分散在$n$行上的$3n$个非零元。
$B_i^{nw}$, $B_i^{ne}$, $B_i^{sw}$和 $B_i^{se}$都只包含一个非零元素, 分别表示第$i$个块内的角点与其西北边、东北边、西南边和东南边上的块内相邻的角点之间的相关系数。 
传统的块预处理方法是通过利用$B_i^{-1}$来做串行的因式分解从而构造出$A$的逆做近似,这种串行的因式分解并不适合与并行应用。
与之相对的,$A$的块对角矩阵是对$A$的一个很好的近似,而且它的逆是可以天然的并行求解得到。
这个块对角矩阵的逆可以表示为 
\begin{eqnarray*}
M^{-1}=    \left [
        \begin{array}{ccccccc}
        B_1^{-1} &   &  \\
         & \ddots&  \\
        &   &  B_{m^2}^{-1} \\
    \end{array}
    \right ]
\end{eqnarray*}
使用 $M$作为预处理子,每个块上的预处理过过程$\textbf{x}
= M^{-1}\textbf{y}$ 通常可以转化成求解稀疏矩阵线性方程组 $B_i \textbf{x}_i = \textbf{y}_i$,
(而不是显式的构造出矩阵$B_i^{-1}$)并且采用LU分解的方法来求解。  
在LU分解已经被初始过的前提下, 用LU分解的方法求解这些线性方程组的复杂度为 $\mathcal{O}(n^4)$,
 


 

\subsection{误差向量传播方法}
\label{sec:precondEVP}
相比于LU分解,使用误差向量传播方法(EVP)求解线性方程$B_i \textbf{x}_i =\textbf{y}_i$ 的计算复杂度是 $\mathcal{O}(n^2)$, 这里$B_i$ 的大小为$n^2\times n^2$。 
据我们所知,EVP方法的计算复杂度是串行求解椭圆方程的算法中最低的一个 \cite{roache1995elliptic}。 
EVP方法及其变化形式被用在很多个海洋模式中,如Sandia海洋模拟系统\cite{dietrich1987ocean}和加拿大版本的DIEcast模式\cite{tseng2011parallel}。 
同时,Tseng和Chien等人还将改进过的基于区域分解的并行EVP方法作为求解器应用在全球海洋模拟中。


\begin {figure}[!t]
\centering
\includegraphics[height=5.0cm]{evp9pmarch1.png}
\caption []{针对九点差分的EVP求解方法。 当$(i,j)$周边除了点 $(i+1,j+1)$之外的其他点的值都已知时, 点$(i+1,j+1)$上的值可以利用点$(i,j)$对应的方程求解。 \label {fig:evp9p}}
\end {figure}

EVP方法的计算流程如下. 我们将隐式求解海表高度的椭圆系统离散化之后所得到的线性方程 \ref{eq:ssh}变换成如下形式:
\begin{eqnarray}
\label{eq:evp9p}
&\eta_{i+1,j+1} = (1/A_{i,j}^{ne} )(\psi_{i,j} - A_{i,j}^0\eta_{i,j}-A_{i,j}^e\eta_{i+1,j} \nonumber\\
&-A_{i,j}^n\eta_{i,j+1}-A_{i-1,j}^{ne}\eta_{i-1,j+1} +A_{i-1,j}^e\eta_{i-1,j}\nonumber\\
&-A_{i-1,j-1}^{ne}\eta_{i-1,j-1}-A_{i,j-1}^n\eta_{i,j-1}- A_{i+1,j-1}^{ne}\eta_{i,j-1} )
\end{eqnarray}
利用这个形式我们就可以在其他周围点的解已知的情况下得到西北角点上的值。 


图 \ref{fig:evp9p} 所示的是在一个小区域上采用Dirichlet边界条件的椭圆方程
 $\mathcal{B}\textbf{x} = \psi$。  We
我们定义紧挨南面和西面的边界的那一层内部的点集为初始猜测层$\textbf{e}$ 
而紧挨着北面和东面的边界的那一层内部的点集定义为目标边界层$\textbf{f}$ (对应到图\ref{fig:evp9p}中有
$\textbf{e}= \{E_1, \dots, E_7\}$, $\textbf{f}= \{F_1, \dots, F_7\}$
)。 
假设我们知道了初始猜测层 $\textbf{e}$上的真解,  
那么总个区域上的真解也就可以通过等式\ref{eq:evp9p}来从西南角向东北角逐次的计算出来。 
这个过程我们称之为行进方法。
比较麻烦的是, 初始猜测层 $\textbf{e}$ 上的真解往往是直到整个椭圆方程都被解出来了以后才能知道。
但是,我们可以得到一组解
$\textbf{x}$在除了边界以外的整个区域上都满足椭圆方程的定义。 
具体做法是通过先在初始猜测层$\textbf{e}$上随机赋予初始值
$\textbf{x}|_\textbf{e}$ 然后通过行进方法得到整个区域上的值。 
这时$E=(\textbf{x} -\eta)|_\textbf{e}$
和$F=(\textbf{x} -\eta)|_\textbf{f}$ 分别表示在
初始猜测层$\textbf{e}$ 和目标边界层$\textbf{f}$上的误差向量。
误差向量$F$ 是可以直接由 目标边界层$\textbf{f}$ 上的边界条件计算得到(这里我们采用的是Dirichlet边界条件)。 
初始猜测层上的误差向量与目标边界层上的误差向量之间的关系可以表示为$F=W*E$.  
这里的关系矩阵$W$可以通过在初始猜测层上依次赋予不同的单位向量(初始猜测层上某一个点赋值为1,其它点赋值均为0)而区域内其它的点都赋予零值后使用行进方法得到。 
这里我们提供一个在零边界条件的椭圆方程中的EVP算法\ref{alg:evp}。 


 
EVP方法主要包括两个步骤: 前处理和求解过程。 
在前处理步中, 主要计算关系矩阵和它的逆。 其计算复杂度为$\mathcal{C}_{pre}=
(2n-5)* 9n^2 + (2n-5)^3 = \mathcal {O} (26n^3)$。 
求解过程的复杂度为$\mathcal{C}_{evp}= 2* 9n^2 + (2n-5)^2 = \mathcal{O} (22n^2)$。 
从这个复杂度的表示可以看出来,EVP方法的计算开销要比LU分解等其它的直接法低很多。 
从计算开销的角度来看, EVP方法能够被应用于实际应用中,因为虽然前处理过程的开销比较大,但是通常只需要在一开始的时候计算一次关系矩阵和它的逆, 往后就可以反复的使用了。

\begin{algorithm}[t!]
\caption{九点的误差向量传播法}
\label{alg:evp}
%\begin{scriptsize}
\begin{algorithmic}[1]
\REQUIRE 包含有$n\times n$个网格点的区域上的残差的$\psi$ , $k = size(\textbf{e})=2n-5$; \\
//\qquad \textit{前处理}
\STATE  $\textbf{x} = \textbf{0}$
\FOR {i = 1, k}
\STATE $\textbf{x}|_\textbf{e}(i) = 1$
\STATE $\textbf{x} = marching(\textbf{x},\textbf{0})$
\STATE $W(i,:) = \textbf{x}|_\textbf{f}$
\STATE $\textbf{x}|_\textbf{e}(i) = 0$
\ENDFOR
\STATE $R = inverse(W)$ \\
//\qquad \textit{求解 }
\STATE $\textbf{x}= marching(\textbf{x},\psi)$
\STATE $F = (\textbf{x} - \eta)|_\textbf{f}$
\STATE $\textbf{x}|_\textbf{e} =\textbf{x}|_\textbf{e} - R*F$
\STATE $\textbf{x} = marching(\textbf{x},\psi)$
\end{algorithmic}
%\end{scriptsize}
\end{algorithm}

\section{试验方法和结果分析}
\label{sec:precondExp}

 
我们首先利用美国国家大气研究中心NCAR-Wyoming超级计算中心
(NWSC) \cite{loft:2015}的黄石超级计算机来评测一下采用了我们的新的正压模态求解器的CESM1.2.0的性能。 
黄石超级计算机由 72,576个
2.6-Ghz英特尔型号为Xeon E5-2670 的``Sandy Bridge'' 处理器构成。 
这些处理器都是通过13.6 GBps InfiniBand网络连通。   
黄石超级计算机上50\%以上的计算资源都是在使用地球系统模式CESM来进行气候模拟, 
因此提高CESM在黄石超级计算机上的效率对于全球气候变化研究来讲十分的有意义\cite{wf2014}。 


由于我们只是改动了海洋模式分量POP中的求解器,所以实验中我们将着重测试POP的性能。 
我们采用的CESM的
``G\_NORMAL\_YEAR''分量配置集,这个配置中只有海洋和海冰分量模式是真实计算的,而大气等分量模式采用气候数据输入。
我们针对两种使用频率最高的海洋模式POP水平网格分辨率来进行实验:
1度 ($320\times 384$) 和0.1度 ($3600\times 2400$).
值得一提的是,CESM1.2.0的默认配置中将黄石超级计算机上的MPI的环境需求限制(MPIMP\_EAGER\_LIMIT)设置为0, 这个变量主要控制在网络通信约会协议被使用之前的最大的消息量。我们发现使用黄石超级计算机上的默认的环境需求限制配置,也就是MP\_EAGER\_LIMIT = 131072,能够显著的改善CESM的运行性能。 



\subsection{低分辨率模拟}
图\ref{fig:runtime1}给出了几个可用的求解器在黄石超级计算机上模拟1度海洋模式POP时正压模态的运行时间。  
采用默认的对角预处理子, 在各种并行度下, P-CSI要比ChronGear的性能好。
在最大的并行度下(768处理器核心),P-CSI将每个模拟天中求解器的的执行时间从0.58s减少到了0.41s,也就是1.4倍的加速比)。 
另外,使用新的EVP块预处理,默认的ChronGear和P-CSI求解器的收敛性都得到提升。 
在768核上, 使用EVP预处理的P-CSI的计算效率为0.37s每模拟天, 相对于默认的对角预处理的ChronGear求解器加速了1.6倍。 

\begin {figure}[!t]
\centering
\includegraphics[height =6.5cm]{NEW1deg_solverruntime}
\caption []{1度POP每模拟天中正压模态的运行时间.\label {fig:runtime1}}
\end {figure}

\begin{table}[!h]
\begin{center}
\caption{在黄石超级计算机上,1度 POP总运行时间的优化比例。 \label{tab:improve_1}}
\begin{tabular}{|l||l|l|l|l|l|}
\hline
进程数 & 48  & 96  & 192 & 384 & 768\\\hline
\hline
ChronGear+EVP & -.5\% & 1.1\%  & 6.5\% & 10.8\%  & 12.1 \% \\\hline
P-CSI+Diagonal  & .7\% &3.9\% &9.3\%  &11.0\% & 12.6 \% \\\hline
P-CSI+EVP	      &-2.4\% & .4\%	& 7.4\%  & 14.4\% & 16.7\%\\\hline
\end{tabular}
\end{center}
\end{table}


正压模态的改进使得整个海洋模式POP的运行时间也随之下降。
表 \ref{tab:improve_1} 给出了三个新的求解器和预处理子的不同搭配相比于原始的对角预处理的ChronGear求解的所带来的性能提升百分比。 
这里给出的时间都是 取之于一个五天的模拟,同时模式的初始化和 I/O 操作都没有考虑在内。 
P-CSI 结合EVP预处理在768核上取得了  16.7\% 的加速效果。
虽然16.7\% 的提高看起来并不显著,但是由于 1度分辨率的海洋模式POP通常的模拟时间尺度都是几个世纪, 这样的一个性能的改进在如此长期的模拟中可以节省数百万个小时的CPU计算时间。 
另外,当海洋模式POP中开启生物化学的模块时,即使是1度的分辨率也需要使用相对较多的处理器核心,因为这个模块需要引进很多新的示踪物,进而导致计算量增加很多。 

\subsection{高分辨率模拟}
\begin {figure*}[t!]
\begin{center}
\includegraphics[height =6.5cm]{NEW01deg_solverruntime}
\hspace{10pt}
\includegraphics[height =6.5cm]{NEW01deg_speedup_ys}
\end{center}
\caption []{ 在美国黄石超级计算机上0.1度 POP中正压模态模拟一天的的执行时间(左), 
在美国黄石超级计算机上0.1度 POP核心部分的模拟速率(右)。\label {fig:runtime01}}
\end {figure*}


现在我们测试在黄石超级计算机上,高分辨率0.1度的海洋模式POP的新的正压求解的可扩展性。 
在这个高分辨下, 海洋模式中网格块的大小和分布得选择,会影响到进程上任务的分发,进而对性能造成很大的影响。 
因此,为了不让这种选择影响到最终的可扩展性结果,我们非常仔细的将每个进程上的网格块的横纵比例统一地设置为3:2,而陆地的比例设置为0.25,同时还利用空间填充曲线的方法来划分网格。 
我们使用POP中默认的时间步长,也就是500个时间步每天(dt\_count = 500)。 
最后,为了统一,所有的求解器都是每十个迭代步做一次收敛判断。 
这里值得一提的是,由于P-CSI的迭代过程的开销相对比较小(相比于海洋模式POP中的收敛性判断), 因此,如果将收敛性判断的频率进一步降低的话, P-CSI的性能可能能够进一步的提高。 
 

如图\ref{fig:runtime01}左边所示,ChronGear的性能在所使用的处理器核心数大于2700时反而会有所下降,
而P-CSI的运行时间在这之后相对比较平缓。 
采用对角预处理的前提下, P-CSI方法在16,875核上将0.1度的海洋模式POP的正压模态的性能提高了4.3倍(从19.0秒下降到4.4秒每模拟天)。 
使用EVP预处理能够进一步的改进ChronGear和P-CSI的性能, 使得这两个算法分别相对于原始的正压模态求解器加速了1.4倍和5.2倍。
在\ref{se:baro}节中,我们证明了原始的正压模态求解器在海洋模式POP的总运行时间所占的比例会随着使用的核心数目的增加而增大。
尤其是在16,875个核上,采用对角预处理的ChronGear求解器占到总运行时间的 50\%以上。 
相对应的,图\ref{fig:StepComp_pcsi}表明,采用了具有 更高可扩展性的EVP预处理的P-CSI求解器后, 同样是在 16,875 核上运行,正压模态的执行时间仅仅占到总运行时间的16\%。 

 
正压求解器性能上的改进有助于提高海洋模式POP的总体性能, 尤其是在大规模并行环境中。
模拟速率(模拟年每天)是一个被广泛采用的评估模式性能的指标, 我们这里采用不考虑模式初始化和I/O开销的 核心模拟速率。这是由于模式初始化只是在模式运行开始的时候执行一次,而模式的I/O开销是随着用户设置的频率的变化而改变的。 
通常专家们认为,如果想要进行长期的气候模拟,那么模拟速度最小要达到五个模式年每天\cite{dennis2012computational}, 图\ref{fig:runtime01} 右边可以看出,模式采用新的P-CSI求解器能够达到比采用ChronGear求解器更高的模拟速率。 
结合EVP预处理的P-CSI求解器在16,875核心上使得核心模拟速率提高了1.7倍,从原来的6.2模拟年年每天到10.5模拟年每天。 
 

\begin {figure}[t!]
\centering

\includegraphics[height=6.5cm]{NEWPOPStepComp_pcsi.eps}
\caption[] { 0.1度海洋模式 POP采用EVP块预处理的P-CSI后的各部分的时间分析。 \label{fig:StepComp_pcsi}}
\end{figure}



为了找到性能提升的主要来源,\ref{fig:component}给出了正压求解器的具体的时间成分分析。 
图\ref{fig:component}中可以看出, P-CSI求解器比原始的ChronGear求解好的主要原因是因为全局归约操作的减少。 
全局归约操作的减少同时也显著的减少了海洋模式POP对操作系统噪声的敏感程度\cite{ferreira},因为更少的归约操作意味着相邻两次全局同步之间的时间变得更长了。 
除此之外,EVP块预处子还可以进一步减少边界通信,因为它是的达到收敛所需要的迭代次数变少了。 
在大规模并行环境下,正压求解器的计算开销相比于全局归约操作和边界通信的开销来说基本上可以忽略,因此,EVP预处理子引入的那多出来的一倍的计算量对总体的运行时间影响不大。
最后我们还要注意到,在处理器核心数小于1200时,全局归约操作的时间是随着进程数的增加而减少的,这与等式\ref{t_pcg}所给出的理论结果是一致的。

 

\subsection{Edison超级计算机测试结果}
Now we run the 0.1度 POP simulations on the Edison supercomputer
to verify that performance improvements are not unique to Yellowstone.
Edison, which is the newest supercomputer at the National Energy
Research Scientific Computing Center (NERSC), consists of 133,824 2.4
GHz Intel
为了证明新的求解器不仅仅在黄石超级计算机上有性能提升,我们还在Edison超级计算机上测试0.1度海洋模式POP的模拟效果。Edison超级计算机是美国能源研究科学计算中心(NERSC)的最新的一台超级计算机,它是由 133,824 个2.4 GHz的英特尔的``Ivy Bridge'' 处理器核心组成,这些处理器核心是通过 8GBps 的Cray
Aries高速网络以蜻蜓拓扑的结构连接起来。  
\begin {figure*}[t!]
\begin{center}
\includegraphics[height =7cm]{01deg_comp_all_gs}
\hspace{10pt}
\includegraphics[height =7cm]{01deg_comp_all_halo}
\end{center}
\vspace{-.2in}
\caption[] {黄石超级计算机上,0.1度海洋模式POP的正压求解器中主要的耗时成分的分析:全局归约(左)和边界通信(右)。  }
\label{fig:component}
%\vspace{-.2in}
\end {figure*}
\begin {figure*}[t!]
\begin{center}
\includegraphics[height=6.5cm]{01deg_solverruntime_edison}
\hspace{10pt}
\includegraphics[height=6.5cm]{01deg_speedup_edison}
\end{center}
\vspace{-.2in}
\caption []{在美国Edison超级计算机上0.1度 POP中正压模态模拟一天的的执行时间(左), 
在美国黄石超级计算机上0.1度 POP核心部分的模拟速率(右)。\label {fig:runtime01_edison}}
\vspace{-.2in}
\end {figure*}

图\ref{fig:runtime01_edison}可以看出,四种配置的正压求解器在Edison超级计算机上的模拟性能与在黄石超级计算机上的模拟性能有相近的特点。 
我们注意到在Edison超级计算机上模拟时,全局通信的时间开销要比在黄石超级计算机上的开销更不稳定一些,主要可能是因为网络资源的竞争造成的\cite{wang2014}。 
这最终导致了ChronGear求解器(不管采用什么预处理子)的运行时间在多次模拟中相差很大,因此我们将这些实验中最好的三个结果取平均值作为它的运行时间。 
由于P-CSI几乎没有全局归约操作(只发生在检查收敛性的时候), 这种多次运行中的差别很小。 

 
在Edison超级计算机上, 采用对角预处理的P-CSI方法使得0.1度的海洋模式POP的正压模态在16,875核心上加速了3.7倍,从26.2秒每模拟天下降到7.0秒每模拟天。 
使用EVP预处理,ChronGear和P-CSI方法的性能都得到提升。 
采用EVP块预处理的P-CSI方法使得正压模态加速了5.6倍。 
 
 
\begin{table}
\begin{center}
\caption {在黄石超级计算机上,0.1度海洋模式POP的核心模拟速率\label{tab:improve_01}}
\begin{tabular}{|l||r|r|r|r|r|r|r|}
\hline
Solver & 470  & 1200   & 2700 & 4220 & 7500 & 10800 & 16875\\\hline
\hline
ChronGear     &0.7 &1.7&3.4  &4.6 &6.0 &6.0 &5.3\\\hline
ChronGear+Evp &0.7 &1.7&3.4  &4.9 &6.6 &6.9 &6.5\\\hline
P-CSI+Diag    &0.7 &1.7&3.5  &5.0 &7.0 &8.3 &8.9\\\hline
P-CSI+Evp     &0.7 &1.7&3.5  &5.0 &7.2 &8.6 &9.3\\
\hline
\end{tabular}
\end{center}
\end{table}

\section{本章小结}
\label{sec:precondConclusion}
