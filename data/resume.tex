\begin{resume}

  \resumeitem{个人简历}

  1988 年 10 月 1 日出生于湖北省通城县。

  2007 年 9 月-- 2011年7月, 哈尔滨工业大学,信息与计算科学,学士。

  2011 年 9 月--至今, 清华大学,计算机科学与技术,攻读博士学位。

  \researchitem{发表的学术论文} % 发表的和录用的合在一起

  % 1. 已经刊载的学术论文(本人是第一作者,或者导师为第一作者本人是第二作者)
  \begin{publications}
    \item Hu, Yong and Huang, Xiaomeng and Baker, Allison H and Tseng, Yu-heng and Bryan, Frank O and Dennis, John M and Yang, Guangwen. Improving the Scalability of the Ocean Barotropic Solver in the Community Earth System Model, The International Conference for High Performance Computing, Networking, Storage, and Analysis, 2015.  (CCF推荐A类会议)
    \item  Hu, Yong and Huang, Xiaomeng and Wang, Xiaoge and Fu, Haohuan and Xu, Shizhen and Ruan, Huabin and Xue, Wei and Yang, Guangwen. A Scalable Barotropic Mode Solver for the Parallel Ocean Program. Euro-Par 2013 Parallel Processing. (CCF推荐C类会议) 
  \end{publications}

  % 2. 尚未刊载,但已经接到正式录用函的学术论文(本人为第一作者,或者
  %    导师为第一作者本人是第二作者)。
  \begin{publications}[before=\publicationskip,after=\publicationskip]
    \item A. H. Baker, Hu Yong, D. M. Hammerling, Y. Tseng, Xiaomeng Huang, F. Bryan. Evaluating Consistency in the Ocean Model Component of the Community Earth System Model. Geoscientific Model Development Disscussions, GMDD 2016. (SCI检索)
  \end{publications}

  % 3. 其他学术论文。可列出除上述两种情况以外的其他学术论文,但必须是
  %    已经刊载或者收到正式录用函的论文。
  \begin{publications}
    \item Xu Shizhen, Huang Xiaomeng, Zhang Yan, Hu Yong, Yang Guangwen. A Customized GPU Acceleration of the Princeton Ocean Model. The 25th IEEE International Conference on Application-specific Systems, Architectures and Processors. (EI检索)
     \item  Xu Shizhen, Huang Xiaomeng, Zhang Yan, Hu Yong, Fu Haohuan,  Yang Guangwen. Porting the Princeton Ocean Model to GPUs. In Algorithms and Architectures for Parallel Processing (pp. 1-14). Springer International Publishing, 2014. (EI检索).
    \item Wang Wencan, Huang Xiaomeng, Fu Haohuan, Hu Yong, Xu Shizhen, Yang Guangwen. CFIO: A Fast I/O Library for Climate Models. In Proc. Of the 11th IEEE International Symposium on Parallel and Distributed Processing with Applications, 2013. (EI检索)
    
  \end{publications}

  % \researchitem{研究成果} % 有就写,没有就删除
  % \begin{achievements}
  %   \item 任天令, 杨轶, 朱一平, 等. 硅基铁电微声学传感器畴极化区域控制和电极连接的
  %     方法: 中国, CN1602118A. (中国专利公开号)
  % \end{achievements}

\end{resume}
