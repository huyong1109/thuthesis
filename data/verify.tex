\chapter{海洋模式正确性验证}
\label{cha:verify}

\section{本章概述}

\section{背景和动机}
\label{sec:verifyBackgroud}

\section{海洋模式正确性验证的特点}
\label{sec:verifyPart1}
\subsection{传统正确性方法}
\label{sec:verifyPart11}

\section{基于集合模拟的统计方法}\label{se:ver}

 
由于海洋动力过程本身具有混沌性,即使在正压求解中引入一个浮点数舍入误差精度级别的扰动,也可能使得最终的模式结果与原始结果截然不同。 
因此,既然我们不能保证引入新的求解器之后所得到的结果与原始结果是二进制一致的,那么我们在将新的P-CSI求解器和EVP预处理正式的加入到海洋模式POP的版本中之前,需要证明加入它们之后不会造成模拟结果的不正确性,更不能改变所得到的气候的状态。 

当我们将海洋模式POP移植到新的机器上时,也会遇到类似的问题,即不同的机器上得到的模拟结果是不能做到二进制一致的。 
目前海洋模式POP中现成验证新的机器是否移植成功的标准是将特定的样例在新的机器上模拟五天,然后计算一下所得到的结果中的海表高度场与美国国家大气研究中心所提供的标准数据集中的海表高度场之间的均方根误差(RMSE)。
这个计算过程提供了一个比较简单的评估地球系统模式CESM在新的机器上的运行结果(这些结果可能是有软件环境或者硬件环境引入的错误的)的方法, 但是我们发现它并不能检测和评估求解器引入的误差。 
比如, 我们跑一些1度的例子,这些例子中正压求解器的收敛条件分别为$10^{-10}$ 到$10^{-16}$之间的值(默认为 $10^{-13}$),然后计算这些例子与使用最严的收敛条件($10^{-16}$)的例子的模拟结果的均方根误差。 
图 \ref{fig:ssh_rmse_t} 中给出了采用不同的收敛条件得到的模拟结果中的温度场的每个月的均方根误差。 
值得一提的是,为了凸显出线性求解器的影响,我们只考虑公海区域的值(POP在几个边缘海内的模拟并不算好)。 
很明显的可以看出,改变求解器的收敛条件所引入的误差并没有在最后的温度场中得到体现(也没有在速度和海表高度等诊断场中的得到明显的体现)。 
收敛条件越松,求解器中引入的误差就会越大。
因此我们希望看到使用最松的两个收敛条件的($10^{-10}$和$10^{-11}$)的实例的运行结果所计算出来的均方根误差比其他实例的要大一些。 
但是,我们的实验证明这并不成立。 
在第12个月到第20月之间,收敛条件为$10^{-10}$的例子的均方根误差在所有实例中是最小的。 


正是因为现成的简单的均方根误差测试并不能检测出来模拟结果的气候态是否被改变,因此我们提出了一种新的基于统计的评估海洋正压求解器的方案。 
这个方法的灵感来源于Allison的文章\cite{baker2014methodology},他们为了验证公共地球系统模式CESM的公共大气模式(CAM)中的数据被压缩之后是否仍然有效,并不是仅仅使用一次模拟的结果,而是利用集合模拟的结果 来作为评估无法做到二进制一致的改动的基准。 
我们的方案和这篇文章所提到的方案很类似\cite{baker2014methodology}, 我们先跑一系列的实例。这些实例除了在初始的海洋温度场中加入了$10^{-14}$量级的随机扰动之外,和默认的配置完全相同。 
这个量级的扰动应该不会造成模拟结果的气候状态变化。 
我们发现40个这个样的实例就足以表现出海洋模式自身的扰动性, 但是由于海洋模式中需要模拟的海洋运动的时间尺度更长, 我们这个集合模拟中的实例也模拟了比大气模式CAM的更长的时间。 
值得一提的是,我们最终决定选择三维的温度场作为评估对象(而不是像现有的评估方式那样选择二维的海表高度),因为我们发现温度场是找到不同例子之间差异的最好的诊断变量。 
 




 
我们用如下的流程来判断一个新的结果是否和指定的集合模拟结果相容。 
设集合模拟在某个时间$T$的关于变量$X$的输出结果集合为 $\mathcal{E}=\{X_1,X_2,...,X_m\}$, 这里
$m$ 表示集合模拟的大小。 
在某个给定点$j$, 我们就可以从集合模拟中得到关于变量$X$的一个集合
  $\{X_1(j),X_2(j),...,X_m(j)\}$.
随着集合模拟的规模的增加,这个数列就能够更加准确的反应出来给定点的状态的分布。 
我们假设在点$j$上这个数列的平均值和方案为别为$\mu (j) $和$\delta (j)$。 
然后假设新的实例的模拟结果为 $\tilde{X}$, 那么以下定义的这个均方根标准分就可以表示这个新的实例和集合模拟结果之间的平均误差:

$$ RMSZ(\tilde{X}, \mathcal{E}) =  \sqrt{\frac{1}{n}\sum_{j=1}^n(\frac{\tilde{X}(j) -\mu (j)}{\delta (j)})^2}$$

我们使用这个基于集合的均方根标准分的指标来重新评估以下不同的求解器收敛条件的影响。 
图\ref{fig:ssh_rmsz_t} 可以看出,与简单的均方根误差不同, 这个新的基于集合模拟的方法能够发现收敛条件越松,结果中的误差就越大。 
现在,使用最松的两个收敛条件的例子得到的均方根Z分数也是最大的,可以明显的与集合模拟的结果区分开来。 
最后,我们用这个基于集合模拟的判断指标来评估一下我们的新的P-CSI求解器。 我们发现使用P-CSI求解或者使用最紧的两个收敛条件所得到的结果与集合模拟的结果是相容的。 
因此,我们确信我们的新的求解器并没有在模式中引入气候模式意义上不可接受的误差。 


\begin{figure}[!t]
\begin{center}
\includegraphics[height=6.5cm]{temp_rmse.eps}
\end{center}
\caption[] {采用不同收敛条件的1度海洋模式POP模拟结果中的温度场的每个月的均方根误差(RMSE)。}
\label{fig:ssh_rmse_t}
\end{figure}
\begin{figure}[!t]
\begin{center}
\includegraphics[height=6.5cm]{temp_rmsz.eps}
\end{center}
\caption[] {采用不同收敛条件的1度海洋模式POP模拟结果中的温度场的每个月的均方根标准分(RMSZ)。黄色的区域表示集合模拟中40个实例结果的均方根标准分的分布。 }
\label{fig:ssh_rmsz_t}
\end{figure}

\section{试验方法和结果分析}
\label{sec:verifyExp}

\section{本章小结}
\label{sec:verifyConclusion}
