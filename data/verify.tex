\chapter{海洋模式正确性验证}
\label{cha:verify}

\section{本章概述}
并行海洋模式 (POP)是公共地球系统模式(CESM)的海洋模式分量,它被广泛的应用于气候研究。 
目前有很多关注如何提高  CESM-POP性能和正确的工作,比如提高数值算法的性能,改进参数化方案,移植到新的架构上,以及增加并行度等。 
由于海洋模式动力过程混沌的特性,即使对代码做一些细微的改动,也无法保证修改后的模式得到的结果与原始结果是二进制一致的。 
并且,判断模式中的改动是不是可以被接受的(也就说是不是在统计的意义上与原始结果一致)并非易事。 
最近的一个工作显示,对大气模式的数据利用基于集合模拟的统计方法取得了很好的效果。
这个基于集合模拟的统计学一致性检测工具的核心是利用集合模拟结果的可变性的量化结果来作为评判未来模拟的标准,并且利用统计学上的区分度来做判断。 
由于海洋模式和大气模式在动力学和时间尺度上有着明显的区别, 我们这里采用了一种需要结合集合模拟的平均状态和方差的统计学方法来对海洋模式的模拟结果进行评测。 
具体来说,我们利用CESM-POP集合模拟的结果的统计学分布来计算新的模式结果在每一个点上的标准分。 
那些标准分大于某一个给定值在所有点中所占的比例就可以反映出新的模式结果与集合模拟结果的统计学差别。 
这里,集合模拟的大小和成员的挑选都很重要。
我们的实验表明,新的海洋模式POP的集合一致性检测工具(POP-ECT)能够区分那些统计学上明显与集合模拟的结果一致的模拟结果和那些明显不一致的模拟结果。 
同时,这个检测工具提供了一个简单、客观和系统的方法,来检测CESM-POP是否有硬件或者软件层引入的错误。
这个工具进一步提升了CESM-POP代码的质量保证。 

公共地球系统模式CESM是一个应用广泛的全耦合气候模拟程序\citep{cesm2013}, 它的模拟结果经常被提交到政府间气候变化委员会的评估报告中\cite{stocker2013ipcc}。
CESM由多个分量模式耦合而成,其中包括大气、海洋、海冰、和陆地等的分量模式。 
这里,我们专注于CESM的海洋模式分量并行海洋模式POP。 POP是由Los Alamos国家实验室研发的大洋环流模式的一个扩展版本。 
CESM-POP求解的是一个采用静力平衡近似和Boussinesq近似的海洋动力学三维原始方程,它描述的海洋模式过程在时间和空间上的尺度都很大。  

 
 
最近CESM-POP中的很多进展都是关心如何减少计算开销\citep{yong2015}, 但是数值模拟程序的中的任何开发都需要软件质量保证,来确保不会在程序中引入错误。 
这种维护科学程序的可信性的质量保证,对于气候模式来说尤为重要。 
因为气候模式的模拟结果会影响到关系国计民生的政策方针\citep{carson2002, easterbrook2011}。 

  
气候模式,比如CESM,通常都代码量大并且程序复杂。 模式中过多的配置选项使得很难穷尽的测试它们 \citep{clune2011, pipitone2012}。 
另外,由于气候模式本身的混沌性,判断一个新的模拟结果中的差别是由某个错误引起的还是由于模式本身的不确定性引起是一件非常具有挑战的事情。 
我们注意到,在模式的初始条件或者中间结果中引入浮点数精度级别的随机扰动都会使得最后只能够的结果产生比较大的偏差。
CESM-POP中的一些新的成果,尤其是那些提高性能的成果(比如利用新的异构加速技术和改进数值方法等),通常都会导致新的模式的输出结果与原来的结果不能保证是二进制一致的。 
对于CESM-POP来说,即使是在同一个超级计算机上采用不同的处理器核数来计算,得到的结果都不会是二进制一致的。 
如果可以直接评估CESM-POP的海洋数据的气候一致性,将会极大的促进模式的的发展,同时使得模式能够更加灵活的应用新的软件或者硬件技术。 


最近由Allison等人\cite{baker2015}研发的CESM集合模拟一致性检测工具(CESM-ECT),通过一个新的基于集合模拟的工具解决了比较气候模式输出结果的难题。 
这个工具能够评判一个新的气候模拟结果(比如引入了软件或者硬件的改动)是否与以原始结果(没有任何改动的版本)为基础构造的被认为是可以接受的集合模拟的结果在统计学意义上是否是一致的。 
但是文章\cite{baker2015}中的这个CESM-ECT工具只评估了CESM中大气分量模式--公共大气模式(CAM)的变量,并且其实验并不是完全耦合的CESM配置(比如说其中海洋分量并不是采用的并行海洋模式POP,而是海洋气候数据模式。这个模式只能够提供海表温度数据,但是不能相应气候分量模式提供的强迫)。 
为了简洁,我们将为CESM的通用的集合模拟统计一致性测试工具称之为CESM-ECT。 
我们将应用于公共大气模式分量的检测方法称之为CAM-ECT,它是我们研发的CESM-ECT工具集中的一个模块。 
值得注意的是,直接将CAM-ECT方法应用于海洋数据并不可行,因为海洋和大气在动力学、空间尺度和时间尺度上的差别都很大。
比如, 海洋模式中气候尺度比大气中的气候尺度要小一到两个量级,海洋,尤其是深海中运动的时间尺度比大气中的要慢很多个量级。
因此, 我们专门为海洋模式中的统计一致性检测开发了一个新的方法,我们称之为POP-ECT。 
尽管这个新的POP-ECT工具也是利用CESM的集合模拟的结果来评估模式的不确定性,它和CAM-ECT是由本质区别的。 
POP-ECT的统计过程中考虑到了海洋中不确定性的空间分布特点。另外,由于海洋达到全球准稳态所需要的时间比大气需要的时间长得多,我们在POP-ECT中采用与CAM-ECT中不一样诊断时间。 
同时,海洋模式中的可用的诊断变量的个数比较少,也使得我们能够采用跟大气模式不一样的方案。 
最后,值得一提的是,本章中所有的实验结果都是采用CESM 的1.2.2版本。
\section{背景和动机}
\label{sec:verifyBackgroud}

\section{海洋模式正确性验证的特点}
\label{sec:verifyPart1}
\subsection{传统正确性方法}
\label{sec:verifyPart11}
目前海洋模式POP中可用的正确性测试工具(我们称之为POP-RMSE),是一个比较简单的评估CESM-POP程序是否成功的移植到了一个新的机器上。 
这个工具的主要目标是发现与新机器上硬件或软件层的问题。 
这个测试中,首先将特定设置的实验在新的机器上运行五个模拟天,然后将其输出结果和美国国家大气研究中所提供 的标准的数据集相比较。
这里计算了两个结果的海表高度场的均方根误差(RMSE)。 比如对于两个包含$n$个值的数据集合$X_0$ 和$X_1$,它们的均方根误差可以表示为
 \begin{equation*}
 RMSE(X_0, X_1) = \sqrt{\frac{1}{n}\sum_{i=1}^n(X_1(i)-X_0(i))^2}.
 \end{equation*}
然后,拿得到的均方根误差的增长曲线与给定的两个样例所得到的曲线相比较。 
这两个样例中,一个是原始的配置,另一个是在求解器中采用比默认收敛条件要高一个量级的收敛条件。 

POP-RMSE中这个简单的方法对于评估CESM在新的机器上的输出结果是很方便的, 但是它比为大气模拟数据设计的CAM-ECT的表达能力要弱得多。 
比如说,POP-RMSE无法评判CESM-POP中最新的线性求解中的改动所带的微小的气候状态的改变 \cite{yong2015}。 
在文章\cite{yong2015}中,作者将默认的预处理共轭梯度法求解器替换成预处理的Chebyshev迭代算法, 以此来提高CESM-POP的计算性能。
尽管新的P-CSI求解器在高分辨率模式中的通信开销比原始的预处理共轭梯度法的要小很多, 证明这个新的求解器不会对海洋的模拟结果造成负面的影响对于这个求解能否被接受是及其重要的。 
因此,在 \cite{yong2015}中, 为了验证POP-RMSE是否能够检测出求解器在不同时间所引入的误差, 我们搜集了1度分辨率的CESM-POP在不同的收敛条件($10^{-10}$ 到$10^{-16}$)下36个月的运行结果。  图   \ref{fig:rmse_temp} 给出的是最紧的收敛条件的实例  ($10^{-16}$) 和其他收敛条件的实例的结果的温度场之间的均方根误差。 
尽管这些例子采用不同的收敛条件,图\ref{fig:rmse_temp} 中很难体现出来这些由于求解器误差所带来的影响  \citep{yong2015}。 

 
由于海洋动力过程本身具有混沌性,即使在正压求解中引入一个浮点数舍入误差精度级别的扰动,也可能使得最终的模式结果与原始结果截然不同。 
因此,既然我们不能保证引入新的求解器之后所得到的结果与原始结果是二进制一致的,那么我们在将新的P-CSI求解器和EVP预处理正式的加入到海洋模式POP的版本中之前,需要证明加入它们之后不会造成模拟结果的不正确性,更不能改变所得到的气候的状态。 

当我们将海洋模式POP移植到新的机器上时,也会遇到类似的问题,即不同的机器上得到的模拟结果是不能做到二进制一致的。 
目前海洋模式POP中现成验证新的机器是否移植成功的标准是将特定的样例在新的机器上模拟五天,然后计算一下所得到的结果中的海表高度场与美国国家大气研究中心所提供的标准数据集中的海表高度场之间的均方根误差(RMSE)。
这个计算过程提供了一个比较简单的评估地球系统模式CESM在新的机器上的运行结果(这些结果可能是有软件环境或者硬件环境引入的错误的)的方法, 但是我们发现它并不能检测和评估求解器引入的误差。 
比如, 我们跑一些1度的例子,这些例子中正压求解器的收敛条件分别为$10^{-10}$ 到$10^{-16}$之间的值(默认为 $10^{-13}$),然后计算这些例子与使用最严的收敛条件($10^{-16}$)的例子的模拟结果的均方根误差。 
图 \ref{fig:ssh_rmse_t} 中给出了采用不同的收敛条件得到的模拟结果中的温度场的每个月的均方根误差。 
值得一提的是,为了凸显出线性求解器的影响,我们只考虑公海区域的值(POP在几个边缘海内的模拟并不算好)。 
很明显的可以看出,改变求解器的收敛条件所引入的误差并没有在最后的温度场中得到体现(也没有在速度和海表高度等诊断场中的得到明显的体现)。 
收敛条件越松,求解器中引入的误差就会越大。
因此我们希望看到使用最松的两个收敛条件的($10^{-10}$和$10^{-11}$)的实例的运行结果所计算出来的均方根误差比其他实例的要大一些。 
但是,我们的实验证明这并不成立。 
在第12个月到第20月之间,收敛条件为$10^{-10}$的例子的均方根误差在所有实例中是最小的。 


正是因为现成的简单的均方根误差测试并不能检测出来模拟结果的气候态是否被改变,因此我们提出了一种新的基于统计的评估海洋正压求解器的方案。 
这个方法的灵感来源于Allison的文章\cite{baker2014methodology},他们为了验证公共地球系统模式CESM的公共大气模式(CAM)中的数据被压缩之后是否仍然有效,并不是仅仅使用一次模拟的结果,而是利用集合模拟的结果 来作为评估无法做到二进制一致的改动的基准。 
我们的方案和这篇文章所提到的方案很类似\cite{baker2014methodology}, 我们先跑一系列的实例。这些实例除了在初始的海洋温度场中加入了$10^{-14}$量级的随机扰动之外,和默认的配置完全相同。 
这个量级的扰动应该不会造成模拟结果的气候状态变化。 
我们发现40个这个样的实例就足以表现出海洋模式自身的扰动性, 但是由于海洋模式中需要模拟的海洋运动的时间尺度更长, 我们这个集合模拟中的实例也模拟了比大气模式CAM的更长的时间。 
值得一提的是,我们最终决定选择三维的温度场作为评估对象(而不是像现有的评估方式那样选择二维的海表高度),因为我们发现温度场是找到不同例子之间差异的最好的诊断变量。 
 




 
我们用如下的流程来判断一个新的结果是否和指定的集合模拟结果相容。 
设集合模拟在某个时间$T$的关于变量$X$的输出结果集合为 $\mathcal{E}=\{X_1,X_2,...,X_m\}$, 这里
$m$ 表示集合模拟的大小。 
在某个给定点$j$, 我们就可以从集合模拟中得到关于变量$X$的一个集合
  $\{X_1(j),X_2(j),...,X_m(j)\}$.
随着集合模拟的规模的增加,这个数列就能够更加准确的反应出来给定点的状态的分布。 
我们假设在点$j$上这个数列的平均值和方案为别为$\mu (j) $和$\delta (j)$。 
然后假设新的实例的模拟结果为 $\tilde{X}$, 那么以下定义的这个均方根标准分就可以表示这个新的实例和集合模拟结果之间的平均误差:

$$ RMSZ(\tilde{X}, \mathcal{E}) =  \sqrt{\frac{1}{n}\sum_{j=1}^n(\frac{\tilde{X}(j) -\mu (j)}{\delta (j)})^2}$$

我们使用这个基于集合的均方根标准分的指标来重新评估以下不同的求解器收敛条件的影响。 
图\ref{fig:ssh_rmsz_t} 可以看出,与简单的均方根误差不同, 这个新的基于集合模拟的方法能够发现收敛条件越松,结果中的误差就越大。 
现在,使用最松的两个收敛条件的例子得到的均方根Z分数也是最大的,可以明显的与集合模拟的结果区分开来。 
最后,我们用这个基于集合模拟的判断指标来评估一下我们的新的P-CSI求解器。 我们发现使用P-CSI求解或者使用最紧的两个收敛条件所得到的结果与集合模拟的结果是相容的。 
因此,我们确信我们的新的求解器并没有在模式中引入气候模式意义上不可接受的误差。 


\begin{figure}[!t]
\begin{center}
\includegraphics[height=6.5cm]{temp_rmse.eps}
\end{center}
\caption[] {采用不同收敛条件的1度海洋模式POP模拟结果中的温度场的每个月的均方根误差(RMSE)。}
\label{fig:ssh_rmse_t}
\end{figure}
\begin{figure}[!t]
\begin{center}
\includegraphics[height=6.5cm]{temp_rmsz.eps}
\end{center}
\caption[] {采用不同收敛条件的1度海洋模式POP模拟结果中的温度场的每个月的均方根标准分(RMSZ)。黄色的区域表示集合模拟中40个实例结果的均方根标准分的分布。 }
\label{fig:ssh_rmsz_t}
\end{figure}


\section{基于集合模拟的统计方法}\label{se:ver}
由于RMSE方法并不能明显的检测出不同的收敛条件所带来的影响,文章 \inlinecite{yong2015} 中采用了一种集合模拟的方法来测试海洋模式POP的输出数据。 
这个工作是基于文章\inlinecite{baker2014}中提出的测试数据压缩的影响的方法(这也是CESM-ECT方法的前身)。
需要注意的是,\inlinecite{baker2014}中提到的集合模拟是通过对初始值进行扰动得到的。 
但是,改变海洋模式中正压模态线性求解器的收敛性判据或者像文章 \inlinecite{yong2015}中所提到的引进新的求解器,都可以近似为强迫场的扰动。 
我们很有必要先研究一下,仅仅通过初始场的扰动所得到的结果状态集是否包含了在每一个时间步上对强迫场进行扰动所生成的状态。
正如文章\inlinecite{caya1998}所做的那样,我们将海洋模式POP的时间迭代过程用下面这个只有几个部件的简单方程来表示:
\begin{equation}
\frac{\partial X(t)}{\partial t} = \alpha X(t) +G,
\label{e:1}
\end{equation}
这里 $G$ 表示一个固定的强迫项,  $\alpha$是一个常数,  $X(t)$是关于时间变量的函数, $X(0) = X_0$。 
方程 (\ref{e:1}) 对应的解析解为:
\begin{equation}
X(t) = X_0e^{\alpha t } -\frac{G}{\alpha}. 
\label{e:2}
\end{equation}
这个例子里面,我们假设$\alpha$ 是一个纯虚数,这样 $\alpha X(t)$ 就是一个震动波(这里我们不考虑减振机制)。 

 
首先,我们考虑一下初值中的扰动:
\begin{equation*}
X_0' =X_0+P,
\end{equation*}
这里 $P$ 是一个常量。方程 (\ref{e:1}) 在采用扰动后的初始值之后的解析解可以表示为 
\begin{equation}
X_P(t) = (X_0+P)e^{\alpha t } -\frac{G}{\alpha}, 
\label{e:Pan}
\end{equation}
也可以表示为
\begin{equation}
X_P(t)-X(t) = Pe^{\alpha t }.
\label{e:Perr}
\end{equation}
方程 (\ref{e:Perr}) 表明原始结果和然后处置条件后得到的结果的差是振荡的。
下面我们考虑当在强迫场中加入一个扰动时: 
\begin{equation*}
G' = G + F,
\end{equation*}
这里 $F$ 是一个常量。  方程(\ref{e:1}) 加入带扰动的强迫项后,其解析解可以表示为
\begin{equation}
X_F(t) = X_0e^{\alpha } -\frac{G+F}{\alpha},
\label{e:Fan}
\end{equation}
也可以等价于
\begin{equation}
X_F(t)-X(t) = \frac{F}{\alpha}. 
\label{e:Ferr}
\end{equation}
方程 (\ref{e:Ferr}) 表明强迫场中加入干扰项后对结果产生的固定的。
因此,我们可以用方程(\ref{e:Perr})的初始场中引入扰动所带来的误差的量级来衡量在方程(\ref{e:Ferr})的强迫场中引入扰动后对结果的影响。 
举个例子,在图  \ref{fig:1Danalytical}中我们描述了一个初始扰动和强迫扰动对方程(\ref{e:2})的最终结果的影响相似的例子。在这个例子中,强迫项的扰动所引起的误差比初始条件扰动所赢得误差还小。
这张图的最上面显示的是方程(\ref{e:Pan}) 和 (\ref{e:Fan})所给出的解析解,下面显示的方程 (\ref{e:Perr}) 和 (\ref{e:Ferr}) 所给出的与 $P$ 和 $F$ 的值相对应的误差。 
因此,我们可以看出,文章\inlinecite{yong2015}中选择使用基于由扰动海洋模式初始的温度场而得到的集合模拟结果来评估求解器所引入的不确定是合理的。


 
文章\inlinecite{yong2015}的具体做法是,首先生成了一个包含有40个运行了36个月的模拟的集合,这些实例与原始实例的差别在于在初始的海洋温度场引入了一个$\mathcal{O}(10^{-14})$量级的扰动。 
然后计算新的模拟结果 $\tilde{{X}}$和这个集合模拟的结果用如下的方式计算均方根标准分。 假设集合$E$所包含的成员的个数记为 $N_{ens}$,此时,对于每一变量$X$,在每一个点$i$,都存在一个包含$N_{ens}$个数的数列。
定义点$i$上的集合模拟结果中对应的数列的所有元素的平均值和标准方差分别为$\mu_i$ 和 $\sigma_i$。 
由于总的网格点数为$n$,那么对于变量$\tilde{{X}}$,新的模拟结果与集合模拟结果$E$之间的均方根标准分定义为

\begin{equation}
 RMSZ(\tilde{X}, E) = \sqrt{\frac{1}{n}\sum_{i=1}^n(\frac{\tilde{x}_i -\mu_i}{\sigma_i})^2}. 
\label{e:rmsz}
\end{equation}
图\ref{fig:rmsz_temp_ens} 给出了与图\ref{fig:rmse_temp}中相同的五个收敛条件的模拟结果的RMSZ分数。
这里,由最宽松的收敛条件引入的误差变得非常的明显。 
基于图\ref{fig:rmsz_temp_ens}的结果,文章\inlinecite{yong2015}证明了新的求解器在CESM-POP中是可以接受的。  


\subsection{公共大气模式的集合模拟一致性检测 }
 
文章 \inlinecite{baker2015}中提到的CESM-ECT检测工具利用集合模拟的结果来对CESM模式的气候态的内在的不确定性进行量化,进而将新的模拟结果(由于软件、硬件或者其他不能保证二进制一致的改动引入的新的结果)与这个集合模拟结果的分布做对比。
这个想法的关键是,如果新的模拟结果的输出数据与集合模拟的结果在统计学上是不可以区分的话,那么这个新的结果就认为是与原始结果“一致的”。统计学上的一致性是模式代码正确性验证的质量保证方面的关键因素\citep{oberkampf2010}。
CESM-ECT方法首先被应用于公共大气模式CAM分量的历史输出数据,最终形成了CAM-ECT工具集。
CAM数据很直观的就被当成了验证的首要目标,因为大气中扰动传播的时间尺度要比在CESM其他分量中的短很多。而且,CAM包含有很多不相关的全局变量。 


 
正如文章\cite{baker2015}所描述的那样,CAM-ECT中使用的集合模拟结果是在可信的机器上使用CESM的被验证过的版本和配置运行得到的。 
这个集合模拟包含有151个长达一年的模拟结果,这些结果之间唯一的不同是在初始的大气温度场中加入了$\mathcal{O}(10^{-14})$ 量级的扰动。 
集合模拟的输出结果只包含了特定变量在每个点上的年平均值。这些变量的个数记为$N_{var}$,它们可以反映出整个大气场(默认情况下 $N_{var}=120$)。这个CAM-ECT工具集是对集合模拟结果中的每一个变量的全球的面积加权平均用主成分分析(PCA)的方法,生成了一个反应集合模拟特征的统计学分布。
这个主成分(PC)分数的分布被保存下来,之后用来与新的模拟实验作对比。 
基于落在指定置信区间(通常取95$\%$)之外的PC分数的个数,可以确定少数的几个新的实例(通常是3个)是否通过测试。 
这里可以调优那几个关于是否能通过测试的参数,从而达到想要的误报率。
\section{试验方法和结果分析}
\label{sec:verifyExp}
\begin{figure}[h]
\includegraphics[height=6.5cm]{figures/temp_rmse_GMD.png}
\caption {Monthly Root Mean Square Error (RMSE) of temperature for experiments with different barotropic solver convergence tolerances. Note that this is a subset of Fig. 12 in \cite{yong2015}.}
\label{fig:rmse_temp}
\end{figure} 


 \begin{figure}[h]
 \begin{center}
 \includegraphics[height=6.5cm]{figures/temp_rmsz_GMD.png}
 \end{center}
 \caption {Monthly Root Mean Square Z-score (RMSZ) of temperature with respect to an ensemble (denoted by yellow) for experiments with barotropic different convergence tolerances. Note that this is subset of Fig. 13 in \cite{yong2015}. }
 \label{fig:rmsz_temp_ens}
 \end{figure} 

\begin {figure}[h]
\centering
\includegraphics[height =6.5cm]{figures/pert-force.png}
\caption{The top panel displays four analytic solutions to Eq. (\ref{e:2}) with the indicated perturbations to the initial conditions (P)和perturbation to the forcing term (F).  Note that all four perturbations have similar effect on the ``original'' (unperturbed) solution.  The bottom panel plots the error between the four perturbations和the original solution. In this example, the error due to perturbing the forcing term (F) is constant和smaller in magnitude that the errors caused by the initial condition perturbations.}
\label {fig:1Danalytical}
\end {figure}

\begin {figure}[h]
\centering
\includegraphics[height =20cm]{figures/gimp_all_sst_std.png}
\caption{The ensemble distribution for the standard deviation of sea surface temperature (SST) at months 1, 12, 24,和36.}
\label{fig:SST_STD_all}
\end {figure}

\begin {figure}[h]
\centering
\includegraphics[height =12cm]{figures/RSM-TEMP-tol.png}
\caption {Response surfaces for Z-score of temperature (TEMP) over time (monthly)和 Z-score tolerance.  Each subplot represents a different barotropic
solver convergence tolerance (labeled above). The color bar indicates the percentage of grid points with a Z-score below the Z-score tolerance (on the y-axis).}
\label{fig:RSM-TEMP-tol}
\end {figure}

\begin {figure}[h]
\centering
\includegraphics[height =12cm]{figures/RSM-SSH-tol.png}
\caption {Response surface for Z-score of sea surface height (SSH) over time (monthly)和Z-score tolerance.  Each subplot represents a different barotropic
solver convergence tolerance (labeled above). The color bar indicates the percentage of grid points with a Z-score below the Z-score tolerance (on the y-axis).}
\label{fig:RSM-SSH-tol}
\end {figure}


\begin{figure}[h]
\centering 
\includegraphics[height=7cm]{figures/prz_tol_combined.png}
\caption {Percentage of grid points with Z-scores for temperature
  (TEMP)和sea surface height (SSH) that exceed the 3.0 tolerance for simulations with various barotropic solver convergence tolerances. }
\label {fig:PRZ-tol}
\end{figure}

\begin{figure}[h]
\includegraphics[height=7.0cm]{figures/temp_cores_zoom_combine.png}
\caption{Percentage of grid points with Z-scores for temperature (TEMP) that exceed the 3.0 tolerance for simulations with various numbers of processor cores. (Note that the left和right subplots contain the same information, with different scales for the y-axis.)}
\label {fig:combine}
\end{figure}

\begin {figure}[h]
\centering
\includegraphics[height =12cm]{figures/RSM-TEMP-param.png}
\caption {Response surfaces for Z-score of temperature (TEMP) over time (monthly)和 Z-score threshold.  The top two subplots represent two different processor core layouts.
The bottom left has a tracer mixing coefficient for convective instability that is 10 times larger than the original,和the bottom right uses a different tracer advection scheme. The color bar indicates the percentage of grid points with a Z-score below the Z-score tolerance (on the y-axis).}
\label{fig:RSM-TEMP-param}
\end {figure}

\begin {figure}[h]
\centering
\includegraphics[height =12cm]{figures/RSM-SSH-param.png}
\caption {Response surfaces for Z-score of sea surface height (SSH) over time (monthly)和 Z-score threshold.  The top two subplots represent two different processor core layouts.
The bottom left has a tracer mixing coefficient for convective instability 10 times larger than the original,和the bottom right uses a different tracer advection scheme. The color bar indicates the percentage of grid points with a Z-score below the Z-score tolerance (on the y-axis).}
\label{fig:RSM-SSH-param}
\end {figure}


\begin{figure}[h]
\centering
\includegraphics[height=7cm]{figures/prz_param_combined.png}
\caption{Percentage of grid points with Z-scores for temperature
  (TEMP)和sea surface height (SSH) that exceed the 3.0 threshold for simulations with an alternative tracer advection scheme (lw\_lim)和several tracer mixing coefficients for convective instability.}
\label {fig:PRZ-temp-param}
\end{figure}


  
\begin {figure}[h]
\centering
\includegraphics[height =20cm]{figures/zscore_sst_combine_crop.png}
\caption{Z-score of sea surface temperature (SST) at month 12 for the
original (default) case, a 384 processor core case, a case with a
  larger tracer mixing coefficient for convective instability,和a case with an alternate tracer advection scheme.}
\label {fig:zscore-combine}
\end {figure}

\begin {figure}[h]
\centering
\includegraphics[height =6.5cm]{figures/ens_size_combined.png}
\caption{The distributions of experimental failure rates based on 1000 tests for variables temperature (TEMP)和sea surface height (SSH).  For each ensemble size, the green bars indicate the maximum和minimum values obtained,和the red boxes indicate the mean.   }
 \label {fig:temp_ens_80}
\end {figure}
\subsection{研究动机}

 
文章 \cite{yong2015} 中给出来的求解器验证性工作很有启发性,而且完全满足目前的需求。 
但是它仍然有几个遗留问题没有解决,这促使我们研发了一个类似于CAM-ECT这样的更强大的针对CESM-POP数据的评测工具。 
第一,一个更通用的海洋模式POP的验证性工具首先考虑的是海洋中空间的变化特点,这个特点比大气中的要更加突出。
CAM-ECT和文章\cite{yong2015} 中的RMSZ方法是利用空间的平均值的差别来作评估,目前还不清楚这样的策略是否有足够的能力来检测海洋中的正确性表达问题以及模式本身的误差。
第二,由于CESM-POP的相互独立的诊断性变量要比大气模式CAM少很多,因此我们需要重新思考了一下一致性检测的量化指标。
第三,不同的衡量指标的选择促使我们进一步检查集合模拟的大小应该如何选择。
在文章\cite{yong2015} 中,四十个实例组成的集合模拟就足以检查出线性求解器中的误差,而这个集合模拟的大小并没有在更通用的环境中进行实验验证。
而且,这个大小要比CAM-ECT中使用的默认大小151要小很多。
最后,海洋和大气中时间尺度的不同也促使我们去研究集合模拟的模拟时间究竟多长最为合适。 
由于海洋中的时间尺度比大气中的要长一些, 直观上讲,CESM-POP中额统计学一致性检测工具需要的集合模拟的模拟时间应该会更长一些。 
值得一提的是,图\ref{fig:rmsz_temp_ens} 似乎又有悖于我们的这个假设。 
我们可以看出来,当收敛条件足够严格的时候(收敛条件小于$10^{-12}$时), 均方根标准分在模式运行几个月之后随着时间的推移而逐渐减小。
这表明,我需要进一步的研究一致性检测所需要的模拟时间的长度。 
实际上, 图\ref{fig:rmsz_temp_ens}可以看出,这个给定的1度分辨率的海洋数据是相对确定的,在求解器中引入的误差足够小(也就是收敛条件足够小的时候)初值场中的误差在模拟了一年之后逐渐的耗散掉了。


%----------------------------------------------------------------------------
%----------------------------------------------------------------------------
%----------------------------------------------------------------------------


\section{CESM-POP的新的一致性检测工具}\label{sec:newtest}

 
在CESM-ECT工具集中扩充CESM-POP一致性检测的工具(我们记为POP-ECT),我们需要一个能够反映海洋模式自身不确定性的恰当的集合模拟结果,并且开发一种针对海洋自身特点的方法。
这里我们首先讨论一下如何构造这个集合模拟结果,然后介绍一下新的测试工具的流程。 
\subsection{POP数据集合的构造}\label{sec:ensemble}

 
我们通过与很多被认可的海洋状态组成的集合作对比,来评估CESM-POP的输出数据与数据集合在统计学上的差别是否明显。
明显则我们认为这个输出数据与给定的数据集合是不同的气候状态。 
因此,将CESM-ECT应用到CESM-POP数据的第一步就是要构造出一个合适的数据集合。 
很明显,数据集合的组成对于测试的有效性来说至关重要,而且这些数据必须在一个被认可的机器上采用被认可的CESM版本来生成。
这里,与CAM相比,海洋模式的诊断变量要少很多,相互独立的诊断变量有:温度(TEMP),海表高度(SSH),盐度(SALT), 以及与模式网格相对应的经向和纬想得速度(分别为UVEL和VVEL)。
这五个变量中,只要海表高度SSH是二维变量,其他都是三维变量。 

 
在POP-ECT中,我们够早了一个由$N_{ens}$模拟结果组成的数据集合 $E$, 记为 $E =\{E_1, E_2, \dots, E_{N_{ens}} \}$。
这些模拟结果与指定的默认版本之间的唯一不同是在初始的海洋温度场中加入$\mathcal{O}(10^{-14})$量级的随机扰动。 
这个初值扰动的大小和双精度浮点数的舍入误差在一个量级,因此它应该不会造成气候上的改变。
这个数据集合的成员的数量必须足够的多才能够生成一个比较有表达力的分布,但是这个数量又要足够的小,因为增加一个成员就以为增加一份计算开销。 
我们将在稍后的第\ref{sec:ens}节中讨论数据集合的大小的选择 。这里的讨论和之后的实验我们都采用 $N_{ens} \;=\; 40$这个配置。
我们的实验表明,如果只是用来做一致性测试,这个大小的集合 足以充分的反应海洋的内在的不确定性。
模拟数据集合包含有在每个点上五个POP诊断变量(TEMP, SSH, SALT, UVEL和VVEL)的月平均数据。
每一个这样的变量$X$都包含有$N_X$个网格点,记为${X} = \{ x_1, x_2, \dots, x_{N_X}\}$, 这里$x_i$ 表示在网格点$i$上变量$X$的月平均值。
我们收集了$T$个月份的数据(即$T$个时间片的月平均数据)。

 
CESM-ECT方法的下一步是量化数据集合的分布特征,进而为行的模拟结果提供评估依据。
这个数据集合的统计特诊描述被存在一个叫做数据集合汇总的文件中,并且这个文件会被贴上生成这些模拟结果的CESM自带的软件标签。
当这个汇总文件一旦生成, 数据集合中$N_{ens}$个模拟结果的历史文件就不需要继续保存了。 
回忆一下CAM数据所使用的方法,CAM-ECT会计算每个变量在每个点上的可用的年平均数据的全球面积加权平均值。
这些计算结果就构成了每个变量的$N_{ens}$个全球平均。 
然而, 这个方法采用的全面积加权平均不能很好的应用在海洋模式数据中,因为海洋模式的不确定性相比于大气模式数据来说,在网格点上的一致性更差一些。
因此,我们需要更多的考虑到空间上的分布特诊。 


比如,我们考虑一个由$N_{ens} \;=\; 40$个使用1度分辨的海洋模式POP作为海洋模式分量的CESM模拟$T\;=\;36$个月得到的模拟结果组成的数据集合。 
图 \ref{fig:SST_STD_all}中给出的是数据集合的1,12,24和36个模拟月份中海表温度(SST)的标准方差的空间分布。
注意到这里海表温度SST仅仅只是三维变量TEMP的最上面一层(更加准确的说,是10米的海洋表层)。
图 \ref{fig:SST_STD_all}可以看出来,标准方差远远不是空间一致分布的,不同的网格点之间的值相差好几个数量级(注意颜色刻度尺的刻度)。 
从1月份到12月份的变化也十分明显,这主要是与模式冷启动时由于流体动力学不稳定性造成的模式不稳定相关。  
在模拟一年之后,在热带区域出现的标准差最大表明集合模拟中热带不稳定波的增长带来了更大的不确定性\citep{legeckis1977}。
由于赤道区域的网格分辨率更细一些($1/3^\circ$左右),这些大的可变性在 赤道地区很容易被强化。 
在几个主要的海洋环流系统区域内,也有几个比较大的标准方差集中区域。
从12个月到36的变化相对较小,这表明相应的物理不稳定性由于海洋的耗散将不会继续增长。
鉴于图 \ref{fig:SST_STD_all}中明显的不确定性,文章\cite{yong2015} 中的均方根标准分策略可能在某些不确定性很小的区域由于方程\ref{e:rmsz}中的分母太小引入不必要的潜在误差。
这里强调一下,图\ref{fig:SST_STD_all} 中不确定性的大小确实是与分辨率有关的,并且,图\ref{fig:SST_STD_all} 中的结果只适应于有耗散的低分辨率海洋模式,比如说目前大部分气候研究中所使用的1度的CESM-POP。


 
综上所述,为了构造出一个在CESM-POP数据上比较健壮的集合统计学一致性检测工具,我们必须得到一个既包含有空间信息,又包含有时间信息的数据集合。
特别的,POP-ECT构造的集合综述文件中第$T$ 个 月份的时间片CESM-POP数据包含有:
\begin{itemize}
 \item $N_{var} \times N_X \times T$  monthly mean values across the ensemble at each grid point $i$ ($\mu_i$)
 \item $N_{var} \times N_X \times T$ standard deviations of ensemble monthly mean values at each grid point $i$ ($\sigma_i$),
 \end{itemize}
 这也就是说,我们保存了指定个月份$T \;=\; 36$ 在所有$N_x$ 个点上的集合平均值和标准方差值(这里 $N_x$ 是变量是二维还是三维的变量有关系)。
 

 最后值得一提的是,由于在淡水和盐度之间没有合适的淡水反馈机制,在没有任何特殊改动的情况,CESM-POP在封闭的内海区域(比如哈德逊湾和地中海)可能会产生一些与实际不符的盐度分布。
当前版本的CESM-POP在非耦合的模拟中强加了一个很强的淡水恢复机制,而在耦合模拟中加入的是边缘海淡水平衡机制。
这些特殊的处理能够保持盐度的平衡,但是对于模式的动力过程造成了虚假的强迫。
因此,本文的工作中我们不去讨论如何在边缘海中正确的做验证,而是将重心限制在开放海域内。

\subsection{验证过程}
 
利用前面所得到的POP-ECT综述文件,我们利用以下的方法来判断一个新的模拟输出结果(比如说改动了代码,或者移植到了新的机器上,或者采用了一个新的编译器选项等)是否和指定的数据集合的分布所描述的海洋气候状态在统计学上是一致。 
由于我们只有五个比较基础的诊断变量,因此我们不需要采用主成分分析的方法。 
这里,我们采用的策略是在每一个给定点上评估新的模拟结果和数据集合之间的标准差。
对于每一个给定点$i$和某一个给定的变量$\tilde{{X}}$, 在给定的时间片$t$上我们采用标准分方法计算$\tilde{{X}}$和数据集合的距离。 
特别的,给定变量$\tilde{{X}}$ 在 $t$时刻的值,$\tilde{{X}} = \{ \tilde{x}_{1,t}, \tilde{x}_{2,t}, \dots, \tilde{x}_{N_X,t}\}$, 那么在给定点$i$ 处变量$\tilde{{X}}$ 在$t$时间的标准分为
\begin{equation*}
Z_{\tilde{x}_{i,t}}=  \frac{\tilde{x}_{i,t} -\mu_{i,t}}{\sigma_{i,t}},
\end{equation*}
这里 $\mu_{i,t}$和$\sigma_{i,t}$分别表示指定的数据集合在点 $i$处,变量$X$在给定的$t$月份的几何平均和标准方差。  

 
下面,为了简洁,对于某一个时间片$t$,我们从相关的变量中去掉时间变量下表。比如说,标准分我们记为$Z_{\tilde{x_i}}$。 
我们假设在每一个点上允许的阈值为$tol_{Z}$,也就是说如果 $Z_{\tilde{x_i}} > tol_{Z}$就表明$i$这个点是一个“不通过”的点。 
由于标准分表示的是一个数与平均值之间的标准方差,所以标准分越大就表明这个新的模拟结果和数据集合所给的气候状态就越远。 
下一步,我们将考察一下所有点中通过标准分标准的点所占的比例。 定义给定变量$\tilde{X}$的标准分通过比例(ZPR)为:
\begin{equation}\label{e:zpr}
ZPR_{\tilde{X}} = \frac{ \#\{i \;|\; \tilde{x_i} \in \tilde{X} \; \land \; |Z_{\tilde{x_i}}| \; \leq \; tol_{Z}\} }{\#\{i \;|\; \tilde{x_i} \in \tilde{X} \} }.
\end{equation}
 
为了给出一个评判给定变量$\tilde{X}$ 是否通过测试的整体性标准,我们对标准分通过比例设置了一个最小阈值($min_{ZPR}$)。 
也就是说,如果$ZPR_{\tilde{X}} \geq min_{ZPR}$成立,则我们认为 $\tilde{X}$通过测试。
默认情况下,标准分的通过阈值为$tol_{Z} \; = \; 3.0$, 而标准分通过比例ZPR的阈值为$min_{ZPR} \; = \; 0.9$。 
换句话说,新的模拟结果中变量$\tilde{X}$要想通过测试,必须保证这个变量在90$\%$以上的点上的值落在在对应点对应变量的集合平均值($\mu$) 的$3.0$ 个标准差之间。
我们对五个独立的诊断变量依次做上面的判断。 
只有当所有的变量都通过测试,我们才认为这个新的模拟结果整体上来说和数据集合在统计意义上是一致的。


 
我们注意到计算得到的标准分随着模拟时间的改变而改变。 
由于海洋运动的时间尺度比较长,本章中大部分的实验我们都将CESM运行36个月。
另外,模拟过程中我们会保存月输出的时间片数据(CAM-ECT中只保存了年平均数据),用来考察数据集合中的海洋状态能否随着时间的延长而稳定下来。 



在后面的章节中我们可以看到,ZPRs通常在几个月的模拟之后便会平稳下来,而且这种平稳的趋势在几个诊断变量中都是相似的。 
因此,除了挑选出合适的标准分阈值以及ZPRs的通过率阈值,我们还选择了一个时间点($t_C$)
来判断新的运行结果(而不是去检查每个月的数据)。 
于是,集合模拟的时间长度不需要比$t_C$更长。 

\subsection{软件工具}

为了使得用户和开发展都能够使用我们的新的针对POP设计的诊断方法,我们将POP-ECT添加到现成的CESM-ECT Python工具集中(pyCECT v2.0)。
这个工具集已经被集成到当前CESM发布的版本中。 
这个CESM-ECT Python工具集包括了生成CESM特定模块的集合模拟综述文件的工具,以及利用这些集合模拟综述文件来进行统计学一致性检测的工具pyCECT。
由于POP-ECT的综述文件和CAM-ECT的综述文件完全不同,我们利用并行的Python代码pyEnsSumPop来生成POP-ECT综述文件。
具体来讲,从CESM-POP集合模拟的输出文件中,pyEnsSumPop并行的生成一个包含有第\ref{sec:ensemble}节中描述的海洋模式一些统计量的集合综述文件。 
CESM软件工程组将会按照需要生成一个新的POP模拟数据的集合,这个数据集合包含有一个注明了具体改动的软件标签。
POP-ECT所生成的集合综述文件和第\ref{sec:code}节提到的CESM的版本标签将被放到未来发展的模式中。 
有了POP-ECT的综述文件,用户或者开发者就可以利用pyCECT Python工具来评估新的模拟数据的一致性。目前可以选择POP-ECT或者CAM-ECT来对结果进行评估。
新的CESM-POP模拟数据可能来自一个新的机器的移植, 一个新的编译器选项的使用,代码的更新,或者输出数据的改变。
pyCECT评估新的海洋模式模拟结果是否与给定的POP-ECT生成的数据集合是统计意义上一致的,并且从总体上给出一个“通过”或者“不通过”的判断。
另外,在指定的检查时间点$t_C$,每一个变量的标准分通过率都会被给出。 

% %-----------------------------------------------------------------------------
% % -----------------------------------------------------------------------------
% %----------------------------------------------------------------------------
\section{实验} \label{sec:exp}

 
这一节的主要目标是通过一 系列的实验来评估我们的新的POP一致性检测工具在CESM-POP的模拟数据上的效果。
我们的这些实验都是有预判结果的,比如说重新考察一下改变正压求解器的收敛条件所带来的效果。
这些实验都是用的CESM 1.2.2版本来运行,使用CESM-POP作为活跃的海洋模式分量,CICE作为活跃的海冰分量,而大气和路面模式都采用数据来驱动。
因此,不会出现逐年响应的事件(比如厄尔尼诺南方涛动),而且赤道太平洋地区的变化可能会受到人为的抑制。
我们这个特殊的分量模式的配置在CESM的官方文档中被记为“G\_NORMAL\_YEA”分量设置。
CESM的网格分辨率采用“T62\_g16”,对应的海洋和海冰模式分量采用是1度($320 \times 384$)分辨率,垂直方向上有60层,以格林兰岛为偏移极点的偏移网格。 
这些模拟都是在美国国家大气研究中心的黄石超级计算机上使用$96$ 个处理器核心(除非特殊说明了处理器核心的数目)运行得到的。 

 对于这些模拟实验,我们不单单是对某一个时间片做了评估,而是对36个月的数据都进行了评估,以此来观察ZPRs随着时间的变化规律,同时也为$t_C$的选择提供参考依据。
为了从ZPRs的角度来刻画标准分和模拟时间的关系,并且为$tol_{Z}$和$min_{ZPR}$的选择提供参考依据,我们采用了反应曲面法(RSM)\citep{box2007}。 
也就是说,我们提供了变量$\tilde{X}$的反应曲面的图标,这个图表中给出了不同的模拟月份中,满足一系列的标准分阈值条件$Z_{\tilde{x_i}} > tol_{Z}$的点的比例的关于给定阈值$tol_{Z}$的累积分布函数。
最后,如前面提到的那样,我们发现,40个模拟结果组成集合模拟对于我们的实验来说已经足够。
但是我们还是在第 \ref{sec:ens}节进一步的探讨了集合成员的个数的选择。  

为了简洁,尽管我们也分析了其他变量,我们只展示了温度(TEMP)和海表高度(SSH)的结果。 
这两个变量在海洋模式系统有很强的代表性。海表高度和海洋海流动力系统密切相关,而温度是由模型的标量传输决定的。
 

\subsection{正压求解器收敛性条件}

 
首先,我们先用我们新的加强版本的CESM-ECT工具来重新评估一下改变正压求解器收敛性条件所带来的影响。
CESM-POP中正压求解器默认的收敛条件是$10^{-13}$, 我们运行一些采用 $10^{-9}$ 到 $10^{-16}$之间值作为收敛条件的模拟,然后输出所有点上的36个月平均数据。
我们预测采用比默认的收敛条件$10^{-13}$更加严格的条件得到的模拟结果应该是与原结果在气候学上是一致的,但是更送一些的收敛条件将会引入一些误差。


 
图 \ref{fig:RSM-TEMP-tol}和Fig. \ref{fig:RSM-SSH-tol}分别给出了温度和海表高度的反应曲面。 
每张图包含有四个反应曲面:上面两个分别是采用原始默认收敛条件($10^{-13}$)和更严格的收敛条件得到的结果,下面两个是更松的收敛条件($10^{-10}$和$5.0*10^{-9}$)得到的结果。 
每一个反应曲面中,X轴表示模拟的月份(从1月到36个月),Y轴表示衡量标准分的分布的阈值$tol_{Z}$。这里阈值就是计算方程中ZPR所用的划定标准分满足某个条件的比例时所用到的阈值。
颜色刻度中ZPR采用的是以10\%为间隔的百分比。 
反应曲面对于评估$tol_{Z}$和$min_{ZPR}$的不同组合非常有效。
比如说,考虑变化求解器收敛条件对温度变量的影响。 图中左上角的 \ref{fig:RSM-TEMP-tol}子图显示,原始的收敛条件($10^{-13}$)下,在所有模拟月份中90\% 的点的标准分都小于2.0。
与之相对的,下面的采用为$10^{-10}$收敛条件的子图表明,在第九个模拟月份之后,90\% 的点的标准分要小于3.0, 而在12个分,只有百分之七十到八十左右的网格点的标准分要小于2.0. 
右下角的子图采用的收敛条件进一步放松 到$5.0*10^{-9}$,它的比较低的ZPR比例表明这个例子中引入较大的误差。
我们继续考察图\ref{fig:RSM-SSH-tol} 中关于海表高度的反应曲面,四个子图分别表示与图\ref{fig:RSM-TEMP-tol}中相同的四个收敛条件,可以看到总体趋势与温度的反应曲面很相似。 
对于采用$10^{-13}$为收敛条件的模拟结果中,在除了第六个月的所有模拟月份中, 90\%的点上标准分都要小于2.0。 
与温度变量的反应曲面类似, 收敛条件为$10^{-10}$ 对应的子图可以出来有明显的误差,这个误差在收敛条件为$5.0*10^{-9}$对应的例子中更加明显。 
温度和海表高度的反应曲面中有一个明显的不同就是,温度得到的曲线随着时间的变化更加光滑。 这主要是由于温度的计算中有一个很很要的扩散过程。

 

如果我们在图\ref{fig:RSM-TEMP-tol}和图\ref{fig:RSM-SSH-tol}中固定标准分的衡量阈值,我们就可以很容易的评估ZPR。 
假设我们采用一个比较保守的选择$tol_{Z} \; = \; 3.0$。 
图 \ref{fig:PRZ-tol}给出了温度和海表高度两个变量中,标准分超出阈值 $tol_{Z} \; = \; 3.0$ 的点的百分比。 
如果我们选择ZPR的阈值为$min_{ZPR} \; = \; 0.9$, 也就对应着图\ref{fig:PRZ-tol}中10\%的不通过率,我们可以很明显的看出收敛条件为$10^{-10}$的例子就处于我们定义通过与不通过的边界线上(因此,这个收敛条件值在实际模拟中不能使用)。 
而采用比$10^{-10}$更严格的收敛条件得到的结果的不通过率更低一些,因此,看起来它们的两个变量跟原始结果所得到的变量在统计学上时一致的。
图\ref{fig:PRZ-tol}中的图标很明晰的证明了,随着收敛条件的放松,超出给定标准分阈值的点会增多。 
这个结果比文章\cite{yong2015}中的结果更加清楚。 

\section{本章小结}
\label{sec:verifyConclusion}
