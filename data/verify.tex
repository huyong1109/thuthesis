\chapter{海洋模式正确性验证}
\label{cha:verify}

\section{本章概述}
并行海洋模式 (POP)是公共地球系统模式(CESM)的海洋模式分量,它被广泛的应用于气候研究。 
目前有很多关注如何提高  CESM-POP性能和正确的工作,比如提高数值算法的性能,改进参数化方案,移植到新的架构上,以及增加并行度等。 
由于海洋模式动力过程混沌的特性,即使对代码做一些细微的改动,也无法保证修改后的模式得到的结果与原始结果是二进制一致的。 
并且,判断模式中的改动是不是可以被接受的(也就说是不是在统计的意义上与原始结果一致)并非易事。 
最近的一个工作显示,对大气模式的数据利用基于集合模拟的统计方法取得了很好的效果。
这个基于集合模拟的统计学一致性检测工具的核心是利用集合模拟结果的可变性的量化结果来作为评判未来模拟的标准,并且利用统计学上的区分度来做判断。 
由于海洋模式和大气模式在动力学和时间尺度上有着明显的区别, 我们这里采用了一种需要结合集合模拟的平均状态和方差的统计学方法来对海洋模式的模拟结果进行评测。 
具体来说,我们利用CESM-POP集合模拟的结果的统计学分布来计算新的模式结果在每一个点上的标准分。 
那些标准分大于某一个给定值在所有点中所占的比例就可以反映出新的模式结果与集合模拟结果的统计学差别。 
这里,集合模拟的大小和成员的挑选都很重要。
我们的实验表明,新的海洋模式POP的集合一致性检测工具(POP-ECT)能够区分那些统计学上明显与集合模拟的结果一致的模拟结果和那些明显不一致的模拟结果。 
同时,这个检测工具提供了一个简单、客观和系统的方法,来检测CESM-POP是否有硬件或者软件层引入的错误。
这个工具进一步提升了CESM-POP代码的质量保证。 

公共地球系统模式CESM是一个应用广泛的全耦合气候模拟程序\citep{cesm2013}, 它的模拟结果经常被提交到政府间气候变化委员会的评估报告中\cite{stocker2013ipcc}。
CESM由多个分量模式耦合而成,其中包括大气、海洋、海冰、和陆地等的分量模式。 
这里,我们专注于CESM的海洋模式分量并行海洋模式POP。 POP是由Los Alamos国家实验室研发的大洋环流模式的一个扩展版本。 
CESM-POP求解的是一个采用静力平衡近似和Boussinesq近似的海洋动力学三维原始方程,它描述的海洋模式过程在时间和空间上的尺度都很大。  

 
 
最近CESM-POP中的很多进展都是关心如何减少计算开销\citep{yong2015}, 但是数值模拟程序的中的任何开发都需要软件质量保证,来确保不会在程序中引入错误。 
这种维护科学程序的可信性的质量保证,对于气候模式来说尤为重要。 
因为气候模式的模拟结果会影响到关系国计民生的政策方针\citep{carson2002, easterbrook2011}。 

  
气候模式,比如CESM,通常都代码量大并且程序复杂。 模式中过多的配置选项使得很难穷尽的测试它们 \citep{clune2011, pipitone2012}。 
另外,由于气候模式本身的混沌性,判断一个新的模拟结果中的差别是由某个错误引起的还是由于模式本身的不确定性引起是一件非常具有挑战的事情。 
我们注意到,在模式的初始条件或者中间结果中引入浮点数精度级别的随机扰动都会使得最后只能够的结果产生比较大的偏差。
CESM-POP中的一些新的成果,尤其是那些提高性能的成果(比如利用新的异构加速技术和改进数值方法等),通常都会导致新的模式的输出结果与原来的结果不能保证是二进制一致的。 
对于CESM-POP来说,即使是在同一个超级计算机上采用不同的处理器核数来计算,得到的结果都不会是二进制一致的。 
如果可以直接评估CESM-POP的海洋数据的气候一致性,将会极大的促进模式的的发展,同时使得模式能够更加灵活的应用新的软件或者硬件技术。 


最近由Allison等人\cite{baker2015}研发的CESM集合模拟一致性检测工具(CESM-ECT),通过一个新的基于集合模拟的工具解决了比较气候模式输出结果的难题。 
这个工具能够评判一个新的气候模拟结果(比如引入了软件或者硬件的改动)是否与以原始结果(没有任何改动的版本)为基础构造的被认为是可以接受的集合模拟的结果在统计学意义上是否是一致的。 
但是文章\cite{baker2015}中的这个CESM-ECT工具只评估了CESM中大气分量模式--公共大气模式(CAM)的变量,并且其实验并不是完全耦合的CESM配置(比如说其中海洋分量并不是采用的并行海洋模式POP,而是海洋气候数据模式。这个模式只能够提供海表温度数据,但是不能相应气候分量模式提供的强迫)。 
为了简洁,我们将为CESM的通用的集合模拟统计一致性测试工具称之为CESM-ECT。 
我们将应用于公共大气模式分量的检测方法称之为CAM-ECT,它是我们研发的CESM-ECT工具集中的一个模块。 
值得注意的是,直接将CAM-ECT方法应用于海洋数据并不可行,因为海洋和大气在动力学、空间尺度和时间尺度上的差别都很大。
比如, 海洋模式中气候尺度比大气中的气候尺度要小一到两个量级,海洋,尤其是深海中运动的时间尺度比大气中的要慢很多个量级。
因此, 我们专门为海洋模式中的统计一致性检测开发了一个新的方法,我们称之为POP-ECT。 
尽管这个新的POP-ECT工具也是利用CESM的集合模拟的结果来评估模式的不确定性,它和CAM-ECT是由本质区别的。 
POP-ECT的统计过程中考虑到了海洋中不确定性的空间分布特点。另外,由于海洋达到全球准稳态所需要的时间比大气需要的时间长得多,我们在POP-ECT中采用与CAM-ECT中不一样诊断时间。 
同时,海洋模式中的可用的诊断变量的个数比较少,也使得我们能够采用跟大气模式不一样的方案。 
最后,值得一提的是,本章中所有的实验结果都是采用CESM 的1.2.2版本。
\section{背景和动机}
\label{sec:verifyBackgroud}

\section{海洋模式正确性验证的特点}
\label{sec:verifyPart1}
\subsection{传统正确性方法}
\label{sec:verifyPart11}
目前海洋模式POP中可用的正确性测试工具(我们称之为POP-RMSE),是一个比较简单的评估CESM-POP程序是否成功的移植到了一个新的机器上。 
这个工具的主要目标是发现与新机器上硬件或软件层的问题。 
这个测试中,首先将特定设置的实验在新的机器上运行五个模拟天,然后将其输出结果和美国国家大气研究中所提供 的标准的数据集相比较。
这里计算了两个结果的海表高度场的均方根误差(RMSE)。 比如对于两个包含$n$个值的数据集合$X_0$ 和$X_1$,它们的均方根误差可以表示为
 \begin{equation*}
 RMSE(X_0, X_1) = \sqrt{\frac{1}{n}\sum_{i=1}^n(X_1(i)-X_0(i))^2}.
 \end{equation*}
然后,拿得到的均方根误差的增长曲线与给定的两个样例所得到的曲线相比较。 
这两个样例中,一个是原始的配置,另一个是在求解器中采用比默认收敛条件要高一个量级的收敛条件。 

POP-RMSE中这个简单的方法对于评估CESM在新的机器上的输出结果是很方便的, 但是它比为大气模拟数据设计的CAM-ECT的表达能力要弱得多。 
比如说,POP-RMSE无法评判CESM-POP中最新的线性求解中的改动所带的微小的气候状态的改变 \cite{yong2015}。 
在文章\cite{yong2015}中,作者将默认的预处理共轭梯度法求解器替换成预处理的Chebyshev迭代算法, 以此来提高CESM-POP的计算性能。
尽管新的P-CSI求解器在高分辨率模式中的通信开销比原始的预处理共轭梯度法的要小很多, 证明这个新的求解器不会对海洋的模拟结果造成负面的影响对于这个求解能否被接受是及其重要的。 
因此,在 \cite{yong2015}中, 为了验证POP-RMSE是否能够检测出求解器在不同时间所引入的误差, 我们搜集了1度分辨率的CESM-POP在不同的收敛条件($10^{-10}$ 到$10^{-16}$)下36个月的运行结果。  图   \ref{fig:rmse_temp} 给出的是最紧的收敛条件的实例  ($10^{-16}$) 和其他收敛条件的实例的结果的温度场之间的均方根误差。 
尽管这些例子采用不同的收敛条件,图\ref{fig:rmse_temp} 中很难体现出来这些由于求解器误差所带来的影响  \citep{yong2015}。 

 
由于海洋动力过程本身具有混沌性,即使在正压求解中引入一个浮点数舍入误差精度级别的扰动,也可能使得最终的模式结果与原始结果截然不同。 
因此,既然我们不能保证引入新的求解器之后所得到的结果与原始结果是二进制一致的,那么我们在将新的P-CSI求解器和EVP预处理正式的加入到海洋模式POP的版本中之前,需要证明加入它们之后不会造成模拟结果的不正确性,更不能改变所得到的气候的状态。 

当我们将海洋模式POP移植到新的机器上时,也会遇到类似的问题,即不同的机器上得到的模拟结果是不能做到二进制一致的。 
目前海洋模式POP中现成验证新的机器是否移植成功的标准是将特定的样例在新的机器上模拟五天,然后计算一下所得到的结果中的海表高度场与美国国家大气研究中心所提供的标准数据集中的海表高度场之间的均方根误差(RMSE)。
这个计算过程提供了一个比较简单的评估地球系统模式CESM在新的机器上的运行结果(这些结果可能是有软件环境或者硬件环境引入的错误的)的方法, 但是我们发现它并不能检测和评估求解器引入的误差。 
比如, 我们跑一些1度的例子,这些例子中正压求解器的收敛条件分别为$10^{-10}$ 到$10^{-16}$之间的值(默认为 $10^{-13}$),然后计算这些例子与使用最严的收敛条件($10^{-16}$)的例子的模拟结果的均方根误差。 
图 \ref{fig:ssh_rmse_t} 中给出了采用不同的收敛条件得到的模拟结果中的温度场的每个月的均方根误差。 
值得一提的是,为了凸显出线性求解器的影响,我们只考虑公海区域的值(POP在几个边缘海内的模拟并不算好)。 
很明显的可以看出,改变求解器的收敛条件所引入的误差并没有在最后的温度场中得到体现(也没有在速度和海表高度等诊断场中的得到明显的体现)。 
收敛条件越松,求解器中引入的误差就会越大。
因此我们希望看到使用最松的两个收敛条件的($10^{-10}$和$10^{-11}$)的实例的运行结果所计算出来的均方根误差比其他实例的要大一些。 
但是,我们的实验证明这并不成立。 
在第12个月到第20月之间,收敛条件为$10^{-10}$的例子的均方根误差在所有实例中是最小的。 


正是因为现成的简单的均方根误差测试并不能检测出来模拟结果的气候态是否被改变,因此我们提出了一种新的基于统计的评估海洋正压求解器的方案。 
这个方法的灵感来源于Allison的文章\cite{baker2014methodology},他们为了验证公共地球系统模式CESM的公共大气模式(CAM)中的数据被压缩之后是否仍然有效,并不是仅仅使用一次模拟的结果,而是利用集合模拟的结果 来作为评估无法做到二进制一致的改动的基准。 
我们的方案和这篇文章所提到的方案很类似\cite{baker2014methodology}, 我们先跑一系列的实例。这些实例除了在初始的海洋温度场中加入了$10^{-14}$量级的随机扰动之外,和默认的配置完全相同。 
这个量级的扰动应该不会造成模拟结果的气候状态变化。 
我们发现40个这个样的实例就足以表现出海洋模式自身的扰动性, 但是由于海洋模式中需要模拟的海洋运动的时间尺度更长, 我们这个集合模拟中的实例也模拟了比大气模式CAM的更长的时间。 
值得一提的是,我们最终决定选择三维的温度场作为评估对象(而不是像现有的评估方式那样选择二维的海表高度),因为我们发现温度场是找到不同例子之间差异的最好的诊断变量。 
 




 
我们用如下的流程来判断一个新的结果是否和指定的集合模拟结果相容。 
设集合模拟在某个时间$T$的关于变量$X$的输出结果集合为 $\mathcal{E}=\{X_1,X_2,...,X_m\}$, 这里
$m$ 表示集合模拟的大小。 
在某个给定点$j$, 我们就可以从集合模拟中得到关于变量$X$的一个集合
  $\{X_1(j),X_2(j),...,X_m(j)\}$.
随着集合模拟的规模的增加,这个数列就能够更加准确的反应出来给定点的状态的分布。 
我们假设在点$j$上这个数列的平均值和方案为别为$\mu (j) $和$\delta (j)$。 
然后假设新的实例的模拟结果为 $\tilde{X}$, 那么以下定义的这个均方根标准分就可以表示这个新的实例和集合模拟结果之间的平均误差:

$$ RMSZ(\tilde{X}, \mathcal{E}) =  \sqrt{\frac{1}{n}\sum_{j=1}^n(\frac{\tilde{X}(j) -\mu (j)}{\delta (j)})^2}$$

我们使用这个基于集合的均方根标准分的指标来重新评估以下不同的求解器收敛条件的影响。 
图\ref{fig:ssh_rmsz_t} 可以看出,与简单的均方根误差不同, 这个新的基于集合模拟的方法能够发现收敛条件越松,结果中的误差就越大。 
现在,使用最松的两个收敛条件的例子得到的均方根Z分数也是最大的,可以明显的与集合模拟的结果区分开来。 
最后,我们用这个基于集合模拟的判断指标来评估一下我们的新的P-CSI求解器。 我们发现使用P-CSI求解或者使用最紧的两个收敛条件所得到的结果与集合模拟的结果是相容的。 
因此,我们确信我们的新的求解器并没有在模式中引入气候模式意义上不可接受的误差。 


\begin{figure}[!t]
\begin{center}
\includegraphics[height=6.5cm]{temp_rmse.eps}
\end{center}
\caption[] {采用不同收敛条件的1度海洋模式POP模拟结果中的温度场的每个月的均方根误差(RMSE)。}
\label{fig:ssh_rmse_t}
\end{figure}
\begin{figure}[!t]
\begin{center}
\includegraphics[height=6.5cm]{temp_rmsz.eps}
\end{center}
\caption[] {采用不同收敛条件的1度海洋模式POP模拟结果中的温度场的每个月的均方根标准分(RMSZ)。黄色的区域表示集合模拟中40个实例结果的均方根标准分的分布。 }
\label{fig:ssh_rmsz_t}
\end{figure}


\section{基于集合模拟的统计方法}\label{se:ver}
由于RMSE方法并不能明显的检测出不同的收敛条件所带来的影响,文章 \inlinecite{yong2015} 中采用了一种集合模拟的方法来测试海洋模式POP的输出数据。 
这个工作是基于文章\inlinecite{baker2014}中提出的测试数据压缩的影响的方法(这也是CESM-ECT方法的前身)。
需要注意的是,\inlinecite{baker2014}中提到的集合模拟是通过对初始值进行扰动得到的。 
但是,改变海洋模式中正压模态线性求解器的收敛性判据或者像文章 \inlinecite{yong2015}中所提到的引进新的求解器,都可以近似为强迫场的扰动。 
我们很有必要先研究一下,仅仅通过初始场的扰动所得到的结果状态集是否包含了在每一个时间步上对强迫场进行扰动所生成的状态。
正如文章\inlinecite{caya1998}所做的那样,我们将海洋模式POP的时间迭代过程用下面这个只有几个部件的简单方程来表示:
\begin{equation}
\frac{\partial X(t)}{\partial t} = \alpha X(t) +G,
\label{e:1}
\end{equation}
这里 $G$ 表示一个固定的强迫项,  $\alpha$是一个常数,  $X(t)$是关于时间变量的函数, $X(0) = X_0$。 
方程 (\ref{e:1}) 对应的解析解为:
\begin{equation}
X(t) = X_0e^{\alpha t } -\frac{G}{\alpha}. 
\label{e:2}
\end{equation}
这个例子里面,我们假设$\alpha$ 是一个纯虚数,这样 $\alpha X(t)$ 就是一个震动波(这里我们不考虑减振机制)。 

 
首先,我们考虑一下初值中的扰动:
\begin{equation*}
X_0' =X_0+P,
\end{equation*}
这里 $P$ 是一个常量。方程 (\ref{e:1}) 在采用扰动后的初始值之后的解析解可以表示为 
\begin{equation}
X_P(t) = (X_0+P)e^{\alpha t } -\frac{G}{\alpha}, 
\label{e:Pan}
\end{equation}
也可以表示为
\begin{equation}
X_P(t)-X(t) = Pe^{\alpha t }.
\label{e:Perr}
\end{equation}
方程 (\ref{e:Perr}) 表明原始结果和然后处置条件后得到的结果的差是振荡的。
下面我们考虑当在强迫场中加入一个扰动时: 
\begin{equation*}
G' = G + F,
\end{equation*}
这里 $F$ 是一个常量。  方程(\ref{e:1}) 加入带扰动的强迫项后,其解析解可以表示为
\begin{equation}
X_F(t) = X_0e^{\alpha } -\frac{G+F}{\alpha},
\label{e:Fan}
\end{equation}
也可以等价于
\begin{equation}
X_F(t)-X(t) = \frac{F}{\alpha}. 
\label{e:Ferr}
\end{equation}
方程 (\ref{e:Ferr}) 表明强迫场中加入干扰项后对结果产生的固定的。
因此,我们可以用方程(\ref{e:Perr})的初始场中引入扰动所带来的误差的量级来衡量在方程(\ref{e:Ferr})的强迫场中引入扰动后对结果的影响。 
举个例子,在图  \ref{fig:1Danalytical}中我们描述了一个初始扰动和强迫扰动对方程(\ref{e:2})的最终结果的影响相似的例子。在这个例子中,强迫项的扰动所引起的误差比初始条件扰动所赢得误差还小。
这张图的最上面显示的是方程(\ref{e:Pan}) 和 (\ref{e:Fan})所给出的解析解,下面显示的方程 (\ref{e:Perr}) 和 (\ref{e:Ferr}) 所给出的与 $P$ 和 $F$ 的值相对应的误差。 
因此,我们可以看出,文章\inlinecite{yong2015}中选择使用基于由扰动海洋模式初始的温度场而得到的集合模拟结果来评估求解器所引入的不确定是合理的。


 
文章\inlinecite{yong2015}的具体做法是,首先生成了一个包含有40个运行了36个月的模拟的集合,这些实例与原始实例的差别在于在初始的海洋温度场引入了一个$\mathcal{O}(10^{-14})$量级的扰动。 
然后计算新的模拟结果 $\tilde{{X}}$和这个集合模拟的结果用如下的方式计算均方根标准分。 假设集合$E$所包含的成员的个数记为 $N_{ens}$,此时,对于每一变量$X$,在每一个点$i$,都存在一个包含$N_{ens}$个数的数列。
定义点$i$上的集合模拟结果中对应的数列的所有元素的平均值和标准方差分别为$\mu_i$ 和 $\sigma_i$。 
由于总的网格点数为$n$,那么对于变量$\tilde{{X}}$,新的模拟结果与集合模拟结果$E$之间的均方根标准分定义为

\begin{equation}
 RMSZ(\tilde{X}, E) = \sqrt{\frac{1}{n}\sum_{i=1}^n(\frac{\tilde{x}_i -\mu_i}{\sigma_i})^2}. 
\label{e:rmsz}
\end{equation}
图\ref{fig:rmsz_temp_ens} 给出了与图\ref{fig:rmse_temp}中相同的五个收敛条件的模拟结果的RMSZ分数。
这里,由最宽松的收敛条件引入的误差变得非常的明显。 
基于图\ref{fig:rmsz_temp_ens}的结果,文章\inlinecite{yong2015}证明了新的求解器在CESM-POP中是可以接受的。  


\subsection{公共大气模式的集合模拟一致性检测 }
 
文章 \inlinecite{baker2015}中提到的CESM-ECT检测工具利用集合模拟的结果来对CESM模式的气候态的内在的不确定性进行量化,进而将新的模拟结果(由于软件、硬件或者其他不能保证二进制一致的改动引入的新的结果)与这个集合模拟结果的分布做对比。
这个想法的关键是,如果新的模拟结果的输出数据与集合模拟的结果在统计学上是不可以区分的话,那么这个新的结果就认为是与原始结果“一致的”。统计学上的一致性是模式代码正确性验证的质量保证方面的关键因素\citep{oberkampf2010}。
CESM-ECT方法首先被应用于公共大气模式CAM分量的历史输出数据,最终形成了CAM-ECT工具集。
CAM数据很直观的就被当成了验证的首要目标,因为大气中扰动传播的时间尺度要比在CESM其他分量中的短很多。而且,CAM包含有很多不相关的全局变量。 


 
正如文章\cite{baker2015}所描述的那样,CAM-ECT中使用的集合模拟结果是在可信的机器上使用CESM的被验证过的版本和配置运行得到的。 
这个集合模拟包含有151个长达一年的模拟结果,这些结果之间唯一的不同是在初始的大气温度场中加入了$\mathcal{O}(10^{-14})$ 量级的扰动。 
集合模拟的输出结果只包含了特定变量在每个点上的年平均值。这些变量的个数记为$N_{var}$,它们可以反映出整个大气场(默认情况下 $N_{var}=120$)。这个CAM-ECT工具集是对集合模拟结果中的每一个变量的全球的面积加权平均用主成分分析(PCA)的方法,生成了一个反应集合模拟特征的统计学分布。
这个主成分(PC)分数的分布被保存下来,之后用来与新的模拟实验作对比。 
基于落在指定置信区间(通常取95$\%$)之外的PC分数的个数,可以确定少数的几个新的实例(通常是3个)是否通过测试。 
这里可以调优那几个关于是否能通过测试的参数,从而达到想要的误报率。
\section{试验方法和结果分析}
\label{sec:verifyExp}
\begin{figure}[h]
\includegraphics[height=6.5cm]{figures/temp_rmse_GMD.png}
\caption {Monthly Root Mean Square Error (RMSE) of temperature for experiments with different barotropic solver convergence tolerances. Note that this is a subset of Fig. 12 in \cite{yong2015}.}
\label{fig:rmse_temp}
\end{figure} 


 \begin{figure}[h]
 \begin{center}
 \includegraphics[height=6.5cm]{figures/temp_rmsz_GMD.png}
 \end{center}
 \caption {Monthly Root Mean Square Z-score (RMSZ) of temperature with respect to an ensemble (denoted by yellow) for experiments with barotropic different convergence tolerances. Note that this is subset of Fig. 13 in \cite{yong2015}. }
 \label{fig:rmsz_temp_ens}
 \end{figure} 

\begin {figure}[h]
\centering
\includegraphics[height =6.5cm]{figures/pert-force.png}
\caption{The top panel displays four analytic solutions to Eq. (\ref{e:2}) with the indicated perturbations to the initial conditions (P)和perturbation to the forcing term (F).  Note that all four perturbations have similar effect on the ``original'' (unperturbed) solution.  The bottom panel plots the error between the four perturbations和the original solution. In this example, the error due to perturbing the forcing term (F) is constant和smaller in magnitude that the errors caused by the initial condition perturbations.}
\label {fig:1Danalytical}
\end {figure}

\begin {figure}[h]
\centering
\includegraphics[height =20cm]{figures/gimp_all_sst_std.png}
\caption{The ensemble distribution for the standard deviation of sea surface temperature (SST) at months 1, 12, 24,和36.}
\label{fig:SST_STD_all}
\end {figure}

\begin {figure}[h]
\centering
\includegraphics[height =12cm]{figures/RSM-TEMP-tol.png}
\caption {Response surfaces for Z-score of temperature (TEMP) over time (monthly)和 Z-score tolerance.  Each subplot represents a different barotropic
solver convergence tolerance (labeled above). The color bar indicates the percentage of grid points with a Z-score below the Z-score tolerance (on the y-axis).}
\label{fig:RSM-TEMP-tol}
\end {figure}

\begin {figure}[h]
\centering
\includegraphics[height =12cm]{figures/RSM-SSH-tol.png}
\caption {Response surface for Z-score of sea surface height (SSH) over time (monthly)和Z-score tolerance.  Each subplot represents a different barotropic
solver convergence tolerance (labeled above). The color bar indicates the percentage of grid points with a Z-score below the Z-score tolerance (on the y-axis).}
\label{fig:RSM-SSH-tol}
\end {figure}


\begin{figure}[h]
\centering 
\includegraphics[height=7cm]{figures/prz_tol_combined.png}
\caption {Percentage of grid points with Z-scores for temperature
  (TEMP)和sea surface height (SSH) that exceed the 3.0 tolerance for simulations with various barotropic solver convergence tolerances. }
\label {fig:PRZ-tol}
\end{figure}

\begin{figure}[h]
\includegraphics[height=7.0cm]{figures/temp_cores_zoom_combine.png}
\caption{Percentage of grid points with Z-scores for temperature (TEMP) that exceed the 3.0 tolerance for simulations with various numbers of processor cores. (Note that the left和right subplots contain the same information, with different scales for the y-axis.)}
\label {fig:combine}
\end{figure}

\begin {figure}[h]
\centering
\includegraphics[height =12cm]{figures/RSM-TEMP-param.png}
\caption {Response surfaces for Z-score of temperature (TEMP) over time (monthly)和 Z-score threshold.  The top two subplots represent two different processor core layouts.
The bottom left has a tracer mixing coefficient for convective instability that is 10 times larger than the original,和the bottom right uses a different tracer advection scheme. The color bar indicates the percentage of grid points with a Z-score below the Z-score tolerance (on the y-axis).}
\label{fig:RSM-TEMP-param}
\end {figure}

\begin {figure}[h]
\centering
\includegraphics[height =12cm]{figures/RSM-SSH-param.png}
\caption {Response surfaces for Z-score of sea surface height (SSH) over time (monthly)和 Z-score threshold.  The top two subplots represent two different processor core layouts.
The bottom left has a tracer mixing coefficient for convective instability 10 times larger than the original,和the bottom right uses a different tracer advection scheme. The color bar indicates the percentage of grid points with a Z-score below the Z-score tolerance (on the y-axis).}
\label{fig:RSM-SSH-param}
\end {figure}


\begin{figure}[h]
\centering
\includegraphics[height=7cm]{figures/prz_param_combined.png}
\caption{Percentage of grid points with Z-scores for temperature
  (TEMP)和sea surface height (SSH) that exceed the 3.0 threshold for simulations with an alternative tracer advection scheme (lw\_lim)和several tracer mixing coefficients for convective instability.}
\label {fig:PRZ-temp-param}
\end{figure}


  
\begin {figure}[h]
\centering
\includegraphics[height =20cm]{figures/zscore_sst_combine_crop.png}
\caption{Z-score of sea surface temperature (SST) at month 12 for the
original (default) case, a 384 processor core case, a case with a
  larger tracer mixing coefficient for convective instability,和a case with an alternate tracer advection scheme.}
\label {fig:zscore-combine}
\end {figure}

\begin {figure}[h]
\centering
\includegraphics[height =6.5cm]{figures/ens_size_combined.png}
\caption{The distributions of experimental failure rates based on 1000 tests for variables temperature (TEMP)和sea surface height (SSH).  For each ensemble size, the green bars indicate the maximum和minimum values obtained,和the red boxes indicate the mean.   }
 \label {fig:temp_ens_80}
\end {figure}

\section{本章小结}
\label{sec:verifyConclusion}
