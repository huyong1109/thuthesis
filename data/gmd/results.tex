
 

\begin{figure} 
\centering
\includegraphics[height=6.5cm]{conditionNumberAspectRatio}
\caption[] {网格横纵比例和系数矩阵条件数之间的关系。 \label{fig:conditionNumberRatio}}
\end{figure}

\begin{figure} 
\centering
\includegraphics[height=6.5cm]{conditionNumberTimestep}
\caption[] {时间步长和系数矩阵条件数之间的关系。 \label{fig:conditionNumberDt}}
\end{figure}

\begin{figure}
\centering
\includegraphics[height=6.5cm]{conditionNumberGridSize}
\caption[] {网格点的数目和系数矩阵条件数之间的关系。 \label{fig:conditionNumbGrid}}
\end{figure}
 
这个结论说明,当网格横纵比例越接近于1时,最大特诊值的上界会随之减少,而最小特征值的下界会随之变大。
也就是说,系数矩阵特征值谱的半径($[\lambda_{min}, \lambda_{max}]$)随着网格横纵比向着1的接近而变小。 
当水平网格的横纵比等于1时,即$ \alpha = \frac{ \Delta y}{ \Delta x} = 1$,我们可以得到$\lambda_{max} \le  4 +\phi$,$\lambda_{min} \ge   \phi$。
这时,系数矩阵的条件数(即 $\kappa=  \lambda_{max}/\lambda_{min}$),由于是由特征值谱的半径的所决定的,也会随着网格横纵比向1的靠近而变小。 
图\ref{fig:conditionNumberRatio}所给出的结果验证了我们的这个结论。图\ref{fig:conditionNumberRatio}中给出的是固定的网格点数$\mathcal{N} = 20\times 20$和固定的$\phi = 0$, 在区间$(1, +\infty)$上,系数矩阵的条件数随着网格横纵比例的增加而增加,而在区间$(0,1)$上,系数矩阵的条件数随着横纵比例的增大而变小。
特征值在当网格横纵比例等于1时取到最小值。
 
特征值的下界是$\phi=\frac{S }{g \tau^2 H}$,这个变量是由时间步长以及水平网格的面积和海底深度的比例确定的。
它表明,特征值的下界会随着时间步长的增大而减小。 
相应的,系数举证的 条件数也就随之增大。
图\ref{fig:conditionNumberDt}中证明了我们的这个推论。
这个图中的实验采用的是固定的网格点数$\mathcal{N} = 20\times 20$,和固定的网格横纵比$\Delta x /{\Delta y} = 1$。 


 
不考虑时间步长的因素(假设$\phi=0$)时,上面的分析表明,当网格的横纵比等于1时,不论网格点的个数是多少, 系数矩阵的特征值谱半径限制在 $(0,4)$ 区间内。
但是,系数矩阵的条件数的变化却可能会很大,因为当网格点的数量$\mathcal{N}$ 不断增加时,最小的特征值会组件的逼近零。 
图\ref{fig:conditionNumbGrid} 中解释了条件数和网格点数目之间的关系。 

 











