% This is LLNCS.DEM the demonstration file of
% the LaTeX macro package from Springer-Verlag
% for Lecture Notes in Computer Science,
% version 2.4 for LaTeX2e as of 16. April 2010s
%
%\documentclass[journal abbreviation, manuscript]{copernicus}
\documentclass[gmd, manuscript]{copernicus}
%
\usepackage{amsmath}
\usepackage{epstopdf}
\DeclareGraphicsExtensions{.eps,.mps,.pdf,.jpg,.PNG}
\DeclareGraphicsRule{*}{pdf}{*}{}
\graphicspath{{./fig/}}

\begin{document}

\linenumbers

%%\title{Scalable Barotropic Solver for High-resolution Ocean Models}
\title{Accelerating the High-resolution Ocean Model Component in the Community Earth System Model}

\author[1,2]{Xiaomeng Huang}
\author[1]{Yong Hu}
\author[3]{Yuheng Tseng}
\author[3]{Allison H. Baker}
\author[3]{Frank O. Bryan}
\author[3]{John Dennis}
\author[1]{Haohuan Fu}
\author[1]{Guangwen Yang}

\affil[1]{Ministry of Education Key Laboratory for Earth System Modeling, and Center for Earth System Science, Tsinghua University, Beijing, 100084, China}
\affil[2]{Laboratory for Regional Oceanography and Numerical Modeling, Qingdao National Laboratory for Marine Science and Technology, Qingdao, 266237, China}
\affil[3]{The National Center for Atmospheric Research, Boulder, CO, USA}
%% The [] brackets identify the author to the corresponding affiliation, 1, 2, 3, etc. should be inserted.

\runningtitle{Accelerating the High-resolution ocean model component in the Community Earth System Model}

\runningauthor{X.M. Huang et al. }

\correspondence{Xiaomeng Huang (hxm@tsinghua.edu.cn), Yuheng Tseng(ytseng@ucar.edu)}
\received{}
\pubdiscuss{} %% only important for two-stage journals

\revised{}
\accepted{}
\published{}
%% These dates will be inserted by the Publication Production Office during the typesetting process.
\firstpage{1}

%title

\maketitle

\begin{abstract}
In the Community Earth System Model (CESM), the ocean model component is computationally expensive for high-resolution grids and is frequently the least scalable component of CESM for certain production experiments. The key problem is that the modified Preconditioned Conjugate Gradient (PCG) used to solve the barotropic mode scales poorly at the high core counts. We design a fast numerical solver to accelerate the high-resolution ocean simulation. The novel solver adopts a Chebyshev type iterative method to reduce the global communication cost in conjunction with an effective block preconditioner to reduce the iterations further. Comparing with the two origin solvers, it significantly reduces the global communication time and realizes a competitive convergence rate. The experimental results show that the simulation speed of the improved ocean model component with 0.1{\degree}  resolution achieves 10.5 simulated years per wall-clock day on 16,875 cores.
\end{abstract}

%----------------------------------------------------------------------------
\introduction  \label{se:int}
Recent progresses on high-resolution global climate models have demonstrated that refining the model resolution is helpful for representing important climate processes so as to facilitate climate prediction. Significant improvements can be achieved in the global simulations of Tropical Instability Waves \citep{roberts2009impact}, El Ni\~no Southern Oscillation (ENSO) \citep{shaffrey2009uk}, the Gulf Stream \citep{chassignet2008gulf, kuwano2010precipitation} , the global water cycle \citep{demory2014role}, and others. Specifically, \cite{gent2010improvements}  and \cite{wehner2014effect} showed that increasing the resolution of atmosphere models produces better mean climate, more accurate depiction of the tropical storm formation, and more realistic extreme daily precipitation.  \cite{bryan2010frontal} and  \cite{graham2014importance} confirmed that increasing the resolution of ocean model to eddy resolving level helps to capture the positive correlation between sea surface hight and surface wind stress, as well as to improve the asymmetry of the ENSO cycle in simulation.

%High-resolution climate simulations require tremendous computing resources.
In the High Resolution Model Intercomparison Project (HighResMIP) application for  the Coupled Model Intercomparison Project phase 6 (CMIP6), the global model resolutions with 25 km or finer at mid-latitudes are proposed to implement the Tier-1 and Tier-2 experiments. Because all climate models participated in CMIP6 are often need to run for hundreds of years, tremendous computing resources are needed to run the high-resolution production simulations so that the simulations are extraordinary costly. In order to run high-resolution climate models more routinely, additional algorithm optimization is required to utilize the large scale computing resources.

Our work focuses on improving the simulation speed of the ocean model component (Parallel Ocean Model, POP) in CESM, whose development is centered at the National Center for Atmospheric Research(NCAR). It is a fully-coupled climate system model, including atmosphere, ocean, sea-ice and land components. Its ocean model component solves the three-dimensional primitive equations with hydrostatic and Boussinesq approximations and divides the time integration into two parts: the baroclinic mode and the barotropic mode  \citep{smith2010parallel}. The baroclinic mode describes the three-dimensional dynamic and thermodynamics processes, and the barotropic mode solves the vertically-integrated momentum and continuity equations in two dimensions.




%----------------------------------------------------------------------------
%improving barotropic

The barotropic solver is the major bottleneck in the POP within the high-resolution CESM because it dominates the total computation time on a large number of cores.
This results from the barotropic solver for calculating free surface, which scales poorly at the high core counts because of a obvious global communication bottleneck inherent with the algorithm.
The implicit free-surface method is used in the barotropic mode because it allows a large time step to efficiently compute the fast gravity mode.
The drawback of this method is requiring solving a large elliptic system of equations.
Conjugate Gradient method (CG) and its variants are popular choices to solve elliptic equations due to calculating free surface implicitly in ocean models like MITgcm\citep{adcroft2014mitgcm}, FVCOM\citep{lai2010nonhydrostatic}, MOM3\citep{pacanowsky1999mom3}, OPA \citep{madec1997ocean} and so on.
However, CG method causes heavy global communication overhead in the existing version of ocean model component \citep{Worley:2011:PCE:2063384.2063457}.
A number of works has been attempted to improve the performance of Conjugate Gradient (CG) algorithm while most of them has been related to reducing the amount of communication between processes and accelerating the computation of each process.
Methods that reduce the number of global reductions over the standard formulation, such as the Chronopoulos-Gear (ChronGear, \cite{dAzevedo1999lapack}) variant used in POP, were early contributions to the field and still popular.
These methods as well as more recent incarnations (e.g.,  \cite{hoemmen2010}) attempt to reduce global communications, but combining them with a sophisticated preconditioner is non-trivial. A nice overview of reducing global communication costs for CG can be found in \cite{ghysels2014}. In addition, recent efforts at improving the performance of CG include a variant that overlaps the global reduction with the matrix-vector computation via a pipelined-approach \citep{ghysels2014}.

Another highlighted method in improving CG method is preconditioning,
which has been shown to effectively reduce the number of iterations in the conjugate gradient method since the 1990s.
The current ChronGear solver in the POP has benefited by using a simple diagonal preconditioner \citep{pini1990simple, reddy2013comparison}.
Some parallelizable methods such as polynomial, approximate-inverse, multigrid, and block preconditionings have drawn much attention recently.
High-order polynomial preconditioning can reduce iterations as effectively as incomplete LU factorization (ILU) and its variants in sequential simulations \cite{benzi2002preconditioning}.
However, the computational overhead for the polynomial preconditioner typically offsets its superiority to the simple diagonal preconditioner (e.g., \cite{meyer1989numerical,smith1992parallel}).
The approximate-inverse preconditioner, while highly parallelizable, requires to solve a linear system several times larger than the original system (e.g., \cite{smith1992parallel,bergamaschi2007numerical}), which makes it less attractive for the POP.

In addition, multigrid is highly scalable and effective for linear systems derived from elliptic systems of equations.
Recent works indicated that  geometric multigrid is promising in atmosphere and ocean modeling (e.g., \cite{muller2014massively}, \cite{matsumura2008non,kanarska2007algorithm}).
However, the geometric multigrid method in global ocean models does not always scale ideally because of the presence of complex topography, non-uniform or anisotropic grids (e.g., \cite{fulton1986multigrid,stuben2001review,tseng2003ghost,matsumura2008non}).
This is the case for the current POP which employs the dipole general orthogonal girds to avoid the polar singularity in the ocean. This leads to an elliptic system with variable coefficients defined on an irregular domain with non-uniform grids.
Algebraic multigrid (AMG) is an alternative to Geometric multigrid to handle complex topography. But, setting the AMG in the parallel environment is more expensive than the iterative solver, which makes it unfavorable as a preconditioner (\cite{muller2014massively}).


Block preconditioning has been shown to be an effective parallel preconditioner (e.g., \cite{concus1985block, white2011block}) and is appealing for the POP because it makes use of the block structure of the coefficient matrix that arises from discretization of the elliptic equations.
This work studies a block preconditioning method based on Error Vector Propagation (EVP) method, which has a priority in efficiency in solving  elliptic equations.

There are also some other approaches to improve the performance of ocean models. Dennis and Tufo proposed a load-balancing algorithm based on space-filling curve\citep{dennis2007inverse, dennis2008scaling}. This algorithm not only eliminates land blocks, but also decreases the communications overhead because of the reduced number of processes. Since reducing the frequency of communication can decrease the computing overhead, \citet{beare1997optimisation} proposed an approach by increasing the number of extra halos and overlapping the communications with the computation to optimize the computing performance of parallel ocean general circulation. Although all these approaches improve performance of ocean models, they do not attempt to remove the global communication bottleneck. The promising preliminary results in our previous work \cite{hu2015improving} with implementing a Chebyshev type method \citep{stiefel1958kernel} in CESM-POP, encouraged us to explore finer linear solvers for high-resolution ocean models.

To improve the scaling of the ocean model component, we abandon the PCG method and design a new ocean barotropic solver which does not include global communication. The new barotropic solver, named as P-CSI, includes a Classical Stiefel Iteration (CSI) method with an effectively block preconditioner based on the Error Vector Propagation (EVP) method \citep{roache1995elliptic}. The P-CSI solver will become the default ocean barotropic solver in the upcoming version of CESM v2.0. This paper is an extended version of our conference paper \citep{hu2015improving} presented at the 27th International Conference for High Performance Computing, Networking, Storage and Analysis (SC2015). Comparing with the conference paper, we focus on the high-resolution ocean model component and  add the theoretical analysis about the computational complexity and convergence of P-CSI. Most of the sections have been rewritten to make our methods easy to understand by climate modelers.

The remainder of this paper is organized as follows. Section \ref{se:baro} reviews the  barotropic mode in the ocean model component and the existing solvers. Sections \ref{se:pcsi} describes the detail design of P-CSI solver. Section \ref{se:Algorithm} analyzes the computational complexity and convergence rate of P-CSI. Section \ref{se:exp} compares the computing performance of the existing solvers and the P-CSI solvers. Finally, conclusions are given in Sections \ref{se:conc}.

\input{barotropic_solver}
%----------------------------------------------------------------------------
\section{Design of the P-CSI Solver} \label{se:pcsi}
%----------------------------------------------------------------------------
The PCG  solver usually are faster than the stationary iterative methods such as Jacobi method, Gauss-Seidel method and the Successive over-relaxation method, because of its fast convergence speed \citep{golub2012matrix}.
However, PCG requires  inner products to determine the step size, which could take much time on large-scale parallelism.
Although ChronGear mitigates the bottleneck of global communications, its performance is still limited by the inherent poor data locality and the remaining global communication operations. To accelerate the solving the linear system in ocean model component, we need to reduce the global communications as much as possible, as well as keep a close convergence rate as PCG.

\subsection{Classical Stiefel iterative method}
The CSI is a special kind of Chebyshev Iterative methods \citep{stiefel1958kernel}, which was revisited by \citeauthor{gutknecht2002chebyshev} (\citeyear{gutknecht2002chebyshev})
and identified to be ideal for massively parallel computers. As early as 1985, Saad et al.~\citeyear{saad1985solving} proposed a generalization of CSI on linearly connected processors and claimed that this approach performs better than the conjugate gradient method when the eigenvalues are known.

In the procedure of P-CSI (details of P-CSI is presented in \ref{algorithm}), the iteration parameters which control the searching directions in iteration steps,
are derived from a stretched Chebyshev function of two extreme eigenvalues.
As a result, this approach requires no inner product operation, potentially avoiding the bottleneck of global reduction (see the workflow of ChronGear and P-CSI in Figure \ref{fig:pcsi_pcg}).
It is clear to see that the P-CSI is more scalable than ChronGear on massively parallel architectures. However, it requires a priori knowledge about the spectrum of coefficient matrix $A$ \citep{gutknecht2002chebyshev}. It is well-known that obtaining the eigenvalues is equivalent to solving a linear system of equation.
Fortunately, the coefficient matrix $A$ and its preconditioned form in the POP are both real symmetric, positive definite matrices. Approximate estimation of the largest and smallest eigenvalues, $\mu$ and $\nu$, respectively, of the preconditioned coefficient matrix is sufficient to ensure the convergence of P-CSI.

To get the eigenvalues estimations, we adopt the efficient Lanczos method ~\citep{Paige1980235} to estimate the extreme eigenvalues.
Since preconditioning is used in our case, we need to estimate the extreme eigenvalues of $M^{-1}A$ matrix.
The procedure of eigenvalues estimations presented in Appendix \ref{algorithm}.
Tests indicate that only a small number of Lanczos steps are necessary to generate eigenvalue estimates of $M^{-1}A$ that result in near-optimal P-CSI convergence. That means that the computational overhead of eigenvalue estimates  is very small in our case.

\begin {figure}[!htbp]
\begin{center}
\includegraphics[width=12cm,height=6cm]{pcsi_pcg.png}
\caption []{Workflow of ChronGear and P-CSI iterations when four processes are used. \label{fig:pcsi_pcg}}
\end{center}
\end {figure}


\subsection{A block EVP preconditioner} \label{se:evp}
Block preconditioning is an promising parallel preconditioner in POP because it makes use of the block structure of the coefficient matrix that arises from discretization of the elliptic equations.
A block preconditioning based on the EVP method is proposed and detailed in \citet{hu2015improving} to improve the parallel performance of the barotropic solver in the POP.
To the best of our knowledge, the EVP approach and its variants are one of the least costly algorithms for solving elliptic equation in serial \citep{roache1995elliptic},
which have been used in several different Ocean models \citep{dietrich1987ocean,sheng1998candie,tseng2011parallel}. The parallel EVP solver is also implemented in \citet{tseng2011parallel}.
Theoretically, the EVP approach is a direct solver, which requires two steps: preprocessing and solving.
In the preprocessing step, the influence coefficient matrix and its inverse are computed, involving a computational complexity of $\mathcal{C}_{pre} = (2n-5)* 9n^2 + (2n-5)^3 = \mathcal {O} (26n^3)$, which is intensive but only needs to computed once at the beginning.
Obtaining the solution in each solving step is inexpensive and requires only $\mathcal{C}_{evp}= 2* 9n^2 + (2n-5)^2 = \mathcal{O} (22n^2)$ \citep{hu2015improving}.
This indicates that the EVP has lower computational cost for the solver step than other direct solvers such as LU.

The EVP method is an efficient option for solving elliptic equations. However, a major drawback of the standard EVP is that, without applying additional modifications, it cannot be used on a large domain due to its error propagation and numerical instabilities in the marching process \citep{roache1995elliptic}.
The fact that the EVP is not well-suited for large domains indeed poses no issue for large-scale parallel computing, where a larger number of processors results in smaller domains.
The serial disadvantage becomes an advantage in parallel computing. On a small block up to the size of $12\times 12$, EVP solves with an acceptable round-off error of $\mathcal{O}(10^{-8})$ when double-precision floating-point is used.
This can even make the EVP ideal for parallel block preconditioning on a large number of cores.

Although the EVP preconditioning doubles the computation in each iteration, it halves both global and boundary communications which dominate in the barotropic execution time at the high core counts. This advantage will be tested and verified in Section \ref{sec:exp-preconditioner}.
The implementation of EVP preconditioning significantly reduces the number of iterations for both the ChronGear and P-CSI solvers \citep{hu2015improving}.

\section{算法分析与比较}\label{se:Algorithm}
 
除了P-CSI,PCG和ChronGear方法的收敛性也同样依赖于系数矩阵的最大最小特征值。
这里,我们首先来挖掘一下影响求解器收敛的特征值的特征,然后给出ChronGear和P-CSI求解器的计算复杂度和收敛速度的理论结果。 

\subsection{特征值谱和条件数}
POP中的系数矩阵 $A$ 是正定对称的\citep{smith2010parallel},因此它的特征是都是正实数\citep{stewart1976positive}。
假设系数举证的特征值谱\citep{golub2012matrix} 是 $\mathcal{S} = \{\lambda_1, \lambda_2, \cdots, \lambda_N\}$,这里 $\lambda_{min} = \lambda_1 \le \lambda_i \le \lambda_\mathcal{N} = \lambda_{max}$( $1<i <\mathcal{N}$, $\aleph$ 为 $A$的大小 )表示$A$的特征值.
 
利用 Gershgorin圆盘定理\citep{bell1965gershgorin}可知,对于任意的 $\lambda \in \mathcal{S}$,都存在一个数对 $(i,j)$ 满足
\begin{align}
&|\lambda -  (A_{i,j}^O + \phi ) | \le \sum_{\chi \in \{NW,NE,SW,SE,W,E,N,S\}}|A_{i,j}^\chi|
\end{align}
由方程\ref{defineA}给出的系数的定义,我们可以得到如下结论 
\begin{align} \label{eigsGersh}
&\lambda_{max} \le  \max (  5\alpha - \frac{1}{\alpha}, \frac{5}{\alpha}- \alpha) +\phi   \\
&\lambda_{min} \ge 2\min (  \alpha - \frac{1}{\alpha},\frac{1} {\alpha} -  \alpha) + \phi
\end{align}

\begin {figure}[!htbp]
\centering
\includegraphics[height=6.5cm]{conditionNumberAspectRatio}
\caption[] {网格横纵比例和系数矩阵条件数之间的关系。 \label{fig:conditionNumberRatio}}
\end{figure}
\begin {figure}[!htbp]
\centering
\includegraphics[height=6.5cm]{conditionNumberTimestep}
\caption[] {时间步长和系数矩阵条件数之间的关系。 \label{fig:conditionNumberDt}}
\end{figure}

\begin {figure}[!htbp]
\centering
\includegraphics[height=6.5cm]{conditionNumberGridSize}
\caption[] {网格点的数目和系数矩阵条件数之间的关系。 \label{fig:conditionNumbGrid}}
\end{figure}
 
这个结论说明,当网格横纵比例越接近于1时,最大特诊值的上界会随之减少,而最小特征值的下界会随之变大。
也就是说,系数矩阵特征值谱的半径($[\lambda_{min}, \lambda_{max}]$)随着网格横纵比向着1的接近而变小。 
当水平网格的横纵比等于1时,即$ \alpha = \frac{ \Delta y}{ \Delta x} = 1$,我们可以得到$\lambda_{max} \le  4 +\phi$,$\lambda_{min} \ge   \phi$。
这时,系数矩阵的条件数(即 $\kappa=  \lambda_{max}/\lambda_{min}$),由于是由特征值谱的半径的所决定的,也会随着网格横纵比向1的靠近而变小。 
图\ref{fig:conditionNumberRatio}所给出的结果验证了我们的这个结论。图\ref{fig:conditionNumberRatio}中给出的是固定的网格点数$\mathcal{N} = 20\times 20$和固定的$\phi = 0$, 在区间$(1, +\infty)$上,系数矩阵的条件数随着网格横纵比例的增加而增加,而在区间$(0,1)$上,系数矩阵的条件数随着横纵比例的增大而变小。
特征值在当网格横纵比例等于1时取到最小值。
 
特征值的下界是$\phi=\frac{S }{g \tau^2 H}$,这个变量是由时间步长以及水平网格的面积和海底深度的比例确定的。
它表明,特征值的下界会随着时间步长的增大而减小。 
相应的,系数举证的 条件数也就随之增大。
图\ref{fig:conditionNumberDt}中证明了我们的这个推论。
这个图中的实验采用的是固定的网格点数$\mathcal{N} = 20\times 20$,和固定的网格横纵比$\Delta x /{\Delta y} = 1$。 


 
不考虑时间步长的因素(假设$\phi=0$)时,上面的分析表明,当网格的横纵比等于1时,不论网格点的个数是多少, 系数矩阵的特征值谱半径限制在 $(0,4)$ 区间内。
但是,系数矩阵的条件数的变化却可能会很大,因为当网格点的数量$\mathcal{N}$ 不断增加时,最小的特征值会组件的逼近零。 
图\ref{fig:conditionNumbGrid} 中解释了条件数和网格点数目之间的关系。 

\subsection{收敛速度} \label{convergence_rate}

 PCG和ChronGear方法的收敛速度都依赖于系数矩阵$A$的条件数。  
\inlinecite{dAzevedo1999lapack}证明了 ChronGear  和PCG在理论上收敛速度是相同的。 
在PCG和ChronGear方法中,每一步迭代的残差都有一个上界\citep{Liesen2004}
\begin{equation}
\frac{||\textbf{x}_k-\textbf{x}^*||_A }{||\textbf{x}_0-\textbf{x}^*||_A}  \le \min_{p\in \mathcal{P}_k, p(0) = 1 }\max_{\lambda \in \mathcal{S}} |p(\lambda)| \label{PcgConvergeRate}
\end{equation}
这里$\textbf{x}_k$ 表示第$k$步迭代后得到的解向量, $\textbf{x}^*$ 表示线性方程组的解析解  (即$\textbf{x}^* = A^{-1}b$),$\lambda$ 表示系数矩阵 $A$的特征值。
利用 第一类Chebyshev 多项式来估计这个最小最大值,可以得到 
\begin{align}
\label{chrongear_convergence}
||\textbf{x}_k-\textbf{x}^*||_A \le  2 (\frac{\sqrt{\kappa}-1}{\sqrt{\kappa}+1})^k ||\textbf{x}_0-\textbf{x}^*||_A
\end{align}
这里  $\kappa =  \kappa_2(A)$ 表示系数矩阵 $A$的条件数。


方程\ref{chrongear_convergence}表明,PCG方法收敛速度的理论上界取决于系数矩阵的条件数。 
PCG方法在求解良态的矩阵(特征值的条件数很小)时比求解病态的矩阵(特征值的条件数很大)时收敛速度更快。 
图\ref{fig:convergence_diag}给出了PCG和ChronGear方法在0.1度海洋模式中的真实的收敛速度。
如这幅图所示,相对残差随着迭代步数的增加而减少。
总共需要一百步左右的迭代步才能够得到一个相对残差为$10^{-13}$的解。
 

 
除了消除了全局归约操作,P-CSI方法的更一个优点是它有和PCG相同量级的收敛速度。 
P-CSI方法的收敛速度可以用残差的形式表示为
\begin{equation}
\textbf{r}_k = P_k(A)\textbf{r}_0 \label{eq:rPjr0}
\end{equation}
这里
$P_k(\zeta) = \frac{\tau_k(\beta-\alpha \zeta)}{\tau_k(\beta)}$ for $ \zeta \in [\nu, \mu]$ ~\citep{stiefel1958kernel} .
$\tau_k(\xi)$ 是Chebyshev多项式,它的表达式为  
\begin{equation}
\tau_k(\xi) =   \frac{1}{2}[(\xi+\sqrt{\xi^2-1})^k+(\xi+\sqrt{\xi^2-1})^{-k}]
\end{equation}
当 $ \xi \in [-1,1]$时, Chebyshev多项式的另一个等价表达式为$\tau_k(\xi) = cos(k\cos^{-1} \xi)$。从这个表达式可以清楚的看出来,当$| \xi | \le 1$时, $|\tau_k(\xi)| \le 1$。$P_k(\zeta)$ 是一个多项式,满足如下形式
\begin{equation}
P_k = \min_{p\in \mathcal{P}_k, p(0) = 1 }\max_{\zeta \in [\nu,\mu]} |p(\zeta)|
\end{equation}
%which is the theoretical bound of the convergence rate  in PCG \ref{PcgConvergeRate}.

假设$A= Q^T\Lambda Q$,这里 $\Lambda$ 表示以  $A$的特征值为对角元素的对角矩阵,而 $Q$ 表示以  $A$的特征向量为列的实正交矩阵。于是,我们可以得到
\begin{equation}
P_k(A) = Q^T P_k(\Lambda)Q = Q^T \left [\begin{array}{cccc}
P_k(\lambda_1) & & &\\
& P_k(\lambda_2) & &\\
& & \ddots &\\
 & & & P_k(\lambda_N)
\end{array} \right ] Q \label{eq:PjMA}
\end{equation}
Assume that the estimated largest and smallest extreme eigenvalues of coefficient matrix 
假设所估计的系数矩阵的最大最小特征值$\nu$和 $\mu$ 满足 $0 < \nu \le \lambda_i \le \mu$ ($i = 1, 2, \cdots, \mathcal{N}$),这时有$|\beta - \alpha \lambda_i| \le 1$, $|P_k(\lambda_i)| \le \tau^{-1}_k (\beta)$.
方程 \ref{eq:rPjr0} 和 \ref{eq:PjMA} 可以推出
\begin{equation}
\label{pcsi_convergence}
\frac{||\textbf{r}_k||_2}{||\textbf{r}_0||_2}  \le  \tau_k^{-1}(\beta) = \frac{2(\beta+\sqrt{\beta^2-1})^k}{1+(\beta+\sqrt{\beta^2-1})^{2k}} \le 2(\frac{\sqrt{\kappa'}-1}{\sqrt{\kappa'}+1})^k
\end{equation}
这里$\kappa' = \frac{\mu}{\nu}$.
 这个等式表明,当系数矩阵特征值的估计比较合理的时候(也就是$\kappa' =\kappa$) , P-CSI的收敛速度和PCG有相同的理论上界。


 
以上分析对于采用了预处理子的情形也同样适用。
唯一的区别是上面分析中提到的系数矩阵是$M^{-1}A$ 而不是$A$。
值得一提的是,PCG,ChronGear和P-CSI算法中预处理后的矩阵实际上是$M^{-1/2}A(E^{-1/2})^T$,它是对称的,并且和$M^{-1}A$ 有着相同的特征值\citep{Shewchuk1994}。
因此,这个预处理后的矩阵的条件数为$\kappa =  \kappa_2(M^{-1/2}A(E^{-1/2})^T)$。这个条件数通常情况下比原始矩阵 $A$的条件数要小。 
 $M$ 与$A$越是相近,$M^{-1}A$的条件数就越小。当$M = A$,$M^{-1}A$的条件数为$\kappa_2(M^{-1 }A ) = 1$。 

 当使用少数几个处理器核心来运行海洋模式分量时,全局通信并不是瓶颈,导致P-CSI方法和ChronGear方法的时间开销比较接近。
但是当采用很多的处理器核心来计算高分辨的海洋模式分量的时候,P-CSI方法的每一个步迭代就应该会比ChronGear方法快很多,因为ChronGear方法中的全局归约操作就会变成一个明显的瓶颈。 


\subsection{计算复杂度}  \label{subse:complex}


 
当使用了EVP预处理时,计算的复杂变成了$\mathcal{T}_c = 37 \frac{\mathcal{N}}{p}$。因此P-CSI方法在使用EVP方法时一次求解过程的复杂度为
\begin{eqnarray}
\label{t_pcsiEvp}
\mathcal{T}_{pcsi-evp}=\mathcal{O}(\mathcal{K}_{pcsi-evp} (37\frac{\mathcal{N}}{p} +8\sqrt{\frac{\mathcal{N} }{p}} + 4\alpha))
\end{eqnarray}

以上式子可以看出来,使用了EVP预处理后,每一步迭代过程中的计算开销翻了一倍。
但是,一次求解的总体时间是下降。 
采用EVP预处理后,迭代步数$\mathcal{K}_{pcsi-evp}$会下降到原来的一半(参见图\ref{fig:convergence_diag})。
其结果是,在大核上运行时,计算时间变得非常的少,而最耗时的通信部分的次数也随着迭代步数的减少而减少了一半左右。





 

\begin{figure} 
\centering
\includegraphics[height=6.5cm]{conditionNumberAspectRatio}
\caption[] {网格横纵比例和系数矩阵条件数之间的关系。 \label{fig:conditionNumberRatio}}
\end{figure}

\begin{figure} 
\centering
\includegraphics[height=6.5cm]{conditionNumberTimestep}
\caption[] {时间步长和系数矩阵条件数之间的关系。 \label{fig:conditionNumberDt}}
\end{figure}

\begin{figure}
\centering
\includegraphics[height=6.5cm]{conditionNumberGridSize}
\caption[] {网格点的数目和系数矩阵条件数之间的关系。 \label{fig:conditionNumbGrid}}
\end{figure}
 
这个结论说明,当网格横纵比例越接近于1时,最大特诊值的上界会随之减少,而最小特征值的下界会随之变大。
也就是说,系数矩阵特征值谱的半径($[\lambda_{min}, \lambda_{max}]$)随着网格横纵比向着1的接近而变小。 
当水平网格的横纵比等于1时,即$ \alpha = \frac{ \Delta y}{ \Delta x} = 1$,我们可以得到$\lambda_{max} \le  4 +\phi$,$\lambda_{min} \ge   \phi$。
这时,系数矩阵的条件数(即 $\kappa=  \lambda_{max}/\lambda_{min}$),由于是由特征值谱的半径的所决定的,也会随着网格横纵比向1的靠近而变小。 
图\ref{fig:conditionNumberRatio}所给出的结果验证了我们的这个结论。图\ref{fig:conditionNumberRatio}中给出的是固定的网格点数$\mathcal{N} = 20\times 20$和固定的$\phi = 0$, 在区间$(1, +\infty)$上,系数矩阵的条件数随着网格横纵比例的增加而增加,而在区间$(0,1)$上,系数矩阵的条件数随着横纵比例的增大而变小。
特征值在当网格横纵比例等于1时取到最小值。
 
特征值的下界是$\phi=\frac{S }{g \tau^2 H}$,这个变量是由时间步长以及水平网格的面积和海底深度的比例确定的。
它表明,特征值的下界会随着时间步长的增大而减小。 
相应的,系数举证的 条件数也就随之增大。
图\ref{fig:conditionNumberDt}中证明了我们的这个推论。
这个图中的实验采用的是固定的网格点数$\mathcal{N} = 20\times 20$,和固定的网格横纵比$\Delta x /{\Delta y} = 1$。 


 
不考虑时间步长的因素(假设$\phi=0$)时,上面的分析表明,当网格的横纵比等于1时,不论网格点的个数是多少, 系数矩阵的特征值谱半径限制在 $(0,4)$ 区间内。
但是,系数矩阵的条件数的变化却可能会很大,因为当网格点的数量$\mathcal{N}$ 不断增加时,最小的特征值会组件的逼近零。 
图\ref{fig:conditionNumbGrid} 中解释了条件数和网格点数目之间的关系。 

 












\conclusions \label{se:conc}
We accelerated the high-resolution ocean model component in the CESM by implementing a new P-CSI ocean barotropic solver. This new solver adopts a Chebyshev type iterative method to avoid the global communication operations in conjunction with an effective EVP preconditioner to improve the computing performance further. Comparing with the origin solvers, it significantly reduces the global reductions and realizes a competitive convergence rate. These approaches and experiences in improving the performance of high-resolution ocean model component in CESM may be helpful to other climate models.

\section{Code availability}
The present P-CSI solver v1.0 is available at https://github.com/hxmhuang/PCSI.  These code are also included in the upcoming CESM public releases v2.0.

\appendix
\section{Algorithms} \label{algorithm}   %% Appendix A

The procedure of PCG is shown as following \citep{smith2010parallel}: \\
 \space \\
Initial guess: $\textbf{x}_0$  \\
Compute residual $\textbf{r}_0 = \textbf{b}- {\textbf{A}}\textbf{x}_0$ \\
Set $\textbf{s}_0 =0$, $\beta_0=1$\\
For $k = 1, 2, \cdots,  k_{max}$  do
\begin{enumerate}
\item $\textbf{r}'_{k-1} = \textbf{M}^{-1}\textbf{r}_{k-1} $  \label{pcg_scale1}
\item $\beta_k = \textbf{r}_{k-1}^T\textbf{r}'_{k-1} $\label{pcg_dot1}
\item $\textbf{s}_k = \textbf{r}'_{k-1} +(\beta_k/\beta_{k-1})\textbf{s}_{k-1} $ \label{pcg_scale2}
\item $\textbf{s}'_k = \textbf{A}\textbf{s}_k   $ \label{pcg_mat}
\item $\alpha_k =\beta_k/(\textbf{s}_k^T\textbf{s}'_k )$\label{pcg_dot2}
\item $\textbf{x}_k= {\textbf{x}}_{k-1} +\alpha_k \textbf{s}_k   $ \label{pcg_scale3}
\item $\textbf{r}_k =\textbf{r}_{k-1} -\alpha_k\textbf{s}'_k  $\label{pcg_scale4}
\item convergence\_check($\textbf{r}_{k}$)
\end{enumerate}
End Do \\

Operations like  $\beta_k/\beta_{k-1}$ in line (\ref{pcg_scale2})) are scaler computations,  while $\alpha_k \textbf{s}_k$ in line (\ref{pcg_scale3}) are vector scalings.
$\textbf{A}\textbf{s}_k$ in line (\ref{pcg_mat})) is a matrix-vector multiplication.
Inner-products of vectors are  $\textbf{r}_{k-1}^T\textbf{r}'_{k-1}$ in line (\ref{pcg_dot1}) and $\textbf{s}_k^T\textbf{s}'_k$  in line (\ref{pcg_dot2})).


The procedure of ChronGear is shown as following : \\
 \space \\
Initial guess: $\textbf{x}_0$  \\
Compute residual $\textbf{r}_0 = \textbf{b}- {\textbf{A}}\textbf{x}_0$ \\
Set $\textbf{s}_0 =0$, $\textbf{p}_0 =0$, $\rho_0=1$, $\sigma_0=0$ \\
For $k = 1, 2, \cdots,  k_{max}$  do
\begin{enumerate}
\item $\textbf{r}'_{k} =\textbf{M}^{-1}\textbf{r}_{k-1} $  \label{cg_scale0}
\item $\textbf{z}_k = \textbf{A}\textbf{r}'_{k}$  \label{cg_mat}
\item $\rho_k = \textbf{r}_{k-1}^T\textbf{r}'_{k}$\label{cg_dot1}
\item $\sigma_k = \textbf{z}_k^T\textbf{r}'_k - \beta_k^2\sigma_{k-1}$\label{cg_sigma}
\item $\beta_k = \rho_k / \rho_{k-1}$\label{cg_beta}
\item $\alpha_k = \rho_k /\sigma_{k}$\label{cg_alpha}
\item $\textbf{s}_k = \textbf{r}'_{k} +\beta_k\textbf{s}_{k-1}$\label{cg_scale1}
\item $\textbf{p}_k = \textbf{z}_{k} +\beta_k\textbf{p}_{k-1}$\label{cg_scale2}
\item $\textbf{x}_k =\textbf{x}_{k-1} +\alpha_k \textbf{s}_k$\label{cg_scale3}
\item $\textbf{r}_k =\textbf{r}_{k-1} -\alpha_k\textbf{p}_k$
\item convergence\_check($\textbf{r}_{k}$)
\end{enumerate}
End Do \\


The pseudo code of the P-CSI algorithm  is shown as \\
 \space \\
Initial guess: $\textbf{x}_0$, estimated eigenvalue boundary $[\nu,\mu]$\\
Set $\alpha =\frac{2}{\mu -\nu}$, $ \beta = \frac{\mu +\nu}{\mu -\nu}$, $\gamma = \frac{\beta}{\alpha}$, $\omega_0 =\frac{ 2}{\gamma}$ \\
Compute residual $\textbf{r}_0 = \textbf{b}- {\textbf{A}}\textbf{x}_0$, $\Delta \textbf{x}_{0} = \gamma^{-1}\textbf{M}^{-1}\textbf{r}_0$, $\textbf{x}_1 =\textbf{x}_0 +\Delta \textbf{x}_{0}$, $\textbf{r}_1 =\textbf{b} -\textbf{A}\textbf{x}_1$ \\
For $k = 1, 2, \cdots,  k_{max}$  do
\begin{enumerate}
\item $\omega_k  = 1/(\gamma - \frac{1}{4\alpha^2}\omega_{k-1})$
\item $\textbf{r}'_{k}=\textbf{M}^{-1}\textbf{r}_{k}$
\item $\Delta \textbf{x}_{k} =\omega_k\textbf{r}'_{k}+(\gamma \omega_k-1)\Delta \textbf{x}_{k-1}$
\item $\textbf{x}_{k+1} =\textbf{x}_{k}+\Delta \textbf{x}_{k}$
\item $\textbf{r}_{k+1} =\textbf{b}- \textbf{A}\textbf{x}_{k+1}$
\item convergence\_check($\textbf{r}_{k}$)
\end{enumerate}
End Do \\


\section{Eigenvalue Estimation}                               %% Appendix A1, A2, etc.
The procedure of Lanczos method to estimate the extreme eigenvalues of $M^{-1}A$ matrix\\
\space \\
Initial guess: $\textbf{r}_0$\\
Set $\textbf{s}_0=\textbf{M}^{-1}\textbf{r}_0$;  $\textbf{q}_1 = \textbf{r}_0/({\textbf{r}_0^T\textbf{s}_0})$; $\textbf{q}_0=\textbf{0}$;  $\beta_0 =0$;\quad  $\mu_0 =0$; $T_0=\emptyset$   \\
For $j = 1, 2, \cdots,  m$  do
\begin{enumerate}
\item $\textbf{p}_j = \textbf{M}^{-1}\textbf{q}_j  $
\item $ \textbf{r}_j =\textbf{A}\textbf{p}_j-\beta_{j-1}\textbf{q}_{j-1} $
\item $\alpha_j =\textbf{p}_j^T\textbf{r}_j $
\item $\textbf{r}_j =\textbf{r}_j-\alpha_{j}\textbf{q}_{j} $
\item $\textbf{s}_j = \textbf{M}^{-1}\textbf{r}_j $
\item $\beta_j = \textbf{r}_j^T\textbf{s}_j $
\item $\textbf{if} \quad\beta_j == 0\quad \textbf{then}\quad \textbf{return}  $
\item $\mu_j = max(\mu_{j-1},\alpha_j+\beta_j+\beta_{j-1})  $\label{lan_gersh}
\item $T_j=tri\_diag(T_{j-1},\alpha_j,\beta_j) $\label{lan_tm}
\item $\nu_j = eigs(T_j, 'smallest')  $ \label{lan_nu}
\item $\textbf{if} |\frac{\mu_j}{\mu_{j-1}} -1 | < \epsilon\quad\textbf{and}\quad|1- \frac{\nu_j}{\nu_{j-1}}|< \epsilon \quad\textbf{then} \quad\textbf{return}    $\label{lanczos_converge}
\item $\textbf{q}_{j+1}= \textbf{r}_j/\beta_j $
\end{enumerate}
End Do \\


In step (\ref{lan_tm}), $T$ is a tridiagonal matrix that contains $\alpha_j(j=1,2,...,m)$ as the diagonal entries and $\beta_j (j=1,2,...,m-1)$ as the off-diagonal entries.
\[ T_{m} =  \left[\begin{array}{ccccc}
\alpha_1 & \beta_1 &   &  &   \\
\beta_1 &\alpha_2 &\beta_2    &   &     \\
& \beta_2 &\ddots &\ddots &\\
& & \ddots& \ddots& \beta_{m-1}\\
& &&  \beta_{m-1}& \alpha_m
\end{array} \right]\]

Let $\xi_{min}$ and $\xi_{max}$ be the smallest and largest eigenvalues of $T_m$, respectively. \citet{Paige1980235} demonstrated that
%\begin{equation}
$\nu \le \xi_{min} \le \nu+\delta_1(m)$ and $\mu -\delta_2(m)  \le \xi_{max} \le \mu$.
%\end{equation}
As $m$ increases, $\delta_1(m)$ and $\delta_2(m)$ will gradually close to zero. Thus, the eigenvalue estimation of $M^{-1}A$ is transformed to solve the eigenvalues of $T_m$.
Step \ref{lan_gersh} in eigenvalues estimation employs the Gershgorin circle theorem to estimate the largest eigenvalue of $T_m$, that is,
%\begin{equation}
$\mu = \max_{1 \le i \le m}\sum^m_{j=1}|T_{ij}|=\max_{1 \le i \le m}(\alpha_i +\beta_{i}+\beta_{i-1})$.
%\end{equation}
The efficient QR algorithm ~\citep{ortega1963llt} with a complexity of $\Theta(m)$ is used to estimate the smallest eigenvalue $\nu$ in step (\ref{lan_tm}).

\section{Block preconditioning}
\begin {figure}[!htbp]
\centering
\includegraphics[width=7cm, height=7.0cm]{blockpreconditioning.eps}
\caption[] {Sparsity pattern of the coefficient matrix developed from nine-point stencils.
The whole domain is divided into $3\times3$ non-overlapping blocks.
Elements in red rectangles are coefficients between points in blocks.
Elements in blue rectangles are coefficients between points from the $i$-th block and its neighbor blocks. \label{fig:blockprecond}}
\end{figure}

To the new block EVP preconditioner, we first briefly review a general block preconditioner, as illustrated by Figure \ref{fig:blockprecond}.  If the linear system of $\mathcal{N} \times \mathcal{N}$ grid points is reordered block-by-block with size of $n\times n$ (e.g., $\mathcal{N}/3\times \mathcal{N}/3$ in Figure \ref{fig:blockprecond}), then coefficient matrix $A$ can be represented by a nine-diagonal block matrix. Each row of this matrix contains nine sub-matrices.  Each $B_i$ (red blocks) is a block matrix containing coefficients of the grid points in the i-th block, which share the same structure as $A$ but have a smaller size ($n^2\times n^2$).  $B_i^e$, $B_i^w$, $B_i^n$ and $B_i^s$ are block matrices containing coefficients of points on east, west, north and south boundaries and the points on their respective neighboring blocks, thus having at most $3n$ nonzero elements distributed on $n$ rows. $B_i^{nw}$, $B_i^{ne}$, $B_i^{sw}$ and $B_i^{se}$ have only one nonzero element, representing the coefficient of corner points and their neighboring points on the northwest, northeast, southwest and southeast blocks.  The traditional block preconditioning method constructs the approximate inverse of $A$ by sequentially factorizing it with approximations of $B_i^{-1}$, which is ill-suited for parallel applications.  In contrast, the inverse of the block diagonal of $A$, which provides a good approximation for $A$, can be calculated naturally in parallel.  The inverse of the diagonal block matrices $M$  is
\begin{eqnarray*}
M^{-1}=    \left [
        \begin{array}{ccccccc}
        B_1^{-1} &   &  \\
         & \ddots&  \\
        &   &  B_{m^2}^{-1} \\
    \end{array}
    \right ]
\end{eqnarray*}

Therefore, the preconditioning process $\textbf{x}= M^{-1}\textbf{y}$ is typically transformed into solving the sparse linear equations $B_i \textbf{x}_i = \textbf{y}_i (i=1,2,...,n)$ for each block and solving the equations by LU decomposition, instead of explicitly constructing the inverse of $B$ matrix. The computational complexity of solving these equations with LU decomposition is $\mathcal{O}(n^4)$. With our block Error Vector
Propagation (EVP) preconditioner, the computational complexity of solving the equations $B_i \textbf{x}_i =\textbf{y}_i$ can be reduces to $\mathcal{O}(n^2)$, where the size of $B_i$ is $n^2\times n^2$.

We rewrite the equation (\ref{eq:sten}) into the following form so that we can march the northeastward solution assuming all other neighboring points are exactly known
\begin{align}
\label{eq:evp9p}
\eta_{i+1,j+1} &= (\psi_{i,j} - A_{i,j}^0\eta_{i,j}-A_{i,j}^e\eta_{i+1,j} -A_{i,j}^n\eta_{i,j+1}-A_{i-1,j}^{ne}\eta_{i-1,j+1}  \nonumber\\
&+A_{i-1,j}^e\eta_{i-1,j} -A_{i-1,j-1}^{ne}\eta_{i-1,j-1}-A_{i,j-1}^n\eta_{i,j-1}- A_{i+1,j-1}^{ne}\eta_{i,j-1} )/A_{i,j}^{ne}
\end{align}

The EVP method works as follows. Figure \ref{fig:evp9p} illustrates a Dirichlet boundary elliptic equation $\mathcal{B}\textbf{x} = \psi$ on a small domain.  We define the interior points next to the south and west boundaries as the initial guess points $\textbf{e}$ and those next to the north and east boundaries are the final boundary points $\textbf{f}$ (e.g., $\textbf{e}= \{E_1, \dots, E_7\}$, $\textbf{f}= \{F_1, \dots, F_7\}$).  If the true solution on $\textbf{e}$ is
known, the exact values over the whole domain can be computed sequentially from southwest to northeast corners, using equation
(\ref{eq:evp9p}). This procedure is named as marching. Unfortunately, the value on $\textbf{e}$ is often not known until the elliptic equation is solved.  However, we can get a solution $\textbf{x}$ satisfying the elliptic equation on the whole domain except on the boundary, by first guessing the value
$\textbf{x}|_\textbf{e}$ on $\textbf{e}$ and then calculating the rest using the marching method.  Then $E=(\textbf{x} -\eta)|_\textbf{e}$
and $F=(\textbf{x} -\eta)|_\textbf{f}$ are error vectors on $\textbf{e}$ and $\textbf{f}$, respectively.  The error vector $F$ is already known since $\textbf{f}$  satisfies Dirichlet boundary condition.  The relationship between the error on initial guess points and the final boundary points can be represented as $F=W*E$.  This influence coefficient matrix $W$ can be formed by marching on the whole domain with unit vectors on the initial guess points and zero residual value in the whole domain.


\begin {figure}[!htbp]
\centering
%\includegraphics[width=0.8\linewidth]{evp9pmarch1.png}
\includegraphics[height=6cm]{evp9pmarch1.png}
\caption []{EVP marching method for nine-point stencil. The solution on point $(i+1,j+1)$ can be calculated using the equation on point $(i,j)$, providing solutions on other neighbor points of point $(i,j)$.  \label {fig:evp9p}}
\end {figure}




%%%%%%%%%%%%%%%%%%%%%%%%%%%%%%%%%%%%%%%%%%%%%%%%%%%%%%%%%%%%%%%%%%%
\begin{acknowledgements}
This work is supported in part by a grant from the National Natural Science Foundation of China (41375102), and the National Grand Fundamental Research 973 Program of China (No. 2014CB347800). Computing resources were provided by the Climate Simulation Laboratory at NCAR's Computational and Information Systems Laboratory (sponsored by the NSF and other agencies) and the National Energy Research Scientific Computing Center, a DOE Office of Science User Facility supported by the Office of Science of the U.S. Department of Energy under Contract No. DE-AC02-05CH11231.
\end{acknowledgements}
\bibliographystyle{copernicus}
\bibliography{gmd_cesmpop_bib}


%%%%%%%%%%%%%%%%%%%%%%%%% figures %%%%%%%%%%%%%%%%%%%%%%%%%%%%%%
%
%
\end{document}

