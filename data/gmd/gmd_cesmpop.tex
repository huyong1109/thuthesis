% This is LLNCS.DEM the demonstration file of
% the LaTeX macro package from Springer-Verlag
% for Lecture Notes in Computer Science,
% version 2.4 for LaTeX2e as of 16. April 2010s
%
%\documentclass[journal abbreviation, manuscript]{copernicus}
\documentclass[gmd, manuscript]{copernicus}
%
\usepackage{amsmath}
\usepackage{epstopdf}
\DeclareGraphicsExtensions{.eps,.mps,.pdf,.jpg,.PNG}
\DeclareGraphicsRule{*}{pdf}{*}{}
\graphicspath{{./fig/}}

\begin{document}

\linenumbers

%%\title{Scalable Barotropic Solver for High-resolution Ocean Models}
\title{Accelerating the High-resolution Ocean Model Component in the Community Earth System Model}

\author[1,2]{Xiaomeng Huang}
\author[1]{Yong Hu}
\author[3]{Yuheng Tseng}
\author[3]{Allison H. Baker}
\author[3]{Frank O. Bryan}
\author[3]{John Dennis}
\author[1]{Haohuan Fu}
\author[1]{Guangwen Yang}

\affil[1]{Ministry of Education Key Laboratory for Earth System Modeling, and Center for Earth System Science, Tsinghua University, Beijing, 100084, China}
\affil[2]{Laboratory for Regional Oceanography and Numerical Modeling, Qingdao National Laboratory for Marine Science and Technology, Qingdao, 266237, China}
\affil[3]{The National Center for Atmospheric Research, Boulder, CO, USA}
%% The [] brackets identify the author to the corresponding affiliation, 1, 2, 3, etc. should be inserted.

\runningtitle{Accelerating the High-resolution ocean model component in the Community Earth System Model}

\runningauthor{X.M. Huang et al. }

\correspondence{Xiaomeng Huang (hxm@tsinghua.edu.cn), Yuheng Tseng(ytseng@ucar.edu)}
\received{}
\pubdiscuss{} %% only important for two-stage journals

\revised{}
\accepted{}
\published{}
%% These dates will be inserted by the Publication Production Office during the typesetting process.
\firstpage{1}

%title

\maketitle

\begin{abstract}
In the Community Earth System Model (CESM), the ocean model component is computationally expensive for high-resolution grids and is frequently the least scalable component of CESM for certain production experiments. The key problem is that the modified Preconditioned Conjugate Gradient (PCG) used to solve the barotropic mode scales poorly at the high core counts. We design a fast numerical solver to accelerate the high-resolution ocean simulation. The novel solver adopts a Chebyshev type iterative method to reduce the global communication cost in conjunction with an effective block preconditioner to reduce the iterations further. Comparing with the two origin solvers, it significantly reduces the global communication time and realizes a competitive convergence rate. The experimental results show that the simulation speed of the improved ocean model component with 0.1{\degree}  resolution achieves 10.5 simulated years per wall-clock day on 16,875 cores.
\end{abstract}

\section{Introduction} \label{se:int}
% no \IEEEPARstart
High-resolution global climate models have become increasingly important
in recent years as a means for understanding climate variability
and projecting future climate change.  The Community Earth System
Model (CESM), whose development is centered at the National Center for
Atmospheric Research (NCAR), is one of the most widely used global
climate models, and its climate projections are a key component in the
Intergovernmental Panel on Climate Change (IPCC) Fifth Assessment
Report (AR5) \cite{stocker2013ipcc}.

% CESM is a fully-coupled climate model system, including atmosphere,
% ocean, sea-ice and land components. \hl{In particular, the ocean component
% requires the modeling of broad spatial and temporal scales in order to
% encompass the relevant dynamics where the ocean energy dominates.  The
% spatial scales range from tens of meters of the sub-mesoscale eddies
% to dimensions of the ocean basins, and the temporal scales range from
% the order of days for gravity waves to centuries for slow planetary
% waves.}  As a result, ocean models have to iterate for a longer time at
% a finer resolution, which makes them computationally intensive
% \cite{Worley:2011:PCE:2063384.2063457,dennis2012computational}.
% However, many recent studies demonstrate that high-resolution ocean
% models are required in order to produce more realistic and accurate
% predictions
% \cite{bryan2010frontal,mcclean2011prototype,graham2014importance}.

CESM is a fully-coupled climate system model, including atmosphere,
ocean, sea-ice and land components. In particular, the ocean component
is required to represent processes across a broad range of spatial and
temporal scales relevant to climate science. Ocean mesoscale eddies
have spatial scales of $\mathcal{O}$(10 - 100 km), one to two orders of magnitude
smaller than the dynamically analogous weather systems in the
atmosphere. The adjustment time scale for the deep ocean is many
centuries up to a few millennia, again orders of magnitude longer than
the corresponding timescales in the atmosphere. The computational
burden of a global eddy-resolving ocean climate model \cite{bryan2010frontal,mcclean2011prototype,graham2014importance} is
thus increased over that for an atmosphere model by the demand for
finer spatial resolution and longer integration times.

Moreover, climate model simulations are often run for decades or
even centuries, but these long-term simulations are typically too
computationally expensive to run at high-resolution.  For example,
most CESM simulations in IPCC AR5 are carried out with a nominal
1\degree\space ocean and a 1\degree\space to 2\degree\space atmosphere model.
Recent increases in both supercomputing resources and
high-resolution satellite observations have motivated
efforts to improve the parallel performance of high-resolution climate
models so that they can be run more routinely (and for less cost).

%Recently, the increasing computational power of supercomputers and
%high-resolution satellite observations have inspired much research
%that focuses on adapting high-resolution climate models for massive
%parallelism.
%As noted, in most production simulations, the least scalable
%component of CESM is the ocean model Parallel Ocean Model (POP)
%\cite{dennis2012computational}.

The Parallel Ocean Model (POP) component of CESM solves the
three-dimensional primitive equations with hydrostatic and Boussinesq
approximations \cite{smith2010parallel} and divides the time
integration into two parts: the baroclinic and the
barotropic modes. The baroclinic mode describes the 
three-dimensional dynamic and thermodynamics processes, and the barotropic
mode solves the vertically-integrated momentum and continuity
equations in two dimensions. The implicit free-surface method is a common choice
in barotropic mode in ocean models because it allows a large time step to
efficiently compute the fast gravity mode.  However, this method
requires solving a large elliptic system of equations, which
scales poorly in POP.  In fact, the poor scaling performance of the
barotropic solver in POP, which is dominated by the communication
overhead \cite{Worley:2011:PCE:2063384.2063457}, is well known, and
its optimization will benefit the entire CESM model
\cite{dennis2012computational}.

The currently recommended linear solver for the barotropic mode in CESM
POP is the Chronopoulos-Gear (ChronGear) method
\cite{dAzevedo1999lapack}, a modified Preconditioned Conjugate
Gradient method (PCG), combined with a diagonal preconditioner.
The required global reduction in the ChronGear method does not scale well and
causes a bottleneck for high-resolution simulations.  To improve the scaling
of POP, and, therefore CESM, we focus on optimizing the barotropic
solver by eliminating global reductions and developing a more
effective preconditioner.  In particular, we make the following
contributions:


%MOVE TO RELATED WORK
%There are currently different alternatives to mitigate the poor behavior of the PCG type of solver in the massive parallelization.
%Some approaches attempt to overlap the communication with computation time\cite{beare1997optimisation}.
%Some use the land elimination and load-balance strategies \cite{dennis2007inverse, dennis2008scaling}
%to reduce the number of processes and the associated overhead of global reduction.
%Although these approaches may improve performance, they do not eliminate the major bottleneck of the global reduction.


\begin{itemize}
\item We develop a new block parallel preconditioner based on the
Error Vector Propagation (EVP) method \cite{roache1995elliptic} designed to
improve solver convergence in the POP barotropic mode.
\item We add a preconditioning interface to the Classical Stiefel Iteration
(CSI) solver explored in \cite{hu2013scalable} and
implement the resulting preconditioned CSI (P-CSI) solver
and new EVP block preconditioner in CESM1.2.0 POP.
\item We demonstrate an improvement in convergence rate for both ChronGear and
P-CSI when using block EVP.
\item We obtain a 5.2x speedup of
  the barotropic mode in high-resolution POP due to the improved scalability
of P-CSI with block EVP preconditioning, greatly improving the
scalability of POP (and ultimately CESM) at large core counts.
\item We develop and apply an ensemble-based statistical method to evaluate the impact
of changing the linear solver in POP and ensure that a consistent ocean climate is produced.
\end{itemize}

The remainder of this paper is organized as follows.
Section \ref{se:baro} discusses POP's barotropic solver and its scalability.
Sections \ref{se:psi} and \ref{se:evp} detail the design
of P-CSI for POP and the development of the block EVP preconditioner.
Section \ref{se:exp} compares the scalability of the ChronGear and
P-CSI solvers.  Section \ref{se:ver} verifies the new P-CSI solver
using the ensemble based statistical method.
Finally,  related work and conclusions are presented in Sections
\ref{se:rel} and \ref{se:conc}, respectively.

%{\textbf better to move related work in the next section.}

% You must have at least 2 lines in the paragraph with the drop letter
% (should never be an issue)

\input{barotropic_solver}
%----------------------------------------------------------------------------
\section{Design of the P-CSI Solver} \label{se:pcsi}
%----------------------------------------------------------------------------
The PCG  solver usually are faster than the stationary iterative methods such as Jacobi method, Gauss-Seidel method and the Successive over-relaxation method, because of its fast convergence speed \citep{golub2012matrix}.
However, PCG requires  inner products to determine the step size, which could take much time on large-scale parallelism.
Although ChronGear mitigates the bottleneck of global communications, its performance is still limited by the inherent poor data locality and the remaining global communication operations. To accelerate the solving the linear system in ocean model component, we need to reduce the global communications as much as possible, as well as keep a close convergence rate as PCG.

\subsection{Classical Stiefel iterative method}
The CSI is a special kind of Chebyshev Iterative methods \citep{stiefel1958kernel}, which was revisited by \citeauthor{gutknecht2002chebyshev} (\citeyear{gutknecht2002chebyshev})
and identified to be ideal for massively parallel computers. As early as 1985, Saad et al.~\citeyear{saad1985solving} proposed a generalization of CSI on linearly connected processors and claimed that this approach performs better than the conjugate gradient method when the eigenvalues are known.

In the procedure of P-CSI (details of P-CSI is presented in \ref{algorithm}), the iteration parameters which control the searching directions in iteration steps,
are derived from a stretched Chebyshev function of two extreme eigenvalues.
As a result, this approach requires no inner product operation, potentially avoiding the bottleneck of global reduction (see the workflow of ChronGear and P-CSI in Figure \ref{fig:pcsi_pcg}).
It is clear to see that the P-CSI is more scalable than ChronGear on massively parallel architectures. However, it requires a priori knowledge about the spectrum of coefficient matrix $A$ \citep{gutknecht2002chebyshev}. It is well-known that obtaining the eigenvalues is equivalent to solving a linear system of equation.
Fortunately, the coefficient matrix $A$ and its preconditioned form in the POP are both real symmetric, positive definite matrices. Approximate estimation of the largest and smallest eigenvalues, $\mu$ and $\nu$, respectively, of the preconditioned coefficient matrix is sufficient to ensure the convergence of P-CSI.

To get the eigenvalues estimations, we adopt the efficient Lanczos method ~\citep{Paige1980235} to estimate the extreme eigenvalues.
Since preconditioning is used in our case, we need to estimate the extreme eigenvalues of $M^{-1}A$ matrix.
The procedure of eigenvalues estimations presented in Appendix \ref{algorithm}.
Tests indicate that only a small number of Lanczos steps are necessary to generate eigenvalue estimates of $M^{-1}A$ that result in near-optimal P-CSI convergence. That means that the computational overhead of eigenvalue estimates  is very small in our case.

\begin {figure}[!htbp]
\begin{center}
\includegraphics[width=12cm,height=6cm]{pcsi_pcg.png}
\caption []{Workflow of ChronGear and P-CSI iterations when four processes are used. \label{fig:pcsi_pcg}}
\end{center}
\end {figure}


\subsection{A block EVP preconditioner} \label{se:evp}
Block preconditioning is an promising parallel preconditioner in POP because it makes use of the block structure of the coefficient matrix that arises from discretization of the elliptic equations.
A block preconditioning based on the EVP method is proposed and detailed in \citet{hu2015improving} to improve the parallel performance of the barotropic solver in the POP.
To the best of our knowledge, the EVP approach and its variants are one of the least costly algorithms for solving elliptic equation in serial \citep{roache1995elliptic},
which have been used in several different Ocean models \citep{dietrich1987ocean,sheng1998candie,tseng2011parallel}. The parallel EVP solver is also implemented in \citet{tseng2011parallel}.
Theoretically, the EVP approach is a direct solver, which requires two steps: preprocessing and solving.
In the preprocessing step, the influence coefficient matrix and its inverse are computed, involving a computational complexity of $\mathcal{C}_{pre} = (2n-5)* 9n^2 + (2n-5)^3 = \mathcal {O} (26n^3)$, which is intensive but only needs to computed once at the beginning.
Obtaining the solution in each solving step is inexpensive and requires only $\mathcal{C}_{evp}= 2* 9n^2 + (2n-5)^2 = \mathcal{O} (22n^2)$ \citep{hu2015improving}.
This indicates that the EVP has lower computational cost for the solver step than other direct solvers such as LU.

The EVP method is an efficient option for solving elliptic equations. However, a major drawback of the standard EVP is that, without applying additional modifications, it cannot be used on a large domain due to its error propagation and numerical instabilities in the marching process \citep{roache1995elliptic}.
The fact that the EVP is not well-suited for large domains indeed poses no issue for large-scale parallel computing, where a larger number of processors results in smaller domains.
The serial disadvantage becomes an advantage in parallel computing. On a small block up to the size of $12\times 12$, EVP solves with an acceptable round-off error of $\mathcal{O}(10^{-8})$ when double-precision floating-point is used.
This can even make the EVP ideal for parallel block preconditioning on a large number of cores.

Although the EVP preconditioning doubles the computation in each iteration, it halves both global and boundary communications which dominate in the barotropic execution time at the high core counts. This advantage will be tested and verified in Section \ref{sec:exp-preconditioner}.
The implementation of EVP preconditioning significantly reduces the number of iterations for both the ChronGear and P-CSI solvers \citep{hu2015improving}.

\section{Algorithm analysis and comparison}\label{se:Algorithm}
Besides P-CSI, the convergence of PCG and ChronGear methods also rely on the two extreme eigenvalues of the coefficient matrix.
Here, we first uncover the characteristics of the eigenvalues which affect the convergence of solvers, and then present the theoretical analysis on computational complexity and convergence rate of the ChronGear and P-CSI solvers.

\subsection{Spectrum and Condition Number }
The coefficient matrix $A$ in POP is symmetric positive-definite \citep{smith2010parallel}, thus its eigenvalues are positive real numbers \citep{stewart1976positive}. Assume that the spectrum \citep{golub2012matrix} of A is $\mathcal{S} = \{\lambda_1, \lambda_2, \cdots, \lambda_N\}$, where  $\lambda_{min} = \lambda_1 \le \lambda_i \le \lambda_\mathcal{N} = \lambda_{max}$( $1<i <\mathcal{N}$, where $\mathcal{N}$ is the size of $A$ ) are eigenvalues of $A$.
%Definition of the coefficients in \ref{defineA} shows that
% \begin{align}
% A_{i,j}^{O} = \sum_{\chi \in \{NW,NE,SW,SE,W,E,N,S\}}A_{i,j}^\chi
% \end{align}
Applying the  Gershgorin circle theorem \citep{bell1965gershgorin}, we know that for any $\lambda \in \mathcal{S}$, there exits a pair $(i,j)$ satisfying
\begin{align}
&|\lambda -  (A_{i,j}^O + \phi ) | \le \sum_{\chi \in \{NW,NE,SW,SE,W,E,N,S\}}|A_{i,j}^\chi|
\end{align}
With the definition of the coefficients in \ref{defineA}, we get
\begin{align} \label{eigsGersh}
&\lambda_{max} \le  \max (  5\alpha - \frac{1}{\alpha}, \frac{5}{\alpha}- \alpha) +\phi   \\
&\lambda_{min} \ge 2\min (  \alpha - \frac{1}{\alpha},\frac{1} {\alpha} -  \alpha) + \phi
\end{align}

\begin {figure}[!htbp]
\centering
\includegraphics[height=6.5cm]{conditionNumberAspectRatio}
\caption[] {Relationship between aspect ratio and the condition number of the coefficient matrix.\label{fig:conditionNumberRatio}}
\end{figure}
\begin {figure}[!htbp]
\centering
\includegraphics[height=6.5cm]{conditionNumberTimestep}
\caption[] {Relationship between time step size and the condition number of the coefficient matrix.\label{fig:conditionNumberDt}}
\end{figure}

\begin {figure}[!htbp]
\centering
\includegraphics[height=6.5cm]{conditionNumberGridSize}
\caption[] {Relationship between the number of grid points and the condition number of the coefficient matrix.\label{fig:conditionNumbGrid}}
\end{figure}
It tells that when the aspect ratio becomes closer to 1, the upper bound of the largest eigenvalue decreases as , while the lower bound of the smallest increases.
That is,  the spectrum radius ($[\lambda_{min}, \lambda_{max}]$) decreases as the aspect ratio becomes closer to 1.
When the aspect ratio of horizontal grid size is equal to one, that is $ \alpha = \frac{ \Delta y}{ \Delta x} = 1$, we get
$\lambda_{max} \le  4 +\phi$,$\lambda_{min} \ge   \phi$.
Then, the condition number (e.g. $\kappa=  \lambda_{max}/\lambda_{min}$), which is determined by the spectrum radius, also decreases when the aspect ratio becomes closer to 1..
This conclusion is supported in Figure \ref{fig:conditionNumberRatio}. Figure \ref{fig:conditionNumberRatio} shows that with constant grid points $\mathcal{N} = 20\times 20$ and a constant $\phi = 0$,  the condition number of the coefficient matrix increases when the aspect ratio increases in the interval $(1, +\infty)$ or decrease in the interval $(0,1)$, and it reaches the minimum when the aspect ratio is 1.

The lower bound of eigenvalues are determined by  $\phi=\frac{S }{g \tau^2 H}$, which is a factor of time step and the ratio between horizontal grid size and the ocean depth. It shows that the lower bound of the eigenvalues decreases when the time step increases. Correspondingly, the condition number of the coefficient matrix increases. This deduction is demonstrated in Figure \ref{fig:conditionNumberDt}, in which experiments are configured with  a constant grid size  $\mathcal{N} = 20\times 20$  and a constant aspect ratio $\Delta x /{\Delta y} = 1$.


Regardless of the time step (e.g. assume that $\phi=0$), the analysis above shows that the spectrum radius is confined in $(0,4)$ when the aspect ratio is 1, no matter what the grid size is.
However, the condition number of the coefficient matrix varies in a large range because the smallest eigenvalue becomes closer to zero when the grid size $\mathcal{N}$ increases. Figure \ref{fig:conditionNumbGrid} explains the relationship between condition numbers and grid sizes.


\subsection{Convergence rate} \label{convergence_rate}

The convergence rate of both PCG and ChronGear relies heavily on the condition number of the coefficient matrix $A$.
\cite{dAzevedo1999lapack} demonstrated that ChronGear has the same theoretical convergence rate as PCG.
In both PCG and ChronGear, the error in each iteration has an upper bound \citep{Liesen2004}
\begin{equation}
\frac{||\textbf{x}_k-\textbf{x}^*||_A }{||\textbf{x}_0-\textbf{x}^*||_A}  \le \min_{p\in \mathcal{P}_k, p(0) = 1 }\max_{\lambda \in \mathcal{S}} |p(\lambda)| \label{PcgConvergeRate}
\end{equation}
where $\textbf{x}_k$ is the solution vector after the k-th iteration, $\textbf{x}^*$ is the solution of the linear equations (that is $\textbf{x}^* = A^{-1}b$), $\lambda$ represents an eigenvalue of $A$.
Applying the Chebyshev polynomials of the first kind to estimate this min-max approximation, we get
\begin{align}
\label{chrongear_convergence}
||\textbf{x}_k-\textbf{x}^*||_A \le  2 (\frac{\sqrt{\kappa}-1}{\sqrt{\kappa}+1})^k ||\textbf{x}_0-\textbf{x}^*||_A
\end{align}
where   $\kappa =  \kappa_2(A)$ is the condition number of the matrix $A$.

Equation \ref{chrongear_convergence} indicates that the theoretical bound of convergence rate of PCG increases with the condition number.
PCG converges faster for well-conditioned matrix (e.g. matrix with small condition number) than ill-conditioned matrix.
%The definition indicates that the convergence rate is actually determined by two extreme eigenvalues of the coefficient matrix.
The actual convergence of PCG and ChronGear in the 0.1\degree\space ocean model component  is presented in Figure \ref{fig:convergence_diag}. As it shows, the relative residual decreases as the iteration number increases. To get a solution with a $10^{-13}$ relative error, about one hundred iterations is needed.

Besides the elimination of global reductions, another feature of P-CSI is the fast convergence rate which has the same order with the one of PCG.
The convergence rate of P-CSI Algorithm in the residual form satisfies
\begin{equation}
\textbf{r}_k = P_k(A)\textbf{r}_0 \label{eq:rPjr0}
\end{equation}
where
$P_k(\zeta) = \frac{\tau_k(\beta-\alpha \zeta)}{\tau_k(\beta)}$ for $ \zeta \in [\nu, \mu]$ ~\citep{stiefel1958kernel} .
$\tau_k(\xi)$ is a Chebyshev polynomial expressed as
\begin{equation}
\tau_k(\xi) =   \frac{1}{2}[(\xi+\sqrt{\xi^2-1})^k+(\xi+\sqrt{\xi^2-1})^{-k}]
\end{equation}
When $ \xi \in [-1,1]$, the Chebyshev polynomial has an equivalent form $\tau_k(\xi) = cos(k\cos^{-1} \xi)$, which clearly shows that $|\tau_k(\xi)| \le 1$ when $| \xi | \le 1$. $P_k(\zeta)$ is the polynomial satisfying that
\begin{equation}
P_k = \min_{p\in \mathcal{P}_k, p(0) = 1 }\max_{\zeta \in [\nu,\mu]} |p(\zeta)|
\end{equation}
%which is the theoretical bound of the convergence rate  in PCG \ref{PcgConvergeRate}.

Assume that $A= Q^T\Lambda Q$, where $\Lambda$ is a diagonal matrix having the eigenvalues of $A$ on the diagonal, and $Q$ is a real orthogonal matrix with the columns which are eigenvectors of $A$.
Then we have
\begin{equation}
P_k(A) = Q^T P_k(\Lambda)Q = Q^T \left [\begin{array}{cccc}
P_k(\lambda_1) & & &\\
& P_k(\lambda_2) & &\\
& & \ddots &\\
 & & & P_k(\lambda_N)
\end{array} \right ] Q \label{eq:PjMA}
\end{equation}
Assume that the estimated largest and smallest extreme eigenvalues of coefficient matrix $\nu$ and $\mu$ satisfies $0 < \nu \le \lambda_i \le \mu$ ($i = 1, 2, \cdots, \mathcal{N}$), then $|\beta - \alpha \lambda_i| \le 1$, $|P_k(\lambda_i)| \le \tau^{-1}_k (\beta)$.
Equation \ref{eq:rPjr0} and \ref{eq:PjMA} indicate that
\begin{equation}
\label{pcsi_convergence}
\frac{||\textbf{r}_k||_2}{||\textbf{r}_0||_2}  \le  \tau_k^{-1}(\beta) = \frac{2(\beta+\sqrt{\beta^2-1})^k}{1+(\beta+\sqrt{\beta^2-1})^{2k}} \le 2(\frac{\sqrt{\kappa'}-1}{\sqrt{\kappa'}+1})^k
\end{equation}
where $\kappa' = \frac{\mu}{\nu}$.
It shows that P-CSI has the same theoretical upper bound of convergence rate as PCG, when the eigenvalues estimation is appropriate (e.g. $\kappa' =\kappa$) .

The analysis above still works for cases when a nontrivial preconditioning is used.
The only difference is that the coefficient matrix is $M^{-1}A$ instead of $A$.
It is worth mentioning that the preconditioned matrix in the PCG, ChronGear and P-CSI algorithms is actually  $M^{-1/2}A(E^{-1/2})^T$, which is symmetric and has the same set of eigenvalues as $M^{-1}A$ \citep{Shewchuk1994}. Thus the condition number of the preconditioned matrix is $\kappa =  \kappa_2(M^{-1/2}A(E^{-1/2})^T)$, which is usually smaller than the condition number of $A$.
The closer $M$ is to $A$, the smaller the condition number of $M^{-1}A$. When $M = A$, then $\kappa_2(M^{-1 }A ) = 1$.

When computing the ocean model component with small core counts, global reductions are not an issue so that the computing performance of P-CSI and ChronGear are close. When computing the high-resolution ocean model component with large core counts, P-CSI should be significantly faster than ChronGear per iteration because global reduction will become an obvious bottleneck in conjugate gradient method.


\subsection{Computational complexity}  \label{subse:complex}
Assume that $p$ is the number of processes and $\mathcal{N} $ is the number of grid points; following similar definition in \citet{hu2015improving}.
The computational complexity of p-CSI is analyzed here and compared with ChronGear.

For each ChronGear solver iteration, computation includes matrix-vector multiplication and vector-vector multiplication. Its complexity $\mathcal{T}_c = \mathcal{O}(15\frac{\mathcal{N}}{p})$, which decreases and has a lower limit of zero when the number of processes increases. Boundary updating occurs in the halo regions for each process, after operations like matrix-vector multiplication and non-diagonal preconditioning, which requires one or more boundary layers. The actual time depends on the network delay and the volume of the halo regions.  With a halo size of $2$, the volume in each boundary is $2\sqrt{\frac{\mathcal{N}}{p}}$ and decreases as the number of processes increases. Boundary communication complexity is $\mathcal{T}_b =\mathcal{O}(4\alpha + 8\sqrt{\frac{\mathcal{N}}{p}})$, where $\alpha$ is point-to-point communication latency per message.
It shows that boundary updating time also decreases as the number of processes increases but has a lower bound.
Global communication contains only one global reduction per iteration which contains a MPI\_allreduce and a masking operation to exclude land points. The cost of the masking operation should decrease with the number of processes $p$ while the cost of the MPI\_allreduce should monotonically increase, thus the global
reduction complexity satisfies $\mathcal{T}_g= \mathcal{O}(2\frac{\mathcal{N}}{p} + \log p )$.

%The expression $\mathcal{T}_g$ should therefore initially decrease followed by a monotonic increase with process count.
%Note that the global reduction has virtually no data exchange since there are only two numbers from each process.
%Let $T_0$ be the time unit of one floating-point operation and $B$ be the number of floating-point numbers transmitted by the network per second from process to process.
%Provided that the processor frequency and network bandwidth are $S_{cpu}$ and $B_{net}$, and that their efficiencies are $R_{cpu}$ and $R_{net}$, then $T_0 = R_{cpu} S_{cpu}^{-1}$, and $B = \frac{1}{8}R_{net}B_{net}$.
Combining all three components, the execution time of one diagonal preconditioned ChronGear solver step can be expressed as:
\begin{eqnarray}
\label{t_pcg}
\mathcal{T}_{cg}=\mathcal{K}_{cg} (\mathcal{T}_c + \mathcal{T}_b+\mathcal{T}_g ) = \mathcal{O}(\mathcal{K}_{cg} (17\frac{\mathcal{N}}{p} +8\sqrt{\frac{\mathcal{N} }{p}}+ 4\alpha +\log p))
%\end{tabular}
\end{eqnarray}
where $\mathcal{K}_{cg}$ is the number of iterations,
which does not change with the number of processes \citep{hu2013scalable}. Equation (\ref{t_pcg}) shows that the time required for computation and boundary updating decreases as the number of processes increases. But the time required for the global reduction increases with increasing numbers of processes. Therefore, we expect the execution time of the ChronGear solver to increase when the number of processors exceeds a certain threshold.

Applying the same analytical method , we get the computational complexity of P-CSI as
\begin{eqnarray}
\label{t_pcsi}
\mathcal{T}_{pcsi}=\mathcal{O}(\mathcal{K}_{pcsi} (15\frac{\mathcal{N}}{p} +8\sqrt{\frac{\mathcal{N} }{p}} + 4\alpha))
\end{eqnarray}
where $K_{pcsi}$ is the number of iterations. The comparison of the complexities of ChronGear and P-CSI shows that P-CSI has a lower time computational complexity than ChronGear, because it does not contain the $\log p$ term inherited with global communications.

In the case with the EVP preconditioning, the computational complexity becomes $\mathcal{T}_c = 37 \frac{\mathcal{N}}{p}$. Thus the computational complexity of P-CSI with EVP preconditioning is
\begin{eqnarray}
\label{t_pcsiEvp}
\mathcal{T}_{pcsi-evp}=\mathcal{O}(\mathcal{K}_{pcsi-evp} (37\frac{\mathcal{N}}{p} +8\sqrt{\frac{\mathcal{N} }{p}} + 4\alpha))
\end{eqnarray}
It shows that the computing time in each iteration doubles with the EVP preconditioning.
However, the total time may still decreases. With EVP preconditioning, the iteration number $\mathcal{K}_{pcsi-evp}$ decreases to almost one half (see Figure \ref{fig:convergence_diag}). As a result, the total number of communications, which is the most time-consuming part on large number of cores, when the computation time becomes relatively small, decreases to about one half.



\subsection{特征值谱和条件数}
\label{solver:Algorithm:condition}

POP中的系数矩阵 $A$ 是正定对称的\cite{smith2010parallel},因此它的特征是都是正实数\cite{stewart1976positive}。
假设系数举证的特征值谱\cite{golub2012matrix} 是 $\mathcal{S} = \{\lambda_1, \lambda_2, \cdots, \lambda_N\}$,这里 $\lambda_{min} = \lambda_1 \le \lambda_i \le \lambda_\mathcal{N} = \lambda_{max}$( $1<i <\mathcal{N}$, $\aleph$ 为 $A$的大小 )表示$A$的特征值.
 
利用 Gershgorin圆盘定理\cite{bell1965gershgorin}可知,对于任意的 $\lambda \in \mathcal{S}$,都存在一个数对 $(i,j)$ 满足
\begin{align}
&|\lambda -  (A_{i,j}^O + \phi ) | \le \sum_{\chi \in \{NW,NE,SW,SE,W,E,N,S\}}|A_{i,j}^\chi|
\end{align}
由方程\ref{defineA}给出的系数的定义,我们可以得到如下结论 
\begin{align} \label{eigsGersh}
&\lambda_{max} \le  \max (  5\alpha - \frac{1}{\alpha}, \frac{5}{\alpha}- \alpha) +\phi   \\
&\lambda_{min} \ge 2\min (  \alpha - \frac{1}{\alpha},\frac{1} {\alpha} -  \alpha) + \phi
\end{align}

\begin {figure} 
\centering
\includegraphics[height=6.5cm]{conditionNumberAspectRatio}
\caption[] {网格横纵比例和系数矩阵条件数之间的关系。 \label{fig:conditionNumberRatio}}
\end{figure}
\begin {figure} 
\centering
\includegraphics[height=6.5cm]{conditionNumberTimestep}
\caption[] {时间步长和系数矩阵条件数之间的关系。 \label{fig:conditionNumberDt}}
\end{figure}

\begin {figure}
\centering
\includegraphics[height=6.5cm]{conditionNumberGridSize}
\caption[] {网格点的数目和系数矩阵条件数之间的关系。 \label{fig:conditionNumbGrid}}
\end{figure}
 
这个结论说明,当网格横纵比例越接近于1时,最大特诊值的上界会随之减少,而最小特征值的下界会随之变大。
也就是说,系数矩阵特征值谱的半径($[\lambda_{min}, \lambda_{max}]$)随着网格横纵比向着1的接近而变小。 
当水平网格的横纵比等于1时,即$ \alpha = \frac{ \Delta y}{ \Delta x} = 1$,我们可以得到$\lambda_{max} \le  4 +\phi$,$\lambda_{min} \ge   \phi$。
这时,系数矩阵的条件数(即 $\kappa=  \lambda_{max}/\lambda_{min}$),由于是由特征值谱的半径的所决定的,也会随着网格横纵比向1的靠近而变小。 
图\ref{fig:conditionNumberRatio}所给出的结果验证了我们的这个结论。图\ref{fig:conditionNumberRatio}中给出的是固定的网格点数$\mathcal{N} = 20\times 20$和固定的$\phi = 0$, 在区间$(1, +\infty)$上,系数矩阵的条件数随着网格横纵比例的增加而增加,而在区间$(0,1)$上,系数矩阵的条件数随着横纵比例的增大而变小。
特征值在当网格横纵比例等于1时取到最小值。
 
特征值的下界是$\phi=\frac{S }{g \tau^2 H}$,这个变量是由时间步长以及水平网格的面积和海底深度的比例确定的。
它表明,特征值的下界会随着时间步长的增大而减小。 
相应的,系数举证的 条件数也就随之增大。
图\ref{fig:conditionNumberDt}中证明了我们的这个推论。
这个图中的实验采用的是固定的网格点数$\mathcal{N} = 20\times 20$,和固定的网格横纵比$\Delta x /{\Delta y} = 1$。 


 
不考虑时间步长的因素(假设$\phi=0$)时,上面的分析表明,当网格的横纵比等于1时,不论网格点的个数是多少, 系数矩阵的特征值谱半径限制在 $(0,4)$ 区间内。
但是,系数矩阵的条件数的变化却可能会很大,因为当网格点的数量$\mathcal{N}$ 不断增加时,最小的特征值会组件的逼近零。 
图\ref{fig:conditionNumbGrid} 中解释了条件数和网格点数目之间的关系。 

 


\subsection{收敛速度} \label{solver:Algorithm:convergence_rate}

 PCG和ChronGear方法的收敛速度都依赖于系数矩阵$A$的条件数。  
论文\inlinecite{dAzevedo1999lapack}证明了 ChronGear  和PCG在理论上收敛速度是相同的。 
在PCG和ChronGear方法中,每一步迭代的残差都有一个上界\cite{Liesen2004}
\begin{equation}
\frac{||\textbf{x}_k-\textbf{x}^*||_A }{||\textbf{x}_0-\textbf{x}^*||_A}  \le \min_{p\in \mathcal{P}_k, p(0) = 1 }\max_{\lambda \in \mathcal{S}} |p(\lambda)| \label{PcgConvergeRate}
\end{equation}
这里$\textbf{x}_k$ 表示第$k$步迭代后得到的解向量, $\textbf{x}^*$ 表示线性方程组的解析解  (即$\textbf{x}^* = A^{-1}b$),$\lambda$ 表示系数矩阵 $A$的特征值。
利用 第一类Chebyshev 多项式来估计这个最小最大值,可以得到 
\begin{align}
\label{chrongear_convergence}
||\textbf{x}_k-\textbf{x}^*||_A \le  2 (\frac{\sqrt{\kappa}-1}{\sqrt{\kappa}+1})^k ||\textbf{x}_0-\textbf{x}^*||_A
\end{align}
这里  $\kappa =  \kappa_2(A)$ 表示系数矩阵 $A$的条件数。


方程\ref{chrongear_convergence}表明,PCG方法收敛速度的理论上界取决于系数矩阵的条件数。 
PCG方法在求解良态的矩阵(特征值的条件数很小)时比求解病态的矩阵(特征值的条件数很大)时收敛速度更快。 
图\ref{fig:convergence_diag}给出了PCG和ChronGear方法在0.1度海洋模式中的真实的收敛速度。
如这幅图所示,相对残差随着迭代步数的增加而减少。
总共需要一百步左右的迭代步才能够得到一个相对残差为$10^{-13}$的解。
 

 
除了消除了全局归约操作,P-CSI方法的更一个优点是它有和PCG相同量级的收敛速度。 
P-CSI方法的收敛速度可以用残差的形式表示为
\begin{equation}
\textbf{r}_k = P_k(A)\textbf{r}_0 \label{eq:rPjr0}
\end{equation}
这里
$P_k(\zeta) = \frac{\tau_k(\beta-\alpha \zeta)}{\tau_k(\beta)}$ for $ \zeta \in [\nu, \mu]$ ~\cite{stiefel1958kernel} .
$\tau_k(\xi)$ 是Chebyshev多项式,它的表达式为  
\begin{equation}
\tau_k(\xi) =   \frac{1}{2}[(\xi+\sqrt{\xi^2-1})^k+(\xi+\sqrt{\xi^2-1})^{-k}]
\end{equation}
当 $ \xi \in [-1,1]$时, Chebyshev多项式的另一个等价表达式为$\tau_k(\xi) = cos(k\cos^{-1} \xi)$。从这个表达式可以清楚的看出来,当$| \xi | \le 1$时, $|\tau_k(\xi)| \le 1$。$P_k(\zeta)$ 是一个多项式,满足如下形式
\begin{equation}
P_k = \min_{p\in \mathcal{P}_k, p(0) = 1 }\max_{\zeta \in [\nu,\mu]} |p(\zeta)|
\end{equation}
%which is the theoretical bound of the convergence rate  in PCG \ref{PcgConvergeRate}.

假设$A= Q^T\Lambda Q$,这里 $\Lambda$ 表示以  $A$的特征值为对角元素的对角矩阵,而 $Q$ 表示以  $A$的特征向量为列的实正交矩阵。于是,我们可以得到
\begin{equation}
P_k(A) = Q^T P_k(\Lambda)Q = Q^T \left [\begin{array}{cccc}
P_k(\lambda_1) & & &\\
& P_k(\lambda_2) & &\\
& & \ddots &\\
 & & & P_k(\lambda_N)
\end{array} \right ] Q \label{eq:PjMA}
\end{equation}
Assume that the estimated largest and smallest extreme eigenvalues of coefficient matrix 
假设所估计的系数矩阵的最大最小特征值$\nu$和 $\mu$ 满足 $0 < \nu \le \lambda_i \le \mu$ ($i = 1, 2, \cdots, \mathcal{N}$),这时有$|\beta - \alpha \lambda_i| \le 1$, $|P_k(\lambda_i)| \le \tau^{-1}_k (\beta)$.
方程 \ref{eq:rPjr0} 和 \ref{eq:PjMA} 可以推出
\begin{equation}
\label{pcsi_convergence}
\frac{||\textbf{r}_k||_2}{||\textbf{r}_0||_2}  \le  \tau_k^{-1}(\beta) = \frac{2(\beta+\sqrt{\beta^2-1})^k}{1+(\beta+\sqrt{\beta^2-1})^{2k}} \le 2(\frac{\sqrt{\kappa'}-1}{\sqrt{\kappa'}+1})^k
\end{equation}
这里$\kappa' = \frac{\mu}{\nu}$.
 这个等式表明,当系数矩阵特征值的估计比较合理的时候(也就是$\kappa' =\kappa$) , P-CSI的收敛速度和PCG有相同的理论上界。


 
以上分析对于采用了预处理子的情形也同样适用。
唯一的区别是上面分析中提到的系数矩阵是$M^{-1}A$ 而不是$A$。
值得一提的是,PCG,ChronGear和P-CSI算法中预处理后的矩阵实际上是$M^{-1/2}A(E^{-1/2})^T$,它是对称的,并且和$M^{-1}A$ 有着相同的特征值\cite{Shewchuk1994}。
因此,这个预处理后的矩阵的条件数为$\kappa =  \kappa_2(M^{-1/2}A(E^{-1/2})^T)$。这个条件数通常情况下比原始矩阵 $A$的条件数要小。 
 $M$ 与$A$越是相近,$M^{-1}A$的条件数就越小。当$M = A$,$M^{-1}A$的条件数为$\kappa_2(M^{-1 }A ) = 1$。 

 当使用少数几个处理器核心来运行海洋模式分量时,全局通信并不是瓶颈,导致P-CSI方法和ChronGear方法的时间开销比较接近。
但是当采用很多的处理器核心来计算高分辨的海洋模式分量的时候,P-CSI方法的每一个步迭代就应该会比ChronGear方法快很多,因为ChronGear方法中的全局归约操作就会变成一个明显的瓶颈。



\subsection{理想实验}
为了验证第\ref{convergence_rate}节中所给出的收敛的理论结果,我们在理想的试验设置中构造了一系列条件数不同的矩阵。
这里,我们不再采用全局固定的网格大小,而是采用均匀的经纬网格。
随着维度$\theta$的变化,沿着经度方向的网格大小为 $\Delta x_j  = \pi R \cos (\theta_j)$。
时间步长设置为$\tau = 10\frac{\Delta y}{c}$, 这里$c = 200m/s$ 表示重力快波的速度\cite{smith2010parallel}。 
我们用PCG和P-CSI方法分别结合单位阵预处理、对角预处理和EVP预处理来求解得到的这些方程。
实验中,EVP预处的块的大小为$5\times5$, 收敛条件设为$tol = 10^{-6}$。 
由于ChronGear 和PCG的收敛速度相同,因此这个图中PCG的收敛结果也是ChronGear的结果。
  

\begin {figure} 
\centering
\includegraphics[height=6.5cm]{iterationGridSize}
\caption[] {网格点数目和求解器的迭代步数之间的关系。\label{fig:iterationGridSize}}
\end{figure}

正如图\ref{fig:iterationGridSize}所示,随着问题规模的增加,系数矩阵变得越来越病态。
所有的求解器都需要更多的迭代步数才能得到相同的相对残差。
对于PCG和P-CSI,收敛速度会随着预处理方法的变化而变化。 
指定问题的规模,使用单位矩阵作为预处理的求解器需要的迭代步数最大,而使用EVP预处理的求解器所需要的迭代步数最小。 
这也说明,使用EVP预处理,矩阵的形态会比使用单位阵预处理或者对角预处理后的形态要好。 
图中还说明,在预处理方法相同的情况下,P-CSI方法的收敛速度要比PCG的慢一些。 



 
\subsection{CSI方法的性能}
 
我们先测试一下采用对角预处理子时,PCG,ChronGear和P-CSI三个求解器的性能。
图\ref{fig:convergence_diag}给出了不同的正压求解器的收敛速度。
图中横轴表示迭代的步数,垂直轴表示求解器在对应迭代步的残差。 
尽管ChronGear方法改变了PCG方法的计算顺序,PCG方法和ChronGear在每一步的迭代速度几乎一致。
P-CSI方法在开始和最后的迭代中收敛速度比较慢,而在中间的迭代步骤中收敛速度变快。
这与系数矩阵的特征值的分布有关系。 


\begin {figure}
\includegraphics[width=10cm,height=6cm]{Convergence_diag.eps}
\caption[] {不同求解器的收敛速度。\label{fig:convergence_diag}}
\end{figure}





\conclusions \label{se:conc}
We accelerated the high-resolution ocean model component in the CESM by implementing a new P-CSI ocean barotropic solver. This new solver adopts a Chebyshev type iterative method to avoid the global communication operations in conjunction with an effective EVP preconditioner to improve the computing performance further. Comparing with the origin solvers, it significantly reduces the global reductions and realizes a competitive convergence rate. These approaches and experiences in improving the performance of high-resolution ocean model component in CESM may be helpful to other climate models.

\section{Code availability}
The present P-CSI solver v1.0 is available at https://github.com/hxmhuang/PCSI.  These code are also included in the upcoming CESM public releases v2.0.

\appendix
\section{Algorithms} \label{algorithm}   %% Appendix A

The procedure of PCG is shown as following \citep{smith2010parallel}: \\
 \space \\
Initial guess: $\textbf{x}_0$  \\
Compute residual $\textbf{r}_0 = \textbf{b}- {\textbf{A}}\textbf{x}_0$ \\
Set $\textbf{s}_0 =0$, $\beta_0=1$\\
For $k = 1, 2, \cdots,  k_{max}$  do
\begin{enumerate}
\item $\textbf{r}'_{k-1} = \textbf{M}^{-1}\textbf{r}_{k-1} $  \label{pcg_scale1}
\item $\beta_k = \textbf{r}_{k-1}^T\textbf{r}'_{k-1} $\label{pcg_dot1}
\item $\textbf{s}_k = \textbf{r}'_{k-1} +(\beta_k/\beta_{k-1})\textbf{s}_{k-1} $ \label{pcg_scale2}
\item $\textbf{s}'_k = \textbf{A}\textbf{s}_k   $ \label{pcg_mat}
\item $\alpha_k =\beta_k/(\textbf{s}_k^T\textbf{s}'_k )$\label{pcg_dot2}
\item $\textbf{x}_k= {\textbf{x}}_{k-1} +\alpha_k \textbf{s}_k   $ \label{pcg_scale3}
\item $\textbf{r}_k =\textbf{r}_{k-1} -\alpha_k\textbf{s}'_k  $\label{pcg_scale4}
\item convergence\_check($\textbf{r}_{k}$)
\end{enumerate}
End Do \\

Operations like  $\beta_k/\beta_{k-1}$ in line (\ref{pcg_scale2})) are scaler computations,  while $\alpha_k \textbf{s}_k$ in line (\ref{pcg_scale3}) are vector scalings.
$\textbf{A}\textbf{s}_k$ in line (\ref{pcg_mat})) is a matrix-vector multiplication.
Inner-products of vectors are  $\textbf{r}_{k-1}^T\textbf{r}'_{k-1}$ in line (\ref{pcg_dot1}) and $\textbf{s}_k^T\textbf{s}'_k$  in line (\ref{pcg_dot2})).


The procedure of ChronGear is shown as following : \\
 \space \\
Initial guess: $\textbf{x}_0$  \\
Compute residual $\textbf{r}_0 = \textbf{b}- {\textbf{A}}\textbf{x}_0$ \\
Set $\textbf{s}_0 =0$, $\textbf{p}_0 =0$, $\rho_0=1$, $\sigma_0=0$ \\
For $k = 1, 2, \cdots,  k_{max}$  do
\begin{enumerate}
\item $\textbf{r}'_{k} =\textbf{M}^{-1}\textbf{r}_{k-1} $  \label{cg_scale0}
\item $\textbf{z}_k = \textbf{A}\textbf{r}'_{k}$  \label{cg_mat}
\item $\rho_k = \textbf{r}_{k-1}^T\textbf{r}'_{k}$\label{cg_dot1}
\item $\sigma_k = \textbf{z}_k^T\textbf{r}'_k - \beta_k^2\sigma_{k-1}$\label{cg_sigma}
\item $\beta_k = \rho_k / \rho_{k-1}$\label{cg_beta}
\item $\alpha_k = \rho_k /\sigma_{k}$\label{cg_alpha}
\item $\textbf{s}_k = \textbf{r}'_{k} +\beta_k\textbf{s}_{k-1}$\label{cg_scale1}
\item $\textbf{p}_k = \textbf{z}_{k} +\beta_k\textbf{p}_{k-1}$\label{cg_scale2}
\item $\textbf{x}_k =\textbf{x}_{k-1} +\alpha_k \textbf{s}_k$\label{cg_scale3}
\item $\textbf{r}_k =\textbf{r}_{k-1} -\alpha_k\textbf{p}_k$
\item convergence\_check($\textbf{r}_{k}$)
\end{enumerate}
End Do \\


The pseudo code of the P-CSI algorithm  is shown as \\
 \space \\
Initial guess: $\textbf{x}_0$, estimated eigenvalue boundary $[\nu,\mu]$\\
Set $\alpha =\frac{2}{\mu -\nu}$, $ \beta = \frac{\mu +\nu}{\mu -\nu}$, $\gamma = \frac{\beta}{\alpha}$, $\omega_0 =\frac{ 2}{\gamma}$ \\
Compute residual $\textbf{r}_0 = \textbf{b}- {\textbf{A}}\textbf{x}_0$, $\Delta \textbf{x}_{0} = \gamma^{-1}\textbf{M}^{-1}\textbf{r}_0$, $\textbf{x}_1 =\textbf{x}_0 +\Delta \textbf{x}_{0}$, $\textbf{r}_1 =\textbf{b} -\textbf{A}\textbf{x}_1$ \\
For $k = 1, 2, \cdots,  k_{max}$  do
\begin{enumerate}
\item $\omega_k  = 1/(\gamma - \frac{1}{4\alpha^2}\omega_{k-1})$
\item $\textbf{r}'_{k}=\textbf{M}^{-1}\textbf{r}_{k}$
\item $\Delta \textbf{x}_{k} =\omega_k\textbf{r}'_{k}+(\gamma \omega_k-1)\Delta \textbf{x}_{k-1}$
\item $\textbf{x}_{k+1} =\textbf{x}_{k}+\Delta \textbf{x}_{k}$
\item $\textbf{r}_{k+1} =\textbf{b}- \textbf{A}\textbf{x}_{k+1}$
\item convergence\_check($\textbf{r}_{k}$)
\end{enumerate}
End Do \\


\section{Eigenvalue Estimation}                               %% Appendix A1, A2, etc.
The procedure of Lanczos method to estimate the extreme eigenvalues of $M^{-1}A$ matrix\\
\space \\
Initial guess: $\textbf{r}_0$\\
Set $\textbf{s}_0=\textbf{M}^{-1}\textbf{r}_0$;  $\textbf{q}_1 = \textbf{r}_0/({\textbf{r}_0^T\textbf{s}_0})$; $\textbf{q}_0=\textbf{0}$;  $\beta_0 =0$;\quad  $\mu_0 =0$; $T_0=\emptyset$   \\
For $j = 1, 2, \cdots,  m$  do
\begin{enumerate}
\item $\textbf{p}_j = \textbf{M}^{-1}\textbf{q}_j  $
\item $ \textbf{r}_j =\textbf{A}\textbf{p}_j-\beta_{j-1}\textbf{q}_{j-1} $
\item $\alpha_j =\textbf{p}_j^T\textbf{r}_j $
\item $\textbf{r}_j =\textbf{r}_j-\alpha_{j}\textbf{q}_{j} $
\item $\textbf{s}_j = \textbf{M}^{-1}\textbf{r}_j $
\item $\beta_j = \textbf{r}_j^T\textbf{s}_j $
\item $\textbf{if} \quad\beta_j == 0\quad \textbf{then}\quad \textbf{return}  $
\item $\mu_j = max(\mu_{j-1},\alpha_j+\beta_j+\beta_{j-1})  $\label{lan_gersh}
\item $T_j=tri\_diag(T_{j-1},\alpha_j,\beta_j) $\label{lan_tm}
\item $\nu_j = eigs(T_j, 'smallest')  $ \label{lan_nu}
\item $\textbf{if} |\frac{\mu_j}{\mu_{j-1}} -1 | < \epsilon\quad\textbf{and}\quad|1- \frac{\nu_j}{\nu_{j-1}}|< \epsilon \quad\textbf{then} \quad\textbf{return}    $\label{lanczos_converge}
\item $\textbf{q}_{j+1}= \textbf{r}_j/\beta_j $
\end{enumerate}
End Do \\


In step (\ref{lan_tm}), $T$ is a tridiagonal matrix that contains $\alpha_j(j=1,2,...,m)$ as the diagonal entries and $\beta_j (j=1,2,...,m-1)$ as the off-diagonal entries.
\[ T_{m} =  \left[\begin{array}{ccccc}
\alpha_1 & \beta_1 &   &  &   \\
\beta_1 &\alpha_2 &\beta_2    &   &     \\
& \beta_2 &\ddots &\ddots &\\
& & \ddots& \ddots& \beta_{m-1}\\
& &&  \beta_{m-1}& \alpha_m
\end{array} \right]\]

Let $\xi_{min}$ and $\xi_{max}$ be the smallest and largest eigenvalues of $T_m$, respectively. \citet{Paige1980235} demonstrated that
%\begin{equation}
$\nu \le \xi_{min} \le \nu+\delta_1(m)$ and $\mu -\delta_2(m)  \le \xi_{max} \le \mu$.
%\end{equation}
As $m$ increases, $\delta_1(m)$ and $\delta_2(m)$ will gradually close to zero. Thus, the eigenvalue estimation of $M^{-1}A$ is transformed to solve the eigenvalues of $T_m$.
Step \ref{lan_gersh} in eigenvalues estimation employs the Gershgorin circle theorem to estimate the largest eigenvalue of $T_m$, that is,
%\begin{equation}
$\mu = \max_{1 \le i \le m}\sum^m_{j=1}|T_{ij}|=\max_{1 \le i \le m}(\alpha_i +\beta_{i}+\beta_{i-1})$.
%\end{equation}
The efficient QR algorithm ~\citep{ortega1963llt} with a complexity of $\Theta(m)$ is used to estimate the smallest eigenvalue $\nu$ in step (\ref{lan_tm}).

\section{Block preconditioning}
\begin {figure}[!htbp]
\centering
\includegraphics[width=7cm, height=7.0cm]{blockpreconditioning.eps}
\caption[] {Sparsity pattern of the coefficient matrix developed from nine-point stencils.
The whole domain is divided into $3\times3$ non-overlapping blocks.
Elements in red rectangles are coefficients between points in blocks.
Elements in blue rectangles are coefficients between points from the $i$-th block and its neighbor blocks. \label{fig:blockprecond}}
\end{figure}

To the new block EVP preconditioner, we first briefly review a general block preconditioner, as illustrated by Figure \ref{fig:blockprecond}.  If the linear system of $\mathcal{N} \times \mathcal{N}$ grid points is reordered block-by-block with size of $n\times n$ (e.g., $\mathcal{N}/3\times \mathcal{N}/3$ in Figure \ref{fig:blockprecond}), then coefficient matrix $A$ can be represented by a nine-diagonal block matrix. Each row of this matrix contains nine sub-matrices.  Each $B_i$ (red blocks) is a block matrix containing coefficients of the grid points in the i-th block, which share the same structure as $A$ but have a smaller size ($n^2\times n^2$).  $B_i^e$, $B_i^w$, $B_i^n$ and $B_i^s$ are block matrices containing coefficients of points on east, west, north and south boundaries and the points on their respective neighboring blocks, thus having at most $3n$ nonzero elements distributed on $n$ rows. $B_i^{nw}$, $B_i^{ne}$, $B_i^{sw}$ and $B_i^{se}$ have only one nonzero element, representing the coefficient of corner points and their neighboring points on the northwest, northeast, southwest and southeast blocks.  The traditional block preconditioning method constructs the approximate inverse of $A$ by sequentially factorizing it with approximations of $B_i^{-1}$, which is ill-suited for parallel applications.  In contrast, the inverse of the block diagonal of $A$, which provides a good approximation for $A$, can be calculated naturally in parallel.  The inverse of the diagonal block matrices $M$  is
\begin{eqnarray*}
M^{-1}=    \left [
        \begin{array}{ccccccc}
        B_1^{-1} &   &  \\
         & \ddots&  \\
        &   &  B_{m^2}^{-1} \\
    \end{array}
    \right ]
\end{eqnarray*}

Therefore, the preconditioning process $\textbf{x}= M^{-1}\textbf{y}$ is typically transformed into solving the sparse linear equations $B_i \textbf{x}_i = \textbf{y}_i (i=1,2,...,n)$ for each block and solving the equations by LU decomposition, instead of explicitly constructing the inverse of $B$ matrix. The computational complexity of solving these equations with LU decomposition is $\mathcal{O}(n^4)$. With our block Error Vector
Propagation (EVP) preconditioner, the computational complexity of solving the equations $B_i \textbf{x}_i =\textbf{y}_i$ can be reduces to $\mathcal{O}(n^2)$, where the size of $B_i$ is $n^2\times n^2$.

We rewrite the equation (\ref{eq:sten}) into the following form so that we can march the northeastward solution assuming all other neighboring points are exactly known
\begin{align}
\label{eq:evp9p}
\eta_{i+1,j+1} &= (\psi_{i,j} - A_{i,j}^0\eta_{i,j}-A_{i,j}^e\eta_{i+1,j} -A_{i,j}^n\eta_{i,j+1}-A_{i-1,j}^{ne}\eta_{i-1,j+1}  \nonumber\\
&+A_{i-1,j}^e\eta_{i-1,j} -A_{i-1,j-1}^{ne}\eta_{i-1,j-1}-A_{i,j-1}^n\eta_{i,j-1}- A_{i+1,j-1}^{ne}\eta_{i,j-1} )/A_{i,j}^{ne}
\end{align}

The EVP method works as follows. Figure \ref{fig:evp9p} illustrates a Dirichlet boundary elliptic equation $\mathcal{B}\textbf{x} = \psi$ on a small domain.  We define the interior points next to the south and west boundaries as the initial guess points $\textbf{e}$ and those next to the north and east boundaries are the final boundary points $\textbf{f}$ (e.g., $\textbf{e}= \{E_1, \dots, E_7\}$, $\textbf{f}= \{F_1, \dots, F_7\}$).  If the true solution on $\textbf{e}$ is
known, the exact values over the whole domain can be computed sequentially from southwest to northeast corners, using equation
(\ref{eq:evp9p}). This procedure is named as marching. Unfortunately, the value on $\textbf{e}$ is often not known until the elliptic equation is solved.  However, we can get a solution $\textbf{x}$ satisfying the elliptic equation on the whole domain except on the boundary, by first guessing the value
$\textbf{x}|_\textbf{e}$ on $\textbf{e}$ and then calculating the rest using the marching method.  Then $E=(\textbf{x} -\eta)|_\textbf{e}$
and $F=(\textbf{x} -\eta)|_\textbf{f}$ are error vectors on $\textbf{e}$ and $\textbf{f}$, respectively.  The error vector $F$ is already known since $\textbf{f}$  satisfies Dirichlet boundary condition.  The relationship between the error on initial guess points and the final boundary points can be represented as $F=W*E$.  This influence coefficient matrix $W$ can be formed by marching on the whole domain with unit vectors on the initial guess points and zero residual value in the whole domain.


\begin {figure}[!htbp]
\centering
%\includegraphics[width=0.8\linewidth]{evp9pmarch1.png}
\includegraphics[height=6cm]{evp9pmarch1.png}
\caption []{EVP marching method for nine-point stencil. The solution on point $(i+1,j+1)$ can be calculated using the equation on point $(i,j)$, providing solutions on other neighbor points of point $(i,j)$.  \label {fig:evp9p}}
\end {figure}




%%%%%%%%%%%%%%%%%%%%%%%%%%%%%%%%%%%%%%%%%%%%%%%%%%%%%%%%%%%%%%%%%%%
\begin{acknowledgements}
This work is supported in part by a grant from the National Natural Science Foundation of China (41375102), and the National Grand Fundamental Research 973 Program of China (No. 2014CB347800). Computing resources were provided by the Climate Simulation Laboratory at NCAR's Computational and Information Systems Laboratory (sponsored by the NSF and other agencies) and the National Energy Research Scientific Computing Center, a DOE Office of Science User Facility supported by the Office of Science of the U.S. Department of Energy under Contract No. DE-AC02-05CH11231.
\end{acknowledgements}
\bibliographystyle{copernicus}
\bibliography{gmd_cesmpop_bib}


%%%%%%%%%%%%%%%%%%%%%%%%% figures %%%%%%%%%%%%%%%%%%%%%%%%%%%%%%
%
%
\end{document}

