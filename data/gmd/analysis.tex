\section{算法分析与比较}\label{se:Algorithm}
 
除了P-CSI,PCG和ChronGear方法的收敛性也同样依赖于系数矩阵的最大最小特征值。
这里,我们首先来挖掘一下影响求解器收敛的特征值的特征,然后给出ChronGear和P-CSI求解器的计算复杂度和收敛速度的理论结果。 

\subsection{特征值谱和条件数}
POP中的系数矩阵 $A$ 是正定对称的\citep{smith2010parallel},因此它的特征是都是正实数\citep{stewart1976positive}。
假设系数举证的特征值谱\citep{golub2012matrix} 是 $\mathcal{S} = \{\lambda_1, \lambda_2, \cdots, \lambda_N\}$,这里 $\lambda_{min} = \lambda_1 \le \lambda_i \le \lambda_\mathcal{N} = \lambda_{max}$( $1<i <\mathcal{N}$, $\aleph$ 为 $A$的大小 )表示$A$的特征值.
 
利用 Gershgorin圆盘定理\citep{bell1965gershgorin}可知,对于任意的 $\lambda \in \mathcal{S}$,都存在一个数对 $(i,j)$ 满足
\begin{align}
&|\lambda -  (A_{i,j}^O + \phi ) | \le \sum_{\chi \in \{NW,NE,SW,SE,W,E,N,S\}}|A_{i,j}^\chi|
\end{align}
由方程\ref{defineA}给出的系数的定义,我们可以得到如下结论 
\begin{align} \label{eigsGersh}
&\lambda_{max} \le  \max (  5\alpha - \frac{1}{\alpha}, \frac{5}{\alpha}- \alpha) +\phi   \\
&\lambda_{min} \ge 2\min (  \alpha - \frac{1}{\alpha},\frac{1} {\alpha} -  \alpha) + \phi
\end{align}

\begin {figure}[!htbp]
\centering
\includegraphics[height=6.5cm]{conditionNumberAspectRatio}
\caption[] {网格横纵比例和系数矩阵条件数之间的关系。 \label{fig:conditionNumberRatio}}
\end{figure}
\begin {figure}[!htbp]
\centering
\includegraphics[height=6.5cm]{conditionNumberTimestep}
\caption[] {时间步长和系数矩阵条件数之间的关系。 \label{fig:conditionNumberDt}}
\end{figure}

\begin {figure}[!htbp]
\centering
\includegraphics[height=6.5cm]{conditionNumberGridSize}
\caption[] {网格点的数目和系数矩阵条件数之间的关系。 \label{fig:conditionNumbGrid}}
\end{figure}
 
这个结论说明,当网格横纵比例越接近于1时,最大特诊值的上界会随之减少,而最小特征值的下界会随之变大。
也就是说,系数矩阵特征值谱的半径($[\lambda_{min}, \lambda_{max}]$)随着网格横纵比向着1的接近而变小。 
当水平网格的横纵比等于1时,即$ \alpha = \frac{ \Delta y}{ \Delta x} = 1$,我们可以得到$\lambda_{max} \le  4 +\phi$,$\lambda_{min} \ge   \phi$。
这时,系数矩阵的条件数(即 $\kappa=  \lambda_{max}/\lambda_{min}$),由于是由特征值谱的半径的所决定的,也会随着网格横纵比向1的靠近而变小。 
图\ref{fig:conditionNumberRatio}所给出的结果验证了我们的这个结论。图\ref{fig:conditionNumberRatio}中给出的是固定的网格点数$\mathcal{N} = 20\times 20$和固定的$\phi = 0$, 在区间$(1, +\infty)$上,系数矩阵的条件数随着网格横纵比例的增加而增加,而在区间$(0,1)$上,系数矩阵的条件数随着横纵比例的增大而变小。
特征值在当网格横纵比例等于1时取到最小值。
 
特征值的下界是$\phi=\frac{S }{g \tau^2 H}$,这个变量是由时间步长以及水平网格的面积和海底深度的比例确定的。
它表明,特征值的下界会随着时间步长的增大而减小。 
相应的,系数举证的 条件数也就随之增大。
图\ref{fig:conditionNumberDt}中证明了我们的这个推论。
这个图中的实验采用的是固定的网格点数$\mathcal{N} = 20\times 20$,和固定的网格横纵比$\Delta x /{\Delta y} = 1$。 


 
不考虑时间步长的因素(假设$\phi=0$)时,上面的分析表明,当网格的横纵比等于1时,不论网格点的个数是多少, 系数矩阵的特征值谱半径限制在 $(0,4)$ 区间内。
但是,系数矩阵的条件数的变化却可能会很大,因为当网格点的数量$\mathcal{N}$ 不断增加时,最小的特征值会组件的逼近零。 
图\ref{fig:conditionNumbGrid} 中解释了条件数和网格点数目之间的关系。 

\subsection{收敛速度} \label{convergence_rate}

 PCG和ChronGear方法的收敛速度都依赖于系数矩阵$A$的条件数。  
\inlinecite{dAzevedo1999lapack}证明了 ChronGear  和PCG在理论上收敛速度是相同的。 
在PCG和ChronGear方法中,每一步迭代的残差都有一个上界\citep{Liesen2004}
\begin{equation}
\frac{||\textbf{x}_k-\textbf{x}^*||_A }{||\textbf{x}_0-\textbf{x}^*||_A}  \le \min_{p\in \mathcal{P}_k, p(0) = 1 }\max_{\lambda \in \mathcal{S}} |p(\lambda)| \label{PcgConvergeRate}
\end{equation}
这里$\textbf{x}_k$ 表示第$k$步迭代后得到的解向量, $\textbf{x}^*$ 表示线性方程组的解析解  (即$\textbf{x}^* = A^{-1}b$),$\lambda$ 表示系数矩阵 $A$的特征值。
利用 第一类Chebyshev 多项式来估计这个最小最大值,可以得到 
\begin{align}
\label{chrongear_convergence}
||\textbf{x}_k-\textbf{x}^*||_A \le  2 (\frac{\sqrt{\kappa}-1}{\sqrt{\kappa}+1})^k ||\textbf{x}_0-\textbf{x}^*||_A
\end{align}
这里  $\kappa =  \kappa_2(A)$ 表示系数矩阵 $A$的条件数。


方程\ref{chrongear_convergence}表明,PCG方法收敛速度的理论上界取决于系数矩阵的条件数。 
PCG方法在求解良态的矩阵(特征值的条件数很小)时比求解病态的矩阵(特征值的条件数很大)时收敛速度更快。 
图\ref{fig:convergence_diag}给出了PCG和ChronGear方法在0.1度海洋模式中的真实的收敛速度。
如这幅图所示,相对残差随着迭代步数的增加而减少。
总共需要一百步左右的迭代步才能够得到一个相对残差为$10^{-13}$的解。
 

 
除了消除了全局归约操作,P-CSI方法的更一个优点是它有和PCG相同量级的收敛速度。 
P-CSI方法的收敛速度可以用残差的形式表示为
\begin{equation}
\textbf{r}_k = P_k(A)\textbf{r}_0 \label{eq:rPjr0}
\end{equation}
这里
$P_k(\zeta) = \frac{\tau_k(\beta-\alpha \zeta)}{\tau_k(\beta)}$ for $ \zeta \in [\nu, \mu]$ ~\citep{stiefel1958kernel} .
$\tau_k(\xi)$ 是Chebyshev多项式,它的表达式为  
\begin{equation}
\tau_k(\xi) =   \frac{1}{2}[(\xi+\sqrt{\xi^2-1})^k+(\xi+\sqrt{\xi^2-1})^{-k}]
\end{equation}
当 $ \xi \in [-1,1]$时, Chebyshev多项式的另一个等价表达式为$\tau_k(\xi) = cos(k\cos^{-1} \xi)$。从这个表达式可以清楚的看出来,当$| \xi | \le 1$时, $|\tau_k(\xi)| \le 1$。$P_k(\zeta)$ 是一个多项式,满足如下形式
\begin{equation}
P_k = \min_{p\in \mathcal{P}_k, p(0) = 1 }\max_{\zeta \in [\nu,\mu]} |p(\zeta)|
\end{equation}
%which is the theoretical bound of the convergence rate  in PCG \ref{PcgConvergeRate}.

假设$A= Q^T\Lambda Q$,这里 $\Lambda$ 表示以  $A$的特征值为对角元素的对角矩阵,而 $Q$ 表示以  $A$的特征向量为列的实正交矩阵。于是,我们可以得到
\begin{equation}
P_k(A) = Q^T P_k(\Lambda)Q = Q^T \left [\begin{array}{cccc}
P_k(\lambda_1) & & &\\
& P_k(\lambda_2) & &\\
& & \ddots &\\
 & & & P_k(\lambda_N)
\end{array} \right ] Q \label{eq:PjMA}
\end{equation}
Assume that the estimated largest and smallest extreme eigenvalues of coefficient matrix 
假设所估计的系数矩阵的最大最小特征值$\nu$和 $\mu$ 满足 $0 < \nu \le \lambda_i \le \mu$ ($i = 1, 2, \cdots, \mathcal{N}$),这时有$|\beta - \alpha \lambda_i| \le 1$, $|P_k(\lambda_i)| \le \tau^{-1}_k (\beta)$.
方程 \ref{eq:rPjr0} 和 \ref{eq:PjMA} 可以推出
\begin{equation}
\label{pcsi_convergence}
\frac{||\textbf{r}_k||_2}{||\textbf{r}_0||_2}  \le  \tau_k^{-1}(\beta) = \frac{2(\beta+\sqrt{\beta^2-1})^k}{1+(\beta+\sqrt{\beta^2-1})^{2k}} \le 2(\frac{\sqrt{\kappa'}-1}{\sqrt{\kappa'}+1})^k
\end{equation}
这里$\kappa' = \frac{\mu}{\nu}$.
 这个等式表明,当系数矩阵特征值的估计比较合理的时候(也就是$\kappa' =\kappa$) , P-CSI的收敛速度和PCG有相同的理论上界。


 
以上分析对于采用了预处理子的情形也同样适用。
唯一的区别是上面分析中提到的系数矩阵是$M^{-1}A$ 而不是$A$。
值得一提的是,PCG,ChronGear和P-CSI算法中预处理后的矩阵实际上是$M^{-1/2}A(E^{-1/2})^T$,它是对称的,并且和$M^{-1}A$ 有着相同的特征值\citep{Shewchuk1994}。
因此,这个预处理后的矩阵的条件数为$\kappa =  \kappa_2(M^{-1/2}A(E^{-1/2})^T)$。这个条件数通常情况下比原始矩阵 $A$的条件数要小。 
 $M$ 与$A$越是相近,$M^{-1}A$的条件数就越小。当$M = A$,$M^{-1}A$的条件数为$\kappa_2(M^{-1 }A ) = 1$。 

 当使用少数几个处理器核心来运行海洋模式分量时,全局通信并不是瓶颈,导致P-CSI方法和ChronGear方法的时间开销比较接近。
但是当采用很多的处理器核心来计算高分辨的海洋模式分量的时候,P-CSI方法的每一个步迭代就应该会比ChronGear方法快很多,因为ChronGear方法中的全局归约操作就会变成一个明显的瓶颈。 


\subsection{计算复杂度}  \label{subse:complex}


 
当使用了EVP预处理时,计算的复杂变成了$\mathcal{T}_c = 37 \frac{\mathcal{N}}{p}$。因此P-CSI方法在使用EVP方法时一次求解过程的复杂度为
\begin{eqnarray}
\label{t_pcsiEvp}
\mathcal{T}_{pcsi-evp}=\mathcal{O}(\mathcal{K}_{pcsi-evp} (37\frac{\mathcal{N}}{p} +8\sqrt{\frac{\mathcal{N} }{p}} + 4\alpha))
\end{eqnarray}

以上式子可以看出来,使用了EVP预处理后,每一步迭代过程中的计算开销翻了一倍。
但是,一次求解的总体时间是下降。 
采用EVP预处理后,迭代步数$\mathcal{K}_{pcsi-evp}$会下降到原来的一半(参见图\ref{fig:convergence_diag})。
其结果是,在大核上运行时,计算时间变得非常的少,而最耗时的通信部分的次数也随着迭代步数的减少而减少了一半左右。



