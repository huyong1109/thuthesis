%----------------------------------------------------------------------------
\introduction  \label{se:int}
Recent progresses on high-resolution global climate models have demonstrated that refining the model resolution is helpful for representing important climate processes so as to facilitate climate prediction. Significant improvements can be achieved in the global simulations of Tropical Instability Waves \citep{roberts2009impact}, El Ni\~no Southern Oscillation (ENSO) \citep{shaffrey2009uk}, the Gulf Stream \citep{chassignet2008gulf, kuwano2010precipitation} , the global water cycle \citep{demory2014role}, and others. Specifically, \cite{gent2010improvements}  and \cite{wehner2014effect} showed that increasing the resolution of atmosphere models produces better mean climate, more accurate depiction of the tropical storm formation, and more realistic extreme daily precipitation.  \cite{bryan2010frontal} and  \cite{graham2014importance} confirmed that increasing the resolution of ocean model to eddy resolving level helps to capture the positive correlation between sea surface hight and surface wind stress, as well as to improve the asymmetry of the ENSO cycle in simulation.

%High-resolution climate simulations require tremendous computing resources.
In the High Resolution Model Intercomparison Project (HighResMIP) application for  the Coupled Model Intercomparison Project phase 6 (CMIP6), the global model resolutions with 25 km or finer at mid-latitudes are proposed to implement the Tier-1 and Tier-2 experiments. Because all climate models participated in CMIP6 are often need to run for hundreds of years, tremendous computing resources are needed to run the high-resolution production simulations so that the simulations are extraordinary costly. In order to run high-resolution climate models more routinely, additional algorithm optimization is required to utilize the large scale computing resources.

Our work focuses on improving the simulation speed of the ocean model component (Parallel Ocean Model, POP) in CESM, whose development is centered at the National Center for Atmospheric Research(NCAR). It is a fully-coupled climate system model, including atmosphere, ocean, sea-ice and land components. Its ocean model component solves the three-dimensional primitive equations with hydrostatic and Boussinesq approximations and divides the time integration into two parts: the baroclinic mode and the barotropic mode  \citep{smith2010parallel}. The baroclinic mode describes the three-dimensional dynamic and thermodynamics processes, and the barotropic mode solves the vertically-integrated momentum and continuity equations in two dimensions.




%----------------------------------------------------------------------------
%improving barotropic

The barotropic solver is the major bottleneck in the POP within the high-resolution CESM because it dominates the total computation time on a large number of cores.
This results from the barotropic solver for calculating free surface, which scales poorly at the high core counts because of a obvious global communication bottleneck inherent with the algorithm.
The implicit free-surface method is used in the barotropic mode because it allows a large time step to efficiently compute the fast gravity mode.
The drawback of this method is requiring solving a large elliptic system of equations.
Conjugate Gradient method (CG) and its variants are popular choices to solve elliptic equations due to calculating free surface implicitly in ocean models like MITgcm\citep{adcroft2014mitgcm}, FVCOM\citep{lai2010nonhydrostatic}, MOM3\citep{pacanowsky1999mom3}, OPA \citep{madec1997ocean} and so on.
However, CG method causes heavy global communication overhead in the existing version of ocean model component \citep{Worley:2011:PCE:2063384.2063457}.
A number of works has been attempted to improve the performance of Conjugate Gradient (CG) algorithm while most of them has been related to reducing the amount of communication between processes and accelerating the computation of each process.
Methods that reduce the number of global reductions over the standard formulation, such as the Chronopoulos-Gear (ChronGear, \cite{dAzevedo1999lapack}) variant used in POP, were early contributions to the field and still popular.
These methods as well as more recent incarnations (e.g.,  \cite{hoemmen2010}) attempt to reduce global communications, but combining them with a sophisticated preconditioner is non-trivial. A nice overview of reducing global communication costs for CG can be found in \cite{ghysels2014}. In addition, recent efforts at improving the performance of CG include a variant that overlaps the global reduction with the matrix-vector computation via a pipelined-approach \citep{ghysels2014}.

Another highlighted method in improving CG method is preconditioning,
which has been shown to effectively reduce the number of iterations in the conjugate gradient method since the 1990s.
The current ChronGear solver in the POP has benefited by using a simple diagonal preconditioner \citep{pini1990simple, reddy2013comparison}.
Some parallelizable methods such as polynomial, approximate-inverse, multigrid, and block preconditionings have drawn much attention recently.
High-order polynomial preconditioning can reduce iterations as effectively as incomplete LU factorization (ILU) and its variants in sequential simulations \cite{benzi2002preconditioning}.
However, the computational overhead for the polynomial preconditioner typically offsets its superiority to the simple diagonal preconditioner (e.g., \cite{meyer1989numerical,smith1992parallel}).
The approximate-inverse preconditioner, while highly parallelizable, requires to solve a linear system several times larger than the original system (e.g., \cite{smith1992parallel,bergamaschi2007numerical}), which makes it less attractive for the POP.

In addition, multigrid is highly scalable and effective for linear systems derived from elliptic systems of equations.
Recent works indicated that  geometric multigrid is promising in atmosphere and ocean modeling (e.g., \cite{muller2014massively}, \cite{matsumura2008non,kanarska2007algorithm}).
However, the geometric multigrid method in global ocean models does not always scale ideally because of the presence of complex topography, non-uniform or anisotropic grids (e.g., \cite{fulton1986multigrid,stuben2001review,tseng2003ghost,matsumura2008non}).
This is the case for the current POP which employs the dipole general orthogonal girds to avoid the polar singularity in the ocean. This leads to an elliptic system with variable coefficients defined on an irregular domain with non-uniform grids.
Algebraic multigrid (AMG) is an alternative to Geometric multigrid to handle complex topography. But, setting the AMG in the parallel environment is more expensive than the iterative solver, which makes it unfavorable as a preconditioner (\cite{muller2014massively}).


Block preconditioning has been shown to be an effective parallel preconditioner (e.g., \cite{concus1985block, white2011block}) and is appealing for the POP because it makes use of the block structure of the coefficient matrix that arises from discretization of the elliptic equations.
This work studies a block preconditioning method based on Error Vector Propagation (EVP) method, which has a priority in efficiency in solving  elliptic equations.

There are also some other approaches to improve the performance of ocean models. Dennis and Tufo proposed a load-balancing algorithm based on space-filling curve\citep{dennis2007inverse, dennis2008scaling}. This algorithm not only eliminates land blocks, but also decreases the communications overhead because of the reduced number of processes. Since reducing the frequency of communication can decrease the computing overhead, \citet{beare1997optimisation} proposed an approach by increasing the number of extra halos and overlapping the communications with the computation to optimize the computing performance of parallel ocean general circulation. Although all these approaches improve performance of ocean models, they do not attempt to remove the global communication bottleneck. The promising preliminary results in our previous work \cite{hu2015improving} with implementing a Chebyshev type method \citep{stiefel1958kernel} in CESM-POP, encouraged us to explore finer linear solvers for high-resolution ocean models.

To improve the scaling of the ocean model component, we abandon the PCG method and design a new ocean barotropic solver which does not include global communication. The new barotropic solver, named as P-CSI, includes a Classical Stiefel Iteration (CSI) method with an effectively block preconditioner based on the Error Vector Propagation (EVP) method \citep{roache1995elliptic}. The P-CSI solver will become the default ocean barotropic solver in the upcoming version of CESM v2.0. This paper is an extended version of our conference paper \citep{hu2015improving} presented at the 27th International Conference for High Performance Computing, Networking, Storage and Analysis (SC2015). Comparing with the conference paper, we focus on the high-resolution ocean model component and  add the theoretical analysis about the computational complexity and convergence of P-CSI. Most of the sections have been rewritten to make our methods easy to understand by climate modelers.

The remainder of this paper is organized as follows. Section \ref{se:baro} reviews the  barotropic mode in the ocean model component and the existing solvers. Sections \ref{se:pcsi} describes the detail design of P-CSI solver. Section \ref{se:Algorithm} analyzes the computational complexity and convergence rate of P-CSI. Section \ref{se:exp} compares the computing performance of the existing solvers and the P-CSI solvers. Finally, conclusions are given in Sections \ref{se:conc}.
