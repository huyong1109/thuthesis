%
% ================= IF YOU HAVE QUESTIONS =======================
% Questions regarding the SIGS styles, SIGS policies and
% procedures, Conferences etc. should be sent to
% Adrienne Griscti (griscti@acm.org)
%
% Technical questions _only_ to
% Gerald Murray (murray@hq.acm.org)
% ===============================================================
%
% For tracking purposes - this is V2.0 - May 2012

\documentclass{sig-alternate-05-2015}
\usepackage{gensymb}  %format of algorithm
\usepackage{algorithm}  %format of algorithm
\usepackage{algorithmic}  %format of algorithm
\renewcommand{\algorithmiccomment}[1]{ \hfill {/* #1 */} }
\usepackage{graphicx}
\usepackage{epstopdf}



\DeclareGraphicsExtensions{.eps,.mps,.pdf,.jpg,.PNG}
\DeclareGraphicsRule{*}{pdf}{*}{}
\graphicspath{{./fig/}}

\usepackage{color}
\newcommand{\hl}[1]{\textcolor{red}{#1}}

\newcommand{\superscript}[1]{\ensuremath{^{\textrm{#1}}}}
\def\sharedaffiliation{\end{tabular}\newline\begin{tabular}{c}}
\def\thua{\superscript{\dag}}
\def\thub{\superscript{\ddag}}
\def\ncara{\superscript{*}}
\def\ncarb{\superscript{\P}}

\begin{document}
%
% --- Author Metadata here ---
\setcopyright{usgovmixed}
\conferenceinfo{SC '15,}{November 15-20, 2015, Austin, TX, USA}
\isbn{978-1-4503-3723-6/15/11}\acmPrice{\$15.00}
\doi{http://dx.doi.org/10.1145/2807591.2807596}


%\conferenceinfo{SC'15}{November 15-20, 2015, Austin, Texas, USA.}
%\CopyrightYear{2007} % Allows default copyright year (20XX) to be over-ridden - IF NEED BE.
%\crdata{0-12345-67-8/90/01}  % Allows default copyright data (0-89791-88-6/97/05) to be over-ridden - IF NEED BE.
% --- End of Author Metadata ---

%\title{Alternate {\ttlit ACM} SIG Proceedings Paper in LaTeX
%Format\titlenote{(Produces the permission block, and
%copyright information). For use with
%SIG-ALTERNATE.CLS. Supported by ACM.}}
%\subtitle{[Extended Abstract]
%\titlenote{A full version of this paper is available as
%\textit{Author's Guide to Preparing ACM SIG Proceedings Using
%\LaTeX$2_\epsilon$\ and BibTeX} at
%\texttt{www.acm.org/eaddress.htm}}}

\title{Improving the Scalability of the Ocean Barotropic Solver \\in the Community Earth System Model}
%
% You need the command \numberofauthors to handle the 'placement
% and alignment' of the authors beneath the title.
%
% For aesthetic reasons, we recommend 'three authors at a time'
% i.e. three 'name/affiliation blocks' be placed beneath the title.
%
% NOTE: You are NOT restricted in how many 'rows' of
% "name/affiliations" may appear. We just ask that you restrict
% the number of 'columns' to three.
%
% Because of the available 'opening page real-estate'
% we ask you to refrain from putting more than six authors
% (two rows with three columns) beneath the article title.
% More than six makes the first-page appear very cluttered indeed.
%
% Use the \alignauthor commands to handle the names
% and affiliations for an 'aesthetic maximum' of six authors.
% Add names, affiliations, addresses for
% the seventh etc. author(s) as the argument for the
% \additionalauthors command.
% These 'additional authors' will be output/set for you
% without further effort on your part as the last section in
% the body of your article BEFORE References or any Appendices.

% \numberofauthors{6}
% \author{
% % 1st. author
% \alignauthor
% Yong Hu\titlenote{
% Ministry of Education Key Laboratory for Earth System Modeling, Center for Earth System Science,
% and Joint Center for Global Change Studies, Beijing, 100875, China. }
% \\
% %  \affaddr{Department of Computer Science \& Technology }\\
%  % \affaddr{Center for Earth System Science } \\
%   \affaddr{Tsinghua University} \\
%   \affaddr{Beijing, China} \\
%   \email{huyong11@tsinghua.edu.cn}
% % 2nd
% \alignauthor
% Xiaomeng Huang\raisebox{9pt}{$\ast$} \\
% %\affaddr{Center for Earth System Science } \\
%  \affaddr{Tsinghua University} \\
%   \affaddr{Beijing, China} \\
%   \email{hxm@tsinghua.edu.cn}
% %3rd
% \alignauthor
% Allison H. Baker\\
% %\affaddr{Computational and Information Systems Laboratory}\\
%    \affaddr{National Center for Atmospheric Research}\\
%    \affaddr{Boulder, CO}\\
%    \email{abaker@ucar.edu}
% %more
% \and
% \alignauthor
% Yu-heng Tseng\\
%  % \affaddr{Climate and Global Dynamics Division} \\
%   \affaddr{National Center for Atmospheric Research}\\
%   \affaddr{Boulder, CO}\\
%   \email{ytseng@ucar.edu}
% \alignauthor
% Frank O. Bryan\\
% % \affaddr{Climate and Global Dynamics Division} \\
%   \affaddr{National Center for Atmospheric Research}\\
%   \affaddr{Boulder, CO}\\
%   \email{bryan@ucar.edu}
% \alignauthor
% John M. Dennis\\
% % \affaddr{Climate and Global Dynamics Division} \\
%   \affaddr{National Center for Atmospheric Research}\\
%   \affaddr{Boulder, CO}\\
%   \email{dennis@ucar.edu}
% \alignauthor
% Guangwen Yang\raisebox{9pt}{$\ast$}\\
% % \affaddr{Center for Earth System Science } \\
% \affaddr{Tsinghua University} \\
% \affaddr{Beijing, China} \\
% \email{ygw@tsinghua.edu.cn}
% }


\numberofauthors{1}
\author{
\alignauthor
Yong Hu\thua, Xiaomeng Huang\thua, Allison H. Baker\ncarb,\\Yu-heng Tseng\ncara, Frank O. Bryan\ncara,  John M. Dennis\ncarb, Guangwen Yang\thua
\sharedaffiliation
  \begin{tabular}{ccc}
    \affaddr{{\thua}  {Center for Earth System Science\ }} & & \affaddr{{\ncara}Climate and Global Dynamics Division {\ }} \\
    \affaddr{Tsinghua University, 100084, and  {\ } } & & \affaddr{{\ncarb}Computational and Information Systems Laboratory{\ }} \\
      \affaddr{ Joint Center for Global Change Studies {\ } } & & \affaddr{National Center for Atmospheric Research} \\
    \affaddr{Beijing, 100875, China} & & \affaddr{Boulder,CO. USA} \\
   \email{\{huyong11,hxm,ygw\}@tsinghua.edu.cn}  & & \email{\{abaker,ytseng,bryan,dennis\}@ucar.edu} \\
  \end{tabular}
}


% \numberofauthors{1}
% \author{
% \alignauthor
% Yong Hu\thua\thub,Yu-heng Tseng\ncara,Allison H. Baker\ncarb, \\Xiaomeng Huang\thub, Frank O. Bryan\ncara, Guangwen Yang\thua\thub
% \sharedaffiliation
%   \begin{tabular}{ccc}
%     \affaddr{{\thua}Department of Computer Science \& Technology {\ }} & & \affaddr{{\ncara}Climate and Global Dynamics Division {\ }} \\
%     \affaddr{{\thub} Center for Earth System Science{\ } } & & \affaddr{{\ncarb}Computational and Information Systems Laboratory{\ }} \\
%     \affaddr{Tsinghua University,Beijing,China} & & \affaddr{National Center for Atmospheric Research} \\
%     \email{huyong11@mails.tsinghua.edu.cn} & & \affaddr{Boulder,CO. USA} \\
%     \email{\{hxm,ygw\}@tsinghua.edu.cn}  & & \email{\{ytseng,abaker,bryan\}@ucar.edu} \\
%   \end{tabular}
% }



%\sharedaffiliation
%  \begin{tabular}{ccc}
%    \affaddr{{\thu}Ministry of Education Key Laboratory for Earth System Modeling{\ }} & & \affaddr{{\ncar}Climate and Global Dynamics Division, Computational and Information Systems Laboratory{\ }} \\
%    \affaddr{Center for Earth System Science, Tsinghua University, Beijing, 100084, China}            & & \affaddr{National Center for Atmospheric Research} \\
%    \affaddr{Tsinghua National Laboratory for Information Science and Technology (TNList)}            & & \affaddr{1850 Table Mesa Dr., Boulder, CO80305, USA} \\
%    \affaddr{United States}                  & & \affaddr{United States} \\
%  \end{tabular}
%


% There's nothing stopping you putting the seventh, eighth, etc.
% author on the opening page (as the 'third row') but we ask,
% for aesthetic reasons that you place these 'additional authors'
% in the \additional authors block, viz.
%\additionalauthors{Additional authors: John Smith (The Th{\o}rv{\"a}ld Group,
%email: {\texttt{jsmith@affiliation.org}}) and Julius P.~Kumquat
%(The Kumquat Consortium, email: {\texttt{jpkumquat@consortium.net}}).}
\date{30 March 2015}
% Just remember to make sure that the TOTAL number of authors
% is the number that will appear on the first page PLUS the

% number that will appear in the \additionalauthors section.



\maketitle
%ABSTRACT needs to be 150 words that we submit on web site - not in
%actual paper
\begin{abstract}
  

  高分辨率气候模拟的需求量与日剧增,它所消耗的计算资源也是极其巨大。 
  在公共地球系统模式(CESM)中, 并行海洋模式(POP)在高分辨网格(比如0.1度)的配置下的计算量极其巨大,在很多真实运行的实例中都是公共地球系统模式CESM中可扩展性最差的一个分量模式。 
  尤其是在正压模态中求解椭圆方程所用到的改进后的预处理共轭梯度法(PCG), 在大核数并行使可扩展很差, 从而成为了高分辨率的模拟中的一个主要瓶颈。 
  此论文中,我们证明了正压求解器中的通信开销在整个海洋模式POP的运行时间中所占的比随着并行核数的增加而增加。 
  为了解决这一问题, 我们在海洋模式POP中实现了一个预处理的Chebyshev迭代方法(称之为P-CSI方法), 它所需要的全局归约操作要比预处理共轭梯度法少很多。 
  我们还研发了一个高效的基于误差向量传播方法的块预处子,它是的P-CSI方法能在比较少的迭代步内收敛。 
  我们还证明了P-CSI由于可扩展性得到改善, 在16,875核上对高分辨海洋模式POP的正压模态加速了5.2倍,并且是的整个POP的模拟速率提高了1.7倍。 
  最后,我们还通过一个基于集合模拟的统计学方法来证明使用了我们的新的求解器后产生的模拟结果是与原始结果相容的。


\end{abstract}


% \begin{abstract}
%   The increasing need for high-resolution climate simulations warrants
%   the investigation of a new preconditioned linear solver for the
%   barotropic mode in the Parallel Ocean Model (POP).  POP is the ocean
%   component of the Community Earth System Model (CESM) framework and
%   is computationally expensive for high-resolution grids (e.g.,
%   0.1\degree) on many cores.  Similar to other free-surface models,
%   POP solves the fast barotropic dynamics with the implicit
%   free-surface method.  This method requires the solution of an
%   elliptic system of equations, and POP solves this elliptic system
%   with a modified Preconditioned Conjugate Gradient (PCG) method.
%   While effective at moderate core counts and resolutions, PCG methods
%   scale poorly on large parallel systems due to the global reductions
%   required by the inner product calculations.  This poor scaling is
%   problematic for high-resolution CESM simulations.  In this work, we
%   implement a more scalable Preconditioned Stiefel Iteration method
%   (P-CSI) into the barotropic mode of POP within CESM, and develop an
%   effective Error Vector Propagation preconditioner to improve the
%   convergence rate.  Our experiments show that P-CSI accelerates the
%   high-resolution (0.1\degree) barotropic solver by 7.1 times on
%   16,875 cores, from 21.84s to 3.06s per simulation day.  This
%   improves the core POP simulation (without I/O and initialization) by
%   1.74 times, from 5.3 to 9.3 simulated years per wall-clock day.
%   Further, to ensure that the use of an alternative solver in POP does
%   not negatively impact the ocean climate, we use an ensemble-based
%   statistical method to evaluate the impact of the new solver.
% \end{abstract}







% A category with the (minimum) three required fields
\category{J.2}{Physical Sciences and Engineering}{Earth and atmospheric sciences}
%A category including the fourth, optional field follows...
\category{D.1.3}{Programming Techniques}{Parallel Programming}

%it's OK not to have general terms
%\terms{Algorithms, Performance}
%\keywords{Massive Parallelism, Preconditioned Conjugate Gradient, Classical Stiefel Iteration, Parallel Ocean Program}
\keywords{parallel computing, linear solver, ocean modeling}
%----------------------------------------------------------------------------
\introduction  \label{se:int}
Recent progresses on high-resolution global climate models have demonstrated that refining the model resolution is helpful for representing important climate processes so as to facilitate climate prediction. Significant improvements can be achieved in the global simulations of Tropical Instability Waves \citep{roberts2009impact}, El Ni\~no Southern Oscillation (ENSO) \citep{shaffrey2009uk}, the Gulf Stream \citep{chassignet2008gulf, kuwano2010precipitation} , the global water cycle \citep{demory2014role}, and others. Specifically, \cite{gent2010improvements}  and \cite{wehner2014effect} showed that increasing the resolution of atmosphere models produces better mean climate, more accurate depiction of the tropical storm formation, and more realistic extreme daily precipitation.  \cite{bryan2010frontal} and  \cite{graham2014importance} confirmed that increasing the resolution of ocean model to eddy resolving level helps to capture the positive correlation between sea surface hight and surface wind stress, as well as to improve the asymmetry of the ENSO cycle in simulation.

%High-resolution climate simulations require tremendous computing resources.
In the High Resolution Model Intercomparison Project (HighResMIP) application for  the Coupled Model Intercomparison Project phase 6 (CMIP6), the global model resolutions with 25 km or finer at mid-latitudes are proposed to implement the Tier-1 and Tier-2 experiments. Because all climate models participated in CMIP6 are often need to run for hundreds of years, tremendous computing resources are needed to run the high-resolution production simulations so that the simulations are extraordinary costly. In order to run high-resolution climate models more routinely, additional algorithm optimization is required to utilize the large scale computing resources.

Our work focuses on improving the simulation speed of the ocean model component (Parallel Ocean Model, POP) in CESM, whose development is centered at the National Center for Atmospheric Research(NCAR). It is a fully-coupled climate system model, including atmosphere, ocean, sea-ice and land components. Its ocean model component solves the three-dimensional primitive equations with hydrostatic and Boussinesq approximations and divides the time integration into two parts: the baroclinic mode and the barotropic mode  \citep{smith2010parallel}. The baroclinic mode describes the three-dimensional dynamic and thermodynamics processes, and the barotropic mode solves the vertically-integrated momentum and continuity equations in two dimensions.




%----------------------------------------------------------------------------
%improving barotropic

The barotropic solver is the major bottleneck in the POP within the high-resolution CESM because it dominates the total computation time on a large number of cores.
This results from the barotropic solver for calculating free surface, which scales poorly at the high core counts because of a obvious global communication bottleneck inherent with the algorithm.
The implicit free-surface method is used in the barotropic mode because it allows a large time step to efficiently compute the fast gravity mode.
The drawback of this method is requiring solving a large elliptic system of equations.
Conjugate Gradient method (CG) and its variants are popular choices to solve elliptic equations due to calculating free surface implicitly in ocean models like MITgcm\citep{adcroft2014mitgcm}, FVCOM\citep{lai2010nonhydrostatic}, MOM3\citep{pacanowsky1999mom3}, OPA \citep{madec1997ocean} and so on.
However, CG method causes heavy global communication overhead in the existing version of ocean model component \citep{Worley:2011:PCE:2063384.2063457}.
A number of works has been attempted to improve the performance of Conjugate Gradient (CG) algorithm while most of them has been related to reducing the amount of communication between processes and accelerating the computation of each process.
Methods that reduce the number of global reductions over the standard formulation, such as the Chronopoulos-Gear (ChronGear, \cite{dAzevedo1999lapack}) variant used in POP, were early contributions to the field and still popular.
These methods as well as more recent incarnations (e.g.,  \cite{hoemmen2010}) attempt to reduce global communications, but combining them with a sophisticated preconditioner is non-trivial. A nice overview of reducing global communication costs for CG can be found in \cite{ghysels2014}. In addition, recent efforts at improving the performance of CG include a variant that overlaps the global reduction with the matrix-vector computation via a pipelined-approach \citep{ghysels2014}.

Another highlighted method in improving CG method is preconditioning,
which has been shown to effectively reduce the number of iterations in the conjugate gradient method since the 1990s.
The current ChronGear solver in the POP has benefited by using a simple diagonal preconditioner \citep{pini1990simple, reddy2013comparison}.
Some parallelizable methods such as polynomial, approximate-inverse, multigrid, and block preconditionings have drawn much attention recently.
High-order polynomial preconditioning can reduce iterations as effectively as incomplete LU factorization (ILU) and its variants in sequential simulations \cite{benzi2002preconditioning}.
However, the computational overhead for the polynomial preconditioner typically offsets its superiority to the simple diagonal preconditioner (e.g., \cite{meyer1989numerical,smith1992parallel}).
The approximate-inverse preconditioner, while highly parallelizable, requires to solve a linear system several times larger than the original system (e.g., \cite{smith1992parallel,bergamaschi2007numerical}), which makes it less attractive for the POP.

In addition, multigrid is highly scalable and effective for linear systems derived from elliptic systems of equations.
Recent works indicated that  geometric multigrid is promising in atmosphere and ocean modeling (e.g., \cite{muller2014massively}, \cite{matsumura2008non,kanarska2007algorithm}).
However, the geometric multigrid method in global ocean models does not always scale ideally because of the presence of complex topography, non-uniform or anisotropic grids (e.g., \cite{fulton1986multigrid,stuben2001review,tseng2003ghost,matsumura2008non}).
This is the case for the current POP which employs the dipole general orthogonal girds to avoid the polar singularity in the ocean. This leads to an elliptic system with variable coefficients defined on an irregular domain with non-uniform grids.
Algebraic multigrid (AMG) is an alternative to Geometric multigrid to handle complex topography. But, setting the AMG in the parallel environment is more expensive than the iterative solver, which makes it unfavorable as a preconditioner (\cite{muller2014massively}).


Block preconditioning has been shown to be an effective parallel preconditioner (e.g., \cite{concus1985block, white2011block}) and is appealing for the POP because it makes use of the block structure of the coefficient matrix that arises from discretization of the elliptic equations.
This work studies a block preconditioning method based on Error Vector Propagation (EVP) method, which has a priority in efficiency in solving  elliptic equations.

There are also some other approaches to improve the performance of ocean models. Dennis and Tufo proposed a load-balancing algorithm based on space-filling curve\citep{dennis2007inverse, dennis2008scaling}. This algorithm not only eliminates land blocks, but also decreases the communications overhead because of the reduced number of processes. Since reducing the frequency of communication can decrease the computing overhead, \citet{beare1997optimisation} proposed an approach by increasing the number of extra halos and overlapping the communications with the computation to optimize the computing performance of parallel ocean general circulation. Although all these approaches improve performance of ocean models, they do not attempt to remove the global communication bottleneck. The promising preliminary results in our previous work \cite{hu2015improving} with implementing a Chebyshev type method \citep{stiefel1958kernel} in CESM-POP, encouraged us to explore finer linear solvers for high-resolution ocean models.

To improve the scaling of the ocean model component, we abandon the PCG method and design a new ocean barotropic solver which does not include global communication. The new barotropic solver, named as P-CSI, includes a Classical Stiefel Iteration (CSI) method with an effectively block preconditioner based on the Error Vector Propagation (EVP) method \citep{roache1995elliptic}. The P-CSI solver will become the default ocean barotropic solver in the upcoming version of CESM v2.0. This paper is an extended version of our conference paper \citep{hu2015improving} presented at the 27th International Conference for High Performance Computing, Networking, Storage and Analysis (SC2015). Comparing with the conference paper, we focus on the high-resolution ocean model component and  add the theoretical analysis about the computational complexity and convergence of P-CSI. Most of the sections have been rewritten to make our methods easy to understand by climate modelers.

The remainder of this paper is organized as follows. Section \ref{se:baro} reviews the  barotropic mode in the ocean model component and the existing solvers. Sections \ref{se:pcsi} describes the detail design of P-CSI solver. Section \ref{se:Algorithm} analyzes the computational complexity and convergence rate of P-CSI. Section \ref{se:exp} compares the computing performance of the existing solvers and the P-CSI solvers. Finally, conclusions are given in Sections \ref{se:conc}.

\section{正压求解器} \label{se:baro}
 
 
我们给出ChronGear方法的具体算法
\ref{alg:pcg}, 以供参考.  ChronGear主要包括三个部分:计算部分,边界通信部分和全局通信部分。
计算部分主要包括矩阵向量乘操作,向量向量乘操作以及向量的伸缩操作(向量与常数相乘)等具有很好的可扩展性的操作。 
边界通信会在每次矩阵向量乘操作之后被调用,其开销随着计算使用的处理器核心数的增加和减少,但是有一个下界。 
但是,我们后面的章节将会证明,每一步迭代过程中的內积操作\ref{pcg_global1}之后需要调用的全局通信,当使用很大的核心数时,将会成为可扩展性的瓶颈。
 

%----------------------------------------------------------------------------
\subsection{通信瓶颈}\label{se:bottleneck}
  
假设使用$p=m^2$个进程,每一个进程恰好分得一个网格块(这在高分辨率POP中是比较常见的配置)。 
这是正压模态的总运行时间就等于ChronGear求解器在每一个块上的运行时间。 
我们记求解器的每一次迭代过程中,计算、边界通信和全局通信操作的耗时分别为$\mathcal{T}_c$,$\mathcal{T}_b$ 和$\mathcal{T}_g$。 
%----------------------------------------------------------------------------


从算法\ref{alg:pcg}中可以得出, 计算开销
$\mathcal{T}_c$主要涉及到四个向量伸缩操作 (算法中步骤\ref{pcg_scale1}, \ref{pcg_scale2} ,\ref{pcg_scale3}, 和
\ref{pcg_scale4}), 两个向量向量乘积操作或內积操作 (算法中步骤\ref{pcg_dot1} 和 \ref{pcg_dot2}), 以及一个矩阵向量乘积操作 (算法步骤\ref{pcg_mat}).  
因此,$\mathcal{T}_c= (4 n^2 +2n^2+ 9n^2)\theta + \mathcal{T}_{p}=15\frac{\mathcal{N}^2}{p}\theta+\mathcal{T}_{p}$,
这里$\theta$ 表示单位时间内浮点操作的次数,$\mathcal{T}_{p}$ 表示预处理的开销。
比如,当采用对角预处理时,$\mathcal{T}_{p} =\frac{\mathcal{N}^2}{p}\theta$。
上面的定义可以看出,当计算核心数增加时,$\mathcal{T}_c$减少,并且以零为下界。

 
当调用了像矩阵向量乘和除了对角预处理以外的预处理等需要一层或一层以上的边界区域的操作时,在每个进程的边界区域上需要进行边界更新操作。 
由于在海洋模式POP中,每个进程都维护着它自己的块以外的两个边界层,因此,即使使用了非对角的预处理,每次迭代过程中仍只需要进行一次边界通信。
边界通信的实际开销取决于网络延迟和边界区域的大小。 当边界区域的层数为2时(这也是POP中的默认配置), 边界区域的每个边的大小为$2n$ ,而且这个大小会随着进程数的增大而变小。 
 
因此,每一步迭代中,边界通信的时间复杂度为$\mathcal{T}_b =4\alpha +(4\times 2n)\beta=4\alpha +(\frac{8\mathcal{N}}{\sqrt{p}})\beta $这里$\alpha$ 表示单条点对点通信消息的延迟,而
$\beta$ 表示从网络中传输一比特需要的时间 (也就是网络带宽的倒数).
边界更新所需要的时间也是随着就算进程的增多而减少,但是有一个下界$4\alpha$。 


Combining all three components, the execution time of one diagonal preconditioned ChronGear solver step can be expressed as:
ChronGear求解器的每一步迭代中仅仅包含了一次全局归约操作。全局规约操作主要由一次MPI\_allreduce和一个去除路地网格点的掩盖操作,因此全局归约操作的开销满足$\mathcal{T}_g= 2\frac{\mathcal{N}^2}{p}\theta + \log p \alpha$ (假设网络采用二叉树结构)。 
路地点的掩盖操作的开销会随着进程数$p$的增大而减小,而MPI\_allreduce的开销将会一直随着进程数的增加而增大。 
注意到每次全局归约操作实际上近似于没有数据的交换,因为每个进程只传递了两个数值。
因此,全局通信的开销主要取决于网络的延迟而不是网络的传输速率。

\begin{eqnarray}
%\begin{tabular}{l}
\label{t_pcg}
&\mathcal{T}_{cg}=\mathcal{K}_{cg} (\mathcal{T}_c + \mathcal{T}_b+\mathcal{T}_g )\nonumber \\
&=\mathcal{K}_{cg} [18 \frac{\mathcal{N}^2}{p}\theta + \frac{8\mathcal{N}}{\sqrt{p}}\beta +(4+\log p)\alpha]
%\end{tabular}
\end{eqnarray}
这里 $K_{cg}$ 表示算法达到指定收敛条件所需要的迭代次数。
这个迭代次数不会随着进程数的增加而改变 \cite{hu2013scalable}。 
方程\ref{t_pcg})表明,计算和边界更新所需要的时间开销都会随着进程数的增加而减少。 
但是,全局归约操作的时间开销则会随着进程数的增多而增加。 
因此,我们可以预测到,当进程数操作某一个值时, ChronGear求解器的运行时将会增加。 
图 \ref{fig:ChronGearCOMP}给出了在美国国际大气研究中心的黄石超级计算(详情见节\ref{se:exp})上,运行一个模拟天时,ChronGear求解器的全局通信和边界通信两个部分的耗时分析。 
这里我们可以看到,当使用数千个处理器核心时,全局归约操作的时间变成了整个求解器耗时的主要部分,并且这个时间还在继续增大。 
 


\begin{figure}[!t]
%\vspace{-10pt}
\begin{center}
	\includegraphics[height=6.5cm]{newChronGear_comp.eps}
\end{center}
\vspace{-.2in}
\caption[] {在美国黄石超级计算机上,0.1度POP的全局归约操作和边界更新操作占模拟一天总时间开销的比例。}
\label{fig:ChronGearCOMP}
\vspace{-.1in}
\end{figure}
 

\section{P-CSI Solver} \label{se:psi}
%----------------------------------------------------------------------------
To improve the scalability of  POP, a barotropic solver that
requires as few global reductions as possible is desired.
%Originally less efficient methods, such as Chebyshev iteration, were reconsidered in POP.
%Chebyshev iteration was revisited by Gutknecht \cite{gutknecht2002chebyshev} in 2002, and was identified as being suitable for massively parallel computers with high communications costs.
In \cite{hu2013scalable}, Hu et al. proposed an appropriate solver based
on Stiefel's CSI method and did a preliminary
evaluation at modest core counts in a stand-alone version of POP.
Here, we further improve CSI by adding a preconditioning interface,
developing an effective preconditioner, and implementing the optimized
P-CSI solver into POP within the CESM framework.
%P-CSI provides interfaces for different kinds of preconditioner.
% As early as 1985, Saad et al.\cite{saad1985solving} implemented a generalization of P-CSI on a linear array of processors and claimed that this generalization is more favorable than conjugate gradient method in some cases when the eigenvalues are known.

%\subsection{Algorithm and Evaluation} \label{se:psialg}

The P-CSI algorithm and its properties are similar to those of the CSI
algorithm detailed in \cite{hu2013scalable}, with the exception of the
additional preconditioner (described in detail in the next
section).  Notably, P-CSI also does not require inner-product operations, potentially improving high core count scalability.
%eliminates the bottleneck of global reduction as in PCG and ChronGear.
%Instead of requiring prior knowledge about the spectrum of the
%coefficient matrix $A$,
The pseudo code for the P-CSI algorithm designed for POP is shown in Algorithm 2.
The total computation time for each iteration in the diagonal preconditioned P-CSI solver is $T_c =\frac{12\mathcal{N}^2}{p}\theta+\mathcal{T}_p =\frac{13\mathcal{N}^2}{p}\theta$,
and the total execution time for each P-CSI solve is
\begin{eqnarray}
\label{t_psi}
\mathcal{T}_{pcsi} = \mathcal{K}_{pcsi}(\mathcal{T}_c + \mathcal{T}_b ) \nonumber \\
= \mathcal{K}_{pcsi}[13\frac{\mathcal{N}^2}{p}\theta+ 4\alpha + \frac{8\mathcal{N}}{ \sqrt{p}}\beta]
\end{eqnarray}
where $K_{pcsi}$ is the number of iterations in one P-CSI solver step.
\begin {figure}[!t]
\begin{center}
\includegraphics[height=6.5cm]{solver_iteration}
\end{center}
\vspace{-.2in}
\caption []{Effect of the number of Lanczos steps on the number of P-CSI iterations and in 1\degree\space POP \label {fig:iter}}
%\vspace{-.1in}
\end {figure}

\begin{algorithm}[!t]
\caption{Preconditioned Classical Stiefel Iteration}
\label{alg:ppsi}
\begin{scriptsize}
\begin{algorithmic}[1]
\REQUIRE Coefficient matrix $\textbf{B}$, preconditioner $\textbf{M}$, initial guess  $\textbf{x}_0$ and right hand side vector $\textbf{b}$ associated with grid block $B_{i,j}$; Estimated eigenvalue boundary $[\nu,\mu]$;  \\
 // \qquad    \textit{do in parallel with all processes}
\STATE $\alpha =\frac{2}{\mu -\nu}$, $ \beta = \frac{\mu +\nu}{\mu -\nu}$, $\gamma = \frac{\beta}{\alpha}$, $\omega_0 =\frac{ 2}{\gamma}$;\quad $k = 0$;
\STATE $\textbf{r}_0 = \textbf{b}-\textbf{B}\textbf{x}_0$; $\Delta \textbf{x}_{0} = \gamma^{-1}\textbf{M}^{-1}\textbf{r}_0$; $\textbf{x}_1 =\textbf{x}_0 +\Delta \textbf{x}_{0}$; $\textbf{r}_1 =\textbf{b} -\textbf{B}\textbf{x}_1$;
\WHILE{$k \leq k_{max}$ }
\STATE $k=k+1$;
\STATE $\omega_k = 1/(\gamma - \frac{1}{4\alpha^2}\omega_{k-1})$; \COMMENT{the iterated function}
\STATE $\textbf{r}'_{k} =\textbf{M}^{-1}\textbf{r}_{k}$; \COMMENT{preconditioning} \
\STATE $\Delta \textbf{x}_{k} =\omega_k\textbf{r}'_{k}+(\gamma \omega_k-1)\Delta \textbf{x}_{k-1}$;
\STATE $\textbf{x}_{k+1} =\textbf{x}_{k}+\Delta \textbf{x}_{k}$;
\STATE $\textbf{r}_{k+1} =\textbf{b}- \textbf{B}\textbf{x}_{k+1}$; \COMMENT{matrix--vector multiplication}
\STATE $update\_halo(\textbf{r}_{k+1})$; \COMMENT{boundary communication}
%\IF { $k \%  n_{c} == 0$ }
%\STATE check convergence;
%\ENDIF
\STATE $convergence\_check(\textbf{r}_{k+1})$;  \COMMENT{check convergence}
\ENDWHILE
\end{algorithmic}
\end{scriptsize}
\end{algorithm}


P-CSI requires approximations of the largest ($\mu$) and
smallest ($\nu$) eigenvalues of the preconditioned
matrix $M^{-1}A$. Coefficient matrix $A$ and its diagonal
preconditioner $M = \Lambda(A)$ are real symmetric positive definite
matrices in POP, therefore these extreme eigenvalues can be
estimated inexpensively.  As in \cite{hu2013scalable} for $A$, here we use the the Lanczos
method \cite{Paige1980235} for $M^{-1}A$, constructing a series of tridiagonal
matrices whose largest and smallest eigenvalues
converge to those of $M^{-1}A$.
%We use the same eigenvalue
%estimation for $M^{-1}A$ (no preconditioner was used in CSI) in
%\cite{hu2013scalable}.
%In particular, a series of tridiagonal
%matrices $T_m (m=1,2,...)$ whose largest and smallest eigenvalues
%converge to those of $M^{-1}A$ are constructed using the Lanczos
%method \cite{Paige1980235}.
In experiments, we found that setting the
Lanczos convergence tolerance $\epsilon$ to $0.15$ works efficiently
in both 1\degree\space and 0.1\degree\space POP with both diagonal
preconditioning and our new block  preconditioner.
Figure \ref{fig:iter} indicates that only a small number of Lanczos steps
are necessary to generate eigenvalue estimates of $M^{-1}A$ that result in near-optimal P-CSI
convergence.
%Generally less than 50 Lanczos steps are necessary to generate
%the estimated eigenvalue of $M^{-1}A$ that result in near-optimal P-CSI convergence.
In practice, the cost of the Lanczos method is similar to
calling the ChronGear solver a few times.


%We note that the overall convergence rate of P-CSI will necessarily be
%slower than that of PCG method (assuming the same preconditioner for
%both), and, as a result, PCSI requires a larger number of iterations
%($K_{pcsi} > K_{cg}$) to convergence.
In contrast with the ChronGear iteration,  the P-CSI iteration
requires no global reduction except for checking convergence.
% as shown in Figure \ref{fig:pcsi_pcg}.
We note that P-CSI requires a larger number of iterations than ChronGear ($K_{pcsi} > K_{cg}$)
in order to obtain the same convergence criteria.
We expect that this will translate into a
higher execution time for P-CSI than ChronGear at smaller core counts
when global reductions are not an issue.  However, for high-resolution
grids when many cores are required, P-CSI should be notably faster than
ChronGear per iteration (see Equations (\ref{t_pcg}) and
(\ref{t_psi})), which would result in a reduction in time to convergence.



%%IF WE NEED MORE SPACE WE CAN LEAVE OUT ALGORITHM 3
%convergence speed of P-CSI reaches its theoretical optimum when $\nu = \lambda_{min}$ and $\mu =\lambda_{max}$.
%Accurate values of $\lambda_{min}$ and $\lambda_{max}$ are difficult to obtain. In addition, any transformation of the coefficient matrix $A$ is ill-advised because $A$ was distributed to processes.
%To utilize the parallelism of POP, we employ Lanczos method  to construct
%In practice, we find that this theoretical optimum has a iteration number close to the one of PCG.


%\begin{algorithm}[!ht]
%\caption{Lanczos-based Eigenvalue Estimation for Preconditioned Matrix}
%\label{alg:lanczos_pre}
%\begin{algorithmic}[1]
%\REQUIRE Coefficient matrix $\textbf{B}$, preconditioner $\textbf{M}$, and random vector $\textbf{r}_0$ associated with grid block $B_{i,j}$; \\
% //\qquad    \textit{do in parallel with all processes}
%\STATE $\textbf{s}_0=\textbf{M}^{-1}\textbf{r}_0$;\quad $\textbf{q}_1 = \textbf{r}_0/({\textbf{r}_0^T\textbf{s}_0})$;\quad $\textbf{q}_0=\textbf{0}$;
%\STATE $T_0=\emptyset$;\quad $\beta_0 =0$;\quad  $\mu_0 =0$;\quad $j=1$;
%\WHILE{$j<k_{max}$}
%\STATE $\textbf{p}_j = \textbf{M}^{-1}\textbf{q}_j$; \quad $\textbf{r}_j=\textbf{B}\textbf{p}_j-\beta_{j-1}\textbf{q}_{j-1}$;
%\STATE $update\_halo(\textbf{r}_j)$;
%\STATE $\tilde{\alpha}_j =\textbf{p}_j^T\textbf{r}_j$; \quad $\alpha_j=global\_sum(\tilde{\alpha}_j)$;
%\STATE $\textbf{r}_j=\textbf{r}_j-\alpha_{j}\textbf{q}_{j}$; \quad $\textbf{s}_j = \textbf{M}^{-1}\textbf{r}_j$;
%\STATE $\tilde{\beta}_j = \textbf{r}_j^T\textbf{s}_j$; \quad $\beta_j=sqrt(global\_sum(\tilde{\beta}_j))$;
%\STATE \textbf{if} $\beta_j == 0$ \textbf{then} \textbf{return}
%\STATE $\mu_j = max(\mu_{j-1},\alpha_j+\beta_j+\beta_{j-1})$; \label{lan_gersh}\\
%\STATE $T_j=tri\_diag(T_{j-1},\alpha_j,\beta_j)$; \COMMENT{Tridiagonal}\label{lan_tm}
%\STATE $\nu_j = eigs(T_j,'smallest')$ ; \label{lan_nu}
%\STATE \textbf{if} $|\frac{\mu_j}{\mu_{j-1}} -1 |< \epsilon\quad\textbf{and}\quad|1- \frac{\nu_j}{\nu_{j-1}}|< \epsilon$ \textbf{then} \textbf{return}; \label{lanczos_converge}
%\STATE $\textbf{q}_{j+1}= \textbf{r}_j/\beta_j$;\quad $j=j+1$;
%\ENDWHILE
%\end{algorithmic}
%\end{algorithm}

%This algorithm assume that $\textbf{A}$ is positive defined symmetrical matrix.  However, it is simple to adjust the algorithm for negative defined matrix as in our case. Just negating both $\textbf{A}$ and its preconditioner.



%\section{Error Vector Propagation (EVP) Preconditioning} \label{se:evp}

%\textbf{I suggest you to make Fig. 5 to use a different color or label to represent the specified boundary conditions}
Figure \ref{fig:evp9p} illustrates a Dirichlet boundary
elliptic equation $\mathcal{B}\textbf{x} = \psi$ on a small domain.  We
define the interior points next to the south and west boundaries as
the initial guess points $\textbf{e}$ and those next to the north and
east boundaries are the final boundary points $\textbf{f}$ (e.g.,
$\textbf{e}= \{E_1, \dots, E_7\}$, $\textbf{f}= \{F_1, \dots, F_7\}$
in Figure \ref{fig:evp9p}).  If the true solution on $\textbf{e}$ is
known, the exact values over the whole domain can be computed
sequentially from southwest to northeast corners, using Equation
\ref{eq:evp9p}. This procedure is referred to as marching.
Unfortunately, the value on $\textbf{e}$ is often not known until the
elliptic equation is solved.  However, we can get a solution
$\textbf{x}$ satisfying the elliptic equation on the whole domain
except on the boundary, by first guessing the value
$\textbf{x}|_\textbf{e}$ on $\textbf{e}$ and then calculating the rest
using the marching method.  Then $E=(\textbf{x} -\eta)|_\textbf{e}$
and $F=(\textbf{x} -\eta)|_\textbf{f}$ are error vectors on
$\textbf{e}$ and $\textbf{f}$, respectively.  The error vector $F$ is
already known since $\textbf{f}$ are boundary points (Dirichlet
boundary condition is imposed).  The relationship between the error on
initial guess points and the final boundary points can be represented
as $F=W*E$.  This influence coefficient matrix $W$ can be formed by
marching on the whole domain with unit vectors on the initial guess
points and zero residual value in the whole domain.  We summarized
the EVP algorithm for an elliptic equation with zero boundary in
Algorithm \ref{alg:evp}.


The EVP method contains two steps: preprocessing and solving. In
the preprocessing step, the influence coefficient matrix and its
inverse are computed, involving a calculation of $\mathcal{C}_{pre}=
(2n-5)* 9n^2 + (2n-5)^3 = \mathcal {O} (26n^3)$.  Obtaining the solution in the solve step requires
$\mathcal{C}_{evp}= 2* 9n^2 + (2n-5)^2 = \mathcal{O} (22n^2)$.  This
estimate indicates that EVP has lower
computational cost for the solver step than other direct solvers such as
LU.
%What is the cost per iteration for PCG - didn't quite get the below.
%or iterative methods such as PCG for elliptic equations
Therefore, EVP can be practical in real applications from a cost standpoint because
preprocessing is only needed once at the beginning to obtain the
influence coefficient matrix and its inverse.
\begin{algorithm}[t!]
\caption{Nine-point Error Vector Propagation method}
\label{alg:evp}
\begin{scriptsize}
\begin{algorithmic}[1]
\REQUIRE Residual $\psi$ associated with a domain containing $n\times n$ grid points, $k = size(\textbf{e})=2n-5$; \\
//\qquad \textit{preprocessing }
\STATE  $\textbf{x} = \textbf{0}$
\FOR {i = 1, k}
\STATE $\textbf{x}|_\textbf{e}(i) = 1$
\STATE $\textbf{x} = marching(\textbf{x},\textbf{0})$
\STATE $W(i,:) = \textbf{x}|_\textbf{f}$
\STATE $\textbf{x}|_\textbf{e}(i) = 0$
\ENDFOR
\STATE $R = inverse(W)$ \\
//\qquad \textit{solving }
\STATE $\textbf{x}= marching(\textbf{x},\psi)$
\STATE $F = (\textbf{x} - \eta)|_\textbf{f}$
\STATE $\textbf{x}|_\textbf{e} =\textbf{x}|_\textbf{e} - R*F$
\STATE $\textbf{x} = marching(\textbf{x},\psi)$
\end{algorithmic}
\end{scriptsize}
\end{algorithm}

\subsection{EVP as a parallel preconditioner}
The EVP method described above is an efficient option for
solving elliptic equations.  However, a major drawback of
EVP is that it cannot solve on a large domain without further
modifications due to numerical instabilities when marching
\cite{roache1995elliptic}.  But on a small domain up to the size of
$12\times 12$, EVP solves with an acceptable round-off error of
$\mathcal{O}(10^{-8})$ when double-precision floating-point is used.  Its
effectiveness on small domains and low-computational cost make EVP an
ideal method for parallel block preconditioning.
%thus making it possible to use the block diagonal of $A$ as a preconditioner.
%\textbf{not sure what the above sentence means mean EVP combined with block diagonal?}
Here, we develop a block preconditioning technique based on the EVP
method in each block to further improve the performance of the
barotropic solver in POP.  Each preconditioner step solves the
elliptic equations $B_i \textbf{x} = \textbf{y} (i=1,...,m^2)$ in
parallel.


The fact that EVP is not well-suited for large domains is not an issue for
large-scale parallel computing, where larger number of processors
result in smaller domains.  Furthermore, for our system of equations,
the coefficients related to north, south, east and west neighbors on
every point are one magnitude order smaller than the others. We found
that removing these coefficients reduces the cost of
EVP preconditioning by about a half without any significant
impact on the convergence rate when used with both ChronGear and
P-CSI.  As a result, the execution time of EVP preconditioning can be
expressed as $\mathcal{T'}_{p} = 14n^2\theta=
14\frac{\mathcal{N}^2}{p}\theta$.  The actual cost of
$\mathcal{T'}_{p}$ depends on the size of the local block, which
decreases as more processor cores are used.
%In the current version of POP, using non-diagonal preconditioning methods requires an additional boundary communication after the preconditioning in each iteration.
%However, we find that the later boundary communication can be skipped by utilizing the fact that the halo size is 2.
Thus, the total execution time for one ChronGear and P-CSI solver step
with block-EVP preconditioning are
\begin{eqnarray}
\label{t_evppcg}
&\mathcal{T'}_{cg}=\mathcal{K'}_{cg} (\mathcal{T'}_c + \mathcal{T'}_b+\mathcal{T'}_g )\nonumber \\
&=\mathcal{K'}_{cg} [31 \frac{\mathcal{N}^2}{p}\theta + \frac{8\mathcal{N}}{\sqrt{p}}\beta +(4+\log p)\alpha],
\end{eqnarray}
and
\begin{eqnarray}
\label{t_evppsi}
\mathcal{T'}_{pcsi} = \mathcal{K'}_{pcsi}(\mathcal{T'}_c + \mathcal{T'}_b ) \nonumber \\
= \mathcal{K'}_{pcsi}[26\frac{\mathcal{N}^2}{p}\theta+ 4\alpha + \frac{8\mathcal{N}}{ \sqrt{p}}\beta],
\end{eqnarray}
respectively.

\begin {figure}[!t]
\centering
\includegraphics[height=6.5cm]{iteration.eps}
\vspace{-.1in}
\caption[] {Average number of iterations for different barotropic solvers. \label{fig:iteration}}
\vspace{-.1in}
\end{figure}

The implementation of EVP preconditioning in POP significantly reduces
the number of iterations required for convergence for both the ChronGear and P-CSI solvers.
In particular,  Figure \ref{fig:iteration} demonstrates that EVP
preconditioning reduces the iteration count by about two-thirds for
both the 1\degree\space and 0.1 \degree\space resolutions, which is comparable to the
approximate-inverse preconditioner proposed in
\cite{smith1992parallel} (which is not implemented in CESM POP).
Although  EVP preconditioning doubles the computation in each iteration, it halves both global and boundary communications
which dominate in the barotropic execution time at large core counts.
Another advantage of EVP preconditioning is the low
preprocessing cost.
%EVP preconditioning matrix is totally parallelized, while traditional decomposition based preconditioning requires serial processing.
In 0.1\degree\space case, the cost of setting up the preconditioning matrix
is less than that of one call to the solver when 512 processor cores
are used, and this cost is further decreased when more processors are used.
Finally, we note that the 0.1\degree\space case requires fewer iterations than the
1\degree\space case, because the higher resolution POP grid has a 
ratio of longitude to latitude grid spacing that is closer to 1, resulting
in a smaller condition number for the coefficient matrix.

\section{Experimental results} \label{se:exp}
%----------------------------------------------------------------------------
%\subsection{Experimental Platform} \label{se:plat}

We first evaluate the performance of our new barotropic solver in
CESM1.2.0 on the Yellowstone supercomputer, located at NCAR-Wyoming
Supercomputing Center (NWSC) \cite{loft:2015}. Yellowstone uses 
2.6-Ghz Intel Xeon E5-2670 ``Sandy Bridge'' processors providing a total of  72,576 cores that are connected
by a 13.6 GBps InfiniBand network.  Obtaining good performance on
Yellowstone is critical as its role is to support atmospheric sciences
and more than 50\% of its usage is due to CESM \cite{wf2014}
%Processors are 2.6-GHz Intel Xeon E5-2670 with Advanced Vector Extensions (AVX). 
To focus on the performance of POP, we use the CESM
``G\_NORMAL\_YEAR'' component set which uses active ocean and sea
ice components (the atmosphere component is data-driven). 
%ocean and ice with COREv2 normal year forcing is used.
We examine  the two most frequently-used POP horizontal grid resolutions: 
1\degree\space ($320\times 384$) and 0.1\degree\space ($3600\times 2400$).
Note that by default, CESM1.2.0 sets Yellowstone's MPI environment
eager limit (MP\_EAGER\_LIMIT), which controls the maximum message size before a rendezvous protocol is used,
to zero.  We discovered that by using the default eager limit on
Yellowstone (MP\_EAGER\_LIMIT = 131072) instead, POP performance significantly improves.


\subsection{Low-resolution simulations}
The execution times for the barotropic mode with available solvers in 1\degree\space POP on
Yellowstone are shown in Figure \ref{fig:runtime1}.  With the default
diagonal preconditioning, P-CSI out-performs ChronGear on all core
counts and reduces the solver execution time from 0.58s to 0.41s per
simulation day (1.4x speedup) at the highest core count (768).
Further, with the new
block EVP preconditioner, convergence is improved for both the ChronGear
and P-CSI solvers at higher core counts.  At 768 cores, P-CSI with EVP achieves 0.37s per simulation
day, which is an 1.6x improvement over the original ChronGear solver
with diagonal preconditioning.
\begin {figure}[!t]
\centering
\includegraphics[height =6.5cm]{NEW1deg_solverruntime}
\vspace{-.25in}
\caption []{Execution times for the barotropic mode in 1\degree\space POP
  for one simulation day.\label {fig:runtime1}}
\vspace{-.1in}
\end {figure}

\begin{table}[!h]
\begin{center}
\caption{Percent improvement of the total execution time for
  1\degree\space POP on Yellowstone. \label{tab:improve_1}}
\begin{scriptsize}
\begin{tabular}{|l||l|l|l|l|l|}
\hline
Number of cores & 48  & 96  & 192 & 384 & 768\\\hline
\hline
ChronGear+EVP & -.5\% & 1.1\%  & 6.5\% & 10.8\%  & 12.1 \% \\\hline
P-CSI+Diagonal  & .7\% &3.9\% &9.3\%  &11.0\% & 12.6 \% \\\hline
P-CSI+EVP	      &-2.4\% & .4\%	& 7.4\%  & 14.4\% & 16.7\%\\\hline
\end{tabular}
\end{scriptsize}
\vspace{-.2in}
\end{center}
\end{table}


The improvement of the barotropic solver reduces the total execution time for the entire
POP model.  Table \ref{tab:improve_1} lists the percentage
improvement of POP for the three new solver/preconditioner options
compared to POP with the diagonal-preconditioned ChronGear solver. 
Times were obtained from a 5-day simulation, with model initialization
and I/O excluded. P-CSI with a block-EVP preconditioner yields a 16.7\% improvement on 768 processor cores.  
While a 16.7\% improvement may seem modest, POP at 1\degree\space resolution is commonly run for multi-century timescales.  Such an improvement may translate into the saving of millions of CPU hours.  Further,
the 1\degree\space resolution needs to be run at (relatively) high core counts
when POP is configured with biogeochemistry mode due to the many additional
tracers required.

\subsection{High-resolution simulations}
\begin {figure*}[t!]
\begin{center}
\includegraphics[height =6.5cm]{NEW01deg_solverruntime}
\hspace{10pt}
\includegraphics[height =6.5cm]{NEW01deg_speedup_ys}
\end{center}
\vspace{-.2in}
\caption []{Execution times for the barotropic mode in 0.1\degree\space POP
  for one simulation day on Yellowstone (left). 
The core simulation rates of 0.1\degree\space POP on Yellowstone (right).\label {fig:runtime01}}
%\vspace{-.2in}
\end {figure*}


Now we test the scalability of the new barotropic solver in
high-resolution 0.1\degree\space POP on Yellowstone.  At this resolution,
the choice of ocean block size and layout, which affects the
distribution of work across processors, has a large impact on
performance.  Therefore, to remove this influence from our scaling
results, we were careful to specify block decompositions for each core
count with the same aspect ratio (3:2) and land ratio (.25) and to use
space-filling curves.  We use the default timestep for
0.1\degree\space POP, which is 500 time steps per day (dt\_count = 500).
Finally, for the sake of consistency, for all solvers we checked for convergence every 10 iterations.
Note that because P-CSI iterations are relatively inexpensive
(compared to performing the POP  convergence check), 
P-CSI performance may improve if the check for convergence occurs less frequently.


As shown in Figure \ref{fig:runtime01} (left),
ChronGear performance begins to degrade after about 2700 cores,
while the execution time for P-CSI becomes relatively flat at that
point. With diagonal preconditioning, P-CSI accelerates the barotropic
mode in 0.1\degree\space POP by 4.3x (from 19.0s to 4.4s per simulation
day) on 16,875 cores.  EVP preconditioning further improves the
performance of both ChronGear and P-CSI, resulting in a speedup of the
original barotropic mode by 1.4x and 5.2x, respectively.
In Section \ref{se:baro}, we demonstrated that the original barotropic
solver takes an increasing percentage of POP execution time as the
number of cores increases. In particular, on 16,875 cores, ChronGear
with diagonal preconditioning accounts for about 50\% of the total
execution time.  In contrast, Figure \ref{fig:StepComp_pcsi}
illustrates the improvement of the barotropic mode with the more-scalable EVP preconditioned
P-CSI solver, which constitutes only about 16\% of the total execution
time on 16,875 cores. 


Improvement of the barotropic solver benefits the
overall performance of POP, especially at large core counts. 
Simulation rate (simulated years per wall-clock day) is a popular
criterion for climate model performance, and here we use the core
simulation rate (i.e., the execution time excluding initialization and I/O costs).
%Figure \ref{fig:runtime01} (right) shows that diagonal preconditioned and EVP
%preconditioned P-CSI solver accelerate the entire POP simulation by
%1.66x and 1.74x, respectively.
A simulation rate of 5 simulated years per wall-clock day is considered
the minimum rate required to run long term climate simulations
\cite{dennis2012computational}, and Figure \ref{fig:runtime01} (right)
shows that P-CSI can attain higher rates than ChronGear.
The EVP-preconditioned P-CSI solver improves the core simulation rate of
POP by 1.7x on 16,875 cores, from 6.2 to 10.5 simulated years per wall-clock day.

\begin {figure}[t!]
\centering
%\vspace{-10pt}
%\includegraphics[width=1.0\linewidth]{POPStepComp_pcsi.eps}
\includegraphics[height=6.5cm]{NEWPOPStepComp_pcsi.eps}
\caption[] {Percentage of execution time in 0.1\degree\space POP using
  P-CSI with block-EVP preconditioning.\label{fig:StepComp_pcsi}}
\end{figure}




To illustrate the source of improvement, more detailed timing
information for the barotropic solvers is provided in Figure
\ref{fig:component}.  Figure \ref{fig:component} indicates that P-CSI
outperforms ChronGear primarily due to fewer global reductions. The
reduction in global reductions will also significantly reduce the
sensitivity of POP to operating system noise \cite{ferreira} by increasing the time between global synchronization.  In
addition, the block-EVP preconditioner reduces the boundary update
costs by reducing the number of iterations required.  Computation
costs for the barotropic solver are negligible compared to the global
reduction and boundary update costs at large core
counts, and, therefore, the extra computations (almost double)
required by the EVP preconditioner have little to no impact.  Finally, note that the global reduction time actually decreases at less than 1200 cores which is consistent with the theoretical results in equation \ref{t_pcg} and \ref{t_psi}.

%Finally,
%note
%that the global reduction time decreases at the smallest core counts
%(less than 1200), which conflicts with the theoretical result in
%equation \ref{t_pcg} and \ref{t_psi}.
%This behavior can be attributed to a load imbalance. 
%When fewer numbers of cores are used in POP, 
%the specified block size is larger on each process. 
%When the blocks containing only land are eliminated, 
%the bigger block sizes lead to greater load imbalances between cores.

%Note that for smaller core counts (e,g., less than 4,000) global
%reduction and boundary communication times do not decrease with the
%EVP preconditioner (instead of diagonal) for P-CSI.  
%At small core counts, the EVP
%preconditioner exacerbates this load imbalance because of the near
%doubling of computation time.  Thus, even though the iteration number
%is reduced to a half in the case of EVP preconditioned P-CSI, each
%communication takes longer to complete.


\subsection{Simulations on Edison}
Now we run the 0.1\degree\space POP simulations on the Edison supercomputer
to verify that performance improvements are not unique to Yellowstone.
Edison, which is the newest supercomputer at the National Energy
Research Scientific Computing Center (NERSC), consists of 133,824 2.4
GHz Intel ``Ivy Bridge'' processor cores connected by an 8GBps Cray
Aries high-speed interconnect with Dragonfly topology.
% and has a peak performance of 2.57 PFLOPS.
\begin {figure*}[t!]
\begin{center}
\includegraphics[height =7cm]{01deg_comp_all_gs}
\hspace{10pt}
\includegraphics[height =7cm]{01deg_comp_all_halo}
\end{center}
\vspace{-.2in}
\caption[] {Execution times for the major components of the
  barotropic solvers in 0.1\degree\space POP on Yellowstone:  global
  reduction (left) and boundary communication (right). }
\label{fig:component}
%\vspace{-.2in}
\end {figure*}
\begin {figure*}[t!]
\begin{center}
\includegraphics[height=6.5cm]{01deg_solverruntime_edison}
\hspace{10pt}
\includegraphics[height=6.5cm]{01deg_speedup_edison}
\end{center}
\vspace{-.2in}
\caption []{Execution times for the barotropic mode in  0.1\degree\space POP
  for one simulation day on Edison (left). 
The core simulation rate of 0.1\degree\space POP on Edison (right).\label {fig:runtime01_edison}}
\vspace{-.2in}
\end {figure*}

Figure \ref{fig:runtime01_edison} shows that simulations on Edison
with the four solver configurations have similar performance
characteristics as on
Yellowstone. 
%ChronGear stops scaling after 2,700 cores, while P-CSI scales well until 4,220 cores are used. 
We note that we encountered much more variability in the global
communication times in our simulations on Edison (as compared to
Yellowstone), likely due to network contention \cite{wang2014}. As a result, the
ChronGear times (with both preconditioners) varied a lot from run to
run, so we took the average of the best three results to represent the
execution time.  Because P-CSI has hardly any global reductions
(only in the convergence check),
the variability in those runs was small. 
%(and on par with the Yellowstone runs).
%It is worth mentioning that global communication is more unstable on
%Edison than on Yellowstone. 
%As a result, the execution time of ChronGear with either a diagonal
%preconditioner or an EVP preconditioner varies a lot from run to run. 
%Here we use the average of the best three results to represent the execution time. 
On Edison,  P-CSI with diagonal preconditioning in  0.1\degree\space POP
accelerates the barotropic mode by 3.7x (from 26.2s to 7.0s per simulation day) on 16,875 cores.
With EVP preconditioning, both ChronGear and P-CSI performance
improves, and the combination of P-CSI and EVP preconditioning results in a 5.6x  speedup. 
 
%\begin{table}
%%\vspace{-10pt}
%\begin{center}
%\begin{scriptsize}
%\caption {Core simulation rates obtained for 0.1\degree\space POP, with the
%  preconditioned ChronGear (CG) and P-CSI solvers. \label{tab:improve_01}}
%\begin{tabular}{|l||r|r|r|r|r|r|r|}
%%\toprule
%\hline
%Solver & 470  & 1200   & 2700 & 4220 & 7500 & 10800 & 16875\\\hline
%\hline
%%CG     &0.71 &1.68&3.38  &4.62 &6.02 &6.07 &5.34\\\hline
%%CG+Evp &0.70 &1.69&3.44  &4.88 &6.62 &6.89 &6.46\\\hline
%%P-CSI+Diag    &0.72 &1.72&3.51  &5.00 &7.04 &8.29 &8.85\\\hline
%%P-CSI+Evp     &0.70 &1.69&3.49  &5.01 &7.23 &8.55 &9.27\\
%CG     &0.7 &1.7&3.4  &4.6 &6.0 &6.0 &5.3\\\hline
%CG+Evp &0.7 &1.7&3.4  &4.9 &6.6 &6.9 &6.5\\\hline
%P-CSI+Diag    &0.7 &1.7&3.5  &5.0 &7.0 &8.3 &8.9\\\hline
%P-CSI+Evp     &0.7 &1.7&3.5  &5.0 &7.2 &8.6 &9.3\\
%\hline
%%\bottomrule
%\end{tabular}
%\end{scriptsize}
%\vspace{-.2in} 
%\end{center}
%\end{table}

\section{Evaluating the new solver} \label{se:ver}
%Currently, there is no standard utility in POP for evaluating the effect of code modifications (such as a new solver) that will yield non bit-for-bit (BFB) results but should still produce the same mean climate.
Due to the chaotic nature of the ocean dynamics, even a round-off difference from the barotropic solver may potentially result in
distinct model solutions.
Therefore, because we cannot guarantee bit-for-bit (BFB) identical results in ocean solutions when a new solver is introduced,
we needed to show that the use of P-CSI with EVP did not result in inaccuracies (or even a changed climate)
before it could be formally incorporated into a POP release.

When POP is ported to a new machine, a similar situation occurs where running the same simulation on the two machines is not expected to produce BFB results.
The existing POP procedure to verify that a port to a new machine was successful involves running a specific case on the new machine for five simulation days, and
then computing the root-mean-square error (RMSE) between the new solution and the standard dataset released by NCAR for the SSH (sea surface height) field.
%There is no direct verification tool for new solvers  in CESM POP currently, but it provides a way to facilitate the evaluation of a successful port on new machines.
%That is, to run a specific case on the new machine for five days, then compute the root-mean-square (RMS) difference of SSH field between the solution on local machine and those released as standard dataset by NCAR.
While this procedure provides a simple criterion for evaluating CESM results on new machines (which may contain errors due to the software or hardware environment),
we found that it was insufficient for detecting and evaluating solver-induced errors.
For example, we ran the 1\degree\space case for three years with different convergence tolerances varying from $10^{-10}$ to $10^{-16}$ in the barotropic solver (default is $10^{-13}$)
and calculated the RMSE between a given case and the most strict tolerance case ($10^{-16}$).
Figure \ref{fig:ssh_rmse_t} shows the RMSE for the temperature field with various convergence tolerances for each month, and clearly
error introduced by modifying the solver convergence tolerance is not revealed in the temperature field (nor was it evident in any of the other diagnostic fields, such as velocity and SSH).
We had expected that the simulations with tolerances of $10^{-10}$ and $10^{-11}$ would have larger RMSE values than the others.
However, this was not the case, and, during months twelve and twenty, the $10^{-10}$ case has almost the smallest RMSE.
%In the last two months, the $10^{-11}$ case has a smaller RMSE than all other cases except the $10^{-10}$ case.
Note that to isolate the effect of the linear solver, we only looked at error in the open seas (POP does not simulate well on several marginal seas).


%While RMSE can reveal an averaged difference between two cases, it is not sufficient to determine whether that difference is indicative of an altered climate.
Because the existing simple RMSE test was insufficient for detecting whether the climate had been altered, we developed an alternative to evaluating the new ocean solver using a statistical approach.  Rather than relying on a single simulation, an ensemble of simulations can better represent the natural variability of the chaotic climate simulation, as described in \cite{baker2014methodology} in the context of data compression for the CESM Community Atmosphere Model (CAM), and be
used as a baseline for evaluating non-BFB modifications.  Similar to \cite{baker2014methodology}, we create an ensemble of simulations which are identical to the default setup except for an order $10^{-14}$ perturbation in the initial ocean temperature. This perturbation size is not expected to produce different climate model states.  We found that an ensemble of size 40 was sufficient for our purposes to represent the variability in the ocean, and we ran longer simulations than for CAM (12-months) due to the longer time-scales present in the ocean. Also note that we ultimately chose to evaluate only the three-dimensional temperature field (instead of the two-dimensional SSH) as we found it to be the most useful diagnostic variable for revealing differences.





%However, it is not a good criterion for new algorithms because it does not take into consideration of the chaotic nature of climate models.
%In order to make the chaotic nature accounted, we employ the methodology as proposed in CESM atmosphere component CAM by Baker et al. \cite{baker2014methodology}. First we conduct an ensemble of runs which are identical to the original case except for a $10^{-14}$ perturbation in the initial temperature. This perturbation is a reasonable round-off error which climate model should be able to tolerate.

We determine whether the new result is consistent with the reference ensemble results as follows.
We define the ensemble output at time $T$ as $\mathcal{E}=\{X_1,X_2,...,X_m\}$, where
$m$ is the size of the ensemble.
At a given point $j$, we have a series of possible results for each variable $X$ from the ensemble $\{X_1(j),X_2(j),...,X_m(j)\}$.
As the ensemble size increases, this series more correctly reflects the distribution of reasonable realization at the given point.
We define the mean and standard deviation of this series at point $j$ as $\mu (j) $ and
$\delta (j)$, respectively.
 %as  $$ \mu (j) = \frac{1}{m}\sum_{i=1}^m X_i(j), $$
%and standard deviation as  $$ \delta (j) = \sqrt{\frac{1}{m} \sum_{i=1}^m (X_i(j)-\mu(j))^2 }.$$
Let the new case have the result $\tilde{X}$, 
then the root-mean-square Z-score indicates the average error between the new case and the ensemble data:

$$ RMSZ(\tilde{X}, \mathcal{E}) =  \sqrt{\frac{1}{n}\sum_{j=1}^n(\frac{\tilde{X}(j) -\mu (j)}{\delta (j)})^2}$$
%The combination of $\mu$ and $\delta$ provides a criterion to test whether an additional case is close to the ensemble or not.
%Set the additional case has the result $\tilde{x}$, define the root-mean-square Z-score
We then re-evaluated the various solver tolerances using the ensemble-based RMSZ measurement.
Figure \ref{fig:ssh_rmsz_t} indicates that, unlike the simple RMSE test, the new ensemble-based method
is able to identify larger errors due to less strict convergence tolerances.  Now
the two cases with the loosest tolerances clearly have RMSZ scores on the same order as the error
they introduced into the solver and are noticeably removed from the ensemble distribution.
This success led us to use the ensemble-based metric to evaluate our new solver and find that the P-CSI results were consist with those of the ensemble (as were the default and stricter tolerances).

\begin{figure}[!t]
\begin{center}
\includegraphics[height=6.5cm]{temp_rmse.eps}
\end{center}
\vspace{-.1in}
\caption[] {Monthly Root Mean Square Error (RMSE) of temperature for cases with different convergence tolerances in 1\degree\space POP.}
\label{fig:ssh_rmse_t}
\vspace{-.1in}
\end{figure}
\begin{figure}[!t]
\begin{center}
\includegraphics[height=6.5cm]{temp_rmsz.eps}
\end{center}
\vspace{-.1in}
\caption[] {Monthly Root Mean Square Z-score for temperature for cases with different convergence tolerances. The yellow area represent the range of RMSZ within the 40-member ensemble.}
\vspace{-.1in}
\label{fig:ssh_rmsz_t}
\end{figure}


 %cases provides a criterion to judge whether the case  is consistent with the them or not.
%Also, for those cases which depart so far away from the ensemble, such as the cases with the first and second largest tolerance, their RMSZ is in the same order as the order of error they introduced to the solver.
%So the error introduced by replacing ChronGear with PCG is in the same magnitude order of improving the convergence tolerance from $10^{-13}$ to $10^{-14}$ or $10^{-16}$.

\section{Related work} \label{se:rel}
%----------------------------------------------------------------------------
%improving barotropic

We briefly review related work in two categories: general efforts to improve parallel CG performance and efforts specific to ocean modeling. 
In the first category, reducing global communication costs for CG has been of interest since the algorithm was parallelized.
A particularly nice overview of this effort can be found in \cite{ghysels2014}.   Methods that reduce the number of global reductions over the standard formulation, such as the ChronGear \cite{dAzevedo1999lapack} variant used in POP, were early contributions to the field and still popular.  Early s-step methods such as that in \cite{chron1989} as well as more recent incarnations (e.g.,  \cite{hoemmen2010}) also reduce global communications, but combining them with a sophisticated preconditioner is non-trivial. In addition, recent efforts at improving the performance of parallel CG include a variant that overlaps the global-reduction with the matrix-vector computation via a pipelined-approach \cite{ghysels2014}. 
We take a different tack along the lines of \cite {gutknecht2002chebyshev} in that we abandon the CG algorithm and replace it by a simpler iterative method that doesn't include global reductions.

Particular to ocean models, a number of efforts have been made to reduce solver communication overhead. In \cite{Worley:2011:PCE:2063384.2063457}, the use of OpenMP parallelism in the barotropic mode is shown to improve performance at large core counts.
Land elimination is another common strategy for reducing communication overhead, and in
\cite{dennis2007inverse,dennis2008scaling}, space-filling curves both improve load-balancing and reduce the number of processes involved in communications by eliminating land blocks.
Further, early attempts to overlap communication with computation 
for parallel ocean general circulation models are proposed in \cite{beare1997optimisation}, as well as 
methods to reduce communication by increasing halo sizes. Although all these approaches may improve performance, they do not eliminate the global reduction bottleneck. 
In fact, the promising preliminary results in our previous work \cite{hu2013scalable}, obtained by replacing CG with a Chebyshev type method in the stand-alone variant of POP, encouraged the work in this manuscript.

Finally, our switch away from CG required the development of an effective preconditioner for the barotropic mode.  
We mention a couple of preconditioning strategies that have been explored to reduce barotropic solver costs on high-resolution grids.  
Polynomial-preconditioning and local appr\-oximate-inverse methods are shown to accelerate CG convergence in a parallel ocean general circulation model in \cite{smith1992parallel}. 
More recently, an incomplete Cholesky preconditioner was added to the Max Planck Institute ocean model (MPIOM) to improve CG performance on large core counts \cite{adamidis2011high}. 



% Preconditioning has been highlighted in the CG method since the 1990s. Many linear systems converge after a few PCG iterations with a suitable preconditioner.  
% However, many of the most effective preconditioning techniques, such as Incomplete Cholesky decomposition and incomplete LU decomposition, are not so effective in ocean models.  
% Parallelizable and special designed preconditioners are required for elliptic equations in ocean models. 
% In 1985, Concus et al. \cite{concus1985block} used the banded approximate of the matrix inverse to precondition CG method on elliptic partial differential equations and achieved higher efficiency than other universal preconditioning method. 
% Smith et al. \cite{smith1992parallel} employed polynomial preconditioning methods and a local approximate-inverse preconditioning method to accelerate the convergence of CG method in a parallel ocean general circulation model. 
% Adamidis et al. \cite{adamidis2011high} implemented an incomplete Cholesky preconditioner in the global ocean/sea-ice model MPIOM to improve the scalability and performance of PCG.
% %Watanabe \cite{Watanabe2006pcg}  used  PCG combined with an overlapping domain decomposition method to improve the convergence and reduce the communications cost between the processor elements.



% Much work has been done on optimizing the performance of solving the elliptic problem required by the implicit free-surface method in ocean models.  
% Most of them fall into two directions.  
% The first is related to decreasing the negative effects of the global communication required by PCG or ChronGear methods.  In ocean models, OpenMP parallelism and land elimination are common strategies to reduce the number of processes and the associated  global communication overhead. Worley et al. \cite{Worley:2011:PCE:2063384.2063457} strongly recommended the OpenMP strategy to reduce the number of processes when a large count of cores are needed for the baroclinic mode.
% Dennis \cite{dennis2007inverse,dennis2008scaling} proposed a load-balancing strategy based on the space-filling curve partitioning algorithms to eliminate land blocks.  This strategy not only reduces the number of processes but also leads to a better load-balance.
% It doubles the simulation rate on approximately 30,000 processors.    
% %MOVE TO RELATED WORK
% There are currently different alternatives to mitigate the poor behavior of the PCG type of solver in the massive parallelization.
% Some approaches attempt to overlap the communication with computation time\cite{beare1997optimisation}.
% Some use the land elimination and load-balance strategies \cite{dennis2007inverse, dennis2008scaling}
% to reduce the number of processes and the associated overhead of global reduction.
% Reducing the frequency of communication also attenuates the overhead in the barotropic mode.
% As early as 1997,  Beare \cite{beare1997optimisation} proposed the performance of parallel ocean general circulation models can be improved by increasing the number of extra halos and overlapping the communications with the computation.
% Although these approaches may improve performance, they do not eliminate the major bottleneck of the global reduction.

%A variant of the standard conjugate gradient method presented by D'Azevedo \cite{dAzevedo1999lapack}, called the Chronopoulos-Gear algorithm, proposed a way to halve the global communication in PCG.  It combines the two separate global reductions into a single global reduction vector by rearranging the conjugate gradient computation procedure, and achieves a one third latency reduction in POP.
% Another way to attenuate the bottleneck of the barotropic solver is preconditioning.
% Preconditioning has been highlighted in the CG method since the 1990s. Many linear systems converge after a few PCG iterations with a suitable preconditioner.  
% However, many of the most effective preconditioning techniques, such as Incomplete Cholesky decomposition and incomplete LU decomposition, are not so effective in ocean models.  
% Parallelizable and special designed preconditioners are required for elliptic equations in ocean models. 
% In 1985, Concus et al. \cite{concus1985block} used the banded approximate of the matrix inverse to precondition CG method on elliptic partial differential equations and achieved higher efficiency than other universal preconditioning method. 
% Smith et al. \cite{smith1992parallel} employed polynomial preconditioning methods and a local approximate-inverse preconditioning method to accelerate the convergence of CG method in a parallel ocean general circulation model. 
% Adamidis et al. \cite{adamidis2011high} implemented an incomplete Cholesky preconditioner in the global ocean/
%sea-ice model MPIOM to improve the scalability and performance of PCG.
%Watanabe \cite{Watanabe2006pcg}  used  PCG combined with an overlapping domain decomposition method to improve the convergence and reduce the communications cost between the processor elements.

% This paper represents a method which improves the barotropic solver by both strategies mentioned above.  P-CSI solver eliminates the global communication which is required by CG like solvers. In the meantime, it supports an EVP preconditioner which accelerates the convergence with an efficiency comparable to other preconditioning methods developed for ocean models. 

%The improvement of the methods described above is limited due to the inherent poor data locality and sequential execution of PCG. 
%Some work has been done to accelerate the PCG solver by employing the developing hybrid  accelerating devices, such as GPUs \cite{cuomo2012pcg} and FPGAs \cite{Shida2007}.
%Cuomo et al. \cite{cuomo2012pcg} introduced the sparse approximate inverses preconditioning method into the numerical global circulation ocean model and implemented it on a GPU using a scientific computing code library.
%Shida et al. \cite{Shida2007} moved the barotropic mode onto FPGAs, and found comparative performance on 100MHz FPGAs as on GHz processors with the appropriate use of internal memory and streaming DMA.
%GPUs and FPGAs are helpful in reducing the global overhead. These devices have stronger computational ability and more memory than common CPU, so fewer devices and less communication are needed for the same scale computing job.


%----------------------------------------------------------------------------
\section{Conclusion} \label{se:conc}

 
由于海洋模式POP中的ChronGear正压求解在大核数上的不良表现, 高分辨的公共地球系统模式CESM的可扩展性也不太好。
这篇文章通过采用一个需要更少全局归约操作的新的求解器和一个专门为正压模态设计的基于误差向量传播方法的块预处理子,最终改进了求解器的性能。 
我们评估了这个最终的求解器,也就是采用EVP块预处理的P-CSI方法,在两个经常用来运行公共地球系统模式的超级计算机上的性能,并且发现新的求解器相对于原始的求解器有了高达5倍的性能提升。
同时,我们通过海洋模式POP中的一个基于集合模拟的一致性检验工具证明了我们的新的求解器不会影响海洋模式的模拟结果。 
这也使得我们的新的求解器最终可以出现在公共地球系统模式的下一个版本中。 
新的求解器将会有助于未来的低分辨率和高分辨率公共地球系统模式的模拟, 尤其是过去被海洋模式POP分量影响到可扩展性的全耦合模式。 

% Low resolutions will also benefit as they are typically run for very long time periods.


% Even though the new barotropic solvers have more benefit in high
% resolution simulations, it also contributes to the low resolution
% simulations.  The P-CSI solver is verified to maintain a stable ocean
% climate within a small tolerance using an ensemble based statistical
% method.  In closing, this paper highlights a scalable and robust
% barotropic solver for free-surface ocean models.




%ACKNOWLEDGMENTS are optional
\section{Acknowledgments}
Computing resources were provided by the Climate
Simulation Laboratory at NCAR's Computational and Information Systems
Laboratory (sponsored by the NSF and other
agencies) and the National Energy Research Scientific Computing
Center, a DOE Office of Science User Facility supported by the Office
of Science of the U.S. Department of Energy under Contract
No. DE-AC02-05CH11231.

This work is supported in part by a grant from the National Natural Science Foundation
of China (41375102), the National Grand Fundamental Research 973 Program of China (No. 2014CB347800), and the National High Technology Development Program of China
(2011AA01A203).


%Generated by bibtex from your ~.bib file.  Run latex,
%then bibtex, then latex twice (to resolve references)
%to create the ~.bbl file.  Insert that ~.bbl file into
%the .tex source file and comment out
%the command \texttt{{\char'134}thebibliography}.
\bibliographystyle{abbrv}
\bibliography{hycs}  % sigproc.bib is the name of the Bibliography in this case
% This next section command marks the start of
% Appendix B, and does not continue the present hierarchy
%\section{More Help for the Hardy}
%The sig-alternate.cls file itself is chock-full of succinct
%and helpful comments.  If you consider yourself a moderately
%experienced to expert user of \LaTeX, you may find reading
%it useful but please remember not to change it.
%\balancecolumns % GM June 2007
% That's all folks!
\end{document}
