%
% ================= IF YOU HAVE QUESTIONS =======================
% Questions regarding the SIGS styles, SIGS policies and
% procedures, Conferences etc. should be sent to
% Adrienne Griscti (griscti@acm.org)
%
% Technical questions _only_ to
% Gerald Murray (murray@hq.acm.org)
% ===============================================================
%
% For tracking purposes - this is V2.0 - May 2012

\documentclass{sig-alternate-05-2015}
\usepackage{gensymb}  %format of algorithm
\usepackage{algorithm}  %format of algorithm
\usepackage{algorithmic}  %format of algorithm
\renewcommand{\algorithmiccomment}[1]{ \hfill {/* #1 */} }
\usepackage{graphicx}
\usepackage{epstopdf}



\DeclareGraphicsExtensions{.eps,.mps,.pdf,.jpg,.PNG}
\DeclareGraphicsRule{*}{pdf}{*}{}
\graphicspath{{./fig/}}

\usepackage{color}
\newcommand{\hl}[1]{\textcolor{red}{#1}}

\newcommand{\superscript}[1]{\ensuremath{^{\textrm{#1}}}}
\def\sharedaffiliation{\end{tabular}\newline\begin{tabular}{c}}
\def\thua{\superscript{\dag}}
\def\thub{\superscript{\ddag}}
\def\ncara{\superscript{*}}
\def\ncarb{\superscript{\P}}

\begin{document}
%
% --- Author Metadata here ---
\setcopyright{usgovmixed}
\conferenceinfo{SC '15,}{November 15-20, 2015, Austin, TX, USA}
\isbn{978-1-4503-3723-6/15/11}\acmPrice{\$15.00}
\doi{http://dx.doi.org/10.1145/2807591.2807596}


%\conferenceinfo{SC'15}{November 15-20, 2015, Austin, Texas, USA.}
%\CopyrightYear{2007} % Allows default copyright year (20XX) to be over-ridden - IF NEED BE.
%\crdata{0-12345-67-8/90/01}  % Allows default copyright data (0-89791-88-6/97/05) to be over-ridden - IF NEED BE.
% --- End of Author Metadata ---

%\title{Alternate {\ttlit ACM} SIG Proceedings Paper in LaTeX
%Format\titlenote{(Produces the permission block, and
%copyright information). For use with
%SIG-ALTERNATE.CLS. Supported by ACM.}}
%\subtitle{[Extended Abstract]
%\titlenote{A full version of this paper is available as
%\textit{Author's Guide to Preparing ACM SIG Proceedings Using
%\LaTeX$2_\epsilon$\ and BibTeX} at
%\texttt{www.acm.org/eaddress.htm}}}

\title{Improving the Scalability of the Ocean Barotropic Solver \\in the Community Earth System Model}
%
% You need the command \numberofauthors to handle the 'placement
% and alignment' of the authors beneath the title.
%
% For aesthetic reasons, we recommend 'three authors at a time'
% i.e. three 'name/affiliation blocks' be placed beneath the title.
%
% NOTE: You are NOT restricted in how many 'rows' of
% "name/affiliations" may appear. We just ask that you restrict
% the number of 'columns' to three.
%
% Because of the available 'opening page real-estate'
% we ask you to refrain from putting more than six authors
% (two rows with three columns) beneath the article title.
% More than six makes the first-page appear very cluttered indeed.
%
% Use the \alignauthor commands to handle the names
% and affiliations for an 'aesthetic maximum' of six authors.
% Add names, affiliations, addresses for
% the seventh etc. author(s) as the argument for the
% \additionalauthors command.
% These 'additional authors' will be output/set for you
% without further effort on your part as the last section in
% the body of your article BEFORE References or any Appendices.

% \numberofauthors{6}
% \author{
% % 1st. author
% \alignauthor
% Yong Hu\titlenote{
% Ministry of Education Key Laboratory for Earth System Modeling, Center for Earth System Science,
% and Joint Center for Global Change Studies, Beijing, 100875, China. }
% \\
% %  \affaddr{Department of Computer Science \& Technology }\\
%  % \affaddr{Center for Earth System Science } \\
%   \affaddr{Tsinghua University} \\
%   \affaddr{Beijing, China} \\
%   \email{huyong11@tsinghua.edu.cn}
% % 2nd
% \alignauthor
% Xiaomeng Huang\raisebox{9pt}{$\ast$} \\
% %\affaddr{Center for Earth System Science } \\
%  \affaddr{Tsinghua University} \\
%   \affaddr{Beijing, China} \\
%   \email{hxm@tsinghua.edu.cn}
% %3rd
% \alignauthor
% Allison H. Baker\\
% %\affaddr{Computational and Information Systems Laboratory}\\
%    \affaddr{National Center for Atmospheric Research}\\
%    \affaddr{Boulder, CO}\\
%    \email{abaker@ucar.edu}
% %more
% \and
% \alignauthor
% Yu-heng Tseng\\
%  % \affaddr{Climate and Global Dynamics Division} \\
%   \affaddr{National Center for Atmospheric Research}\\
%   \affaddr{Boulder, CO}\\
%   \email{ytseng@ucar.edu}
% \alignauthor
% Frank O. Bryan\\
% % \affaddr{Climate and Global Dynamics Division} \\
%   \affaddr{National Center for Atmospheric Research}\\
%   \affaddr{Boulder, CO}\\
%   \email{bryan@ucar.edu}
% \alignauthor
% John M. Dennis\\
% % \affaddr{Climate and Global Dynamics Division} \\
%   \affaddr{National Center for Atmospheric Research}\\
%   \affaddr{Boulder, CO}\\
%   \email{dennis@ucar.edu}
% \alignauthor
% Guangwen Yang\raisebox{9pt}{$\ast$}\\
% % \affaddr{Center for Earth System Science } \\
% \affaddr{Tsinghua University} \\
% \affaddr{Beijing, China} \\
% \email{ygw@tsinghua.edu.cn}
% }


\numberofauthors{1}
\author{
\alignauthor
Yong Hu\thua, Xiaomeng Huang\thua, Allison H. Baker\ncarb,\\Yu-heng Tseng\ncara, Frank O. Bryan\ncara,  John M. Dennis\ncarb, Guangwen Yang\thua
\sharedaffiliation
  \begin{tabular}{ccc}
    \affaddr{{\thua}  {Center for Earth System Science\ }} & & \affaddr{{\ncara}Climate and Global Dynamics Division {\ }} \\
    \affaddr{Tsinghua University, 100084, and  {\ } } & & \affaddr{{\ncarb}Computational and Information Systems Laboratory{\ }} \\
      \affaddr{ Joint Center for Global Change Studies {\ } } & & \affaddr{National Center for Atmospheric Research} \\
    \affaddr{Beijing, 100875, China} & & \affaddr{Boulder,CO. USA} \\
   \email{\{huyong11,hxm,ygw\}@tsinghua.edu.cn}  & & \email{\{abaker,ytseng,bryan,dennis\}@ucar.edu} \\
  \end{tabular}
}


% \numberofauthors{1}
% \author{
% \alignauthor
% Yong Hu\thua\thub,Yu-heng Tseng\ncara,Allison H. Baker\ncarb, \\Xiaomeng Huang\thub, Frank O. Bryan\ncara, Guangwen Yang\thua\thub
% \sharedaffiliation
%   \begin{tabular}{ccc}
%     \affaddr{{\thua}Department of Computer Science \& Technology {\ }} & & \affaddr{{\ncara}Climate and Global Dynamics Division {\ }} \\
%     \affaddr{{\thub} Center for Earth System Science{\ } } & & \affaddr{{\ncarb}Computational and Information Systems Laboratory{\ }} \\
%     \affaddr{Tsinghua University,Beijing,China} & & \affaddr{National Center for Atmospheric Research} \\
%     \email{huyong11@mails.tsinghua.edu.cn} & & \affaddr{Boulder,CO. USA} \\
%     \email{\{hxm,ygw\}@tsinghua.edu.cn}  & & \email{\{ytseng,abaker,bryan\}@ucar.edu} \\
%   \end{tabular}
% }



%\sharedaffiliation
%  \begin{tabular}{ccc}
%    \affaddr{{\thu}Ministry of Education Key Laboratory for Earth System Modeling{\ }} & & \affaddr{{\ncar}Climate and Global Dynamics Division, Computational and Information Systems Laboratory{\ }} \\
%    \affaddr{Center for Earth System Science, Tsinghua University, Beijing, 100084, China}            & & \affaddr{National Center for Atmospheric Research} \\
%    \affaddr{Tsinghua National Laboratory for Information Science and Technology (TNList)}            & & \affaddr{1850 Table Mesa Dr., Boulder, CO80305, USA} \\
%    \affaddr{United States}                  & & \affaddr{United States} \\
%  \end{tabular}
%


% There's nothing stopping you putting the seventh, eighth, etc.
% author on the opening page (as the 'third row') but we ask,
% for aesthetic reasons that you place these 'additional authors'
% in the \additional authors block, viz.
%\additionalauthors{Additional authors: John Smith (The Th{\o}rv{\"a}ld Group,
%email: {\texttt{jsmith@affiliation.org}}) and Julius P.~Kumquat
%(The Kumquat Consortium, email: {\texttt{jpkumquat@consortium.net}}).}
\date{30 March 2015}
% Just remember to make sure that the TOTAL number of authors
% is the number that will appear on the first page PLUS the

% number that will appear in the \additionalauthors section.



\maketitle
%ABSTRACT needs to be 150 words that we submit on web site - not in
%actual paper
\begin{abstract}
  High-resolution climate simulations are increasingly in demand and
  require tremendous computing resources. In the Community
  Earth System Model (CESM), the Parallel Ocean Model (POP) is
  computationally expensive for high-resolution grids (e.g.,
  0.1\degree) and is frequently the least scalable component of CESM for certain
  production simulations. In particular, the modified Preconditioned
  Conjugate Gradient (PCG), used to solve the elliptic system of
  equations in the barotropic mode, scales poorly at the high core
  counts, which is problematic for high-resolution simulations. In
  this work, we demonstrate that the communication costs in the
  barotropic solver occupy an increasing portion of the total POP
  execution time as core counts are increased. To mitigate this
  problem, we implement a preconditioned Chebyshev-type iterative
  method in POP (called P-CSI), which requires far fewer global
  reductions than PCG.  We also develop an effective block
  preconditioner based on the Error Vector Propagation Method to
  attain a competitive convergence rate for P-CSI.  We demonstrate
  that the improved scalability of P-CSI results in a 5.2x speedup of
  the barotropic mode in high-resolution POP on 16,875 cores, which
  yields a 1.7x speedup of the overall POP simulation.  Further,
  we ensure that the new solver produces an ocean climate consistent with the original
  one via an ensemble-based statistical method.
\end{abstract}


% \begin{abstract}
%   The increasing need for high-resolution climate simulations warrants
%   the investigation of a new preconditioned linear solver for the
%   barotropic mode in the Parallel Ocean Model (POP).  POP is the ocean
%   component of the Community Earth System Model (CESM) framework and
%   is computationally expensive for high-resolution grids (e.g.,
%   0.1\degree) on many cores.  Similar to other free-surface models,
%   POP solves the fast barotropic dynamics with the implicit
%   free-surface method.  This method requires the solution of an
%   elliptic system of equations, and POP solves this elliptic system
%   with a modified Preconditioned Conjugate Gradient (PCG) method.
%   While effective at moderate core counts and resolutions, PCG methods
%   scale poorly on large parallel systems due to the global reductions
%   required by the inner product calculations.  This poor scaling is
%   problematic for high-resolution CESM simulations.  In this work, we
%   implement a more scalable Preconditioned Stiefel Iteration method
%   (P-CSI) into the barotropic mode of POP within CESM, and develop an
%   effective Error Vector Propagation preconditioner to improve the
%   convergence rate.  Our experiments show that P-CSI accelerates the
%   high-resolution (0.1\degree) barotropic solver by 7.1 times on
%   16,875 cores, from 21.84s to 3.06s per simulation day.  This
%   improves the core POP simulation (without I/O and initialization) by
%   1.74 times, from 5.3 to 9.3 simulated years per wall-clock day.
%   Further, to ensure that the use of an alternative solver in POP does
%   not negatively impact the ocean climate, we use an ensemble-based
%   statistical method to evaluate the impact of the new solver.
% \end{abstract}







% A category with the (minimum) three required fields
\category{J.2}{Physical Sciences and Engineering}{Earth and atmospheric sciences}
%A category including the fourth, optional field follows...
\category{D.1.3}{Programming Techniques}{Parallel Programming}

%it's OK not to have general terms
%\terms{Algorithms, Performance}
%\keywords{Massive Parallelism, Preconditioned Conjugate Gradient, Classical Stiefel Iteration, Parallel Ocean Program}
\keywords{parallel computing, linear solver, ocean modeling}
\section{Introduction} \label{se:int}
% no \IEEEPARstart
High-resolution global climate models have become increasingly important
in recent years as a means for understanding climate variability
and projecting future climate change.  The Community Earth System
Model (CESM), whose development is centered at the National Center for
Atmospheric Research (NCAR), is one of the most widely used global
climate models, and its climate projections are a key component in the
Intergovernmental Panel on Climate Change (IPCC) Fifth Assessment
Report (AR5) \cite{stocker2013ipcc}.

% CESM is a fully-coupled climate model system, including atmosphere,
% ocean, sea-ice and land components. \hl{In particular, the ocean component
% requires the modeling of broad spatial and temporal scales in order to
% encompass the relevant dynamics where the ocean energy dominates.  The
% spatial scales range from tens of meters of the sub-mesoscale eddies
% to dimensions of the ocean basins, and the temporal scales range from
% the order of days for gravity waves to centuries for slow planetary
% waves.}  As a result, ocean models have to iterate for a longer time at
% a finer resolution, which makes them computationally intensive
% \cite{Worley:2011:PCE:2063384.2063457,dennis2012computational}.
% However, many recent studies demonstrate that high-resolution ocean
% models are required in order to produce more realistic and accurate
% predictions
% \cite{bryan2010frontal,mcclean2011prototype,graham2014importance}.

CESM is a fully-coupled climate system model, including atmosphere,
ocean, sea-ice and land components. In particular, the ocean component
is required to represent processes across a broad range of spatial and
temporal scales relevant to climate science. Ocean mesoscale eddies
have spatial scales of $\mathcal{O}$(10 - 100 km), one to two orders of magnitude
smaller than the dynamically analogous weather systems in the
atmosphere. The adjustment time scale for the deep ocean is many
centuries up to a few millennia, again orders of magnitude longer than
the corresponding timescales in the atmosphere. The computational
burden of a global eddy-resolving ocean climate model \cite{bryan2010frontal,mcclean2011prototype,graham2014importance} is
thus increased over that for an atmosphere model by the demand for
finer spatial resolution and longer integration times.

Moreover, climate model simulations are often run for decades or
even centuries, but these long-term simulations are typically too
computationally expensive to run at high-resolution.  For example,
most CESM simulations in IPCC AR5 are carried out with a nominal
1\degree\space ocean and a 1\degree\space to 2\degree\space atmosphere model.
Recent increases in both supercomputing resources and
high-resolution satellite observations have motivated
efforts to improve the parallel performance of high-resolution climate
models so that they can be run more routinely (and for less cost).

%Recently, the increasing computational power of supercomputers and
%high-resolution satellite observations have inspired much research
%that focuses on adapting high-resolution climate models for massive
%parallelism.
%As noted, in most production simulations, the least scalable
%component of CESM is the ocean model Parallel Ocean Model (POP)
%\cite{dennis2012computational}.

The Parallel Ocean Model (POP) component of CESM solves the
three-dimensional primitive equations with hydrostatic and Boussinesq
approximations \cite{smith2010parallel} and divides the time
integration into two parts: the baroclinic and the
barotropic modes. The baroclinic mode describes the 
three-dimensional dynamic and thermodynamics processes, and the barotropic
mode solves the vertically-integrated momentum and continuity
equations in two dimensions. The implicit free-surface method is a common choice
in barotropic mode in ocean models because it allows a large time step to
efficiently compute the fast gravity mode.  However, this method
requires solving a large elliptic system of equations, which
scales poorly in POP.  In fact, the poor scaling performance of the
barotropic solver in POP, which is dominated by the communication
overhead \cite{Worley:2011:PCE:2063384.2063457}, is well known, and
its optimization will benefit the entire CESM model
\cite{dennis2012computational}.

The currently recommended linear solver for the barotropic mode in CESM
POP is the Chronopoulos-Gear (ChronGear) method
\cite{dAzevedo1999lapack}, a modified Preconditioned Conjugate
Gradient method (PCG), combined with a diagonal preconditioner.
The required global reduction in the ChronGear method does not scale well and
causes a bottleneck for high-resolution simulations.  To improve the scaling
of POP, and, therefore CESM, we focus on optimizing the barotropic
solver by eliminating global reductions and developing a more
effective preconditioner.  In particular, we make the following
contributions:


%MOVE TO RELATED WORK
%There are currently different alternatives to mitigate the poor behavior of the PCG type of solver in the massive parallelization.
%Some approaches attempt to overlap the communication with computation time\cite{beare1997optimisation}.
%Some use the land elimination and load-balance strategies \cite{dennis2007inverse, dennis2008scaling}
%to reduce the number of processes and the associated overhead of global reduction.
%Although these approaches may improve performance, they do not eliminate the major bottleneck of the global reduction.


\begin{itemize}
\item We develop a new block parallel preconditioner based on the
Error Vector Propagation (EVP) method \cite{roache1995elliptic} designed to
improve solver convergence in the POP barotropic mode.
\item We add a preconditioning interface to the Classical Stiefel Iteration
(CSI) solver explored in \cite{hu2013scalable} and
implement the resulting preconditioned CSI (P-CSI) solver
and new EVP block preconditioner in CESM1.2.0 POP.
\item We demonstrate an improvement in convergence rate for both ChronGear and
P-CSI when using block EVP.
\item We obtain a 5.2x speedup of
  the barotropic mode in high-resolution POP due to the improved scalability
of P-CSI with block EVP preconditioning, greatly improving the
scalability of POP (and ultimately CESM) at large core counts.
\item We develop and apply an ensemble-based statistical method to evaluate the impact
of changing the linear solver in POP and ensure that a consistent ocean climate is produced.
\end{itemize}

The remainder of this paper is organized as follows.
Section \ref{se:baro} discusses POP's barotropic solver and its scalability.
Sections \ref{se:psi} and \ref{se:evp} detail the design
of P-CSI for POP and the development of the block EVP preconditioner.
Section \ref{se:exp} compares the scalability of the ChronGear and
P-CSI solvers.  Section \ref{se:ver} verifies the new P-CSI solver
using the ensemble based statistical method.
Finally,  related work and conclusions are presented in Sections
\ref{se:rel} and \ref{se:conc}, respectively.

%{\textbf better to move related work in the next section.}

% You must have at least 2 lines in the paragraph with the drop letter
% (should never be an issue)

\section{Barotropic solver} \label{se:baro}
 

 


For reference, the ChronGear method is provided in Algorithm
\ref{alg:pcg}.  ChronGear contains three major parts: computation,
boundary updating, 和 global reduction.  Computation involves
matrix-vector 和 vector-vector multiplications 和 vector scaling,
all of which exhibit good scalability.  The cost of the boundary communication, which
is required to update the halo area after the matrix-vector
multiplication, is bounded 和 not problematic at the target core counts.  We will however illustrate in the subsequent section, that cost of the global reduction, which is required by the inner product in step \ref{pcg_global1} does however become problematic at large core counts.

%The boundary communication that is required to update the halo area after the matrix-vector multiplication is also non-problematic as the amount of communication required is bounded. The most time-consuming part is the global reduction process required by the inner product in step \ref{pcg_global1}.

%----------------------------------------------------------------------------
\subsection{Communication bottleneck}\label{se:bottleneck}
Assume $p=m^2$ processes are used, 和 each process has exactly one
grid block (a typical choice for high-resolution POP). Then the total time of the barotropic mode is equal to the
execution time of the ChronGear solver on any block.  For each solver
iteration, we choose $\mathcal{T}_c$, $\mathcal{T}_b$ 和
$\mathcal{T}_g$ to be the cost of the computation, boundary
updating, 和 global reduction, respectively.
%----------------------------------------------------------------------------


From Algorithm \ref{alg:pcg}, the computational cost,
$\mathcal{T}_c$, contains four vector-scaling operations (steps
\ref{pcg_scale1}, \ref{pcg_scale2} ,\ref{pcg_scale3}, 和
\ref{pcg_scale4}), two vector-vector multiplication operations for
inner products (steps \ref{pcg_dot1} 和 \ref{pcg_dot2}), 和 one
matrix-vector multiplication (step \ref{pcg_mat}).  Therefore,
$\mathcal{T}_c= (4 n^2 +2n^2+ 9n^2)\theta + \mathcal{T}_{p}
=15\frac{\mathcal{N}^2}{p}\theta+\mathcal{T}_{p}$, where $\theta$ is
the time unit per floating-point operation 和 $\mathcal{T}_{p}$ is
the cost of preconditioning.  For example, $\mathcal{T}_{p} =
\frac{\mathcal{N}^2}{p}\theta$ for a diagonal preconditioner. When the
number of processes increases, $\mathcal{T}_c$ decreases 和 has a
lower limit of zero.

Boundary updating occurs in the halo regions for each process, after
operations like matrix-vector multiplication 和 non-diagonal
preconditioning, which require one or more boundary layers.
Because every process keeps its own block 和 two extra
halo layers in POP,  only one boundary update is needed per
iteration even when a non-diagonal preconditioner is used.
The actual time depends on the network delay 和 the volume of the
halo regions.  With a halo size of $2$, the volume in each boundary
is $2n$ 和 decreases as the number of processes increases.
%It is worth mentioning that the preconditioning process usually requires at least one boundary layer,
%except for the diagonal preconditioner, which means that values on the outmost points are not updated after preconditioning.
The total boundary updating time for each iteration is then $\mathcal{T}_b =4\alpha +(4\times 2n)\beta=4\alpha +(\frac{8\mathcal{N}}{\sqrt{p}})\beta $,
where $\alpha$ is point-to-point communication latency per message 和
$\beta$ is the transfer time per byte (inverse of b和width).
The boundary updating time also decreases as the number of processes increases but has a lower bound of $4\alpha$.


ChronGear contains only one global reduction per iteration which contains a MPI\_allreduce 和 a masking operation to exclude l和 points,  thus the global
reduction time satisfies $\mathcal{T}_g= 2\frac{\mathcal{N}^2}{p}\theta + \log p \alpha$ (assuming that
a binomial tree approach is used).  The cost of the masking operation should decrease with the number of processes $p$ while the cost of the MPI\_allreduce should monotonically increase.
%The expression $\mathcal{T}_g$ should therefore initially decrease followed by a monotonic increase with process count.
Note that the global
reduction has virtually no data exchange since there are only two
numbers from each process.
%Let $T_0$ be the time unit of one floating-point operation 和 $B$ be the number of floating-point numbers transmitted by the network per second from process to process.
%Provided that the processor frequency 和 network b和width are $S_{cpu}$ 和 $B_{net}$, 和 that their efficiencies are $R_{cpu}$ 和 $R_{net}$, then $T_0 = R_{cpu} S_{cpu}^{-1}$, 和 $B = \frac{1}{8}R_{net}B_{net}$.
Combining all three components, the execution time of one diagonal preconditioned ChronGear solver step can be expressed as:
\begin{eqnarray}
%\begin{tabular}{l}
\label{t_pcg}
&\mathcal{T}_{cg}=\mathcal{K}_{cg} (\mathcal{T}_c + \mathcal{T}_b+\mathcal{T}_g )\nonumber \\
&=\mathcal{K}_{cg} [18 \frac{\mathcal{N}^2}{p}\theta + \frac{8\mathcal{N}}{\sqrt{p}}\beta +(4+\log p)\alpha]
%\end{tabular}
\end{eqnarray}
where $K_{cg}$ is the number of iterations,
%in one ChronGear step
which
does not change with the number of processes \cite{hu2013scalable}.
Equation (\ref{t_pcg}) shows that the time required for computation
和 boundary updating decreases as the number of processes increases.
But the time required for the global reduction increases with increasing
numbers of processes. Therefore, we expect the execution time of the
ChronGear solver to increase when the number of processors exceeds a
certain threshold.
Figure \ref{fig:ChronGearCOMP} gives timings for the global reduction 和
boundary (halo) updating components of
the ChronGear solver for one simulation day on the Yellowstone machine at
NCAR (machine details given in Section \ref{se:exp}).
%Indeed, the scaling behavior of ChronGear in the
%0.1\degree POP is consistent with the above analysis
%(Figure \ref{fig:ChronGearCOMP}).
Note that the execution time of global reduction
becomes dominant 和 increases when more than a couple thous和 cores are used.


\begin{figure}[!t]
%\vspace{-10pt}
\begin{center}
	\includegraphics[height=6.5cm]{newChronGear_comp.eps}
\end{center}
\vspace{-.2in}
\caption[] {Timing for the global reduction 和 halo updating components of the ChronGear
  solver in 0.1\degree\space POP for one simulation day on Yellowstone.}
\label{fig:ChronGearCOMP}
\vspace{-.1in}
\end{figure}
%However,it still inherits the poor scalability from PCG.
%We tested the diagonal preconditioned ChronGear solver in  0.1\degree POP, 和 found that the scaling behavior is consistent with the above analysis. As shown in Fig.\ref{fig:ChronGearCOMP}, the execution time of global reduction becomes dominates in the ChronGear solver 和 increases when more than 2634 cores are used.


\section{P-CSI求解器} \label{se:psi}

为了提高海洋模式POP的可扩展性,一个好的正压模态求解器应该具备的特点是需要尽可能少的全局归约操作。 
在文章\cite{hu2013scalable}中,胡勇等人提出了一个基于Stiefel的CSI方法的求解器。这个求解器被现在了单独版本的POP中。 
这里,我们将通过添加一个预处理的结构,进一步改进CSI方法。
这个改进后的方法我们称之为P-CSI方法。 我们将这个方法实现在地球系统模式CESM框架下的海洋模式分量POP中。 
早在1985年, Saad 等人\cite{saad1985solving}将传统的CSI的方法应用于向量机上,并且指出当方程中矩阵的特征值已知的条件下, 这种方法比共轭梯度法更适用于某些并行情况。 


P-CSI算法和它的性质与CSI方法相似,只是多加了一个预处理接口(将在下一章详细讨论)。 
值得一提的是,P-CSI方法与CSI方法一样,在每一步迭代过程中不需要做內积操作, 因此它即使在大核数上也能够保持较好的可扩展性。
为海洋模式POP设计的P-CSI求解器的伪码在算法\ref{alg:ppsi}中。
使用对角预处理的P-CSI方法的每一步迭代的计算时间为$T_c =\frac{12\mathcal{N}^2}{p}\theta+\mathcal{T}_p =\frac{13\mathcal{N}^2}{p}\theta$, 
而P-CSI求解器的总运行时间为如下
\begin{eqnarray}
\label{t_psi}
\mathcal{T}_{pcsi} = \mathcal{K}_{pcsi}(\mathcal{T}_c + \mathcal{T}_b ) \nonumber \\
= \mathcal{K}_{pcsi}[13\frac{\mathcal{N}^2}{p}\theta+ 4\alpha + \frac{8\mathcal{N}}{ \sqrt{p}}\beta]
\end{eqnarray}
这里$K_{pcsi}$ 表示P-CSI方法收敛到给定条件时所需要的迭代步数。 
\begin {figure}[!t]
\begin{center}
\includegraphics[height=6.5cm]{solver_iteration}
\end{center}
\vspace{-.2in}
\caption []{1度POP中,Lanczos方法的迭代次数对P-CSI迭代步数的影响。 \label{fig:iter}}
%\vspace{-.1in}
\end {figure}

\begin{algorithm}[!t]
\caption{ 预处理的传统Stiefel迭代算法}
\label{alg:ppsi}
\begin{scriptsize}
\begin{algorithmic}[1]
\REQUIRE 与网格块$B_{i,j}$相对应的系数矩阵 $\textbf{B}$, 预处理子$\textbf{M}$, 初始值$\textbf{x}_0$和方程右端向量$\textbf{b}$ ;预估的特征值区间$[\nu,\mu]$;  \\
 // \qquad    \textit{所有进程并行执行}
\STATE $\alpha =\frac{2}{\mu -\nu}$, $ \beta = \frac{\mu +\nu}{\mu -\nu}$, $\gamma = \frac{\beta}{\alpha}$, $\omega_0 =\frac{ 2}{\gamma}$;\quad $k = 0$;
\STATE $\textbf{r}_0 = \textbf{b}-\textbf{B}\textbf{x}_0$; $\Delta \textbf{x}_{0} = \gamma^{-1}\textbf{M}^{-1}\textbf{r}_0$; $\textbf{x}_1 =\textbf{x}_0 +\Delta \textbf{x}_{0}$; $\textbf{r}_1 =\textbf{b} -\textbf{B}\textbf{x}_1$;
\WHILE{$k \leq k_{max}$ }
\STATE $k=k+1$;
\STATE $\omega_k = 1/(\gamma - \frac{1}{4\alpha^2}\omega_{k-1})$; \COMMENT{迭代函数}
\STATE $\textbf{r}'_{k} =\textbf{M}^{-1}\textbf{r}_{k}$; \COMMENT{预处理} \
\STATE $\Delta \textbf{x}_{k} =\omega_k\textbf{r}'_{k}+(\gamma \omega_k-1)\Delta \textbf{x}_{k-1}$;
\STATE $\textbf{x}_{k+1} =\textbf{x}_{k}+\Delta \textbf{x}_{k}$;
\STATE $\textbf{r}_{k+1} =\textbf{b}- \textbf{B}\textbf{x}_{k+1}$; \COMMENT{矩阵向量乘}
\STATE $update\_halo(\textbf{r}_{k+1})$; \COMMENT{边界通信}
\IF { $k \%  n_{c} == 0$ }
\STATE 收敛性检查;
\ENDIF
\STATE $convergence\_check(\textbf{r}_{k+1})$;  \COMMENT{收敛性检查}
\ENDWHILE
\end{algorithmic}
\end{scriptsize}
\end{algorithm}


P-CSI需要对预处理后的系数矩阵$M^{-1}A$的最大特征值$\mu$和最小特征值$\nu$进行估计。 
海洋模式POP中的系数矩阵$A$ 和它的对角预处理矩阵$M = \Lambda(A)$都是实对称矩阵, 因此预处理后的矩阵的最小最大特征值并不难估计。 
论文\cite{hu2013scalable} 中给出$A$的特征值的估计,这里我们采用Lanczos方法\cite{Paige1980235}来对$M^{-1}A$的最大最小特征值进行估计。
实验中,我们发现将Lanczos方法的收敛条件因子$\epsilon$ 设成$0.15$在不同的分辨率(1度或者0.1度)以及不同的预处理子(单位阵预处理、对角预处理和EVP预处理)的情况下都能达到较好的效果。 
图\ref{fig:iter} 可以看出只需要很少的Lanczos步骤就可以估计出比较恰当的$M^{-1}A$的最大最小特征值,从而使得P-CSI方法能够在较短的迭代步数内收敛。
在实际运行中,我们发现Lanczos方法的开销与调用数次ChronGear求解的开销相当。 

与ChronGear的迭代过程相比,P-CSI除了做收敛性检查之外不需要任何的全局归约操作。 
我们注意到,为了达到相同的收敛条件,P-CSI算法需要的迭代步骤比ChronGear的略多($K_{pcsi} > K_{cg}$)。 
以上的两点现象,我们可以解读出来,在当并行使用的处理器核心数比较小的时候,P-CSI应该会比ChronGear的开销更大一些。
因为处理器核心较少的时候,全局归约操作的开销并不算大。 
但是,当采用高分辨的网格时,需要用到更多的处理器核心,这是P-CSI的每一步迭代的开销应该会比ChronGear的要明显的快一些(参见公式\ref{t_pcg})和
(\ref{t_psi})。
而这将进一步减少P-CSI算法到达收敛的时间。 

当满足条件$\nu = \lambda_{min}$和$\mu =\lambda_{max}$时,P-CSI的收敛速度将达到最优值。 
但是$\lambda_{min}$和$\lambda_{max}$是比较难以估计的。
更重要的是,我们不能对系数矩阵$A$随意的做变换,因为 $A$ 是分布在各个进程上的。
为了能够利用POP的并行特性,我们使用Lanczos方法来构造 出一系列三对角矩阵$T_m (m=1,2,...)$,这些矩阵的最大最小特征值逐渐的向$M^{-1}A$的最大最小特征值逼近。
To utilize the parallelism of POP, we employ Lanczos method  to construct
实际运行中, 我们发现合适的选择Lanczos迭代的步数,我们能够得到对最大最小特诊一个比较好的估计,这个估计值能够使得P-CSI方法与预处理共轭梯度法有相近的收敛速度。  


\begin{algorithm}[!ht]
\caption{基于Lanczos方法的针对预处理矩阵的特征值估计方法}
\label{alg:lanczos_pre}
\begin{algorithmic}[1]
\REQUIRE 网格块$B_{i,j}$ 的系数矩阵$\textbf{B}$,预处理子$\textbf{M}$和随机向量$\textbf{r}_0$; \\
 //\qquad    \textit{所有进行并行执行}
\STATE $\textbf{s}_0=\textbf{M}^{-1}\textbf{r}_0$;\quad $\textbf{q}_1 = \textbf{r}_0/({\textbf{r}_0^T\textbf{s}_0})$;\quad $\textbf{q}_0=\textbf{0}$;
\STATE $T_0=\emptyset$;\quad $\beta_0 =0$;\quad  $\mu_0 =0$;\quad $j=1$;
\WHILE{$j<k_{max}$}
\STATE $\textbf{p}_j = \textbf{M}^{-1}\textbf{q}_j$; \quad $\textbf{r}_j=\textbf{B}\textbf{p}_j-\beta_{j-1}\textbf{q}_{j-1}$;
\STATE $update\_halo(\textbf{r}_j)$;
\STATE $\tilde{\alpha}_j =\textbf{p}_j^T\textbf{r}_j$; \quad $\alpha_j=global\_sum(\tilde{\alpha}_j)$;
\STATE $\textbf{r}_j=\textbf{r}_j-\alpha_{j}\textbf{q}_{j}$; \quad $\textbf{s}_j = \textbf{M}^{-1}\textbf{r}_j$;
\STATE $\tilde{\beta}_j = \textbf{r}_j^T\textbf{s}_j$; \quad $\beta_j=sqrt(global\_sum(\tilde{\beta}_j))$;
\STATE \textbf{if} $\beta_j == 0$ \textbf{then} \textbf{return}
\STATE $\mu_j = max(\mu_{j-1},\alpha_j+\beta_j+\beta_{j-1})$; \label{lan_gersh}\\
\STATE $T_j=tri\_diag(T_{j-1},\alpha_j,\beta_j)$; \COMMENT{三对角矩阵}\label{lan_tm}
\STATE $\nu_j = eigs(T_j,'smallest')$ ; \label{lan_nu}
\STATE \textbf{if} $|\frac{\mu_j}{\mu_{j-1}} -1 |< \epsilon\quad\textbf{and}\quad|1- \frac{\nu_j}{\nu_{j-1}}|< \epsilon$ \textbf{then} \textbf{return}; \label{lanczos_converge}
\STATE $\textbf{q}_{j+1}= \textbf{r}_j/\beta_j$;\quad $j=j+1$;
\ENDWHILE
\end{algorithmic}
\end{algorithm}

这个算法需要$\textbf{A}$ 是正定对称矩阵。海洋模式POP中直接得到的系数矩阵是负定对称矩阵,此时只需将方程两边同时乘以单位矩阵和-1,即可将系数矩阵转换成正定对称矩阵。 
 


%\section{Error Vector Propagation (EVP) Preconditioning} \label{se:evp}
\section{A Block-EVP preconditioner} \label{se:evp}

%Equations (\ref{t_pcg}) and (\ref{t_psi}) show that
The total
execution time of the barotropic solver is the product of the number
of iterations and the execution time per iteration.  With increasing
numbers of cores, the execution time of computation in each iteration
decreases, but the execution time of communication increases.  To
reduce communication costs, preconditioning is commonly used to reduce
the number of iterations to convergence, assuming the cost of
preconditioning is reasonable.
%Three major factors affecting a
%preconditioner's cost for large sparse linear systems are efficiency,
%scalability and preprocessing requirements.
While the current ChronGear
solver in POP has benefited greatly by using a simple diagonal
preconditioner \cite{pini1990simple, reddy2013comparison},
further improvement to its convergence rate
would significantly reduce the associated communication cost and
improve scalability.
In fact, the performance of both P-CSI and the existing ChronGear solvers could be
further improved with a more effective preconditioner.

%The data distribution and communication cost can only tolerate lower level of factorization like ILU(0). L and U have the same nonzero structure as the lower and upper parts of A respectively, which often results in a crude approximation has a very bad effect on convergence speed.
%More accurate ILU factorizations is not suit for ocean model, because it requires more communications and computations.
%Also, the preprocessing cost to compute the factors is higher.

%In 1985, Concus et al. \cite{concus1985block} used the banded approximate of the matrix inverse as a preconditioner,
%and achieved higher efficiency than other preconditioners on elliptic partial differential equations.
%NOTE YH what is "local approximate inverse of the true operator"?
%Smith et al. \cite{smith1992parallel} employed a polynomial preconditioning method and a local approximate-inverse method.
%POP supports a preconditioner consisting of a 9-point operator which is a local approximate-inverse of the true operator.

\subsection{Block preconditioning}

% The most widely used preconditioning techniques in sequential simulations are the incomplete factorization methods
% like incomplete LU factorization (ILU) and its variants \cite{benzi2002preconditioning}.
% However, they are ill-suited for parallel computers because they require a recursive operation which limits parallelization.

Some parallelizable preconditioning methods such as polynomial,
approximate-inverse, multigrid, and block preconditioning have
drawn much attention recently.  High-order polynomial preconditioning
can reduce iterations as effectively as incomplete LU factorization
(ILU) and its variants \cite{benzi2002preconditioning} in sequential
simulations.  However, the computational overhead for polynomial
preconditioners typically offsets its superiority to diagonal
preconditioning (e.g., \cite{meyer1989numerical,smith1992parallel}),
as a $k$th-order polynomial preconditioner requires $k$ matrix
vector multiplications in each iteration. Approximate-inverse
preconditioners, while highly parallelizable, require solving
a linear system several times larger than the original system 
\cite{smith1992parallel,bergamaschi2007numerical}, which makes it less
attractive for POP than a simple diagonal preconditioner.  
%Algebraic multigrid (AMG) is highly scalable and effective for linear systems
%derived from elliptic systems of equations (e.g.,
%\cite{baker2012scaling}), but a previous investigation into using an AMG
%preconditioner in POP found the setup costs to
%be too high to be competitive with diagonal preconditioning.
Multigrid is highly scalable and generally effective for linear
systems derived from elliptic systems of equations. Recent works
indicate that geometric multigrid (GMG) is promising in atmosphere
\cite{muller2014massively} and ocean
\cite{matsumura2008non,kanarska2007algorithm} modeling when uniform
grids and simple topography are involved. However,  in global ocean
models, the presence of 
complex topography (such as islands, straits, passages and coastal complexities) combined with non-uniform or anisotropic grids
%non-smooth topographic boundaries and non-uniform or anisotropic grids 
result in less than ideal scaling for simple GMG methods
\cite{matsumura2008non,fulton1986multigrid,tseng2003ghost,stuben2001review}.
Note that in CESM-POP, general dipole orthogonal grids are used to
avoid the polar singularity, and only the ocean part of the earth
surface is simulated with masked lands. These choices lead to an
elliptic system with variable coefficients defined on an irregular
domain with non-uniform grids, making the effective use of GMG
non-trivial.  The scenario is even worse for the high resolution ocean
grid where thousands of islands and narrow passages 
are not representable at a grid coarsening of even one or two levels
(e.g., the Bering Strait). For complex geometries, algebraic multigrid (AMG) is often a viable
alternative to GMG.  A drawback of AMG, though, is that in some cases,
the setup cost can exceed that of iterative solver itself, making it
inferior to a well-preconditioned CG method (e.g.,
\cite{muller2014massively}), particularly when the number of CG
iterations is reasonably low as for CESM-POP (Figure
\ref{fig:iteration} in the next section). 
%However, the setup of AMG is too costly
%that its cost sometimes exceeds that of iterative solver itself, which
%makes it inferior to properly preconditioned CG method in some cases
%\cite{muller2014massively}.  
Further, because POP requires solving the linear
equation tens and hundreds of times per simulation day in the low and
high resolution versions, respectively, and thousands of
simulation years are needed in the typical simulation, the
costly setup of AMG is prohibitive.
%is too costly to be competitive with diagonal preconditioning.
 % Multigrid is
% another popular preconditioning method which is highly scalable and
% efficient in preconditioning the sparse linear systems developed from
% elliptic equations \cite{baker2012scaling}.
% But the cost of setting
% up the multigrid preconditioning is so high that it is not worth using
% multigrid for iterative solvers in problems like POP, which converge
% in tens
%or hundreds of iterations.
Finally, block preconditioning has been shown to be an effective
parallel preconditioner (e.g., \cite{concus1985block, white2011block})
and is appealing for POP because it makes use of the block structure of the
coefficient matrix that arises from discretization of the
elliptic equations.
% factorizes the matrix block-by-block instead of point-by-point.

\begin {figure}
\centering
\includegraphics[height=5.0cm]{blockpreconditioning.eps}
\caption[] {Sparsity pattern of the coefficient matrix developed from nine-point stencils.
The whole domain is divided into $3\times3$ non-overlapping blocks.
Elements in red rectangles are coefficients between points in blocks.
Elements in blue rectangles are coefficients between points from the $i$-th block and its neighbor blocks. \label{fig:blockprecond}}
\vspace{-.2in}
\end{figure}

To facilitate describing the new block EVP preconditioner, we first
briefly review a general block preconditioner, as illustrated by Figure
\ref{fig:blockprecond}.  If the linear system of $\mathcal{N} \times
\mathcal{N}$ grid points is reordered block-by-block with size of $n\times n$
(e.g., $\mathcal{N}/3\times \mathcal{N}/3$ in
Figure \ref{fig:blockprecond}), then coefficient matrix $A$ can be
represented by a nine-diagonal block matrix.
Each row of this matrix contains nine sub-matrices.
% (Fig \ref{fig:blockprecond}). %(arranged in their spatial relationship)
Each $B_i$ (red blocks) is a block matrix containing coefficients of
the grid points in the i-th block, which share the same structure as $A$ but
have a smaller size ($n^2\times n^2$).  $B_i^e$, $B_i^w$, $B_i^n$ and
$B_i^s$ are block matrices containing coefficients of points
on east, west, north and south boundaries and the points on their
respective neighboring blocks, thus having at most $3n$ nonzero
elements distributed on $n$ rows. $B_i^{nw}$, $B_i^{ne}$, $B_i^{sw}$
and $B_i^{se}$ have only one nonzero element, representing the
coefficient of corner points and their neighboring points on the
northwest, northeast, southwest and southeast blocks.  The traditional
block preconditioning method constructs the approximate inverse of $A$
by sequentially factorizing it with approximations of $B_i^{-1}$,
which is ill-suited for parallel applications.  In contrast, the
inverse of the block diagonal of $A$, which provides a good
approximation for $A$, can be calculated naturally in parallel.  The
inverse of the diagonal block matrices is
\begin{eqnarray*}
M^{-1}=    \left [
        \begin{array}{ccccccc}
        B_1^{-1} &   &  \\
         & \ddots&  \\
        &   &  B_{m^2}^{-1} \\
    \end{array}
    \right ]
\end{eqnarray*}
Using $M$ as a preconditioner, the preconditioning process $\textbf{x}
= M^{-1}\textbf{y}$ is typically transformed into solving the sparse
linear equations $B_i \textbf{x}_i = \textbf{y}_i$ for each block,
(instead of explicitly constructing $B_i^{-1}$) and solving via LU decomposition.
The arithmetic complexity of solving these equations with LU decomposition is $\mathcal{O}(n^4)$,
providing that the LU decomposition is previously initialized.


% This parallel block
% preconditioning is preferable in parallel application due to less
% computation to solve the equations with LU decomposition than
% multiplying $\textbf{y}$ by $M^{-1}$.

\subsection{Error Vector Propagation method}
In contrast with LU decomposition, the arithmetic complexity of solving the equations $B_i \textbf{x}_i =\textbf{y}_i$ is $\mathcal{O}(n^2)$,
where $B_i$ is $n^2\times n^2$, when using the Error Vector
Propagation (EVP) method,
which to the best of our knowledge is one of the least costly algorithms
for solving elliptic equations in serial \cite{roache1995elliptic}.
%The EVP method marches over the whole domain based on the discretized equations with an arithmetic complexity $\mathcal{O}(n^2)$.
%EVP is an alternative of fast direct solvers for elliptic equations,
The EVP method and its variants have been used in several ocean models
(e.g., Sandia Ocean Modeling System \cite{dietrich1987ocean} and
CANadian version of DIEcast \cite{sheng1998candie}).  Further, Tseng and Chien
\cite{tseng2011parallel} employed a modified parallel EVP
method based on domain-decomposition as a solver for the global ocean
simulation.

\begin {figure}[!t]
\centering
%\includegraphics[width=0.8\linewidth]{evp9pmarch1.png}
\includegraphics[height=5.0cm]{evp9pmarch1.png}
\caption []{EVP marching method for nine-point stencil. The solution on point $(i+1,j+1)$ can be calculated using the equation on point $(i,j)$, providing solutions on other neighbor points of point $(i,j)$.  \label {fig:evp9p}}
\vspace{-.2in}
\end {figure}

The EVP method works as follows. We discretize Equation \ref{eq:ssh}
(the implicit elliptic system of equations for SSH)
into the following form so that we can march the solution northeastward assuming all other neighboring points are exactly known :
\begin{eqnarray}
\label{eq:evp9p}
&\eta_{i+1,j+1} = (1/A_{i,j}^{ne} )(\psi_{i,j} - A_{i,j}^0\eta_{i,j}-A_{i,j}^e\eta_{i+1,j} \nonumber\\
&-A_{i,j}^n\eta_{i,j+1}-A_{i-1,j}^{ne}\eta_{i-1,j+1} +A_{i-1,j}^e\eta_{i-1,j}\nonumber\\
&-A_{i-1,j-1}^{ne}\eta_{i-1,j-1}-A_{i,j-1}^n\eta_{i,j-1}- A_{i+1,j-1}^{ne}\eta_{i,j-1} )
\end{eqnarray}
%\textbf{I suggest you to make Fig. 5 to use a different color or label to represent the specified boundary conditions}
Figure \ref{fig:evp9p} illustrates a Dirichlet boundary
elliptic equation $\mathcal{B}\textbf{x} = \psi$ on a small domain.  We
define the interior points next to the south and west boundaries as
the initial guess points $\textbf{e}$ and those next to the north and
east boundaries are the final boundary points $\textbf{f}$ (e.g.,
$\textbf{e}= \{E_1, \dots, E_7\}$, $\textbf{f}= \{F_1, \dots, F_7\}$
in Figure \ref{fig:evp9p}).  If the true solution on $\textbf{e}$ is
known, the exact values over the whole domain can be computed
sequentially from southwest to northeast corners, using Equation
\ref{eq:evp9p}. This procedure is referred to as marching.
Unfortunately, the value on $\textbf{e}$ is often not known until the
elliptic equation is solved.  However, we can get a solution
$\textbf{x}$ satisfying the elliptic equation on the whole domain
except on the boundary, by first guessing the value
$\textbf{x}|_\textbf{e}$ on $\textbf{e}$ and then calculating the rest
using the marching method.  Then $E=(\textbf{x} -\eta)|_\textbf{e}$
and $F=(\textbf{x} -\eta)|_\textbf{f}$ are error vectors on
$\textbf{e}$ and $\textbf{f}$, respectively.  The error vector $F$ is
already known since $\textbf{f}$ are boundary points (Dirichlet
boundary condition is imposed).  The relationship between the error on
initial guess points and the final boundary points can be represented
as $F=W*E$.  This influence coefficient matrix $W$ can be formed by
marching on the whole domain with unit vectors on the initial guess
points and zero residual value in the whole domain.  We summarized
the EVP algorithm for an elliptic equation with zero boundary in
Algorithm \ref{alg:evp}.


The EVP method contains two steps: preprocessing and solving. In
the preprocessing step, the influence coefficient matrix and its
inverse are computed, involving a calculation of $\mathcal{C}_{pre}=
(2n-5)* 9n^2 + (2n-5)^3 = \mathcal {O} (26n^3)$.  Obtaining the solution in the solve step requires
$\mathcal{C}_{evp}= 2* 9n^2 + (2n-5)^2 = \mathcal{O} (22n^2)$.  This
estimate indicates that EVP has lower
computational cost for the solver step than other direct solvers such as
LU.
%What is the cost per iteration for PCG - didn't quite get the below.
%or iterative methods such as PCG for elliptic equations
Therefore, EVP can be practical in real applications from a cost standpoint because
preprocessing is only needed once at the beginning to obtain the
influence coefficient matrix and its inverse.
\begin{algorithm}[t!]
\caption{Nine-point Error Vector Propagation method}
\label{alg:evp}
\begin{scriptsize}
\begin{algorithmic}[1]
\REQUIRE Residual $\psi$ associated with a domain containing $n\times n$ grid points, $k = size(\textbf{e})=2n-5$; \\
//\qquad \textit{preprocessing }
\STATE  $\textbf{x} = \textbf{0}$
\FOR {i = 1, k}
\STATE $\textbf{x}|_\textbf{e}(i) = 1$
\STATE $\textbf{x} = marching(\textbf{x},\textbf{0})$
\STATE $W(i,:) = \textbf{x}|_\textbf{f}$
\STATE $\textbf{x}|_\textbf{e}(i) = 0$
\ENDFOR
\STATE $R = inverse(W)$ \\
//\qquad \textit{solving }
\STATE $\textbf{x}= marching(\textbf{x},\psi)$
\STATE $F = (\textbf{x} - \eta)|_\textbf{f}$
\STATE $\textbf{x}|_\textbf{e} =\textbf{x}|_\textbf{e} - R*F$
\STATE $\textbf{x} = marching(\textbf{x},\psi)$
\end{algorithmic}
\end{scriptsize}
\end{algorithm}

\subsection{EVP as a parallel preconditioner}
The EVP method described above is an efficient option for
solving elliptic equations.  However, a major drawback of
EVP is that it cannot solve on a large domain without further
modifications due to numerical instabilities when marching
\cite{roache1995elliptic}.  But on a small domain up to the size of
$12\times 12$, EVP solves with an acceptable round-off error of
$\mathcal{O}(10^{-8})$ when double-precision floating-point is used.  Its
effectiveness on small domains and low-computational cost make EVP an
ideal method for parallel block preconditioning.
%thus making it possible to use the block diagonal of $A$ as a preconditioner.
%\textbf{not sure what the above sentence means mean EVP combined with block diagonal?}
Here, we develop a block preconditioning technique based on the EVP
method in each block to further improve the performance of the
barotropic solver in POP.  Each preconditioner step solves the
elliptic equations $B_i \textbf{x} = \textbf{y} (i=1,...,m^2)$ in
parallel.


The fact that EVP is not well-suited for large domains is not an issue for
large-scale parallel computing, where larger number of processors
result in smaller domains.  Furthermore, for our system of equations,
the coefficients related to north, south, east and west neighbors on
every point are one magnitude order smaller than the others. We found
that removing these coefficients reduces the cost of
EVP preconditioning by about a half without any significant
impact on the convergence rate when used with both ChronGear and
P-CSI.  As a result, the execution time of EVP preconditioning can be
expressed as $\mathcal{T'}_{p} = 14n^2\theta=
14\frac{\mathcal{N}^2}{p}\theta$.  The actual cost of
$\mathcal{T'}_{p}$ depends on the size of the local block, which
decreases as more processor cores are used.
%In the current version of POP, using non-diagonal preconditioning methods requires an additional boundary communication after the preconditioning in each iteration.
%However, we find that the later boundary communication can be skipped by utilizing the fact that the halo size is 2.
Thus, the total execution time for one ChronGear and P-CSI solver step
with block-EVP preconditioning are
\begin{eqnarray}
\label{t_evppcg}
&\mathcal{T'}_{cg}=\mathcal{K'}_{cg} (\mathcal{T'}_c + \mathcal{T'}_b+\mathcal{T'}_g )\nonumber \\
&=\mathcal{K'}_{cg} [31 \frac{\mathcal{N}^2}{p}\theta + \frac{8\mathcal{N}}{\sqrt{p}}\beta +(4+\log p)\alpha],
\end{eqnarray}
and
\begin{eqnarray}
\label{t_evppsi}
\mathcal{T'}_{pcsi} = \mathcal{K'}_{pcsi}(\mathcal{T'}_c + \mathcal{T'}_b ) \nonumber \\
= \mathcal{K'}_{pcsi}[26\frac{\mathcal{N}^2}{p}\theta+ 4\alpha + \frac{8\mathcal{N}}{ \sqrt{p}}\beta],
\end{eqnarray}
respectively.

\begin {figure}[!t]
\centering
\includegraphics[height=6.5cm]{iteration.eps}
\vspace{-.1in}
\caption[] {Average number of iterations for different barotropic solvers. \label{fig:iteration}}
\vspace{-.1in}
\end{figure}

The implementation of EVP preconditioning in POP significantly reduces
the number of iterations required for convergence for both the ChronGear and P-CSI solvers.
In particular,  Figure \ref{fig:iteration} demonstrates that EVP
preconditioning reduces the iteration count by about two-thirds for
both the 1\degree\space and 0.1 \degree\space resolutions, which is comparable to the
approximate-inverse preconditioner proposed in
\cite{smith1992parallel} (which is not implemented in CESM POP).
Although  EVP preconditioning doubles the computation in each iteration, it halves both global and boundary communications
which dominate in the barotropic execution time at large core counts.
Another advantage of EVP preconditioning is the low
preprocessing cost.
%EVP preconditioning matrix is totally parallelized, while traditional decomposition based preconditioning requires serial processing.
In 0.1\degree\space case, the cost of setting up the preconditioning matrix
is less than that of one call to the solver when 512 processor cores
are used, and this cost is further decreased when more processors are used.
Finally, we note that the 0.1\degree\space case requires fewer iterations than the
1\degree\space case, because the higher resolution POP grid has a 
ratio of longitude to latitude grid spacing that is closer to 1, resulting
in a smaller condition number for the coefficient matrix.

\section{实验结果} \label{se:exp}


 
我们首先利用美国国家大气研究中心NCAR-Wyoming超级计算中心
(NWSC) \cite{loft:2015}的黄石超级计算机来评测一下采用了我们的新的正压模态求解器的CESM1.2.0的性能。 
黄石超级计算机由 72,576个
2.6-Ghz英特尔型号为Xeon E5-2670 的``Sandy Bridge'' 处理器构成。 
这些处理器都是通过13.6 GBps InfiniBand网络连通。   
黄石超级计算机上50\%以上的计算资源都是在使用地球系统模式CESM来进行气候模拟, 
因此提高CESM在黄石超级计算机上的效率对于全球气候变化研究来讲十分的有意义\cite{wf2014}。 


由于我们只是改动了海洋模式分量POP中的求解器,所以实验中我们将着重测试POP的性能。 
我们采用的CESM的
``G\_NORMAL\_YEAR''分量配置集,这个配置中只有海洋和海冰分量模式是真实计算的,而大气等分量模式采用气候数据输入。
我们针对两种使用频率最高的海洋模式POP水平网格分辨率来进行实验:
1度 ($320\times 384$) 和0.1度 ($3600\times 2400$).
值得一提的是,CESM1.2.0的默认配置中将黄石超级计算机上的MPI的环境需求限制(MPIMP\_EAGER\_LIMIT)设置为0, 这个变量主要控制在网络通信约会协议被使用之前的最大的消息量。我们发现使用黄石超级计算机上的默认的环境需求限制配置,也就是MP\_EAGER\_LIMIT = 131072,能够显著的改善CESM的运行性能。 



\subsection{低分辨率模拟}
图\ref{fig:runtime1}给出了几个可用的求解器在黄石超级计算机上模拟1度海洋模式POP时正压模态的运行时间。  
采用默认的对角预处理子, 在各种并行度下, P-CSI要比ChronGear的性能好。
在最大的并行度下(768处理器核心),P-CSI将每个模拟天中求解器的的执行时间从0.58s减少到了0.41s,也就是1.4倍的加速比)。 
另外,使用新的EVP块预处理,默认的ChronGear和P-CSI求解器的收敛性都得到提升。 
在768核上, 使用EVP预处理的P-CSI的计算效率为0.37s每模拟天, 相对于默认的对角预处理的ChronGear求解器加速了1.6倍。 

\begin {figure}[!t]
\centering
\includegraphics[height =6.5cm]{NEW1deg_solverruntime}
\caption []{1度POP每模拟天中正压模态的运行时间.\label {fig:runtime1}}
\end {figure}

\begin{table}[!h]
\begin{center}
\caption{在黄石超级计算机上,1度 POP总运行时间的优化比例。 \label{tab:improve_1}}
\begin{tabular}{|l||l|l|l|l|l|}
\hline
进程数 & 48  & 96  & 192 & 384 & 768\\\hline
\hline
ChronGear+EVP & -.5\% & 1.1\%  & 6.5\% & 10.8\%  & 12.1 \% \\\hline
P-CSI+Diagonal  & .7\% &3.9\% &9.3\%  &11.0\% & 12.6 \% \\\hline
P-CSI+EVP	      &-2.4\% & .4\%	& 7.4\%  & 14.4\% & 16.7\%\\\hline
\end{tabular}
\end{center}
\end{table}


正压模态的改进使得整个海洋模式POP的运行时间也随之下降。
表 \ref{tab:improve_1} 给出了三个新的求解器和预处理子的不同搭配相比于原始的对角预处理的ChronGear求解的所带来的性能提升百分比。 
这里给出的时间都是 取之于一个五天的模拟,同时模式的初始化和 I/O 操作都没有考虑在内。 
P-CSI 结合EVP预处理在768核上取得了  16.7\% 的加速效果。
虽然16.7\% 的提高看起来并不显著,但是由于 1度分辨率的海洋模式POP通常的模拟时间尺度都是几个世纪, 这样的一个性能的改进在如此长期的模拟中可以节省数百万个小时的CPU计算时间。 
另外,当海洋模式POP中开启生物化学的模块时,即使是1度的分辨率也需要使用相对较多的处理器核心,因为这个模块需要引进很多新的示踪物,进而导致计算量增加很多。 

\subsection{高分辨率模拟}
\begin {figure*}[t!]
\begin{center}
\includegraphics[height =6.5cm]{NEW01deg_solverruntime}
\hspace{10pt}
\includegraphics[height =6.5cm]{NEW01deg_speedup_ys}
\end{center}
\caption []{ 在美国黄石超级计算机上0.1度 POP中正压模态模拟一天的的执行时间(左), 
在美国黄石超级计算机上0.1度 POP核心部分的模拟速率(右)。\label {fig:runtime01}}
\end {figure*}


现在我们测试在黄石超级计算机上,高分辨率0.1度的海洋模式POP的新的正压求解的可扩展性。 
在这个高分辨下, 海洋模式中网格块的大小和分布得选择,会影响到进程上任务的分发,进而对性能造成很大的影响。 
因此,为了不让这种选择影响到最终的可扩展性结果,我们非常仔细的将每个进程上的网格块的横纵比例统一地设置为3:2,而陆地的比例设置为0.25,同时还利用空间填充曲线的方法来划分网格。 
我们使用POP中默认的时间步长,也就是500个时间步每天(dt\_count = 500)。 
最后,为了统一,所有的求解器都是每十个迭代步做一次收敛判断。 
这里值得一提的是,由于P-CSI的迭代过程的开销相对比较小(相比于海洋模式POP中的收敛性判断), 因此,如果将收敛性判断的频率进一步降低的话, P-CSI的性能可能能够进一步的提高。 
 

如图\ref{fig:runtime01}左边所示,ChronGear的性能在所使用的处理器核心数大于2700时反而会有所下降,
而P-CSI的运行时间在这之后相对比较平缓。 
采用对角预处理的前提下, P-CSI方法在16,875核上将0.1度的海洋模式POP的正压模态的性能提高了4.3倍(从19.0秒下降到4.4秒每模拟天)。 
使用EVP预处理能够进一步的改进ChronGear和P-CSI的性能, 使得这两个算法分别相对于原始的正压模态求解器加速了1.4倍和5.2倍。
在\ref{se:baro}节中,我们证明了原始的正压模态求解器在海洋模式POP的总运行时间所占的比例会随着使用的核心数目的增加而增大。
尤其是在16,875个核上,采用对角预处理的ChronGear求解器占到总运行时间的 50\%以上。 
相对应的,图\ref{fig:StepComp_pcsi}表明,采用了具有 更高可扩展性的EVP预处理的P-CSI求解器后, 同样是在 16,875 核上运行,正压模态的执行时间仅仅占到总运行时间的16\%。 

 
正压求解器性能上的改进有助于提高海洋模式POP的总体性能, 尤其是在大规模并行环境中。
模拟速率(模拟年每天)是一个被广泛采用的评估模式性能的指标, 我们这里采用不考虑模式初始化和I/O开销的 核心模拟速率。这是由于模式初始化只是在模式运行开始的时候执行一次,而模式的I/O开销是随着用户设置的频率的变化而改变的。 
通常专家们认为,如果想要进行长期的气候模拟,那么模拟速度最小要达到五个模式年每天\cite{dennis2012computational}, 图\ref{fig:runtime01} 右边可以看出,模式采用新的P-CSI求解器能够达到比采用ChronGear求解器更高的模拟速率。 
结合EVP预处理的P-CSI求解器在16,875核心上使得核心模拟速率提高了1.7倍,从原来的6.2模拟年年每天到10.5模拟年每天。 
 

\begin {figure}[t!]
\centering

\includegraphics[height=6.5cm]{NEWPOPStepComp_pcsi.eps}
\caption[] { 0.1度海洋模式 POP采用EVP块预处理的P-CSI后的各部分的时间分析。 \label{fig:StepComp_pcsi}}
\end{figure}



为了找到性能提升的主要来源,\ref{fig:component}给出了正压求解器的具体的时间成分分析。 
图\ref{fig:component}中可以看出, P-CSI求解器比原始的ChronGear求解好的主要原因是因为全局归约操作的减少。 
全局归约操作的减少同时也显著的减少了海洋模式POP对操作系统噪声的敏感程度\cite{ferreira},因为更少的归约操作意味着相邻两次全局同步之间的时间变得更长了。 
除此之外,EVP块预处子还可以进一步减少边界通信,因为它是的达到收敛所需要的迭代次数变少了。 
在大规模并行环境下,正压求解器的计算开销相比于全局归约操作和边界通信的开销来说基本上可以忽略,因此,EVP预处理子引入的那多出来的一倍的计算量对总体的运行时间影响不大。
最后我们还要注意到,在处理器核心数小于1200时,全局归约操作的时间是随着进程数的增加而减少的,这与等式\ref{t_pcg}所给出的理论结果是一致的。

 

\subsection{Edison超级计算机测试结果}
Now we run the 0.1度 POP simulations on the Edison supercomputer
to verify that performance improvements are not unique to Yellowstone.
Edison, which is the newest supercomputer at the National Energy
Research Scientific Computing Center (NERSC), consists of 133,824 2.4
GHz Intel
为了证明新的求解器不仅仅在黄石超级计算机上有性能提升,我们还在Edison超级计算机上测试0.1度海洋模式POP的模拟效果。Edison超级计算机是美国能源研究科学计算中心(NERSC)的最新的一台超级计算机,它是由 133,824 个2.4 GHz的英特尔的``Ivy Bridge'' 处理器核心组成,这些处理器核心是通过 8GBps 的Cray
Aries高速网络以蜻蜓拓扑的结构连接起来。  
\begin {figure*}[t!]
\begin{center}
\includegraphics[height =7cm]{01deg_comp_all_gs}
\hspace{10pt}
\includegraphics[height =7cm]{01deg_comp_all_halo}
\end{center}
\vspace{-.2in}
\caption[] {黄石超级计算机上,0.1度海洋模式POP的正压求解器中主要的耗时成分的分析:全局归约(左)和边界通信(右)。  }
\label{fig:component}
%\vspace{-.2in}
\end {figure*}
\begin {figure*}[t!]
\begin{center}
\includegraphics[height=6.5cm]{01deg_solverruntime_edison}
\hspace{10pt}
\includegraphics[height=6.5cm]{01deg_speedup_edison}
\end{center}
\vspace{-.2in}
\caption []{在美国Edison超级计算机上0.1度 POP中正压模态模拟一天的的执行时间(左), 
在美国黄石超级计算机上0.1度 POP核心部分的模拟速率(右)。\label {fig:runtime01_edison}}
\vspace{-.2in}
\end {figure*}

图\ref{fig:runtime01_edison}可以看出,四种配置的正压求解器在Edison超级计算机上的模拟性能与在黄石超级计算机上的模拟性能有相近的特点。 
我们注意到在Edison超级计算机上模拟时,全局通信的时间开销要比在黄石超级计算机上的开销更不稳定一些,主要可能是因为网络资源的竞争造成的\cite{wang2014}。 
这最终导致了ChronGear求解器(不管采用什么预处理子)的运行时间在多次模拟中相差很大,因此我们将这些实验中最好的三个结果取平均值作为它的运行时间。 
由于P-CSI几乎没有全局归约操作(只发生在检查收敛性的时候), 这种多次运行中的差别很小。 

 
在Edison超级计算机上, 采用对角预处理的P-CSI方法使得0.1度的海洋模式POP的正压模态在16,875核心上加速了3.7倍,从26.2秒每模拟天下降到7.0秒每模拟天。 
使用EVP预处理,ChronGear和P-CSI方法的性能都得到提升。 
采用EVP块预处理的P-CSI方法使得正压模态加速了5.6倍。 
 
 
\begin{table}
\begin{center}
\caption {在黄石超级计算机上,0.1度海洋模式POP的核心模拟速率\label{tab:improve_01}}
\begin{tabular}{|l||r|r|r|r|r|r|r|}
\hline
Solver & 470  & 1200   & 2700 & 4220 & 7500 & 10800 & 16875\\\hline
\hline
ChronGear     &0.7 &1.7&3.4  &4.6 &6.0 &6.0 &5.3\\\hline
ChronGear+Evp &0.7 &1.7&3.4  &4.9 &6.6 &6.9 &6.5\\\hline
P-CSI+Diag    &0.7 &1.7&3.5  &5.0 &7.0 &8.3 &8.9\\\hline
P-CSI+Evp     &0.7 &1.7&3.5  &5.0 &7.2 &8.6 &9.3\\
\hline
\end{tabular}
\end{center}
\end{table}

\section{Evaluating the new solver} \label{se:ver}
%Currently, there is no standard utility in POP for evaluating the effect of code modifications (such as a new solver) that will yield non bit-for-bit (BFB) results but should still produce the same mean climate.
Due to the chaotic nature of the ocean dynamics, even a round-off difference from the barotropic solver may potentially result in
distinct model solutions.
Therefore, because we cannot guarantee bit-for-bit (BFB) identical results in ocean solutions when a new solver is introduced,
we needed to show that the use of P-CSI with EVP did not result in inaccuracies (or even a changed climate)
before it could be formally incorporated into a POP release.

When POP is ported to a new machine, a similar situation occurs where running the same simulation on the two machines is not expected to produce BFB results.
The existing POP procedure to verify that a port to a new machine was successful involves running a specific case on the new machine for five simulation days, and
then computing the root-mean-square error (RMSE) between the new solution and the standard dataset released by NCAR for the SSH (sea surface height) field.
%There is no direct verification tool for new solvers  in CESM POP currently, but it provides a way to facilitate the evaluation of a successful port on new machines.
%That is, to run a specific case on the new machine for five days, then compute the root-mean-square (RMS) difference of SSH field between the solution on local machine and those released as standard dataset by NCAR.
While this procedure provides a simple criterion for evaluating CESM results on new machines (which may contain errors due to the software or hardware environment),
we found that it was insufficient for detecting and evaluating solver-induced errors.
For example, we ran the 1\degree\space case for three years with different convergence tolerances varying from $10^{-10}$ to $10^{-16}$ in the barotropic solver (default is $10^{-13}$)
and calculated the RMSE between a given case and the most strict tolerance case ($10^{-16}$).
Figure \ref{fig:ssh_rmse_t} shows the RMSE for the temperature field with various convergence tolerances for each month, and clearly
error introduced by modifying the solver convergence tolerance is not revealed in the temperature field (nor was it evident in any of the other diagnostic fields, such as velocity and SSH).
We had expected that the simulations with tolerances of $10^{-10}$ and $10^{-11}$ would have larger RMSE values than the others.
However, this was not the case, and, during months twelve and twenty, the $10^{-10}$ case has almost the smallest RMSE.
%In the last two months, the $10^{-11}$ case has a smaller RMSE than all other cases except the $10^{-10}$ case.
Note that to isolate the effect of the linear solver, we only looked at error in the open seas (POP does not simulate well on several marginal seas).


%While RMSE can reveal an averaged difference between two cases, it is not sufficient to determine whether that difference is indicative of an altered climate.
Because the existing simple RMSE test was insufficient for detecting whether the climate had been altered, we developed an alternative to evaluating the new ocean solver using a statistical approach.  Rather than relying on a single simulation, an ensemble of simulations can better represent the natural variability of the chaotic climate simulation, as described in \cite{baker2014methodology} in the context of data compression for the CESM Community Atmosphere Model (CAM), and be
used as a baseline for evaluating non-BFB modifications.  Similar to \cite{baker2014methodology}, we create an ensemble of simulations which are identical to the default setup except for an order $10^{-14}$ perturbation in the initial ocean temperature. This perturbation size is not expected to produce different climate model states.  We found that an ensemble of size 40 was sufficient for our purposes to represent the variability in the ocean, and we ran longer simulations than for CAM (12-months) due to the longer time-scales present in the ocean. Also note that we ultimately chose to evaluate only the three-dimensional temperature field (instead of the two-dimensional SSH) as we found it to be the most useful diagnostic variable for revealing differences.





%However, it is not a good criterion for new algorithms because it does not take into consideration of the chaotic nature of climate models.
%In order to make the chaotic nature accounted, we employ the methodology as proposed in CESM atmosphere component CAM by Baker et al. \cite{baker2014methodology}. First we conduct an ensemble of runs which are identical to the original case except for a $10^{-14}$ perturbation in the initial temperature. This perturbation is a reasonable round-off error which climate model should be able to tolerate.

We determine whether the new result is consistent with the reference ensemble results as follows.
We define the ensemble output at time $T$ as $\mathcal{E}=\{X_1,X_2,...,X_m\}$, where
$m$ is the size of the ensemble.
At a given point $j$, we have a series of possible results for each variable $X$ from the ensemble $\{X_1(j),X_2(j),...,X_m(j)\}$.
As the ensemble size increases, this series more correctly reflects the distribution of reasonable realization at the given point.
We define the mean and standard deviation of this series at point $j$ as $\mu (j) $ and
$\delta (j)$, respectively.
 %as  $$ \mu (j) = \frac{1}{m}\sum_{i=1}^m X_i(j), $$
%and standard deviation as  $$ \delta (j) = \sqrt{\frac{1}{m} \sum_{i=1}^m (X_i(j)-\mu(j))^2 }.$$
Let the new case have the result $\tilde{X}$, 
then the root-mean-square Z-score indicates the average error between the new case and the ensemble data:

$$ RMSZ(\tilde{X}, \mathcal{E}) =  \sqrt{\frac{1}{n}\sum_{j=1}^n(\frac{\tilde{X}(j) -\mu (j)}{\delta (j)})^2}$$
%The combination of $\mu$ and $\delta$ provides a criterion to test whether an additional case is close to the ensemble or not.
%Set the additional case has the result $\tilde{x}$, define the root-mean-square Z-score
We then re-evaluated the various solver tolerances using the ensemble-based RMSZ measurement.
Figure \ref{fig:ssh_rmsz_t} indicates that, unlike the simple RMSE test, the new ensemble-based method
is able to identify larger errors due to less strict convergence tolerances.  Now
the two cases with the loosest tolerances clearly have RMSZ scores on the same order as the error
they introduced into the solver and are noticeably removed from the ensemble distribution.
This success led us to use the ensemble-based metric to evaluate our new solver and find that the P-CSI results were consist with those of the ensemble (as were the default and stricter tolerances).

\begin{figure}[!t]
\begin{center}
\includegraphics[height=6.5cm]{temp_rmse.eps}
\end{center}
\vspace{-.1in}
\caption[] {Monthly Root Mean Square Error (RMSE) of temperature for cases with different convergence tolerances in 1\degree\space POP.}
\label{fig:ssh_rmse_t}
\vspace{-.1in}
\end{figure}
\begin{figure}[!t]
\begin{center}
\includegraphics[height=6.5cm]{temp_rmsz.eps}
\end{center}
\vspace{-.1in}
\caption[] {Monthly Root Mean Square Z-score for temperature for cases with different convergence tolerances. The yellow area represent the range of RMSZ within the 40-member ensemble.}
\vspace{-.1in}
\label{fig:ssh_rmsz_t}
\end{figure}


 %cases provides a criterion to judge whether the case  is consistent with the them or not.
%Also, for those cases which depart so far away from the ensemble, such as the cases with the first and second largest tolerance, their RMSZ is in the same order as the order of error they introduced to the solver.
%So the error introduced by replacing ChronGear with PCG is in the same magnitude order of improving the convergence tolerance from $10^{-13}$ to $10^{-14}$ or $10^{-16}$.

\input{relatedwork}

%----------------------------------------------------------------------------
\section{Conclusion} \label{se:conc}

The scalability of high-resolution CESM climate simulations has been
impeded by poor performance of the ChronGear barotropic solver in POP at
large core counts.  This paper improves solver performance by
reducing communication costs via an alternative solver with fewer
global reductions and by improving convergence via the development of
block EVP preconditioner particularly well-suited to the barotropic
mode.  The performance of the resulting solver, P-CSI with block-EVP,
is evaluated on two machines commonly-used for CESM simulations, and
solver speedup is as high as 5x.  Confidence that the solver did
not adversely impact the ocean simulation was ensured by adapting an
ensemble-based consistency strategy to the POP, allowing for the solver's
inclusion in a future CESM release.  The new barotropic solver
will clearly benefit both future low- and high-resolution CESM simulations,
particularly for the fully-coupled model whose scalability has historically been inhibited
by POP.

% Low resolutions will also benefit as they are typically run for very long time periods.


% Even though the new barotropic solvers have more benefit in high
% resolution simulations, it also contributes to the low resolution
% simulations.  The P-CSI solver is verified to maintain a stable ocean
% climate within a small tolerance using an ensemble based statistical
% method.  In closing, this paper highlights a scalable and robust
% barotropic solver for free-surface ocean models.




%ACKNOWLEDGMENTS are optional
\section{Acknowledgments}
Computing resources were provided by the Climate
Simulation Laboratory at NCAR's Computational and Information Systems
Laboratory (sponsored by the NSF and other
agencies) and the National Energy Research Scientific Computing
Center, a DOE Office of Science User Facility supported by the Office
of Science of the U.S. Department of Energy under Contract
No. DE-AC02-05CH11231.

This work is supported in part by a grant from the National Natural Science Foundation
of China (41375102), the National Grand Fundamental Research 973 Program of China (No. 2014CB347800), and the National High Technology Development Program of China
(2011AA01A203).


%Generated by bibtex from your ~.bib file.  Run latex,
%then bibtex, then latex twice (to resolve references)
%to create the ~.bbl file.  Insert that ~.bbl file into
%the .tex source file and comment out
%the command \texttt{{\char'134}thebibliography}.
\bibliographystyle{abbrv}
\bibliography{hycs}  % sigproc.bib is the name of the Bibliography in this case
% This next section command marks the start of
% Appendix B, and does not continue the present hierarchy
%\section{More Help for the Hardy}
%The sig-alternate.cls file itself is chock-full of succinct
%and helpful comments.  If you consider yourself a moderately
%experienced to expert user of \LaTeX, you may find reading
%it useful but please remember not to change it.
%\balancecolumns % GM June 2007
% That's all folks!
\end{document}
