\section{实验结果} \label{se:exp}


 
我们首先利用美国国家大气研究中心NCAR-Wyoming超级计算中心
(NWSC) \cite{loft:2015}的黄石超级计算机来评测一下采用了我们的新的正压模态求解器的CESM1.2.0的性能。 
黄石超级计算机由 72,576个
2.6-Ghz英特尔型号为Xeon E5-2670 的``Sandy Bridge'' 处理器构成。 
这些处理器都是通过13.6 GBps InfiniBand网络连通。   
黄石超级计算机上50\%以上的计算资源都是在使用地球系统模式CESM来进行气候模拟, 
因此提高CESM在黄石超级计算机上的效率对于全球气候变化研究来讲十分的有意义\cite{wf2014}。 


由于我们只是改动了海洋模式分量POP中的求解器,所以实验中我们将着重测试POP的性能。 
我们采用的CESM的
``G\_NORMAL\_YEAR''分量配置集,这个配置中只有海洋和海冰分量模式是真实计算的,而大气等分量模式采用气候数据输入。
我们针对两种使用频率最高的海洋模式POP水平网格分辨率来进行实验:
1度 ($320\times 384$) 和0.1度 ($3600\times 2400$).
值得一提的是,CESM1.2.0的默认配置中将黄石超级计算机上的MPI的环境需求限制(MPIMP\_EAGER\_LIMIT)设置为0, 这个变量主要控制在网络通信约会协议被使用之前的最大的消息量。我们发现使用黄石超级计算机上的默认的环境需求限制配置,也就是MP\_EAGER\_LIMIT = 131072,能够显著的改善CESM的运行性能。 



\subsection{低分辨率模拟}
图\ref{fig:runtime1}给出了几个可用的求解器在黄石超级计算机上模拟1度海洋模式POP时正压模态的运行时间。  
采用默认的对角预处理子, 在各种并行度下, P-CSI要比ChronGear的性能好。
在最大的并行度下(768处理器核心),P-CSI将每个模拟天中求解器的的执行时间从0.58s减少到了0.41s,也就是1.4倍的加速比)。 
另外,使用新的EVP块预处理,默认的ChronGear和P-CSI求解器的收敛性都得到提升。 
在768核上, 使用EVP预处理的P-CSI的计算效率为0.37s每模拟天, 相对于默认的对角预处理的ChronGear求解器加速了1.6倍。 

\begin {figure}[!t]
\centering
\includegraphics[height =6.5cm]{NEW1deg_solverruntime}
\caption []{1度POP每模拟天中正压模态的运行时间.\label {fig:runtime1}}
\end {figure}

\begin{table}[!h]
\begin{center}
\caption{在黄石超级计算机上,1度 POP总运行时间的优化比例。 \label{tab:improve_1}}
\begin{tabular}{|l||l|l|l|l|l|}
\hline
进程数 & 48  & 96  & 192 & 384 & 768\\\hline
\hline
ChronGear+EVP & -.5\% & 1.1\%  & 6.5\% & 10.8\%  & 12.1 \% \\\hline
P-CSI+Diagonal  & .7\% &3.9\% &9.3\%  &11.0\% & 12.6 \% \\\hline
P-CSI+EVP	      &-2.4\% & .4\%	& 7.4\%  & 14.4\% & 16.7\%\\\hline
\end{tabular}
\end{center}
\end{table}


正压模态的改进使得整个海洋模式POP的运行时间也随之下降。
表 \ref{tab:improve_1} 给出了三个新的求解器和预处理子的不同搭配相比于原始的对角预处理的ChronGear求解的所带来的性能提升百分比。 
这里给出的时间都是 取之于一个五天的模拟,同时模式的初始化和 I/O 操作都没有考虑在内。 
P-CSI 结合EVP预处理在768核上取得了  16.7\% 的加速效果。
虽然16.7\% 的提高看起来并不显著,但是由于 1度分辨率的海洋模式POP通常的模拟时间尺度都是几个世纪, 这样的一个性能的改进在如此长期的模拟中可以节省数百万个小时的CPU计算时间。 
另外,当海洋模式POP中开启生物化学的模块时,即使是1度的分辨率也需要使用相对较多的处理器核心,因为这个模块需要引进很多新的示踪物,进而导致计算量增加很多。 

\subsection{高分辨率模拟}
\begin {figure*}[t!]
\begin{center}
\includegraphics[height =6.5cm]{NEW01deg_solverruntime}
\hspace{10pt}
\includegraphics[height =6.5cm]{NEW01deg_speedup_ys}
\end{center}
\caption []{ 在美国黄石超级计算机上0.1度 POP中正压模态模拟一天的的执行时间(左), 
在美国黄石超级计算机上0.1度 POP核心部分的模拟速率(右)。\label {fig:runtime01}}
\end {figure*}


现在我们测试在黄石超级计算机上,高分辨率0.1度的海洋模式POP的新的正压求解的可扩展性。 
在这个高分辨下, 海洋模式中网格块的大小和分布得选择,会影响到进程上任务的分发,进而对性能造成很大的影响。 
因此,为了不让这种选择影响到最终的可扩展性结果,我们非常仔细的将每个进程上的网格块的横纵比例统一地设置为3:2,而陆地的比例设置为0.25,同时还利用空间填充曲线的方法来划分网格。 
我们使用POP中默认的时间步长,也就是500个时间步每天(dt\_count = 500)。 
最后,为了统一,所有的求解器都是每十个迭代步做一次收敛判断。 
这里值得一提的是,由于P-CSI的迭代过程的开销相对比较小(相比于海洋模式POP中的收敛性判断), 因此,如果将收敛性判断的频率进一步降低的话, P-CSI的性能可能能够进一步的提高。 
 

如图\ref{fig:runtime01}左边所示,ChronGear的性能在所使用的处理器核心数大于2700时反而会有所下降,
而P-CSI的运行时间在这之后相对比较平缓。 
采用对角预处理的前提下, P-CSI方法在16,875核上将0.1度的海洋模式POP的正压模态的性能提高了4.3倍(从19.0秒下降到4.4秒每模拟天)。 
使用EVP预处理能够进一步的改进ChronGear和P-CSI的性能, 使得这两个算法分别相对于原始的正压模态求解器加速了1.4倍和5.2倍。
在\ref{se:baro}节中,我们证明了原始的正压模态求解器在海洋模式POP的总运行时间所占的比例会随着使用的核心数目的增加而增大。
尤其是在16,875个核上,采用对角预处理的ChronGear求解器占到总运行时间的 50\%以上。 
相对应的,图\ref{fig:StepComp_pcsi}表明,采用了具有 更高可扩展性的EVP预处理的P-CSI求解器后, 同样是在 16,875 核上运行,正压模态的执行时间仅仅占到总运行时间的16\%。 

 
正压求解器性能上的改进有助于提高海洋模式POP的总体性能, 尤其是在大规模并行环境中。
模拟速率(模拟年每天)是一个被广泛采用的评估模式性能的指标, 我们这里采用不考虑模式初始化和I/O开销的 核心模拟速率。这是由于模式初始化只是在模式运行开始的时候执行一次,而模式的I/O开销是随着用户设置的频率的变化而改变的。 
通常专家们认为,如果想要进行长期的气候模拟,那么模拟速度最小要达到五个模式年每天\cite{dennis2012computational}, 图\ref{fig:runtime01} 右边可以看出,模式采用新的P-CSI求解器能够达到比采用ChronGear求解器更高的模拟速率。 
结合EVP预处理的P-CSI求解器在16,875核心上使得核心模拟速率提高了1.7倍,从原来的6.2模拟年年每天到10.5模拟年每天。 
 

\begin {figure}[t!]
\centering

\includegraphics[height=6.5cm]{NEWPOPStepComp_pcsi.eps}
\caption[] { 0.1度海洋模式 POP采用EVP块预处理的P-CSI后的各部分的时间分析。 \label{fig:StepComp_pcsi}}
\end{figure}



为了找到性能提升的主要来源,\ref{fig:component}给出了正压求解器的具体的时间成分分析。 
图\ref{fig:component}中可以看出, P-CSI求解器比原始的ChronGear求解好的主要原因是因为全局归约操作的减少。 
全局归约操作的减少同时也显著的减少了海洋模式POP对操作系统噪声的敏感程度\cite{ferreira},因为更少的归约操作意味着相邻两次全局同步之间的时间变得更长了。 
除此之外,EVP块预处子还可以进一步减少边界通信,因为它是的达到收敛所需要的迭代次数变少了。 
在大规模并行环境下,正压求解器的计算开销相比于全局归约操作和边界通信的开销来说基本上可以忽略,因此,EVP预处理子引入的那多出来的一倍的计算量对总体的运行时间影响不大。
最后我们还要注意到,在处理器核心数小于1200时,全局归约操作的时间是随着进程数的增加而减少的,这与等式\ref{t_pcg}所给出的理论结果是一致的。

 

\subsection{Edison超级计算机测试结果}
Now we run the 0.1度 POP simulations on the Edison supercomputer
to verify that performance improvements are not unique to Yellowstone.
Edison, which is the newest supercomputer at the National Energy
Research Scientific Computing Center (NERSC), consists of 133,824 2.4
GHz Intel
为了证明新的求解器不仅仅在黄石超级计算机上有性能提升,我们还在Edison超级计算机上测试0.1度海洋模式POP的模拟效果。Edison超级计算机是美国能源研究科学计算中心(NERSC)的最新的一台超级计算机,它是由 133,824 个2.4 GHz的英特尔的``Ivy Bridge'' 处理器核心组成,这些处理器核心是通过 8GBps 的Cray
Aries高速网络以蜻蜓拓扑的结构连接起来。  
\begin {figure*}[t!]
\begin{center}
\includegraphics[height =7cm]{01deg_comp_all_gs}
\hspace{10pt}
\includegraphics[height =7cm]{01deg_comp_all_halo}
\end{center}
\vspace{-.2in}
\caption[] {黄石超级计算机上,0.1度海洋模式POP的正压求解器中主要的耗时成分的分析:全局归约(左)和边界通信(右)。  }
\label{fig:component}
%\vspace{-.2in}
\end {figure*}
\begin {figure*}[t!]
\begin{center}
\includegraphics[height=6.5cm]{01deg_solverruntime_edison}
\hspace{10pt}
\includegraphics[height=6.5cm]{01deg_speedup_edison}
\end{center}
\vspace{-.2in}
\caption []{在美国Edison超级计算机上0.1度 POP中正压模态模拟一天的的执行时间(左), 
在美国黄石超级计算机上0.1度 POP核心部分的模拟速率(右)。\label {fig:runtime01_edison}}
\vspace{-.2in}
\end {figure*}

图\ref{fig:runtime01_edison}可以看出,四种配置的正压求解器在Edison超级计算机上的模拟性能与在黄石超级计算机上的模拟性能有相近的特点。 
我们注意到在Edison超级计算机上模拟时,全局通信的时间开销要比在黄石超级计算机上的开销更不稳定一些,主要可能是因为网络资源的竞争造成的\cite{wang2014}。 
这最终导致了ChronGear求解器(不管采用什么预处理子)的运行时间在多次模拟中相差很大,因此我们将这些实验中最好的三个结果取平均值作为它的运行时间。 
由于P-CSI几乎没有全局归约操作(只发生在检查收敛性的时候), 这种多次运行中的差别很小。 

 
在Edison超级计算机上, 采用对角预处理的P-CSI方法使得0.1度的海洋模式POP的正压模态在16,875核心上加速了3.7倍,从26.2秒每模拟天下降到7.0秒每模拟天。 
使用EVP预处理,ChronGear和P-CSI方法的性能都得到提升。 
采用EVP块预处理的P-CSI方法使得正压模态加速了5.6倍。 
 
 
\begin{table}
\begin{center}
\caption {在黄石超级计算机上,0.1度海洋模式POP的核心模拟速率\label{tab:improve_01}}
\begin{tabular}{|l||r|r|r|r|r|r|r|}
\hline
Solver & 470  & 1200   & 2700 & 4220 & 7500 & 10800 & 16875\\\hline
\hline
ChronGear     &0.7 &1.7&3.4  &4.6 &6.0 &6.0 &5.3\\\hline
ChronGear+Evp &0.7 &1.7&3.4  &4.9 &6.6 &6.9 &6.5\\\hline
P-CSI+Diag    &0.7 &1.7&3.5  &5.0 &7.0 &8.3 &8.9\\\hline
P-CSI+Evp     &0.7 &1.7&3.5  &5.0 &7.2 &8.6 &9.3\\
\hline
\end{tabular}
\end{center}
\end{table}
