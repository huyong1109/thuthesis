\section{正压求解器} \label{se:baro}
 
 
我们给出ChronGear方法的具体算法
\ref{alg:pcg}, 以供参考.  ChronGear主要包括三个部分:计算部分,边界通信部分和全局通信部分。
计算部分主要包括矩阵向量乘操作,向量向量乘操作以及向量的伸缩操作(向量与常数相乘)等具有很好的可扩展性的操作。 
边界通信会在每次矩阵向量乘操作之后被调用,其开销随着计算使用的处理器核心数的增加和减少,但是有一个下界。 
但是,我们后面的章节将会证明,每一步迭代过程中的內积操作\ref{pcg_global1}之后需要调用的全局通信,当使用很大的核心数时,将会成为可扩展性的瓶颈。
 

%----------------------------------------------------------------------------
\subsection{通信瓶颈}\label{se:bottleneck}
  
假设使用$p=m^2$个进程,每一个进程恰好分得一个网格块(这在高分辨率POP中是比较常见的配置)。 
这是正压模态的总运行时间就等于ChronGear求解器在每一个块上的运行时间。 
我们记求解器的每一次迭代过程中,计算、边界通信和全局通信操作的耗时分别为$\mathcal{T}_c$,$\mathcal{T}_b$ 和$\mathcal{T}_g$。 
%----------------------------------------------------------------------------


从算法\ref{alg:pcg}中可以得出, 计算开销
$\mathcal{T}_c$主要涉及到四个向量伸缩操作 (算法中步骤\ref{pcg_scale1}, \ref{pcg_scale2} ,\ref{pcg_scale3}, 和
\ref{pcg_scale4}), 两个向量向量乘积操作或內积操作 (算法中步骤\ref{pcg_dot1} 和 \ref{pcg_dot2}), 以及一个矩阵向量乘积操作 (算法步骤\ref{pcg_mat}).  
因此,$\mathcal{T}_c= (4 n^2 +2n^2+ 9n^2)\theta + \mathcal{T}_{p}=15\frac{\mathcal{N}^2}{p}\theta+\mathcal{T}_{p}$,
这里$\theta$ 表示单位时间内浮点操作的次数,$\mathcal{T}_{p}$ 表示预处理的开销。
比如,当采用对角预处理时,$\mathcal{T}_{p} =\frac{\mathcal{N}^2}{p}\theta$。
上面的定义可以看出,当计算核心数增加时,$\mathcal{T}_c$减少,并且以零为下界。

 
当调用了像矩阵向量乘和除了对角预处理以外的预处理等需要一层或一层以上的边界区域的操作时,在每个进程的边界区域上需要进行边界更新操作。 
由于在海洋模式POP中,每个进程都维护着它自己的块以外的两个边界层,因此,即使使用了非对角的预处理,每次迭代过程中仍只需要进行一次边界通信。
边界通信的实际开销取决于网络延迟和边界区域的大小。 当边界区域的层数为2时(这也是POP中的默认配置), 边界区域的每个边的大小为$2n$ ,而且这个大小会随着进程数的增大而变小。 
 
因此,每一步迭代中,边界通信的时间复杂度为$\mathcal{T}_b =4\alpha +(4\times 2n)\beta=4\alpha +(\frac{8\mathcal{N}}{\sqrt{p}})\beta $这里$\alpha$ 表示单条点对点通信消息的延迟,而
$\beta$ 表示从网络中传输一比特需要的时间 (也就是网络带宽的倒数).
边界更新所需要的时间也是随着就算进程的增多而减少,但是有一个下界$4\alpha$。 


Combining all three components, the execution time of one diagonal preconditioned ChronGear solver step can be expressed as:
ChronGear求解器的每一步迭代中仅仅包含了一次全局归约操作。全局规约操作主要由一次MPI\_allreduce和一个去除路地网格点的掩盖操作,因此全局归约操作的开销满足$\mathcal{T}_g= 2\frac{\mathcal{N}^2}{p}\theta + \log p \alpha$ (假设网络采用二叉树结构)。 
路地点的掩盖操作的开销会随着进程数$p$的增大而减小,而MPI\_allreduce的开销将会一直随着进程数的增加而增大。 
注意到每次全局归约操作实际上近似于没有数据的交换,因为每个进程只传递了两个数值。
因此,全局通信的开销主要取决于网络的延迟而不是网络的传输速率。

\begin{eqnarray}
%\begin{tabular}{l}
\label{t_pcg}
&\mathcal{T}_{cg}=\mathcal{K}_{cg} (\mathcal{T}_c + \mathcal{T}_b+\mathcal{T}_g )\nonumber \\
&=\mathcal{K}_{cg} [18 \frac{\mathcal{N}^2}{p}\theta + \frac{8\mathcal{N}}{\sqrt{p}}\beta +(4+\log p)\alpha]
%\end{tabular}
\end{eqnarray}
这里 $K_{cg}$ 表示算法达到指定收敛条件所需要的迭代次数。
这个迭代次数不会随着进程数的增加而改变 \cite{hu2013scalable}。 
方程\ref{t_pcg})表明,计算和边界更新所需要的时间开销都会随着进程数的增加而减少。 
但是,全局归约操作的时间开销则会随着进程数的增多而增加。 
因此,我们可以预测到,当进程数操作某一个值时, ChronGear求解器的运行时将会增加。 
图 \ref{fig:ChronGearCOMP}给出了在美国国际大气研究中心的黄石超级计算(详情见节\ref{se:exp})上,运行一个模拟天时,ChronGear求解器的全局通信和边界通信两个部分的耗时分析。 
这里我们可以看到,当使用数千个处理器核心时,全局归约操作的时间变成了整个求解器耗时的主要部分,并且这个时间还在继续增大。 
 


\begin{figure}[!t]
%\vspace{-10pt}
\begin{center}
	\includegraphics[height=6.5cm]{newChronGear_comp.eps}
\end{center}
\vspace{-.2in}
\caption[] {在美国黄石超级计算机上,0.1度POP的全局归约操作和边界更新操作占模拟一天总时间开销的比例。}
\label{fig:ChronGearCOMP}
\vspace{-.1in}
\end{figure}
 
