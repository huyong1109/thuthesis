\section{P-CSI求解器} \label{se:psi}

为了提高海洋模式POP的可扩展性,一个好的正压模态求解器应该具备的特点是需要尽可能少的全局归约操作。 
在文章\cite{hu2013scalable}中,胡勇等人提出了一个基于Stiefel的CSI方法的求解器。这个求解器被现在了单独版本的POP中。 
这里,我们将通过添加一个预处理的结构,进一步改进CSI方法。
这个改进后的方法我们称之为P-CSI方法。 我们将这个方法实现在地球系统模式CESM框架下的海洋模式分量POP中。 
早在1985年, Saad 等人\cite{saad1985solving}将传统的CSI的方法应用于向量机上,并且指出当方程中矩阵的特征值已知的条件下, 这种方法比共轭梯度法更适用于某些并行情况。 


P-CSI算法和它的性质与CSI方法相似,只是多加了一个预处理接口(将在下一章详细讨论)。 
值得一提的是,P-CSI方法与CSI方法一样,在每一步迭代过程中不需要做內积操作, 因此它即使在大核数上也能够保持较好的可扩展性。
为海洋模式POP设计的P-CSI求解器的伪码在算法\ref{alg:ppsi}中。
使用对角预处理的P-CSI方法的每一步迭代的计算时间为$T_c =\frac{12\mathcal{N}^2}{p}\theta+\mathcal{T}_p =\frac{13\mathcal{N}^2}{p}\theta$, 
而P-CSI求解器的总运行时间为如下
\begin{eqnarray}
\label{t_psi}
\mathcal{T}_{pcsi} = \mathcal{K}_{pcsi}(\mathcal{T}_c + \mathcal{T}_b ) \nonumber \\
= \mathcal{K}_{pcsi}[13\frac{\mathcal{N}^2}{p}\theta+ 4\alpha + \frac{8\mathcal{N}}{ \sqrt{p}}\beta]
\end{eqnarray}
这里$K_{pcsi}$ 表示P-CSI方法收敛到给定条件时所需要的迭代步数。 
\begin {figure}[!t]
\begin{center}
\includegraphics[height=6.5cm]{solver_iteration}
\end{center}
\vspace{-.2in}
\caption []{1度POP中,Lanczos方法的迭代次数对P-CSI迭代步数的影响。 \label{fig:iter}}
%\vspace{-.1in}
\end {figure}

\begin{algorithm}[!t]
\caption{ 预处理的传统Stiefel迭代算法}
\label{alg:ppsi}
\begin{scriptsize}
\begin{algorithmic}[1]
\REQUIRE 与网格块$B_{i,j}$相对应的系数矩阵 $\textbf{B}$, 预处理子$\textbf{M}$, 初始值$\textbf{x}_0$和方程右端向量$\textbf{b}$ ;预估的特征值区间$[\nu,\mu]$;  \\
 // \qquad    \textit{所有进程并行执行}
\STATE $\alpha =\frac{2}{\mu -\nu}$, $ \beta = \frac{\mu +\nu}{\mu -\nu}$, $\gamma = \frac{\beta}{\alpha}$, $\omega_0 =\frac{ 2}{\gamma}$;\quad $k = 0$;
\STATE $\textbf{r}_0 = \textbf{b}-\textbf{B}\textbf{x}_0$; $\Delta \textbf{x}_{0} = \gamma^{-1}\textbf{M}^{-1}\textbf{r}_0$; $\textbf{x}_1 =\textbf{x}_0 +\Delta \textbf{x}_{0}$; $\textbf{r}_1 =\textbf{b} -\textbf{B}\textbf{x}_1$;
\WHILE{$k \leq k_{max}$ }
\STATE $k=k+1$;
\STATE $\omega_k = 1/(\gamma - \frac{1}{4\alpha^2}\omega_{k-1})$; \COMMENT{迭代函数}
\STATE $\textbf{r}'_{k} =\textbf{M}^{-1}\textbf{r}_{k}$; \COMMENT{预处理} \
\STATE $\Delta \textbf{x}_{k} =\omega_k\textbf{r}'_{k}+(\gamma \omega_k-1)\Delta \textbf{x}_{k-1}$;
\STATE $\textbf{x}_{k+1} =\textbf{x}_{k}+\Delta \textbf{x}_{k}$;
\STATE $\textbf{r}_{k+1} =\textbf{b}- \textbf{B}\textbf{x}_{k+1}$; \COMMENT{矩阵向量乘}
\STATE $update\_halo(\textbf{r}_{k+1})$; \COMMENT{边界通信}
\IF { $k \%  n_{c} == 0$ }
\STATE 收敛性检查;
\ENDIF
\STATE $convergence\_check(\textbf{r}_{k+1})$;  \COMMENT{收敛性检查}
\ENDWHILE
\end{algorithmic}
\end{scriptsize}
\end{algorithm}


P-CSI需要对预处理后的系数矩阵$M^{-1}A$的最大特征值$\mu$和最小特征值$\nu$进行估计。 
海洋模式POP中的系数矩阵$A$ 和它的对角预处理矩阵$M = \Lambda(A)$都是实对称矩阵, 因此预处理后的矩阵的最小最大特征值并不难估计。 
论文\cite{hu2013scalable} 中给出$A$的特征值的估计,这里我们采用Lanczos方法\cite{Paige1980235}来对$M^{-1}A$的最大最小特征值进行估计。
实验中,我们发现将Lanczos方法的收敛条件因子$\epsilon$ 设成$0.15$在不同的分辨率(1度或者0.1度)以及不同的预处理子(单位阵预处理、对角预处理和EVP预处理)的情况下都能达到较好的效果。 
图\ref{fig:iter} 可以看出只需要很少的Lanczos步骤就可以估计出比较恰当的$M^{-1}A$的最大最小特征值,从而使得P-CSI方法能够在较短的迭代步数内收敛。
在实际运行中,我们发现Lanczos方法的开销与调用数次ChronGear求解的开销相当。 

与ChronGear的迭代过程相比,P-CSI除了做收敛性检查之外不需要任何的全局归约操作。 
我们注意到,为了达到相同的收敛条件,P-CSI算法需要的迭代步骤比ChronGear的略多($K_{pcsi} > K_{cg}$)。 
以上的两点现象,我们可以解读出来,在当并行使用的处理器核心数比较小的时候,P-CSI应该会比ChronGear的开销更大一些。
因为处理器核心较少的时候,全局归约操作的开销并不算大。 
但是,当采用高分辨的网格时,需要用到更多的处理器核心,这是P-CSI的每一步迭代的开销应该会比ChronGear的要明显的快一些(参见公式\ref{t_pcg})和
(\ref{t_psi})。
而这将进一步减少P-CSI算法到达收敛的时间。 

当满足条件$\nu = \lambda_{min}$和$\mu =\lambda_{max}$时,P-CSI的收敛速度将达到最优值。 
但是$\lambda_{min}$和$\lambda_{max}$是比较难以估计的。
更重要的是,我们不能对系数矩阵$A$随意的做变换,因为 $A$ 是分布在各个进程上的。
为了能够利用POP的并行特性,我们使用Lanczos方法来构造 出一系列三对角矩阵$T_m (m=1,2,...)$,这些矩阵的最大最小特征值逐渐的向$M^{-1}A$的最大最小特征值逼近。
To utilize the parallelism of POP, we employ Lanczos method  to construct
实际运行中, 我们发现合适的选择Lanczos迭代的步数,我们能够得到对最大最小特诊一个比较好的估计,这个估计值能够使得P-CSI方法与预处理共轭梯度法有相近的收敛速度。  


\begin{algorithm}[!ht]
\caption{基于Lanczos方法的针对预处理矩阵的特征值估计方法}
\label{alg:lanczos_pre}
\begin{algorithmic}[1]
\REQUIRE 网格块$B_{i,j}$ 的系数矩阵$\textbf{B}$,预处理子$\textbf{M}$和随机向量$\textbf{r}_0$; \\
 //\qquad    \textit{所有进行并行执行}
\STATE $\textbf{s}_0=\textbf{M}^{-1}\textbf{r}_0$;\quad $\textbf{q}_1 = \textbf{r}_0/({\textbf{r}_0^T\textbf{s}_0})$;\quad $\textbf{q}_0=\textbf{0}$;
\STATE $T_0=\emptyset$;\quad $\beta_0 =0$;\quad  $\mu_0 =0$;\quad $j=1$;
\WHILE{$j<k_{max}$}
\STATE $\textbf{p}_j = \textbf{M}^{-1}\textbf{q}_j$; \quad $\textbf{r}_j=\textbf{B}\textbf{p}_j-\beta_{j-1}\textbf{q}_{j-1}$;
\STATE $update\_halo(\textbf{r}_j)$;
\STATE $\tilde{\alpha}_j =\textbf{p}_j^T\textbf{r}_j$; \quad $\alpha_j=global\_sum(\tilde{\alpha}_j)$;
\STATE $\textbf{r}_j=\textbf{r}_j-\alpha_{j}\textbf{q}_{j}$; \quad $\textbf{s}_j = \textbf{M}^{-1}\textbf{r}_j$;
\STATE $\tilde{\beta}_j = \textbf{r}_j^T\textbf{s}_j$; \quad $\beta_j=sqrt(global\_sum(\tilde{\beta}_j))$;
\STATE \textbf{if} $\beta_j == 0$ \textbf{then} \textbf{return}
\STATE $\mu_j = max(\mu_{j-1},\alpha_j+\beta_j+\beta_{j-1})$; \label{lan_gersh}\\
\STATE $T_j=tri\_diag(T_{j-1},\alpha_j,\beta_j)$; \COMMENT{三对角矩阵}\label{lan_tm}
\STATE $\nu_j = eigs(T_j,'smallest')$ ; \label{lan_nu}
\STATE \textbf{if} $|\frac{\mu_j}{\mu_{j-1}} -1 |< \epsilon\quad\textbf{and}\quad|1- \frac{\nu_j}{\nu_{j-1}}|< \epsilon$ \textbf{then} \textbf{return}; \label{lanczos_converge}
\STATE $\textbf{q}_{j+1}= \textbf{r}_j/\beta_j$;\quad $j=j+1$;
\ENDWHILE
\end{algorithmic}
\end{algorithm}

这个算法需要$\textbf{A}$ 是正定对称矩阵。海洋模式POP中直接得到的系数矩阵是负定对称矩阵,此时只需将方程两边同时乘以单位矩阵和-1,即可将系数矩阵转换成正定对称矩阵。 
 

