
\subsection{EVP并行预处理子}
 
前面描述的EVP方法是求解椭圆方程的一个比较好的解决方案。 
但是,EVP方法的一个缺陷是如果它的原始版本不能求解很大的区域上问题,否则在行进过程中会出现数值不稳定的现象\cite{roache1995elliptic}。
但是在比较小的区域上,比如大小为$12\times 12$的小区域,采用双精度浮点数计算的话,EVP求解得到的结果的误差大概是$\mathcal{O}(10^{-8})$ 。实际应用中,这个误差仍在可接受的范围之内。 
EVP在小区域上的高效性和比较小的计算复杂度,使得它成为并行块预处理的一个比较理想的选择。 
这里,我们提出了一个基于块上EVP方法的块预处理技术来进一步提高海洋模式POP中正压模态求解器的性能。
每一个预处理步就是并行的求解各个快上的椭圆方程 $B_i \textbf{x} = \textbf{y} (i=1,...,m^2)$组。 


 
事实上,EVP方法不适用于大区域的缺陷在大规模并行计算的条件下并不是一个问题,因为大规模并行使得每个进程分得的区域变成了一个个小块。
另外,我们得到的方程组中,与东西南北四个方向上的相邻点的系数比与其他方向上的相邻点的系数要小一到两个数量级。 
我们发现,将这些小的系数去掉(赋值为零),能够将EVP预处理的开销减少一半,同时并不影响对ChronGear和P-CSI等迭代算法的预处理效果。 
因此, EVP预处理开销的复杂度可以表示为$\mathcal{T'}_{p} = 14n^2\theta=
14\frac{\mathcal{N}^2}{p}\theta$。 
$\mathcal{T'}_{p}$的实际开销取决于局部块的大小,而这个大小是随着所使用的核数的增加而变小的。
结合了EVP块预处理之后, ChronGear和P-CSI求解器执行一次的总开销分别为 
\begin{eqnarray}
\label{t_evppcg}
&\mathcal{T'}_{cg}=\mathcal{K'}_{cg} (\mathcal{T'}_c + \mathcal{T'}_b+\mathcal{T'}_g )\nonumber \\
&=\mathcal{K'}_{cg} [31 \frac{\mathcal{N}^2}{p}\theta + \frac{8\mathcal{N}}{\sqrt{p}}\beta +(4+\log p)\alpha],
\end{eqnarray}
和
\begin{eqnarray}
\label{t_evppsi}
\mathcal{T'}_{pcsi} = \mathcal{K'}_{pcsi}(\mathcal{T'}_c + \mathcal{T'}_b ) \nonumber \\
= \mathcal{K'}_{pcsi}[26\frac{\mathcal{N}^2}{p}\theta+ 4\alpha + \frac{8\mathcal{N}}{ \sqrt{p}}\beta],
\end{eqnarray}
 
\begin {figure}[!t]
\centering
\includegraphics[height=6.5cm]{iteration.eps}
\vspace{-.1in}
\caption[] {Average number of iterations for different barotropic solvers. \label{fig:iteration}}
\vspace{-.1in}
\end{figure}


在海洋模式中采用EVP并行块预处理显著的减少了ChronGear和P-CSI求解器到达收敛条件所需要的迭代步数。
如图\ref{fig:iteration}所示, EVP预处理将1度和0.1度分辨率的海洋模式中的迭代算法的迭代次数减小了约三分之二,这一结果 跟近似逆预处子\cite{smith1992parallel} (这一预处理子并没有在地球系统模式CESM的海洋模式分量POP中实现)的效果相当。
尽管EVP预处子是的每一步迭代过程中的的计算量增加了一倍,它使得在正压模态在大核数上运行时占主要运行时间的全局通信和边界通信的次数也减少了一半。 
EVP预处理子的另外一个优势是它的前处理过程的开销比较小。 
在高分辨率0.1度的例子中, 使用512核并行时,计算预处理矩阵的开销比调用一次求解器的开销还要小。 并且,这一开销会随着所使用的处理器核心数的增加而减少,因为并行度的增加会导致每个进程上的块变小。 
值得一提的是, 0.1度海洋模式中迭代算法所需要的迭代步数比1度的模式中的少,这是因为高分辨的海洋模式的网格在经纬两个方向上的网格大小比更加接近于1, 这也就导致了所得到的系数矩阵的条件数更小一些。 