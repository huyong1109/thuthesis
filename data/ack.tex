% 如果使用声明扫描页,将可选参数指定为扫描后的 PDF 文件名,例如:
% \begin{ack}[scan-statement.pdf]
\begin{acknowledgement}
  由衷感谢我的导师杨广文教授在我读博期间对我科研和生活上的指导和关怀。
  杨老师在课题研究和论文撰写的每一个阶段,都给予我充分的指导和关心。
  杨老师知识渊博,治学严谨, 是我科研工作中学习的榜样。
  他严以律己、宽以待人的高尚品格对我的影响尤为大,将使我终身受益。
  除了科研,杨老师在生活上给予了我很多的关心。
  他经常询问我生活中是否有困难,鼓励和劝诫我们年轻人要利用空余时间多参加体育活动。
  杨老师还带领我们在科研之余组织足球比赛和游泳活动。
  频繁的体育锻炼,不仅强健了我的身体,更使得我在艰苦的博士生涯中能够保持良好的心态,勇敢的面对各种压力和挑战。 
  再次感谢杨老师一直以来对我在学术科研和生活中的鼓励和帮助,使得我能顺利的完成博士学业。

  特别感谢我的副指导老师黄小猛副教授,黄老师极富创造性思维,在科研工作中给了我很多指导和启发。 
  黄老师治学态度严谨、一丝不苟,我写的每一篇文章黄老师都会字句斟酌的帮我修改。好几个春节假期,黄老师都是在跟我一起紧张的修改论文中度过的。
  黄老师对待学生非常诚恳,亦师亦友。 
  黄老师鼓励和支持我多次出国交流,极大地提高了我的科研水平。
  再次感谢黄老师的指导和帮助,使得我能够在广阔的知识海洋中找到前进的方向。


  感谢付昊桓老师、刘利老师和王晓鸽老师在科研工作和论文撰写中给予的指导。 他们给我提供了许多宝贵意见,使得我的论文能够得以顺利发表。
  感谢甘霖、徐世真、唐强、倪裕芳、褚阳、魏万敬等实验室的同学的帮助和支持,你们使得我的博士研究生生活丰富多彩。 

  特别感谢我的家人一直以来对我的精神上的鼓励和经济上的支持,我的父母时时鼓励我按照自己的想法勇敢的去闯。
  感谢我的未婚妻徐逸筠一直以来对我的扶持,是她的出现使得我的博士生涯不再只有苦涩的科研。
  
  感谢清华大学为我创造了良好的科研和生活环境。
  最后,向所有帮助和支持过我的人们表示衷心感谢!

\end{acknowledgement}
