% !Mode:: "TeX:UTF-8"
%!TEX root = ../main.tex
%!TEX program = xelatex
% 如果使用声明扫描页,将可选参数指定为扫描后的 PDF 文件名,例如:
% \begin{ack}[scan-statement.pdf]
\begin{acknowledgement}
  由衷感谢我的导师杨广文教授对我科研和生活上的指导与关怀。
  杨老师在我选择研究方向、论文撰写与投稿等各个方面都给予了充分的指导和关心。
  杨老师知识渊博、治学严谨,是我科研工作中学习的榜样。
  除了科研,杨老师在生活上给予了我很多的关心。
  他经常询问我生活中是否有困难,鼓励和劝诫我多利用空余时间参加体育活动。
  杨老师还带领我们经常组织足球比赛和游泳活动。
  频繁的体育锻炼,不仅强健了我的身体,更使得我在艰苦的博士生涯中能够保持良好的心态,勇敢的面对各种压力和挑战。 
  再次感谢杨老师一直以来对我在学术科研和生活中的鼓励和帮助,使得我能顺利的完成博士学业。

  特别感谢我的指导老师黄小猛副教授。黄老师极富创造性思维,在科研工作中给了我很多指导和启发。 
  黄老师治学态度严谨、一丝不苟。
  我写的每一篇文章黄老师都会字句斟酌的帮我修改。
  犹记得2012年的除夕夜,黄老师都是在跟我一起紧张的修改论文中度过的。
  黄老师对待学生非常诚恳,亦师亦友。 
  黄老师鼓励和支持我多次出国交流,极大地提高了我的科研水平。
  再次感谢黄老师的指导和帮助,使得我能够在广阔的知识海洋中找到前进的方向。


  感谢付昊桓老师、刘利老师、王晓鸽老师、王斌老师和薛巍老师在科研工作给予的指导。 
  他们用宽广的知识面帮我提升了研究的层次。
  感谢美国国家大气研中心的曾于恒老师,Frank Bryan和Allison Baker等人在我出国交换期间的指导和帮助。
  是他们让我在异国他乡也能够感受到家的温暖。
  感谢阮华斌、耿益峰、祝美琪和汪文灿等师兄,以及甘霖、徐世真、唐强、倪裕芳、褚阳、魏万敬等实验室同学的帮助和支持,他们积极活泼的生活方式使得我博士生活丰富多彩。 

  
  特别感谢我的家人一直以来对我的精神上的鼓励和经济上的支持。
  我的父母时时鼓励我按照自己的想法勇敢的去闯,他们的支持免去了我读博的后顾之忧。
  我的未婚妻徐逸筠与我相识相知相爱于我的博士期间,并且即将和我一起走进婚姻殿堂。她的出现给我的博士生涯增添了另一道色彩。
  
  感谢清华大学为我创造了良好的科研和生活环境。在清华大学读博士是我终生难忘的一段经历。
  最后,向所有帮助和支持过我的人们表示衷心感谢!

\end{acknowledgement}
