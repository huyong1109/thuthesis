\chapter{基于Chebyshev迭代的高可扩展正压求解技术}
\label{cha:barosSolver}

\section{本章概述}

公共地球系统模式中的海洋分量并行海洋模式(the Parallel Ocean Model, POP) 求解的是采用静力平衡近似和Boussinesq近似的描述海洋运动的三维原始方程组。 
它将时间积分分成两个部分:斜压模态和正压模态。 
斜压模态描述的是三维的动力学和热动力学过程,而正压模态是求解原始方程垂直积分后得到的二维动量方程和连续性方程。
POP在正压模态中采用隐式的时间差分方案,得到了一个关于海表高度的椭圆微分方程。 
POP将这个椭圆方程空间离散化之后得到一个线性系统$Ax=b$,并且使用共轭梯度法来进行求解。
然而,这个方程的求解过程在POP中可扩展性很差。
已有研究表明,POP正压求解可扩展性较差的根源是全局通信\cite{Worley:2011:PCE:2063384.2063457}。 
如果能优化POP的正压求解过程,将会使得POP乃至整个公共地球系统模式的性能有较大的提升\cite{dennis2012computational}。 


本章的主要内容是通过分析正压模态的通信瓶颈,设计一个适合并行海洋模式正压模态的高可扩展的求解器。 
本章首先从海洋模式的原始方程导出正压模态的理论模型,进而分析离散化之后得到的线性方程组的基本性质,同时给出方程组系数矩阵的条件数与网格横纵比例、时间步长以及网格点数目等之间的关系。 
其次,我们建立了一个量化海洋模式正压模态中求解线性方程组的共轭梯度法的可扩展性瓶颈的模型。
基于这个模型,我们确定了共轭梯度法在分布式并行系统上的可扩展性不好的原因是每一步迭代过程中都需要时间开销很大的全局归约操作。 这个模型还表明,以前被视为效率不如共预处理轭梯度法的传统的Stiefel迭代方法(CSI)在某些大规模并行环境中反而更有优势。
最后,我们在海洋模式POP中实现了基于CSI方法的正压求解器。 相比于共轭梯度法,CSI算法的迭代过程中的迭代参数不是由前一步迭代后的残差的內积计算得到的,而是由系数矩阵$A$的特征值谱决定。 
我们采用Lanczos方法解决了估计大规模矩阵$A$的特征值这一难题。 
通过Lanczos方法可以构造出一个规模很小的三对角矩阵,它的特征值逐渐的逼近系数矩阵的$A$的部分特征值。 
使用CSI方法替换原来的共轭梯度法,可以完全消除海洋模式正压模态中的归约操作以及其所导致的较差的可扩展性。 
实验表明,新的CSI方法比原来的共轭梯度法有更好可扩展性。在15,000核上,CSI方法使得正压模态的求解速度提高了4.7倍。





\section{海洋模式正压模态}
\label{solver:baro}

海洋模式是由描述海洋运动基本定律的微分方程组所构成的系统。由于这个系统是高度非线性的,很难得到有实用价值的解析解。我们需要采用数值方法求解,即将地球划分成一个三维网格,然后在这个三维网格上进行各种物理量的迭代演算。演算过程中加入所需要的边界和强迫场,这样就能够近似的模拟海洋的运动和变化过程。
为了完整的推导我们的新的正压求解器,这里我们简要的描述一下海洋模式POP中的控制方程。
海洋模式中的原始动量和连续性方程可以表达如下:
\begin{align}
&\frac{\partial }{\partial t} \textbf{u} +\mathcal{L}(\textbf{u}) + f\times \textbf{u} = - \frac{1}{\rho_0}\nabla p +F_H(\textbf{u}) +F_V(\textbf{u}) \label{eq:momen}\\
&\mathcal{L}(1) = 0 \label{eq:continuous}
\end{align}
这里$\mathcal{L}(\alpha ) = \frac{\partial }{\partial x} (u\alpha)  +\frac{\partial }{\partial y} (v\alpha) +\frac{\partial }{\partial z} (w\alpha)$,它在$\alpha =1$时与平流算子等价,$x, y, z$ 分别是水平和垂直坐标变量,$\textbf{u} = [u, v]^T$ 表示水平速度向量,$w$ 表示垂直速度,$f$ 表示科氏力系数, $p$ 和 $\rho_0$ 分别表示压力和海水密度,$F_H$ 和 $F_V$分别表示水平和垂直耗散项 \cite{smith2010parallel}。  


%----------------------------------------------------------------------------
\subsection{正压模态原始方程} \label{solver:mode}

为了更加高效的求解三维的原始方程,海洋模式中的时间积分被分解为两个模态。 一个是求解三维动力学和热动力学过程的斜压模态,另一个则是求解二维海表高度(SSH)变化的正压模态。
正压模态的海表高度方程是对三维原始方程做垂直积分之后得到的,主要作用是将速度场中较快的部分分离出来,使用二维的正压模态来实现更快的求解。

在静力平衡近似条件下,POP中海底深度为$z$的位置上的压力可以分解为两个部分:  
\begin{align}
\displaystyle p = p_h + p_s = \int^0_z g\rho dz +p_s
\end{align}
这里$p_h$表示静力平衡压力,$p_s$表示由于海表自由面波动而引起的海表压力,$g$表示由于地球重力加速度。 
正压模态的控制方程是由原始动量方程和连续性方程(\ref{eq:momen},\ref{eq:continuous}) 从海洋底部到海洋表面垂直积分而得到的:
\begin{align}
&\displaystyle \frac{\partial \textbf{U} }{\partial t}  = -g \nabla \eta + F  \label{eq:baro_mon}\\
&\displaystyle \frac{\partial \eta }{\partial t} = - \nabla \cdot H\textbf{U} + q_w  \label{eq:baro_con}
\end{align}
这里  $\textbf{U} =  \frac{1}{H+\eta}\int_{-H}^\eta dz \textbf{u}(z) \approx \frac{1}{H }\int_{-H}^0 dz \textbf{u}(z)$是正压速度的垂直积分,
$\eta = p_s/{\rho_0g}$表示海表高度,$H$表示海洋底部的深度,$q_w$表示单位海表面积的活水通量,而$F$表示对海洋动量方程(\ref{eq:momen})中除了与时间趋势和海表压力梯度相关的所有项做垂直积分所得到的量。
 
垂直积分时,需要用到垂直方向上的边界条件:
\begin{align}
\label{eq:bound_w}
w = \left\{ \begin{array}{ll}
\frac{\partial}{\partial t} \eta  +\textbf{u}\cdot\nabla \eta - q_w, & z = \eta  \\
0, & z = -H
\end{array} \right.
\end{align}
 
值得一提的是,为了简化求解过程,正压连续性方程 (\ref{eq:baro_con}) 中忽略掉了边界条件中的小项 $\nabla \eta$\cite{smith2010parallel}。
为了能够使用更长的时间步长,海洋模式POP的正压模态中采用了隐式格式,得到一个椭圆系统。 
首先通过时间差分,空间上仍使用微分形式,方程  (\ref{eq:baro_mon})和(\ref{eq:baro_con}) 可化为
\begin{align}
&\displaystyle \frac{ \textbf{U}^{n+1} - \textbf{U}^{n-1}}{\tau}  = -g \nabla \eta^{n+1} + F \label{eq:udt} \\
&\displaystyle \frac{\eta^{n+1} - \eta^n }{\tau}  = - \nabla \cdot H\textbf{U} ^{n+1}+ q_w \label{eq:etadt}
\end{align}
这里$\tau$表示与差分格式所使用的时间步长。 
用方程(\ref{eq:udt})中下一时刻的正压速度来替换方程(\ref{eq:etadt})中的正压速度,就得到了一个关于海表高度$\eta$的椭圆系统
\begin{equation}
\label{eq:sshdiscret}
     [-\nabla\cdot H \nabla + \frac{1}{g  \tau^2}]\eta^{n+1}
           = -\nabla\cdot H[\frac{\textbf{U}^{n-1}}{g \tau} + \frac{F}{g}] + \frac{\eta^n}{g\tau^2} +\frac{q_w}{g\tau}
\end{equation}
 
为了简单,我们将椭圆方程(\ref{eq:sshdiscret})重新标记为
\begin{equation}
\label{eq:ssh}
[-\nabla \cdot H\nabla +\frac{1}{g  \tau^2}]\eta^{n+1} = \psi(\eta^n,\eta^{n-1},\tau)
\end{equation}
这里 $\psi$表示一个关于当前时刻和上一时刻$\eta$的状态的函数。
 

\subsection{正压线性方程组}
\label{solver:baroproperty}
\begin{figure}%[!htbp]
\centering
\includegraphics[height=7cm]{grid_domain.pdf}
\caption[] {海洋模式分量中网格划分\label{fig:grid1}}
\end{figure}

如图\ref{fig:grid1}所示,POP在水平方向上采用Arakawa B网格\cite{smith2010parallel},并且采用九点差分格式对方程(\ref{eq:ssh})进行离散。得到如下离散格式:
\begin{align}
    & \nabla\cdot H \nabla \eta  =\frac{1}{\Delta y}\delta_x \overline{[\Delta y H  \delta_x\overline{\eta}^y]}^y +\frac{1}{\Delta x}\delta_y \overline{[\Delta x H  \delta_y\overline{\eta}^x]}^x \label{eq:nabla2}
  \end{align}

这里 $\Delta_\xi$和$\delta_\xi$  ($\xi \in \{x, y\} $)分别为有限差分和它们相应的偏导, $\delta_\xi (\cdot) $和$\overline{(\cdot)}^\xi $ 分别表示有限差分算子和平均算子:  
\begin{align}
&\delta_\xi \varphi = [\varphi (\xi+\Delta_\xi/2) -\varphi(\xi-\Delta_\xi/2)]/\Delta_\xi \\
&\overline{\varphi}^\xi  =[\varphi (\xi+\Delta_\xi/2) +\varphi(\xi-\Delta_\xi/2)]/2
\end{align}
%----------------------------------------------------------------------------
 
 

为了避免极点问题,POP采用偏移的或者三极点的广义正交网格。这使得方程的系数非常的繁杂。
为了能够简单而不失偏颇,我们这里给出系数矩阵在均匀的网格和给定的不变的海底深度$H$的情况下的显式表达式。
表达式(\ref{eq:nabla2})此时可以表示为
\begin{align}
 [\nabla\cdot H \nabla \eta]_{i,j}&= -\frac{H}{S_{i,j}}(A_{i,j}^O\eta_{i,j}+A_{i,j}^{NW}\eta_{i-1,j+1}+A_{i,j}^N\eta_{i,j+1} \nonumber\\
 &+A_{i,j}^{NE}\eta_{i+1,j+1}+A_{i,j}^W\eta_{i-1,j} +A_{i,j}^E\eta_{i+1,j} \nonumber\\
& +A_{i,j}^{SW}\eta_{i-1,j-1} +A_{i,j}^S\eta_{i,j-1}+ A_{i,j}^{SE}\eta_{i+1,j-1})
\end{align}
这里$S_{i,j}  = \Delta x\Delta y$, $A_{i,j}^{\chi } ( \chi \in \mathcal{Q} = \{NW,NE, SW, SE, W, E, N, S\})$ 分别表示网格点 $(i,j)$和它自己、以及它的邻居点之间的在 九点差分格式中的系数(\ref{eq:nabla2})。 这些系数由$\Delta x$,$\Delta y$,$\tau$和$H$决定。
\begin{equation} \label{defineA}
\begin{aligned}
&\alpha  = \frac{ \Delta y}{ \Delta x }, \quad \beta  = 1/\alpha \\
&A_{i,j}^{NW} = A_{i,j}^{NE} =A_{i,j}^{SW} = A_{i,j}^{SE} = - (\alpha  +  \beta  )/4 \\
&A_{i,j}^{W} = A_{i,j}^{E} = (  \beta  -\alpha  )/2 \\
&A_{i,j}^{N} = A_{i,j}^{S} = (\alpha  -\beta )/2 \\
&A_{i,j}^{O} =   \alpha   +\beta  \\
\end{aligned}
\end{equation}


%&\alpha_1 = \frac{2\Delta y}{2\Delta x_j +\Delta x_{j+1}2}\\
%& \alpha_2= \frac{2\Delta y}{ \Delta x_j +\Delta x_{j-1}  } \\
%&\alpha_3 = \frac{\Delta y(\Delta x_{j-1}+2\Delta x_j +\Delta x_{j+1})}{2 (\Delta x_j +\Delta x_{j-1})(\Delta x_j +\Delta x_{j+1})} \\
%& \beta_1 = \frac{(\Delta x_j +\Delta x_{j+1})}{8 \Delta y} \\
%& \beta_2 = \frac{(\Delta x_j +\Delta x_{j-1})}{8 \Delta y} \\
%& \beta_3 = \frac{(\Delta x_{j-1}+2\Delta x_j +\Delta x_{j+1})}{8 \Delta y} \\
%&A_{i,j}^{NW} = A_{i,j}^{NE} = \alpha_1 + \beta_1 \\
%&A_{i,j}^{SW} = A_{i,j}^{SE} = \alpha_2 +  \beta_2 \\
%&A_{i,j}^{W} = A_{i,j}^{E} = \alpha_3 - 2 \beta_3 \\
%&A_{i,j}^{N} = -2\alpha_1  + 2 \beta_1 \\
%&A_{i,j}^{S} = -2\alpha_2  + 2 \beta_2  \\
%&A_{i,j}^{O} = -2\alpha_3  -4 \beta_3  \\
%Then,the coefficients between the given point $(i,j)$和its other neighbors can be computed from $A^n$,$A^e$和$A^{ne}$ on its neighbors。
我们假设全球区域上的网格大小为$N\times N$,则$1\le i\le N$,$1\le i\le N$。  
方程\ref{eq:ssh}在给定点$(i,j)$上的离散格式可以表示为  
\begin{align}
\label{eq:sten}
&(A_{i,j}^O+\phi ) \eta_{i,j}+A_{i,j}^{NW}\eta_{i-1,j+1}+A_{i,j}^N\eta_{i,j+1} +A_{i,j}^{NE}\eta_{i+1,j+1}+A_{i,j}^W\eta_{i-1,j}  \nonumber\\
& +A_{i,j}^E\eta_{i+1,j} +A_{i,j}^{SW}\eta_{i-1,j-1} +A_{i,j}^S\eta_{i,j-1}+ A_{i,j }^{SE}\eta_{i+1,j-1}= \frac{S_{i,j}}{H}\psi_{i,j}
\end{align}
这里$\phi = \frac{S_{i,j}}{g \tau^2H}$是与单位网格大小、海底深度和时间步长有关的一个变量。

\begin{figure}
\centering
\includegraphics[height=7cm]{SparsePatternSample}
\caption[] {大小为$30\times 15$的网格上椭圆微分方程采用九点差分格式所得到的线性方程组的系数矩阵的稀疏模式。 \label{fig:spy}}
\end{figure}
因此椭圆方程(\ref{eq:sshdiscret})就变成了一个关于$\eta$的线性方程组。我们记这个方程组为 $Ax= b$,其中$A$表示由系数 $A_{i,j}^{\chi } ( \chi \in \mathcal{Q}, i,j \in {1,\cdots, N)}$ 构成的块对角矩阵。
方程组的规模随着网格分辨率的提高而变大。对于一度的海洋模式,得到方程组包含有$\aleph = 138,240$个未知量,而0.1度分辨率的POP中,方程组中未知量的个数$\aleph$ 为$8,640,000$。这意味着如果用单精度浮点数将0.1度POP正压模态线性方程组的系数矩阵显式的存储起来需要大概300TB的空间。
值得庆幸的是,海洋模式POP正压模态中得到的线性方程组是稀疏的。
方程(\ref{eq:sten})还表明 $A$ 每一行只有九个非零元素,也就是说$A$是一个稀疏矩阵。 图\ref{fig:spy}展示了大小为$30\times 15$的网格点上得到的系数矩阵$A$的稀疏模式,可以看出来$A$具有带状的对角形式,它只有在对角及其附近的条带内有少量的非零元,而其他点的值都是零。
$A$的元素总个数为202,500,但是只有3870个非零元素。非零元在总元素中的占比为1.9\%左右。
方程( \ref{defineA})和 (\ref{eq:sten})表明$A$是由水平网格的大小、海底深度和时间步长等决定的。
它们之间的关系将在第\ref{solver:Algorithm:condition}节中详细讨论。

\subsection{特征值谱和条件数}
\label{solver:Algorithm:condition}

POP正压模态中线性方程组的系数矩阵 $A$ 是正定对称的\cite{smith2010parallel},因此它的特征值都是正实数\cite{stewart1976positive}。
假设系数矩阵的特征值谱\cite{golub2012matrix} 是 $\mathcal{S} = \{\lambda_1, \lambda_2, \cdots, \lambda_\aleph\}$,这里 $\lambda_{min} = \lambda_1 \le \lambda_k \le \lambda_\aleph = \lambda_{max}$( $1<k <\aleph $,$\aleph=N\times N$ 为 $A$的大小,即网格点的总数目 )表示$A$的所有特征值。

利用 Gershgorin圆盘定理\cite{bell1965gershgorin}可知,对于任意的 $\lambda \in \mathcal{S}$,都存在一个数对 $(i,j)$ 满足
\begin{align}
&|\lambda -  (A_{i,j}^O + \phi ) | \le \sum_{\chi \in \mathcal{Q} = \{NW,NE,SW,SE,W,E,N,S\}}|A_{i,j}^\chi|
\end{align}
由方程(\ref{defineA})给出的系数的定义,我们可以得到如下结论 
\begin{align} \label{eigsGersh}
&\lambda_{max} \le  \max (  5\alpha - \frac{1}{\alpha}, \frac{5}{\alpha}- \alpha) +\phi   \\
&\lambda_{min} \ge 2\min (  \alpha - \frac{1}{\alpha},\frac{1} {\alpha} -  \alpha) + \phi
\end{align}

以上结论说明,当网格横纵比例($\alpha  = \frac{ \Delta y}{ \Delta x }$)越接近于1时,最大特征值的上界会随之减少,而最小特征值的下界会随之变大。
也就是说,系数矩阵特征值谱的半径($[\lambda_{min}, \lambda_{max}]$)随着网格横纵比向着1的接近而变小。 
当水平网格的横纵比等于1时,即$ \alpha = \frac{ \Delta y}{ \Delta x} = 1$,我们可以得到$\lambda_{max} \le  4 +\phi$,$\lambda_{min} \ge   \phi$。
这时,系数矩阵的条件数(即 $\kappa=  \lambda_{max}/\lambda_{min}$),由于是由特征值谱的半径所决定的,也会随着网格横纵比向1的靠近而变小。 

\begin{figure}[ht]
\centering
\includegraphics[height=7cm]{conditionNumberAspectRatio}
\caption[] {固定网格点数$\aleph = 20\times 20$和$\phi = 0$时,正压线性方程组的系数矩阵的条件数与网格横纵比例的关系。 \label{fig:conditionNumberRatio}}
\end{figure}
图\ref{fig:conditionNumberRatio}所给出的结果验证了我们的这个结论。图\ref{fig:conditionNumberRatio}的横坐标表示网格的横纵比例,纵坐标表示当使用固定的网格点数$\aleph = 20\times 20$和固定的$\phi = 0$时,对应的系数矩阵的条件数。 
可以看出,在区间$(1, +\infty)$上,系数矩阵的条件数随着网格横纵比例的增加而增加,而在区间$(0,1)$上,系数矩阵的条件数随着网格横纵比例的增大而变小。
系数矩阵的条件数在网格横纵比例等于1时取到最小值。
 
\begin{figure}[ht]
\centering
\includegraphics[height=7cm]{conditionNumberTimestep}
\caption[] {固定网格点数$\aleph= 20\times 20$和网格横纵比$\Delta x /{\Delta y} = 1$时,正压线性方程组的系数矩阵条件数与时间步长的关系。 \label{fig:conditionNumberDt}}
\end{figure}
特征值的下界是$\phi=\frac{S }{g \tau^2 H}$,这个变量是由时间步长和水平网格的面积与海底深度的比所确定的。
它表明,特征值的下界会随着时间步长的增大而减小。 
相应的,当特征值的上界相对固定时,系数矩阵的条件数也就随之增大。
图\ref{fig:conditionNumberDt}证明了这个推论。
这个图的横坐标表示正压求解时所采用的时间步长,纵坐标表示使用固定的网格点数$\aleph= 20\times 20$,和固定的网格横纵比$\Delta x /{\Delta y} = 1$时,对应的系数矩阵的条件数。可以看出来,时间步长越长,矩阵的条件数越大。在时间步长很小时,矩阵的条件数随着时间步长的增大而急剧增大,当时间步长大于0.5秒时,矩阵的条件数几乎保持不变。
这主要是由于当时间步长过大时,网格大小和横纵比对矩阵条件数的影响要远大于时间步长的影响。

\begin{figure}[ht]
\centering
\includegraphics[height=7cm]{conditionNumberGridSize}
\caption[] {固定$\phi=0$且网格的横纵比等于1时,正压线性方程组的系数矩阵条件数与网格点的数目的关系。 \label{fig:conditionNumbGrid}}
\end{figure}
 
不考虑时间步长的因素(假设$\phi=0$)时,上面的分析表明,当网格的横纵比等于1时,不论网格点的个数是多少,系数矩阵的特征值谱半径限制在 $(0,4)$ 区间内。
但是,系数矩阵的条件数的变化却可能会很大,因为当网格点的数量目$\aleph$ 不断增加时,最小的特征值会逐渐的逼近零。 
图\ref{fig:conditionNumbGrid} 中解释了条件数和网格点数目之间的关系。 
图中横坐标表示网格点的总数目$\aleph$ ,纵坐标表示在不考虑时间步长,并且使用固定的网格横纵比例1的情况下,对应的系数矩阵的条件数。
可以看出来,随着网格点数的增多,系数矩阵的条件数不断变大。



%%%%%%%%%%%%%%%%%%%%%%%%%%%%%%%%%%%%%%%%%%%%%%%%%%%%%%%%%%%%%%%%%%%
\section{海洋模式正压求解器}
\label{solver:bottleneck}
%----------------------------------------------------------------------------

POP将全球区域划分为若干个小块,并将这些块分发给每一个进程。
每个进程只负责计算它所分得的块内网格点上的迭代过程,并且维护一个与周围进程进行数据交换的边界区域。 
为了简洁,在本文中,我们假设整个区域的网格大小为$N\times N$,它被划分为$m\times m$个大小为$n\times n$ ($n=N/m$)的小块。 
假使块$B(k,l)$上对应的系数矩阵和解分别为$\textbf{B}$和$\tilde{\textbf{x}}$,那么$\textbf{B}$就是系数矩阵$A$的一个大小为$n^2\times n^2$的对角子矩阵,并且每一行最多只有九个非零元\cite{hu2013scalable}。 
于是,矩阵向量乘操作 $\textbf{B}\tilde{\textbf{x}}$只包含有$9n^2$次浮点数乘法操作,而不是$n^2\times n^2$次操作。 
 
%----------------------------------------------------------------------------
\subsection{CG求解器} 
\label{solver:pcg}
 
海洋模式POP的正压模态中使用传统的共轭梯度法结合一个对角预处理子$M = \Lambda(A)$ 作为它的默认求解器。 
对角预处理是最简单的一种预处理方式,作用是去掉椭圆方程中各项奇异性给系数矩阵带来的影响。
下一章中,我们在讨论更为复杂的预处理时再考虑通用预处理的情况。
这个默认求解器在小规模并行时效率很高。 
对角预处理的共轭梯度法的过程如下: 

\begin{algorithm}[h]
\caption{共轭梯度法}
\label{alg:pcg}
\begin{algorithmic}[1]
\REQUIRE  与网格块$B_{i,j}$相应的系数矩阵$\textbf{B}$,初始解 $\textbf{x}_0$ 和 $\textbf{b}$  \\
//\qquad    \textit{所有进程并行执行}
\STATE $\textbf{r}_0 = \textbf{b}-\textbf{B}\textbf{x}_0$, $\textbf{s}_0 =0$;\quad $\beta_0=1$, $k=0$;
\WHILE{$k \leq k_{max}$ }
\STATE $k=k+1$; 
\STATE $\textbf{r}'_{k-1} =\textbf{M}^{-1}\textbf{r}_{k-1}$; \COMMENT{对角预处理} \label{AlgPcgPrecond}
\STATE $\tilde{\beta}_k = \textbf{r}_{k-1}^T\textbf{r}'_{k-1}$;\COMMENT{局部內积} \label{AlgPcgInner1}
\STATE $\beta_k = global\_sum(\tilde{\beta}_k)$; \COMMENT{全局规约操作} \label{AlgPcgGlobal1}
\STATE $\textbf{s}_k = \textbf{r}'_{k-1} +(\beta_k/\beta_{k-1})\textbf{s}_{k-1}$;\label{AlgPcgVec1}
\STATE $\textbf{s}'_k = \textbf{B}\textbf{s}_k$; \COMMENT {矩阵向量乘}\label{AlgPcgAx}
\STATE $update\_halo(\textbf{s}'_k$); \COMMENT{ 边界通信}\label{AlgPcgBound}
\STATE $\tilde{\alpha}_k = \textbf{s}_k^T\textbf{s}'_k$;\COMMENT{局部內积} \label{AlgPcgInner2}
\STATE $\alpha_k =\beta_k/ global\_sum(\tilde{\alpha}_k)$;\quad \COMMENT{全局规约操作}\label{AlgPcgGlobal2}
\STATE $\textbf{x}_k =\textbf{x}_{k-1} +\alpha_k \textbf{s}_k$;\label{AlgPcgVec2}
\STATE $\textbf{r}_k =\textbf{r}_{k-1} -\alpha_k\textbf{s}'_k$;\label{AlgPcgVec3}
\IF{ $k \% K_{c} == 0$}
\STATE \textbf{if} $||\textbf{r}_{k+1}|| \le \epsilon$  \textbf{return} ;\COMMENT{每 $K_c$ 检查一次是否收敛}
\ENDIF
\ENDWHILE
\end{algorithmic}
\end{algorithm}

 
正如算法\ref{alg:pcg}所示,共轭梯度法主要包含有以下三个部分: 计算,边界通信,全局通信。
计算部分主要包括矩阵向量乘操作,向量向量乘操作以及向量的伸缩操作(向量与常数相乘)等具有很好的可扩展性的操作。 
边界通信会在每次矩阵向量乘操作之后被调用,其开销随着计算使用的处理器核心数的增加而减少,但是有一个下界。 
每一步迭代过程中的內积操作之后需要调用的全局通信。后面的章节将会说明,当使用很大的核心数时,全局通信将会成为可扩展性的瓶颈。
 
%----------------------------------------------------------------------------

\subsection{共轭梯度法复杂度及评估}
\label{solver:pcgComplex}
假设使用$p=m^2$个进程来计算正压模态,也就是每个进程正好计算一个子块。 
正压模态的运行时间等于共轭梯度法在每个进程上的执行时间。 
假设$\mathcal{T}_c$,$\mathcal{T}_b$和$\mathcal{T}_g$分别表示每个迭代步中计算,边界交换和全局归约操作的时间。


算法\ref{alg:pcg}中,计算主要涉及到步骤\ref{AlgPcgPrecond}、\ref{AlgPcgVec1}、\ref{AlgPcgVec2}和 \ref{AlgPcgVec3}中的四个向量伸缩操作,步骤\ref{AlgPcgInner1}和\ref{AlgPcgInner2}中的两个向量向量乘积操作,和步骤\ref{AlgPcgAx}中的矩阵向量乘积。 
因此计算的复杂度为 $\mathcal{O} (4 n^2 +2n^2+ 9n^2)$。 
假设处理器处理单位浮点数所需要的时间为$\theta$,则计算的时间开销可以表示为
$\mathcal{T}_c=  (4 n^2 +2n^2+ 9n^2)\theta = 15n^2\theta  = 15\frac{N^2}{p}\theta $。 
很明显 $\mathcal{T}_c$ 会随着进程数的减少而减少,同时以零为下界。 


每个进程都需要与其相邻点进行边界交换,这个过程时间开销主要取决于网络延迟和边界缓存区域的大小。 
在海洋模式POP中,边界缓存区域的带宽默认设置为2,因此一次边界通信的数据量大小为 $2n$。 
很明显,这个数据量会随着进程数的增加而减少。 
每个进程需要与其周围的四个进程进行数据交换,因此一次边界更新所需要的时间为$\mathcal{T}_b =4\varpi + (4\times 2n)\vartheta=4\varpi + \frac{8N}{\sqrt{p}}\vartheta$,这里$\varpi$表示单条点对点通信消息的延迟,$\vartheta$ 表示从网络中传输单位比特需要的时间 (也就是网络带宽的倒数)。 
可以看出来,这个更新时间是随着进程数的增加而减少的,并且它有一个由网络延迟决定的大小为$2\varpi$的下界。



每次內积操作之后需要调用一次全局归约操作,归约操作实质上就是把每个进程上所计算得到的局部的內积值收集起来,
因此归约操作的数据传输的时间相比收集操作的时间是可以忽略的。 
全局规约操作主要由一次MPI\_allreduce和一个去除陆地网格点的掩盖操作组成,因此全局归约操作的开销满足$\mathcal{T}_g= 2\frac{\mathcal{N}^2}{p}\theta + \mathcal{G}(p)\varpi$  
这里$\mathcal{G}(\cdot)$是一个与指定架构中网络拓扑结构相关的函数。 
在最优的情况下,采用理想立方体或者树状网络中的$\mathcal{G}(\cdot)$是一个对数函数。 
因此,不管在什么情况下,$\mathcal{G}(p)\varpi$会随着进程数$p$的增加而单调的变大。 
 
一次共轭梯度法的执行时间可以表示为:
\begin{equation}
\label{t_pcg}
\mathcal{T}_{CG} = \mathcal{T}_c + \mathcal{T}_b + \mathcal{T}_g
= 15\frac{N^2}{p}\theta + 4\varpi + \frac{8N}{\sqrt{p}}\vartheta+2\frac{N^2}{p}\theta + \mathcal{G}(p)\varpi
\end{equation}
总个共轭梯度法的求解过程的执行时间可以表示为
\begin{equation}
\label{t_pcg_all}
t_{CG} = \mathcal{K}_{CG}\cdot \mathcal{T}_{CG}
\end{equation}
这里$\mathcal{K}_{CG}$表示共轭梯度法达到收敛所需要的迭代步数,它不会随着进程数的增加而改变。
我们将在第\ref{solver:CG:convergence_rate}节中详细分析共轭梯度法的收敛速度。
由方程(\ref{t_pcg})可知,计算和边界更新所需要的时间会随着进程数的增大而减少,但是全局归约操作的时间却会随着进程数的增加而增大。
因此,整个共轭梯度法的运行时间在当进程数大于某个数值时,会随之增大而增大。


 
我们在清华大学的探索100集群上,使用1度的海洋模式POP (360$\times$240 个网格点)做了一系列的实验。 
探索100包含有740个计算节点,每个计算节点都有两个2.93 GHz 英特尔Xeon X5670 6核处理器以及24/48 GB的内存。 
我们采用了一个并行程序的性能评测和追踪的工具--TAU\cite{shende2006tau},来分析CG求解器的时间开销。 
然后利用实验中每个正压步中不同分量的平均耗时和评测模型中的时间进行对比。 
在评测模型中,我们将$\mathcal{G}(\cdot)$设为一个线性函数,$\varpi$设为 $2\times 10^{-7}$,$\theta$ 设为$2.5\times 10^{-9}$ , $\vartheta$  设为$2.0\times 10^{-10}$。 


\begin{figure}[ht]
\centering
\includegraphics[height = 7cm]{evaluate_PCG}
\caption[] {1度POP中CG求解器的每一步迭代中各部分耗时分析\label{fig:pcg_ratio}}
\end{figure}
如图\ref{fig:pcg_ratio}所示,评测模型的结果和真实实验的结果吻合得非常好。 
实验中正压模态的计算时间确实是和进程数成反比的。 
实验中的边界更新时间在进程数超过100时基本上保持不变,这是因为当进程数比较大的时候,边界通信的消息大小变得非常小,以至于数据传输的时间相比于网络延迟可以忽略掉。 
全局归约操作的时间随着进程数的增加而成比例的变大,并且当使用的进程数大于100时,全局归约操作成为了正压模态中最为耗时的部分。 
这个结果和我们方程(\ref{t_pcg})给出的理论分析的结果是一致的,即全局归约操作是共轭梯度法可扩展性差的根本原因。 

\subsection{共轭梯度法收敛速度} \label{solver:CG:convergence_rate}

不同于直接方法,迭代算法的时间开销并不是固定的。
上一节式\ref{t_pcg_all}指出,CG求解器求解一次的总时间开销等于每步迭代的时间开销与达到收敛所需要的迭代步数的乘积。
因此,在实际求解过程中,共轭梯度法的迭代步数越多,CG求解器的时间开销越大。 
迭代法每一次求解所需要的迭代步数并不确定,但是,通常给出迭代算法收敛速度的估计。
在共轭梯度法中,每一步迭代的残差都有一个上界\cite{Liesen2004}
\begin{equation}
\frac{||\textbf{x}_k-\textbf{x}^*||_A }{||\textbf{x}_0-\textbf{x}^*||_A}  \le \min_{p\in \mathcal{P}_k, p(0) = 1 }\max_{\lambda \in \mathcal{S}} |p(\lambda)| \label{PcgConvergeRate}
\end{equation}
这里$\textbf{x}_k$表示第$k$步迭代后得到的解向量,$\textbf{x}^*$表示线性方程组的解析解(即$\textbf{x}^* = A^{-1}b$),$\lambda$ 表示系数矩阵$A$的特征值。

利用第一类Chebyshev 多项式来估计这个最小最大值,可以得到 
\begin{align}
\label{pcg_convergence}
\frac{||\textbf{x}_k-\textbf{x}^*||_A}{||\textbf{x}_0-\textbf{x}^*||_A}\le  2 (\frac{\sqrt{\kappa}-1}{\sqrt{\kappa}+1})^k
\end{align}
这里$\kappa =  \kappa_2(A)$表示系数矩阵$A$的条件数。
式(\ref{pcg_convergence})表明,共轭梯度法的收敛速度依赖于系数矩阵$A$的条件数。

 
\section{CSI求解器}
\label{solver:csi}

为了解决共轭梯度法的可扩展性瓶颈,新的求解器需要的全局归约操作越少越好。 
我们在海洋模式中重新考虑一些原本被认为是不如共轭梯度法的方法,比如Chebyshev迭代方法。
早在1985年,Saad等人\cite{saad1985solving}将传统的CSI的方法应用于向量机上,并且指出当方程中矩阵的特征值已知的条件下,这种方法比共轭梯度法更适用于某些并行情况。  
Gutknecht \cite{gutknecht2002chebyshev} 在2002年有一篇文章专门的重新研究了一下Chebyshev迭代在通信开销较大的大规模并行环境中性能。 
传统的Stiefel迭代(CSI)就是Chebyshev迭代方法中的一种。 


我们使用CSI方法在海洋模式POP中实现了一个并行正压求解器,称之为CSI求解器。
我们在算法\ref{alg:csi}中给出了CSI的迭代过程。
与共轭梯度法不同的是,CSI方法 在每一步迭代过程中不需要做內积操作,因此它在大核数上也能够保持较好的可扩展性。
CSI需要对预处理后的系数矩阵$A$的最大特征值$\mu$和最小特征值$\nu$进行估计。 
众所周知,计算稀疏矩阵的特征值要比求解线性方程组本身还要难。 
值得庆幸的是,海洋模式POP中的系数矩阵$A$都是实对称矩阵,因此系数矩阵的最小最大特征值并不难估计。 
 
\begin{algorithm}[h]
\caption{传统Stiefel迭代算法}
\label{alg:csi}
\begin{algorithmic}[1]
\REQUIRE 与网格块$B_{i,j}$相对应的系数矩阵 $\textbf{B}$,初始值$\textbf{x}_0$和方程右端向量$\textbf{b}$,预估的特征值区间$[\nu,\mu]$;  \\
 // \qquad    \textit{所有进程并行执行}
\STATE $\alpha =\frac{2}{\mu -\nu}$, $ \beta = \frac{\mu +\nu}{\mu -\nu}$, $\gamma = \frac{\beta}{\alpha}$, $\omega_0 =\frac{ 2}{\gamma}$;\quad $k = 0$;
\STATE $\textbf{r}_0 = \textbf{b}-\textbf{B}\textbf{x}_0$; $\textbf{x}_1 =\textbf{x}_0 -\gamma^{-1}\textbf{r}_0$; $\textbf{r}_1 =\textbf{b} -\textbf{B}\textbf{x}_1$;
\WHILE{$k \leq k_{max}$ }
\STATE $k=k+1$;
\STATE $\omega_k = 1/(\gamma - \frac{1}{4\alpha^2}\omega_{k-1})$; \COMMENT{迭代函数} \label{AlgCsiIter}
\STATE $\Delta \textbf{x}_{k} =\omega_k\textbf{r}_{k-1}+(\gamma \omega_k-1)\Delta \textbf{x}_{k-1}$; \label{AlgCsiVec1}
\STATE $\textbf{x}_{k} =\textbf{x}_{k-1}+\Delta \textbf{x}_{k-1}$; \label{AlgCsiVec2}
\STATE $\textbf{r}_{k} =\textbf{b}- \textbf{B}\textbf{x}_{k}$;\label{AlgCsiVec3}
\STATE $update\_halo(\textbf{r}_k)$; \COMMENT{边界通信}\label{AlgCsiBound}
\IF{ $k \% K_{c} == 0$}
\STATE \textbf{if} $||\textbf{r}_{k+1}|| \le \epsilon$  \textbf{return} ;\COMMENT{每 $K_c$ 检查一次是否收敛}
\ENDIF\ENDWHILE
\end{algorithmic}
\end{algorithm}

如算法\ref{alg:csi}所示,CSI方法和共轭梯度法有相似的迭代过程,迭代过程中有相似的计算和边界通信过程。
但是CSI方法将共轭梯度法中的两次向量內积操作及其相应的向量向量乘积的计算替换成了系数矩阵$A$的最大最小特征值的一个函数迭代。 我们接下来先分析CSI算法的复杂度,然后给出特征值估计的方法。

\subsection{CSI方法复杂度及评估} \label{solver:Algorithm:complex}

跟第\ref{solver:pcgComplex}节中共轭梯度方法复杂度的分析一样,我们也可以给出CSI求解器的计算和通信时间开销。
CSI求解器的计算时间可表示为
$\mathcal{T}_c =  (2 n^2 + 9n^2)\theta = 11n^2\theta =\frac{11N^2}{p}\theta$。 
由于CSI方法和共轭梯度法求解器所使用的边界区域是一样的,因此边界更新的操作的开销也是 $\mathcal{T}_b =4\varpi + \frac{8N}{\sqrt{P}}\vartheta$。 CSI方法除了收敛性判断之外本身并不包含全局归约操作。 
因此,每一步CSI迭代的时间开销可表示为:  
\begin{equation}
\label{t_csi}
\mathcal{T}_{CSI} = \mathcal{T}_c + \mathcal{T}_b
= 11\frac{N^2}{p}\theta + 4\varpi + \frac{8N}{\sqrt{p}}\vartheta
\end{equation}
不考虑收敛性判断的情况下,整个CSI求解器执行一次的时间开销为
\begin{equation}
\label{t_csi_all}
t_{CSI} = \mathcal{K}_{CSI}\cdot \mathcal{T}_{CSI}
\end{equation}
这里$K_{CSI}$表示CSI求解器达到收敛所需要的迭代步数。
在章\ref{solver:CSI:convergence_rate}节中,我们将会证明在相同的收敛条件下,$\mathcal{K}_{CSI}$通常比$\mathcal{K}_{CG}$稍微大一些。



\begin {figure}%[htbp]
\centering
\includegraphics[height=7cm]{evaluate_CSI}
\caption[] { 1度POP中CSI求解器的每一步迭代中各部分耗时分析}
\label{fig:cst_ratio}
\end{figure}
同样,我们利用TAU测试了1度的海洋模式POP在清华大学的探索100集群上时间开销,并且与理论模型进行比较。
如图\ref{fig:cst_ratio}所示,理论模型预测的CSI的可扩展性的表现与真实试验中的结果基本一致。
可以看出,CSI的迭代步中计算和通信的时间开销与CG求解器基本相同,CSI一个迭代步中的计算时间略小于共轭梯度法。
这一点符合两个求解器的理论模型的分析结果(参见式(\ref{t_pcg})和(\ref{t_csi}) )。
CSI求解器的迭代过程中,不涉及到全局通信操作,取而代之的是一个关于系数矩阵最大最小特征值的函数计算。
后者的开销遥远小于前者,因此,总体而言,CSI算法每一步迭代的时间开销要小于共轭梯度法。 

CSI的收敛速度比共轭梯度法稍慢,在小规模并行时,它的执行时间比共轭梯度法的要长一些。 
但是,正如式(\ref{t_pcg}) 和(\ref{t_csi})所示,由于没有十分耗时的全局归约操作,CSI方法单个迭代过程的时间要比共轭梯度法的短。 
在当进程数超过某一个值时,整个共轭梯度法的求解过程 要比CSI所消耗的时间长。 


\subsection{CSI方法收敛速度} \label{solver:CSI:convergence_rate}

 
与共轭梯度法一样,CSI方法的收敛速度也依赖于系数矩阵$A$的条件数。  
除了消除了全局归约操作,P-CSI方法的更一个优点是它有和PCG相同量级的收敛速度。 
P-CSI方法的收敛速度可以用残差的形式表示为
\begin{equation}
\textbf{r}_k = P_k(A)\textbf{r}_0 \label{eq:rPjr0}
\end{equation}
这里
$P_k(\zeta) = \frac{\tau_k(\beta-\alpha \zeta)}{\tau_k(\beta)}$($ \zeta \in [\nu, \mu]$)~\cite{stiefel1958kernel}。
$\tau_k(\xi)$ 是Chebyshev多项式,它的表达式为  
\begin{equation*}
\tau_k(\xi) =   \frac{1}{2}[(\xi+\sqrt{\xi^2-1})^k+(\xi+\sqrt{\xi^2-1})^{-k}]
\end{equation*}
当 $ \xi \in [-1,1]$时,Chebyshev多项式的另一个等价表达式为$\tau_k(\xi) = cos(k\cos^{-1} \xi)$。从这个表达式可以清楚的看出来,当$| \xi | \le 1$时,$|\tau_k(\xi)| \le 1$。
多项式$P_k(\zeta)$满足如下条件
\begin{equation}
P_k = \min_{p\in \mathcal{P}_k, p(0) = 1 }\max_{\zeta \in [\nu,\mu]} |p(\zeta)|
\end{equation}


假设$A= Q^T\Lambda Q$,这里 $\Lambda$ 表示以$A$的特征值为对角元素的对角矩阵,而 $Q$ 表示以  $A$的特征向量为列的实正交矩阵。于是,我们可以得到
\begin{equation}
P_k(A) = Q^T P_k(\Lambda)Q = Q^T \left [\begin{array}{cccc}
P_k(\lambda_1) & & &\\
& P_k(\lambda_2) & &\\
& & \ddots &\\
 & & & P_k(\lambda_\aleph)
\end{array} \right ] Q \label{eq:PjMA}
\end{equation}
 
假设所估计的系数矩阵的最大最小特征值$\nu$和$\mu$满足$0 < \nu \le \lambda_i \le \mu$ ($i = 1, 2, \cdots, \aleph$),这时有$|\beta - \alpha \lambda_i| \le 1$, $|P_k(\lambda_i)| \le \tau^{-1}_k (\beta)$。
方程 \ref{eq:rPjr0} 和 \ref{eq:PjMA} 可以推出
\begin{equation}
\label{csi_convergence}
\frac{||\textbf{r}_k||_2}{||\textbf{r}_0||_2}  \le  \tau_k^{-1}(\beta) = \frac{2(\beta+\sqrt{\beta^2-1})^k}{1+(\beta+\sqrt{\beta^2-1})^{2k}} \le 2(\frac{\sqrt{\kappa'}-1}{\sqrt{\kappa'}+1})^k
\end{equation}
这里$\kappa' = \frac{\mu}{\nu}$。
这个等式表明,当系数矩阵特征值的估计比较合理的时候(也就是$\kappa' =\kappa$),CSI的收敛速度和CG有相同的理论上界。

\begin {figure}
\vspace{10pt}
\centering
\includegraphics[height=7cm]{Convergence_diag}
\caption[] {0.1度POP中采用不同求解器时,迭代残差与迭代步数之间的关系。\label{fig:convergence_diag}}
\end{figure}
式(\ref{pcg_convergence})和(\ref{csi_convergence})表明,共轭梯度法和CSI方法的收敛速度的理论上界取决于系数矩阵的条件数。 
这两个方法求解良态方程(系数矩阵的条件数很小)的收敛速度比求解病态方程(系数矩阵的条件数很大)的快很多。
这里,我们通过海洋模式POP中的实际测试来验证一下这个理论结果。 
在0.1度的海洋模式POP中,我们测试了三个不同的正压求解器在做了简单的对角化预处理之后的收敛情况。
这三个求解器分别为CG求解器,ChronGear求解器(CG求解器的优化版本,将在第\ref{cha:precond}章进行详细介绍),和CSI求解器。
图\ref{fig:convergence_diag}给出了不同的正压求解器0.1度海洋模式中的真实的收敛速度。
图中横轴表示迭代的步数,垂直轴表示求解器在对应迭代步时的残差。 
如图中所示,三个求解器的相对残差随着迭代步数的增加而变小。
总共需要一百多迭代步才能够得到一个相对残差为$10^{-13}$的解。
其中,CSI求解器的残差下降的速度要比共轭梯度法及其变形ChronGear求解器稍慢。
尽管ChronGear方法改变了CG方法的计算顺序,CG方法和ChronGear在每一步的迭代速度几乎一致。
CSI方法在开始和最后的迭代中收敛速度比较慢,而在中间的迭代步骤中收敛速度变快。
这主要与系数矩阵的特征值的分布有关系。 


\subsection{特征值估计}
\label{solver:eigs}
我们在海洋模式POP中设计了基于Lanczos方法的特征值估计方法,它可以给出CSI求解器所需要的最大最小特征值的一个较好估计。
由第\ref{solver:CSI:convergence_rate}节的分析可知,当特征值的估计是准确的时,即$\nu = \lambda_{min}$,$\mu =\lambda_{max}$,CSI方法的收敛速度达到理论上的最优值。 
但是得到最大最小特征值$\lambda_{min}$ 和 $\lambda_{max}$ 的精确值是很困难的。
另外,由于系数矩阵$A$ 是分布在各个进程上的,因此不建议对它做任何形式的变形。  
为了能够利用POP的并行特性,我们使用Lanczos方法来构造出一系列三对角矩阵$T_k(k=1,2,...)$,这些矩阵的最大最小特征值逐渐的向$A$的最大最小特征值逼近。
基于Lanczos方法的特征值估计的步骤如算法\ref{alg:lanczos}所示。 
\begin{algorithm}
\caption{ 基于Lanczos方法的特征值估计}
\label{alg:lanczos}
\begin{algorithmic}[1]
\REQUIRE  网格块$B_{i,j}$ 的系数矩阵$\textbf{B}$和随机向量$\textbf{r}_0$;\\
 //\qquad    \textit{所有进程并行执行}
\STATE $\textbf{q}_1 = \textbf{r}_0/||\textbf{r}_0||$;\quad $\textbf{q}_0=\textbf{0}$;\quad $T_0=\emptyset$;\quad $\beta_0 =0$;\quad  $\mu_0 =0$;\quad $j=1$;
\WHILE{$j<k_{max}$} 
\STATE $\textbf{r}_j=\textbf{B}\textbf{q}_j-\beta_{j-1}\textbf{q}_{j-1}$;\quad $update\_halo(\textbf{r}_j)$;
\STATE $\tilde{\alpha}_j =\textbf{q}_j^T\textbf{r}_j$;\quad $\alpha_j=global\_sum(\tilde{\alpha}_j)$; 
\STATE $\textbf{r}_j=\textbf{r}_j-\alpha_{j}\textbf{q}_{j}$;
%\IF{$\textbf{r}_j \neq \textbf{0}$} 
\STATE $\tilde{\beta}_j = \textbf{r}_j^T\textbf{r}_j$; \quad $\beta_j=sqrt(global\_sum(\tilde{\beta}_j))$;
\STATE \textbf{if} $\beta_j == 0$ \textbf{then} \textbf{return}
\STATE $\mu_j = max(\mu_{j-1}, \quad \alpha_j+\beta_j+\beta_{j-1})$; \label{lanczos_gersh} \COMMENT{Gershgorin圆盘定理}\\
%\ENDIF
%\ENDFOR \\
%//\qquad    \textit{do on master node}
\STATE $T_k=tri\_diag(T_{k-1},\alpha_j,\beta_j)$; \quad $\nu_k = eigs(T_k,'smallest')$ ; \label{lanczos_tridiag} \COMMENT{三对角矩阵}
\STATE \textbf{if} $|\frac{\mu_k}{\mu_{k-1}} -1 |< \varepsilon\quad\textbf{and}\quad|1- \frac{\nu_k}{\nu_{k-1}}|< \varepsilon$ \textbf{then} \textbf{return}
\STATE $\textbf{q}_{j+1}= \textbf{r}_j/\beta_j$;\quad $j=j+1$;
\ENDWHILE
\end{algorithmic}
\end{algorithm}
 
在算法\ref{alg:lanczos}的第\ref{lanczos_tridiag}步中, $T_k$ 是一个三对角矩阵。它以 $\alpha_i (i=1,2,...,k)$ 为对角线元素, $\beta_i (i=1,2,...,k-1)$ 为次对角元素。 
\[ T_{k} = \left(\begin{array}{ccccc}
\alpha_1 & \beta_1  &  &&  \\
\beta_1 & \alpha_2  & \ddots &&  \\
  & \ddots & \ddots & \ddots&  \\
  &    &\ddots  &\ddots& \beta_{k-1} \\
 &    &  &\beta_{k-1}& \beta_{k} 
\end{array} \right)\]

假设$\xi_{min}$和$\xi_{max}$分别为$T_k$的最小的和最大的特征值。
Paige\cite{Paige1980235} 给出了以下结论
\begin{equation}\label{lanczos_minmax}
\begin{aligned} 
&\lambda_{min} \le \xi_{min} \le \lambda_{min}+\delta_1(k) \\
&\lambda_{max}-\delta_2(k)  \le \xi_{max} \le \lambda_{max}
\end{aligned}
\end{equation}
这里$\delta_1(k)$ 和 $\delta_2(k)$ 随着 $k$的增大而逐渐的向零逼近。 因此,系数矩阵 $A$ 的特征值的估计就可以转换成求解三对角矩阵$T_k$的特征值。 
算法\ref{alg:lanczos}中第\ref{lanczos_gersh}步应用了Gershgorin圆盘定理来估计$T_k$的最大特征值,也就是 
\begin{equation}
\mu = \max_{1 \le i \le k}\sum^m_{j=1}|T_{ij}|=\max_{1 \le i \le k}(\beta_{i-1}+\alpha_i +\beta_{i})
\end{equation}
在算法\ref{alg:lanczos} 的第\ref{lanczos_tridiag}中,采用了高效的QR算法\cite{ortega1963llt}来估计最小特征值$\nu$,它的复杂度为$\mathcal{O}(k)$。
 
\begin {figure}%[htbp]
\centering
\includegraphics[height=7cm]{eigen_lanczos}
\caption[] {Lanczos迭代步数和估计得到的特征值之间的关系\label{fig:lanczos_eigs}}
\end{figure}
实验中,我们发现当Lanczos迭代的步数选择合适时,我们能够得到最大最小特征值的一个比较好的估计,这个估计值能够使得CSI方法与共轭梯度法有相近的收敛速度。  
我们在1度POP中,使用Lanczos方法求解CSI方法所需要的最大最小特征值。 
图\ref{fig:lanczos_eigs}中可以看出,随着Lanczos迭代步数的增加,所估计得到的最大特征值逐渐增大,但是在十几步迭代之后便很快收敛到某个上界。
而所估计的最小特征值随着Lanczos迭代步数的增加而不断变小,并且慢慢收敛到某一下界。 
式(\ref{lanczos_minmax})的结论指出,Lanczos估计得到的最大最小特征值收敛于系数矩阵的最大最小特征值。
因此,总体上只需要几十步Lanczos方法便可以得到系数矩阵的最大最小特征值的较好估计。

\begin {figure}%[htbp]
\centering
\includegraphics[height=7cm]{scan_lanczos}
\caption[] {Lanczos迭代步数和求解器达到给定收敛条件所需要的迭代步数之间关系\label{fig:lanczos_iter}}
\end{figure}
图\ref{fig:lanczos_iter}给出的是CSI方法达到给定收敛所需要的迭代步数与Lanczos迭代步数之间的关系。
可以看出,随着Lanczos迭代步数的增加,CSI方法达到收敛所需要的迭代步数随着下降,但是随着迭代步数的继续增加, CSI方法所需要的迭代步数反而有所上升。 
在这个例子中,只需要十几步的Lanczos迭代,所估计得到的最大最小特征值就能够使得CSI方法达到最快的收敛速度。
在最快的情况下,CSI的迭代速度与共轭梯度法的相当。


%%%%%%%%%%%%%%%%%%%%%%%%%%%%%%%%%%%%%%%%%%%%%%%%%%%%%%%%%%%%%%%%%%%
\section{结果与分析} 
\label{solver:exp}

本章第\ref{solver:Algorithm:condition}节,给出了正压模态线性方程组系数矩阵的条件数与网格点总数目、网格横纵比例以及时间步长之间等正压模态的配置的关系。
第\ref{solver:CG:convergence_rate}节和第\ref{solver:CG:convergence_rate}节的理论结论表明,共轭梯度法和CSI方法的收敛速度的理论上界都是由系数矩阵的条件数决定。
因此,网格点总数目等因素决定了正压模态求解器的收敛速度。
第\ref{solver:exp:ideal}节实验中,我们首先通过一组理想实验来说明正压求解器收敛速度与正压模态配置的参数之间的关系。 

本章针对海洋模式正压模态中原有的CG求解器在大规模并行时可扩展性差的问题,提出了新的求解方案--CSI求解器。
第\ref{solver:exp:csi}节实验中,我们在大规模并行平台上测试了新的CSI求解器相比于原始CG求解器的性能。
为了验证CSI方法的正确性,我们第\ref{solver:exp:diff}节实验中给出CSI求解过程在海洋模式POP中的误差。
我们将在第\ref{cha:verify}章中,给出更加严格的正确性验证。 



\subsection{求解器的收敛速度}\label{solver:exp:ideal}

为了验证第\ref{solver:CG:convergence_rate}节和第\ref{solver:CSI:convergence_rate}节中所给出的收敛的理论结果,我们利用理想的试验设置构造了一系列条件数不同的矩阵。
这里,我们不再采用全局固定的网格大小,而是采用均匀的经纬网格。
随着纬度$\theta$的变化,经度方向上的网格大小为 $\Delta x_j  = \pi R \cos (\theta_j)$。
时间步长设置为$\tau = 10\frac{\Delta y}{c}$,这里$c = 200m/s$ 表示重力快波的速度\cite{smith2010parallel}。 
我们用CG和CSI方法来求解得到的这些方程。
实验中,收敛条件设为$tol = 10^{-6}$。 
  

\begin{figure} 
\vspace{5pt}
\centering
\includegraphics[height=7cm]{iterationGridSize1}
\caption[] {网格点数目和求解器的迭代步数之间的关系。\label{fig:iterationGridSize}}
\end{figure}

正如图\ref{fig:iterationGridSize}所示,随着问题规模的增加,系数矩阵变得越来越病态。
所有的求解器都需要更多的迭代步数才能得到相同的相对残差。 
图中还可以看出来,CSI方法的收敛速度要比CG的慢一些。 



 
\subsection{求解器的可扩展性} \label{solver:exp:csi}

为了测试算法的可扩展性,我们在中国的国家超算中心的神威蓝光超级计算机上测试了0.1度的海洋模式POP(3600 $\times$ 2400个网格点)。
神威蓝光是由8,704个SW1600处理器由40Gb 无限带宽网络连接组成。
每一个处理器包含16个1.1GHz的处理器核心以及16GB的内存。 

\begin {figure}
\centering
\includegraphics[height=8cm]{scale_pcg_csi}
\caption []{ 0.1度POP中CG和CSI求解器的可扩展性 \label {fig:scale}}
\end {figure}
我们分别测试了CG和CSI两个版本的求解器在不同的处理器核数(100到 15,000)和不同的收敛条件($\epsilon = 10^{-8}$ 到 $\epsilon = 10^{-16}$)时的性能。
海洋模式POP中的收敛准则是 $||r||_2<\epsilon \bar{a}$,这里 $\bar{a}$ 表示总面积的均方根。 
在单独版本的POP中,默认设置为 $\epsilon = 10^{-12}$。 
如图\ref{fig:scale}所示,PCG 和 CSI 在小于  1,000核时可扩展性都很好。但是当使用超过1,000 核时,CSI相对于PCG的优势就很明显了。 
当收敛条件为 $10^{-12}$时,在100核上CSI的执行时间只有PCG的87\%,这个比例在15,000核上下降到了 21\% 。
但是PCG方法对于收敛条件没有CSI方法敏感。 
当时收敛条件从$10^{-8}$变化到$10^{-16}$ 时,PCG方法的迭代步数从原来的20步增长到281步,但是CSI方法的迭代步数从33步增长到了1,434步。 
但是,PCG方法在收敛速度上的优势并没有使得PCG方法完全的比CSI方法快。 
只有当收敛条件是最为严格的$\epsilon = 10^{-16}$并且当使用的核心数少于1000核时,PCG的执行时间才会比CSI的短。
其他情况下,PCG方法中的全局归约操作的开销并不能被迭代步数的减少给抵消掉。
在1,500核上,当收敛条件$10^{-8}$ 和 $10^{-12}$时,PCG的执行时间是CSI的4.7倍,而当收敛条件为$10^{-16}$时,这个倍数仍然高达3.3。
由于全局归约操作的瓶颈,当使用的核心数大于1,000时,PCG求解器的执行时间随着所使用的核数的增加而增加。 
相对应的,CSI算法一直到10,000核以上都能保持比较好的可扩展性。

%----------------------------------------------------------------------------
\subsection{误差分析} \label{solver:exp:diff}

\begin{table}
\centering
\caption[] {采用PCG和CSI作为正压求解器的两个版本之间的海表高度的比较   \label{tab:err}}
\begin{tabular}{l l@{\quad}l@{\quad}l@{\quad}l} 
\toprule
模拟时间   & 一步  & 一天    & 一个月 &三个月\\
\hline
\multicolumn{1}{l}{总步数 } &\multicolumn{1}{c@{\quad}}{  1} &\multicolumn{1}{c@{\quad}}{  45} &\multicolumn{1}{c@{\quad}}{ 14053}	&\multicolumn{1}{c@{\quad}}{40800}\\
%\hline
最大相对误差 & 1.5016E-3&2.2181E-5& 1.2885E-2&1.4114E-1\\
%\hline
平均相对误差 &3.0223E-6&5.2424E-7& 2.6125E-5&7.8872E-4\\
\bottomrule
\end{tabular}
\end{table}
为了证明CSI方法并没有在海洋模式POP中引入误差,我们在清华大学的探索100集群(具体描述请参见第\ref{solver:pcgComplex}节)上对1度POP做了一组实验,分别使用PCG和CSI方法作为正压模态的求解器运行一段时间。 
表格\ref{tab:err}给出的是这两个版本得到的结果的海表高度的差别。
采用CSI和PCG的两个版本之间的平均误差相对于海表高度的绝对值非常的小。
有意思的是,我们注意到,两者之间最大的误差主要出现在海岸线上。 这些地方比较尖锐的边界很容易造成差分格式的不稳定性。
随着实验时间的延长,两个版本之间的误差也随着增大。 这并不是说求解器之间的误差在变大,而是由于海洋模式中湍流的累积效应将最初的小误差逐渐积累放大。 



\section{本章小结}
\label{solver:Conclusion}

目前已经有很多的工作在研究如何提高海洋模式POP中正压模态的性能。 
但是他们的大部分都没有从根本上解决POP正压模态中的可扩展性瓶颈。 
本章首先从海洋模式的原始方程组出发,建立了正压模态中海表高度的椭圆偏微分方程,
然后经过离散化之后得到了对应的离散后的线性方程组。
通过理论分析和实验验证,我们得到了影响这个线性方程组性质的关键因素是网格横纵比例,时间步长以及分辨率等。

共轭梯度法是求解海洋模式正压模态的常见方法,本章针对POP中的共轭梯度法的通信瓶颈,提出了一个评估海洋模式POP正压模态的模型,通过这个模型,我们对共轭梯度法的可扩展性进行量化,并且发现,正压模态中全局通信是其可扩展性的瓶颈。
这个模型从理论上证明了CSI方法在并行环境中的优越性。 

CSI方法没有耗时的全局通信操作,但是需要预估系数矩阵的最大最小特征值。
我们利用Lanczos方法,可以高效地对系数矩阵两个极端的特征值进行估计。
通过对收敛速度的证明,我们发现新的CSI方法和原始的共轭梯度法有相同量级的收敛速度。
进一步分析两个算法的复杂度,我们预测,CSI求解器由于没有可扩展性差的全局通信操作,在大规模并行是应该会比CG求解器的速度快。

最后,我们将CSI方法实现在海洋模式POP中并进行测试。实验结果表明,CSI方法比原始的共轭梯度法有更好的可扩展性。
同时,我们通过比较新的求解器与原始求解器所得到的结果,证明了新的求解器不会在海洋模式中引入显著的误差。
总之,这篇文章提出了一个比传统的共轭梯度法更有前景的求解椭圆方程的方法。 

