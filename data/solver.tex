\chapter{海洋模式正压求解器}
\label{cha:barosSolver}

\section{本章概述}

公共地球系统模式中的海洋分量模式,并行海洋模式POP(the Parallel Ocean Model) 求解的是采用静力平衡近似和Boussinesq近似的三维原始方程组。 
它将时间积分分成两个部分:斜压模态和正压模态。 
斜压模态描述的是三维的动力学和热动力学过程,而正压模态是求解二维的垂直积分后的动量方程和连续性方程。
海洋模式POP将其正压模态中的椭圆方程离散化之后得到一个线性系统$Ax=b$,并且使用共轭梯度法来进行求解。
然而,这个方程的求解过程在并行海洋模式中可扩展性很差。
事实上,已有研究表明,并行海洋模式的正压求解的比较差的可扩展性主要是由于通信的开销导致的\cite{Worley:2011:PCE:2063384.2063457}。 
如果能优化并行海洋模式的正压求解,将会使得整个公共地球系统模式的性能有较大的提升\cite{dennis2012computational}。 


本章的主要内容是通过分析正压模态的通信瓶颈,提出了一个能够提高并行海洋模式POP中正压模态求解器的扩展性的新的策略。 
由于每一步迭代过程中都需要时间开销很大的全局归约操作, 导致共轭梯度法在分布式并行系统上的可扩展性不好。 
本章首先建立了一个量化预处理共轭梯度法可扩展性瓶颈的模拟。
基于这个模型, 我们认为以前被视为效率不如共预处理轭梯度法的传统的Stiefel迭代方法(CSI),在大规模并行环境中是更有前景的。
相比于预处理共轭梯度法,CSI算法的迭代过程中的迭代参数不是由前一步迭代后的残差的內积计算得到的,而是利用系数矩阵$A$的特征值谱决定。 
我们采用Lanczos方法 解决了估计大规模矩阵$A$的特征值的难题。 
通过Lanczos方法可以构造出一个很小规模的三对角矩阵,它的特征值逐渐的逼近系数矩阵的$A$的部分特征值。 
通过使用CSI方法替换原来的预处理共轭梯度法发, 海洋模式正压模态中的规矩归约操作以及其所导致的较差的可扩展性就被去掉了。 
在0.1度分辨率的海洋模式POP中,采用了CSI方法之后正压模态在15,000核心上取得了5倍左右的加速效果,运行时间从原来的41.96秒每天下降到6.67秒每天。 





\section{海洋模式正压模态求解器}
\label{solver:baro}

为了完整的推导我们的新的P-CSI求解器,这里我们简要的描述一下海洋模式POP中的控制方程。
海洋模式中的原始动量和连续性方程可以表达如下:
\begin{align}
&\frac{\partial }{\partial t} \textbf{u} +\mathcal{L}(\textbf{u}) + f\times \textbf{u} = - \frac{1}{\rho_0}\nabla p +F_H(\textbf{u}) +F_V(\textbf{u}) \label{eq:momen}\\
&\mathcal{L}(1) = 0 \label{eq:continuous}
\end{align}
这里$\mathcal{L}(\alpha ) = \frac{\partial }{\partial x} (u\alpha)  +\frac{\partial }{\partial y} (v\alpha) +\frac{\partial }{\partial z} (w\alpha)$, 它在$\alpha =1$时与平流算子等价, $x, y, z$ 分别是水平和垂直坐标变量, $\textbf{u} = [u, v]^T$ 水平速度向量, $w$ 垂直速度, $f$ 表示科氏力系数,  $p$ 和 $\rho_0$ 分别表示压力和密度, $F_H$ 和 $F_V$分别表示水平和垂直耗散项 \cite{smith2010parallel}。  


%----------------------------------------------------------------------------
\section{正压模态} \label{solver:mode}

为了更加高效的求解三维的原始方程,海洋模式分量将时间积分分解为两个模态。 一个是求解三维动力学和热动力学过程的斜压模态,另一个则是求解二维海表高度(SSH)的变化。
正压模态的海表高度方程是对三维原始方程做垂直积分之后得到的,主要作用是将速度场中较快的部分分离出来,使用二维的正压模态来实现更快的求解。
在静力平衡近似条件下,POP中海底深度为$z$的位置上的压力可以分解为两个部分:  
\begin{align}
\displaystyle p = p_h + p_s = \int^0_z g\rho dz +p_s
\end{align}
这里$p_h$表示静力平衡压力,$p_s$表示由于海表自由面波动而引起的海表压力。 
正压模态的控制方程是由原始动量方程和连续性方程(\ref{eq:momen},\ref{eq:continuous}) 从海洋底部到海洋表面垂直积分而得到的:
\begin{align}
&\displaystyle \frac{\partial \textbf{U} }{\partial t}  = -g \nabla \eta + F  \label{eq:baro_mon}\\
&\displaystyle \frac{\partial \eta }{\partial t} = - \nabla \cdot H\textbf{U} + q_w  \label{eq:baro_con}
\end{align}
这里  $\textbf{U} =  \frac{1}{H+\eta}\int_{-H}^\eta dz \textbf{u}(z) \approx \frac{1}{H }\int_{-H}^0 dz \textbf{u}(z)$是正压速度的垂直积分,
$\eta = p_s/{\rho_0g}$是海表高度,$H$是海洋底部的深度,$q_w$单位海表面积的活水通量,$g$表示由于地球动力引起的加速度而$F$表示对海洋动量方程中除了与时间趋势和海表压力梯度相关的所有项做垂直积分所得到的量(参见方程\ref{eq:momen})。
 
垂直方向上的边界条件是
\begin{align}
\label{eq:bound_w}
w = \left\{ \begin{array}{ll}
\frac{\partial}{\partial t} \eta  +\textbf{u}\cdot\nabla \eta - q_w, & z = \eta  \\
0, & z = -H
\end{array} \right.
\end{align}
 
值得一提的是,为了简化求解过程,正压连续性方程 (\ref{eq:baro_con}) 中忽略掉了边界条件中的小项 $\nabla \eta$\cite{smith2010parallel}。
为了能够使用更长的时间步长,海洋模式分量的正压模态中采用了隐式格式,并且将椭圆方程简化成一个线性系统。 通过时间差分,方程  (\ref{eq:baro_mon})和(\ref{eq:baro_con}) 可化为
\begin{align}
&\displaystyle \frac{ \textbf{U}^{n+1} - \textbf{U}^{n-1}}{\tau}  = -g \nabla \eta + F \label{eq:udt} \\
&\displaystyle \frac{\eta^{n+1} - \eta^n }{\tau}  = - \nabla \cdot H\textbf{U} + q_w \label{eq:etadt}
\end{align}
这里 $\tau$ 与差分格式相关的时间步长。 
用方程(\ref{eq:udt})中下一时刻的正压速度来替换方程(\ref{eq:etadt})中的正压速度, 就得到了一个关于海表高度$\eta$的椭圆系统
\begin{equation}
\label{eq:sshdiscret}
     [-\nabla\cdot H \nabla + \frac{1}{g  \tau^2}]\eta^{n+1}
           = -\nabla\cdot H[\frac{\textbf{U}^{n-1}}{g \tau} + \frac{F}{g}] + \frac{\eta^n}{g\tau^2} +\frac{q_w}{g\tau}
\end{equation}
 
为了简单,我们将椭圆方程(\ref{eq:sshdiscret})重新标记为
\begin{equation}
\label{eq:ssh}
[-\nabla \cdot H\nabla +\frac{1}{g  \tau^2}]\eta^{n+1} = \psi(\eta^n,\eta^{n-1},\tau)
\end{equation}
这里 $\psi$表示一个关于当前时刻和上一时刻$\eta$状态的函数。
 

\begin{figure}%[!htbp]
\centering
\includegraphics[width=12cm, height=6.5cm]{grid_domain.pdf}
\caption[] {海洋模式分量中网格划分\label{fig:grid1}}
\end{figure}


 

\subsection{正压线性方程组}
\label{solver:baroproperty}

如图\ref{fig:grid1}所示, POP采用在水平方向上采用Arakawa B网格\cite{smith2010parallel},并且采用九点差分格式对方程(\ref{eq:ssh})进行离散。得到如下离散格式:
\begin{align}
    & \nabla\cdot H \nabla \eta  =\frac{1}{\Delta y}\delta_x \overline{[\Delta y H  \delta_x\overline{\eta}^y]}^y +\frac{1}{\Delta x}\delta_y \overline{[\Delta x H  \delta_y\overline{\eta}^x]}^x \label{eq:nabla2}
  \end{align}

这里 $\Delta_\xi$和$\delta_\xi$  ($\xi \in \{x, y\} $)分别为有限差分和它们相应的偏导,  $\delta_\xi (\cdot) $和$\overline{(\cdot)}^\xi $ 分别表示有限差分算子和平均算子。  
\begin{align}
&\delta_\xi \psi = [\psi (\xi+\Delta_\xi/2) -\psi(\xi-\Delta_\xi/2)]/\Delta_\xi \\
&\overline{\psi}^\xi  =[\psi (\xi+\Delta_\xi/2) +\psi(\xi-\Delta_\xi/2)]/2
\end{align}
%----------------------------------------------------------------------------


为了避免极点问题,POP采用偏移的或者三极点的广义正交网格。这使得方程的系数非常的繁杂。
为了能够简单而不失偏颇,我们将给出系数矩阵在均匀的网格和给定的不变的海底深度$H$的情况下的显式表达式。
 

表达式\ref{eq:nabla2}此时为
\begin{align}
 [\nabla\cdot H \nabla \eta]_{i,j}&= -\frac{H}{S_{i,j}}(A_{i,j}^O\eta_{i,j}+A_{i,j}^{NW}\eta_{i-1,j+1}+A_{i,j}^N\eta_{i,j+1} \nonumber\\
 &+A_{i,j}^{NE}\eta_{i+1,j+1}+A_{i,j}^W\eta_{i-1,j} +A_{i,j}^E\eta_{i+1,j} \nonumber\\
& +A_{i,j}^{SW}\eta_{i-1,j-1} +A_{i,j}^S\eta_{i,j-1}+ A_{i,j}^{SE}\eta_{i+1,j-1})
\end{align}
这里$S_{i,j}  = \Delta x\Delta y$, $A_{i,j}^{\chi } ( \chi \in \mathcal{Q} = \{NW,NE, SW, SE, W, E, N, S\})$ 分别表示网格点 $(i,j)$和它自己、以及它的邻居点之间的在 九点差分格式中的系数(\ref{eq:nabla2})。 这些系数由$\Delta x$, $\Delta y$, $\tau$和$H$决定。
\begin{equation} \label{defineA}
\begin{aligned}
&\alpha  = \frac{ \Delta y}{ \Delta x }, \quad \beta  = 1/\alpha \\
&A_{i,j}^{NW} = A_{i,j}^{NE} =A_{i,j}^{SW} = A_{i,j}^{SE} = - (\alpha  +  \beta  )/4 \\
&A_{i,j}^{W} = A_{i,j}^{E} = (  \beta  -\alpha  )/2 \\
&A_{i,j}^{N} = A_{i,j}^{S} = (\alpha  -\beta )/2 \\
&A_{i,j}^{O} =   \alpha   +\beta  \\
\end{aligned}
\end{equation}


%&\alpha_1 = \frac{2\Delta y}{2\Delta x_j +\Delta x_{j+1}2}\\
%& \alpha_2= \frac{2\Delta y}{ \Delta x_j +\Delta x_{j-1}  } \\
%&\alpha_3 = \frac{\Delta y(\Delta x_{j-1}+2\Delta x_j +\Delta x_{j+1})}{2 (\Delta x_j +\Delta x_{j-1})(\Delta x_j +\Delta x_{j+1})} \\
%& \beta_1 = \frac{(\Delta x_j +\Delta x_{j+1})}{8 \Delta y} \\
%& \beta_2 = \frac{(\Delta x_j +\Delta x_{j-1})}{8 \Delta y} \\
%& \beta_3 = \frac{(\Delta x_{j-1}+2\Delta x_j +\Delta x_{j+1})}{8 \Delta y} \\
%&A_{i,j}^{NW} = A_{i,j}^{NE} = \alpha_1 + \beta_1 \\
%&A_{i,j}^{SW} = A_{i,j}^{SE} = \alpha_2 +  \beta_2 \\
%&A_{i,j}^{W} = A_{i,j}^{E} = \alpha_3 - 2 \beta_3 \\
%&A_{i,j}^{N} = -2\alpha_1  + 2 \beta_1 \\
%&A_{i,j}^{S} = -2\alpha_2  + 2 \beta_2  \\
%&A_{i,j}^{O} = -2\alpha_3  -4 \beta_3  \\
%Then, the coefficients between the given point $(i,j)$和its other neighbors can be computed from $A^n$, $A^e$和$A^{ne}$ on its neighbors。
方程\ref{eq:ssh}在给定点$(i,j)$上的离散格式则可以表示为  
\begin{align}
\label{eq:sten}
&(A_{i,j}^O+\phi ) \eta_{i,j}+A_{i,j}^{NW}\eta_{i-1,j+1}+A_{i,j}^N\eta_{i,j+1} +A_{i,j}^{NE}\eta_{i+1,j+1}+A_{i,j}^W\eta_{i-1,j}  \nonumber\\
& +A_{i,j}^E\eta_{i+1,j} +A_{i,j}^{SW}\eta_{i-1,j-1} +A_{i,j}^S\eta_{i,j-1}+ A_{i,j }^{SE}\eta_{i+1,j-1}= \frac{S_{i,j}}{H}\psi_{i,j}
\end{align}
这里$\phi = \frac{S_{i,j}}{g  \tau^2H}$ 是与单位网格大小、海底深度和时间步长有关的一个变量。

\begin{figure}
\centering
\includegraphics[height=8cm]{SparsePatternSample}
\caption[] {大小为$30\times 15$的网格上所得到的系数矩阵的稀疏模式。 \label{fig:spy}}
\end{figure}
因此椭圆方程(\ref{eq:sshdiscret})就变成了关于$\eta$的一个线性方程组,也就是 $Ax= b$。 其中$A$表示由系数 $A^*$构成的快对角矩阵。
方程 \ref{defineA}和 \ref{eq:sten}表明$A$是有水平网格大小、海底深度和时间步长决定。 
他们之间具体的关系将在第\textbf{ref}节中深入讨论。
%In the POP, only the nonzero elements are stored。
方程(\ref{eq:sten})同时还表明 $A$ 每一行只有九个非零元素, 也就是说$A$是一个系数矩阵。 图\ref{fig:spy}展示了$A$的稀疏模式。

\subsection{特征值谱和条件数}
\label{solver:Algorithm:condition}

POP中的系数矩阵 $A$ 是正定对称的\cite{smith2010parallel},因此它的特征是都是正实数\cite{stewart1976positive}。
假设系数矩阵的特征值谱\cite{golub2012matrix} 是 $\mathcal{S} = \{\lambda_1, \lambda_2, \cdots, \lambda_N\}$,这里 $\lambda_{min} = \lambda_1 \le \lambda_i \le \lambda_\mathcal{N} = \lambda_{max}$( $1<i <\mathcal{N}$, $\aleph$ 为 $A$的大小 )表示$A$的特征值.

利用 Gershgorin圆盘定理\cite{bell1965gershgorin}可知,对于任意的 $\lambda \in \mathcal{S}$,都存在一个数对 $(i,j)$ 满足
\begin{align}
&|\lambda -  (A_{i,j}^O + \phi ) | \le \sum_{\chi \in \{NW,NE,SW,SE,W,E,N,S\}}|A_{i,j}^\chi|
\end{align}
由方程\ref{defineA}给出的系数的定义,我们可以得到如下结论 
\begin{align} \label{eigsGersh}
&\lambda_{max} \le  \max (  5\alpha - \frac{1}{\alpha}, \frac{5}{\alpha}- \alpha) +\phi   \\
&\lambda_{min} \ge 2\min (  \alpha - \frac{1}{\alpha},\frac{1} {\alpha} -  \alpha) + \phi
\end{align}\\

 
这个结论说明,当网格横纵比例越接近于1时,最大特诊值的上界会随之减少,而最小特征值的下界会随之变大。
也就是说,系数矩阵特征值谱的半径($[\lambda_{min}, \lambda_{max}]$)随着网格横纵比向着1的接近而变小。 
当水平网格的横纵比等于1时,即$ \alpha = \frac{ \Delta y}{ \Delta x} = 1$,我们可以得到$\lambda_{max} \le  4 +\phi$,$\lambda_{min} \ge   \phi$。
这时,系数矩阵的条件数(即 $\kappa=  \lambda_{max}/\lambda_{min}$),由于是由特征值谱的半径的所决定的,也会随着网格横纵比向1的靠近而变小。 

\begin{figure}[ht]
\centering
\includegraphics[height=8cm]{conditionNumberAspectRatio}
\caption[] {网格横纵比例和系数矩阵条件数之间的关系。 \label{fig:conditionNumberRatio}}
\end{figure}




图\ref{fig:conditionNumberRatio}所给出的结果验证了我们的这个结论。图\ref{fig:conditionNumberRatio}中给出的是固定的网格点数$\mathcal{N} = 20\times 20$和固定的$\phi = 0$, 在区间$(1, +\infty)$上,系数矩阵的条件数随着网格横纵比例的增加而增加,而在区间$(0,1)$上,系数矩阵的条件数随着横纵比例的增大而变小。
特征值在当网格横纵比例等于1时取到最小值。
 
\begin{figure}[ht]
\centering
\includegraphics[height=8cm]{conditionNumberTimestep}
\caption[] {时间步长和系数矩阵条件数之间的关系。 \label{fig:conditionNumberDt}}
\end{figure}



特征值的下界是$\phi=\frac{S }{g \tau^2 H}$,这个变量是由时间步长以及水平网格的面积和海底深度的比例确定的。
它表明,特征值的下界会随着时间步长的增大而减小。 
相应的,系数举证的 条件数也就随之增大。
图\ref{fig:conditionNumberDt}中证明了我们的这个推论。
这个图中的实验采用的是固定的网格点数$\mathcal{N} = 20\times 20$,和固定的网格横纵比$\Delta x /{\Delta y} = 1$。 

\begin{figure}[ht]
\centering
\includegraphics[height=8cm]{conditionNumberGridSize}
\caption[] {网格点的数目和系数矩阵条件数之间的关系。 \label{fig:conditionNumbGrid}}
\end{figure}
 
不考虑时间步长的因素(假设$\phi=0$)时,上面的分析表明,当网格的横纵比等于1时,不论网格点的个数是多少, 系数矩阵的特征值谱半径限制在 $(0,4)$ 区间内。
但是,系数矩阵的条件数的变化却可能会很大,因为当网格点的数量$\mathcal{N}$ 不断增加时,最小的特征值会组件的逼近零。 
图\ref{fig:conditionNumbGrid} 中解释了条件数和网格点数目之间的关系。 

方程(\ref{eq:ssh})在二维广义正交网格上使用九点差分格式进行离散。  
POP将全球区域划分为若干个小块,并将这些块分发给每一个进程。
每个进程只负责计算它所分得的块内网格点上的迭代过程,并且维护一个与周围进程进行数据交换的边界区域。 
我们假设全球区域上的网格大小为$\mathcal{N}\times
\mathcal{N}$且被划分为$m\times m$ 个大小为$n\times n$ ($n=\mathcal{N}/m$)的小块.
我们定义$B$ 和 $\tilde{x}$ 分别为与给定块相对应系数矩阵(也就是 $A$的一个大小为 $n^2\times n^2$的子矩阵)和向量。
这个块上的子矩阵和子向量的乘积操作$B\tilde{x}$ 需要$9n^2$ 次计算 \cite{hu2013scalable}.

%%%%%%%%%%%%%%%%%%%%%%%%%%%%%%%%%%%%%%%%%%%%%%%%%%%%%%%%%%%%%%%%%%%
\section{正压模态及其瓶颈分析}
\label{solver:bottleneck}
%----------------------------------------------------------------------------


为了简洁, 在本文中,我们假设整个区域的网格大小为$N\times N$,它被划分为$m\times m$个大小为$n\times n$ ($n=N/m$)的小块。 
假使块$B(k,l)$上对应的系数矩阵为$\tilde{A}$, 那么$\tilde{A}$就是系数矩阵$A$的一个大小为$n^2\times n^2$的对角子矩阵,并且每一行最多只有九个非零元。 
于是,矩阵向量乘积操作 $\tilde{A}\tilde{x}$只包含有$9n^2$次浮点数乘法操作,而不是$n^2\times n^2$次操作。 

%----------------------------------------------------------------------------
\subsection{PCG求解器} 
\label{solver:pcg}
 
海洋模式POP的正压模态中使用传统的共轭梯度法结合一个对角预处理子$M = \Lambda(A)$ 作为它的默认求解器。 
这个默认求解器在小规模并行时效率很高。 
对角预处理的共轭梯度法的过程如下\ref{alg:pcg}: 

\begin{algorithm}[h]
\caption{共轭梯度法}
\label{alg:pcg}
\begin{algorithmic}[1]
\REQUIRE  与网格块$B_{i,j}$相应的系数矩阵$\textbf{B}$,  初始解 $\textbf{x}_0$ 和 $\textbf{b}$  \\
//\qquad    \textit{所有进程并行执行}
\STATE $\textbf{r}_0 = \textbf{b}-\textbf{B}\textbf{x}_0$, $\textbf{s}_0 =0$;\quad $\beta_0=1$, $k=0$;
\WHILE{$k \leq k_{max}$ }
\STATE $k=k+1$;\quad $\textbf{r}'_{k-1} =\textbf{M}^{-1}\textbf{r}_{k-1}$;\quad \COMMENT{对角预处理}
\STATE $\tilde{\beta_k} = \textbf{r}_{k-1}^T\textbf{r}'_{k-1}$;\quad $\beta_k = global\_sum(\tilde{\beta_k})$; \COMMENT{全局规约操作}
\STATE $\textbf{s}_k = \textbf{r}'_{k-1} +(\beta_k/\beta_{k-1})\textbf{s}_{k-1}$;\quad $\textbf{s}'_k = \tilde{\textbf{A}}\textbf{s}_k$; \COMMENT {矩阵向量乘}
\STATE $update\_halo(\textbf{s}'_k$); \COMMENT{ 边界通信}
\STATE $\tilde{\alpha_k} = \textbf{s}_k^T\textbf{s}'_k$;\quad $\alpha_k =\beta_k/ global\_sum(\tilde{\alpha_k})$;\quad \COMMENT{全局规约操作}
\STATE $\textbf{x}_k =\tilde{\textbf{x}}_{k-1} +\alpha_k \textbf{s}_k$;\quad $\textbf{r}_k =\textbf{r}_{k-1} -\alpha_k\textbf{s}'_k$;
\IF{ $k \% n_{c} == 0$}
\STATE \textbf{if} $||\textbf{r}_{k+1}|| \le \epsilon$  \textbf{return} ;\COMMENT{每 $n_c$ 检查一次是否收敛}
\ENDIF
\ENDWHILE
\end{algorithmic}
\end{algorithm}

 
正如算法\ref{alg:pcg}所示, 预处理共轭梯度法主要包含有以下下个部分: 计算,边界通信,全局通信。 
 
%----------------------------------------------------------------------------
\subsection{PCG复杂度模型}
\label{solver:pcgComplex}
假使使用$P=m^2$个进程来计算正压模态,也就是每个进程正好计算一个子块。 
正压模态的运行时间与预处理共轭梯度法梯度法在每个进程上的执行时间。 
假设 $T_c$, $T_b$ 和 $T_g$分别表示每个迭代步中计算,边界交换和全局归约操作的时间。 

 算法\ref{alg:pcg}中,计算主要涉及到步骤3, 5和 8中的四个向量伸缩操作,步骤4和7中的两个向量向量乘积操作,和步骤5中的矩阵向量乘积。 
 因此 $T_c= \Theta (4 n^2 +2n^2+ 9n^2) = \Theta (15n^2) =\Theta(15\frac{N^2}{P})$。 
很明显 $T_c$ 会随着进程数的减少而减少, 同时以0位下界。 


每个进程都需要与其相邻点进行边界交换, 这个过程时间开销主要取决于网络延迟和边界缓存区域的大小。 
边界缓存区域的带宽默认设置为2, 因此一次边界通信的数据量大小为 $2n$。 
很明显,这个数据量会随着进程数的增加而减少。 
每个进程需要与其周围的四个进程进行数据交换, 因此一次边界更新所需要的时间为$T_b =2\times4T_{delay} +\Theta (2\times4\times 2n)=8T_{delay} +\Theta (\frac{16N}{\sqrt{P}})$。 
可以看出来,这个更新时间是随着进程数的增加而减少的,并且它有一个由网络延迟决定的下界 $8T_{delay}$。 

 
每次內积操作之后需要调用一次全局归约操作, 归约操作实质上就是把每个进程上所计算得到的局部的內积值收集起来,
因此归约操作的数据传输的时间相比收集操作的时间是可以忽略的。 
归约操作的时间, 包括了启动延迟和网络阻塞,满足$T_g= T_{init}+ c_g\cdot \mathcal{G}(P)$。 
这里$T_{init}$ 和$c_g$ 表示并行环境中的常量, 而$\mathcal{G}(\cdot)$是一个与指定架构中网络拓扑结构相关的函数。 
比如, 理想立方体网络中的$\mathcal{G}(\cdot)$是一个对数函数。 
总之,$T_g$ 会随着进程数$P$的增加而单调的变大。 
 

 
设$T_0$表示一次浮点数操作所需要的时间, $B$表示网络中每秒中点对点能够传递浮点数的个数。 
那么一次预处理共轭梯度法的执行时间可以表示为:

\begin{equation}
\label{t_pcg}
T_{pcg} = T_c + T_b + T_g
= 15 T_0\frac{N^2}{P} + 8T_{delay} + \frac{16N}{B\cdot\sqrt{P}}+T_{init} +c_g\mathcal{G}(P)
\end{equation}
总个预处理共轭梯度法的求解过程的执行时间可以表示为$t_{pcg} = K_{pcg}\cdot T_{pcg}$, 这里$K_{pcg}$表示预处理共轭梯度法达到收敛所需要的迭代步数,它不会随着进程数的增加而改变。 
方程\ref{t_pcg}可以出来, 计算和边界更新所需要的时间会随着进程数的增大而减少, 
但是全局归约操作的时间却会随着进程数的增加而增大。 
因此,整个预处理共轭梯度法的运行时间在当进程数大于某个数值时,会随之增大而增大。 

\begin{figure}[ht]
\centering
\includegraphics[height = 8cm]{evaluate_PCG}
\caption[] {1 度 POP中预处理共轭梯度法的时间成分分析\label{fig:pcg_ratio}}
\end{figure}

 
我们在清华大学的探索100集群上,使用1度的海洋模式POP (360$\times$240 个网格点)做了一系列的实验。 
探索100包含有740个计算节点,每个计算节点都有两个2.93 GHz 英特尔Xeon X5670 6核处理器以及24/48 GB的内存。 
我们采用了一个并行程序的性能评测和追踪的工具--TAU\cite{shende2006tau} 来测一下预处理共轭梯度法求解器的三个部分的时间。 
我们还将实验中一个正压步中每个分量的平均耗时和评测模型中的时间进行对比。 
在评测模型中, 我们将$\mathcal{G}(\cdot)$设为一个线性函数, $c_g$  设为 $2\times 10^{-7}$, $T_0$ 设为$2.5\times 10^{-9}$ ,  $B$  设为 $5.0\times 10^{9}$。 


如图\ref{fig:pcg_ratio}所示, 评测模型的结果和真实实验的结果吻合得非常好。 
实验中正压模态的计算时间确实是和进程数成反比的。 
实验中的边界更新时间再当进程数超过100的时候基本上保持不变,这是因为当进程数比较大的时候, 边界通信的消息大小变得非常小,以至于数据传输的时间相比于网络延迟可以忽略掉。 
全局归约操作的时间随着进程数的增加而成比例的变大,并且当使用的进程数大于100时,全局归约操作的时间变成正压模态中的主要开销。 
这个结果和我们方程(\ref{t_pcg})给出的理论分析的结果是一致的,即全局归约操作是预处理共轭梯度法可扩展性差额根本原因。 

为了解决预处理共轭梯度法的可扩展性瓶颈,新的求解器需要的全局归约操作越少越好。 
我们在海洋模式中重新考虑一些原本被认为是不如预处理共轭梯度法的方法,比如Chebyshev迭代方法。 
Gutknecht \cite{gutknecht2002chebyshev} 在2002年有一篇文章专门的重新研究了一下Chebyshev迭代在通信开销较大的大规模并行环境中性能。 
传统的Stiefel迭代(CSI)就是Chebyshev迭代方法中的一种。 


 
\section{CSI迭代法}
\label{solver:csi}
与预处理共轭梯度法不同的是, P-CSI方法 在每一步迭代过程中不需要做內积操作, 因此它即使在大核数上也能够保持较好的可扩展性。
P-CSI需要对预处理后的系数矩阵$M^{-1}A$的最大特征值$\mu$和最小特征值$\nu$进行估计。 
众所周知,计算稀疏矩阵的特征值要比求解线性方程组本身还要难。 
值得庆幸的是, 
海洋模式POP中的系数矩阵$A$ 和它的对角预处理矩阵$M = \Lambda(A)$都是实对称矩阵, 因此预处理后的矩阵的最小最大特征值并不难估计。 
 
\begin{algorithm}[h]
\caption{传统Stiefel迭代算法}
\label{alg:csi}
\begin{algorithmic}[1]
\REQUIRE 与网格块$B_{i,j}$相对应的系数矩阵 $\textbf{B}$, 预处理子$\textbf{M}$, 初始值$\textbf{x}_0$和方程右端向量$\textbf{b}$ ;预估的特征值区间$[\nu,\mu]$;  \\
 // \qquad    \textit{所有进程并行执行}
\STATE $\alpha =\frac{2}{\mu -\nu}$, $ \beta = \frac{\mu +\nu}{\mu -\nu}$, $\gamma = \frac{\beta}{\alpha}$, $\omega_0 =\frac{ 2}{\gamma}$;\quad $k = 0$;
\STATE $\textbf{r}_0 = \textbf{b}-\tilde{\textbf{A}}\textbf{x}_0$; $\textbf{x}_1 =\textbf{x}_0 -\gamma^{-1}\textbf{r}_0$; $\textbf{r}_1 =\textbf{b} -\tilde{\textbf{A}}\textbf{x}_1$;
\WHILE{$k \leq k_{max}$ }
\STATE $k=k+1$;\quad $\omega_k = 1/(\gamma - \frac{1}{4\alpha^2}\omega_{k-1})$; \COMMENT{迭代函数}
\STATE $\Delta \textbf{x}_{k} =\omega_k\textbf{r}_{k-1}+(\gamma \omega_k-1)\Delta \textbf{x}_{k-1}$;
\STATE $\textbf{x}_{k} =\textbf{x}_{k-1}+\Delta \textbf{x}_{k-1}$; \quad  $\textbf{r}_{k} =b- \tilde{\textbf{A}}\textbf{x}_{k}$;
\STATE $update\_halo(\textbf{r}_k)$; \COMMENT{边界通信}
\IF{ $k \% n_{c} == 0$}
\STATE \textbf{if} $||\textbf{r}_{k+1}|| \le \epsilon$  \textbf{return} ;\COMMENT{每 $n_c$ 检查一次是否收敛}
\ENDIF\ENDWHILE
\end{algorithmic}
\end{algorithm}

\subsection{算法及评估}

如算法\ref{alg:csi}所示, CSI方法和预处理共轭梯度法有相似的迭代过程,
但是将预处理共轭梯度法中的两次向量內积操作及其相应的向量向量乘积的计算替换成了系数矩阵$A$的最大最小特诊值的一个函数迭代。 
CSI求解器的计算时间可表示为
$T_c = \Theta (4 n^2 + 9n^2) = \Theta (13n^2) =\Theta(\frac{13N^2}{P})$。 
由于CSI方法和预处理共轭梯度法求解器所使用的边界区域是一样的, 因此边界更新的操作的开销也是 $T_b =8T_{delay} +\Theta (16 \frac{N}{\sqrt{P}})$。 CSI方法除了收敛性判断之外本身并不包含全局归约操作。 
因此,每一步CSI迭代的时间开销可表示为:  
\begin{equation}
\label{t_csi}
T_{csi} = T_c + T_b
= 11T_0 \frac{N^2}{P}+ 8T_{delay} + \frac{16N}{B \cdot\sqrt{P}}
\end{equation}
不考虑收敛性判断的情况下, 整个CSI求解器执行一次的时间开销为$t_{csi} = K_{csi}\cdot T_{csi}$, 这里$K_{csi}$ 表示CSI求解器达到收敛所需要的迭代步数。 在相同的收敛条件下, 它通常比$K_{pcg}$稍微大一些。 

如图\ref{fig:cst_ratio}所示, 这个模型真是的预测CSI的可扩展性的表现。 
CSI的收敛速度比预处理共轭梯度法稍慢, 它在小规模并行是的执行时间比预处理共轭梯度法的要长一些。 
但是,正如方程\ref{t_pcg} 和\ref{t_csi} 所示, 由于没有十分耗时的全局归约操作, CSI方法单个迭代过程的时间要比预处理共轭梯度法的短。 
在当进程数超过每一个值时,整个预处理共轭梯度法的求结果过程 要比CSI求解器消耗的时间长。 

 
\begin {figure}%[htbp]
\centering
\includegraphics[height=8cm]{evaluate_CSI}
\caption[] { 1度 POP中一步CSI求解器的成分分析}
\label{fig:cst_ratio}
\end{figure}

\subsection{特征值估计}
\label{solver:eigs}
基于Lanczos-based的特征值估计的过程如算法\ref{alg:lanczos}.
当特征值的估计是准确的时,即$\nu = \lambda_{min}$ , $\mu =\lambda_{max}$,CSI方法的收敛速度达到理论上的最优值。 
但是得到最大最小特征值$\lambda_{min}$ 和 $\lambda_{max}$ 的精确值是很困难的。
另外,由于系数矩阵$A$ 是分布在各个进程上的,因此不建议对它做任何形式的变形。  
为了能够利用POP的并行特性,我们使用Lanczos方法来构造 出一系列三对角矩阵$T_m (m=1,2,...)$,这些矩阵的最大最小特征值逐渐的向$M^{-1}A$的最大最小特征值逼近。
基于Lanczos方法的特征值估计的步骤如算法\ref{alg:lanczos}所示。 
\begin{algorithm}
\caption{ 基于Lanczos方法的特征值估计}
\label{alg:lanczos}
\begin{algorithmic}[1]
\REQUIRE  网格块$B_{i,j}$ 的系数矩阵$\textbf{B}$和随机向量$\textbf{r}_0$;\\
 //\qquad    \textit{所有进行并行执行}
\STATE $\textbf{q}_1 = \textbf{r}_0/||\textbf{r}_0||$;\quad $\textbf{q}_0=\textbf{0}$;\quad $T_0=\emptyset$;\quad $\beta_0 =0$;\quad  $\mu_0 =0$;\quad $j=1$;
\WHILE{$j<k_{max}$} 
%\STATE $\textbf{q}_1 = \textbf{r}_0/||\textbf{r}_0||$; \quad $\beta_0\textbf{q}_0=0$;\quad  $\tilde{\mu}_k =0$; \quad $j=0$;
%\FOR{$j = 1, 2,...,k$}
\STATE $\textbf{r}_j=\tilde{\textbf{A}}\textbf{q}_j-\beta_{j-1}\textbf{q}_{j-1}$;\quad $update\_halo(\textbf{r}_j)$;
\STATE $\tilde{\alpha}_j =\textbf{q}_j^T\textbf{r}_j$;\quad $\alpha_j=global\_sum(\tilde{\alpha}_j)$; 
\STATE $\textbf{r}_j=\textbf{r}_j-\alpha_{j}\textbf{q}_{j}$;
%\IF{$\textbf{r}_j \neq \textbf{0}$} 
\STATE $\tilde{\beta}_j = \textbf{r}_j^T\textbf{r}_j$; \quad $\beta_j=sqrt(global\_sum(\tilde{\beta}_j))$;
\STATE \textbf{if} $\beta_j == 0$ \textbf{then} \textbf{return}
\STATE $\mu_j = max(\mu_{j-1}, \quad \alpha_j+\beta_j+\beta_{j-1})$; \label{lanczos_gersh} \COMMENT{Gershgorin圆盘定理}\\
%\ENDIF
%\ENDFOR \\
%//\qquad    \textit{do on master node}
\STATE $T_k=tri\_diag(T_{k-1},\alpha_j,\beta_j)$; \quad $\nu_k = eigs(T_k,'smallest')$ ; \label{lanczos_tridiag} \COMMENT{三对角矩阵}
\STATE \textbf{if} $|\frac{\mu_k}{\mu_{k-1}} -1 |< \epsilon\quad\textbf{和}\quad|1- \frac{\nu_k}{\nu_{k-1}}|< \epsilon$ \textbf{then} \textbf{return}
\STATE $\textbf{q}_{j+1}= \textbf{r}_j/\beta_j$;\quad $j=j+1$;
\ENDWHILE
\end{algorithmic}
\end{algorithm}
 
在算法\ref{alg:lanczos}的第\ref{lanczos_tridiag}步中, $T_m$ 是一个三对角矩阵。它以 $\alpha_i (i=1,2,...,m)$ 为对角线元素,  $\beta_i (i=1,2,...,m-1)$ 为次对角元素。 

\[ T_{m} = tridiag\left(\begin{array}{ccccccc}
&\beta_1 && \bullet & &\beta_{m-1}&    \\
\alpha_1 & &\alpha_2 && \bullet &&\alpha_{m}\\
&\beta_1 && \bullet & & \beta_{m-1}&
\end{array} \right)\]

假设 $\xi_{min}$ 和 $\xi_{max}$ 分别为 $T_m$的最小的和最大的特征值。 Paige\cite{Paige1980235} 证明了以下结论

\begin{align}
&\lambda_{min} \le \xi_{min} \le \lambda_{min}+\delta_1(m) \\
&\lambda_{max}-\delta_2(m)  \le \xi_{max} \le \lambda_{max}
\end{align}

这里$\delta_1(m)$ 和 $\delta_2(m)$ 随着 $m$的增大而逐渐的向零逼近。 因此,系数矩阵 $A$ 的特征值的估计就可以转换成求解三对角矩阵  $T_m$的特征值。 
算法\ref{alg:lanczos} 中第\ref{lanczos_gersh}步应用了 Gershgorin 圆盘定理来估计 $T_m$的最大特征值,也就是 
\begin{align}
\mu = \max_{1 \le i \le m}\sum^m_{j=1}|T_{ij}|=\max_{1 \le i \le m}(\beta_{i-1}+\alpha_i +\beta_{i})
\end{align}
 
\begin {figure}%[htbp]
\centering
\includegraphics[height=6.5cm]{eigen_lanczos}
\caption[] { Lanczos迭代步数和特征值估计的关系\label{fig:lanczos_eigs}}
\end{figure}
\begin {figure}%[htbp]
\centering
\includegraphics[height=6.5cm]{scan_lanczos}
\caption[] { Lanczos迭代步数和求解器迭代步数的关系\label{fig:lanczos_iter}}
\end{figure}
在算法\ref{alg:lanczos} 中第\ref{lanczos_tridiag}中,采用了高效的QR算法\cite{ortega1963llt}来估计最小特征值$\nu$, 它的复杂度为$\Theta(m)$ 。 
如图\ref{fig:lanczos_eigs}和\ref{fig:lanczos_iter}所示,我们发现合适的选择Lanczos迭代的步数,我们能够得到对最大最小特诊一个比较好的估计,这个估计值能够使得CSI方法与预处理共轭梯度法有相近的收敛速度。  


\section{算法分析与比较}
\label{solver:Algorithm}
 
除了P-CSI,PCG和ChronGear方法的收敛性也同样依赖于系数矩阵的最大最小特征值。
这里,我们首先来挖掘一下影响求解器收敛的特征值的特征,然后给出ChronGear和P-CSI求解器的计算复杂度和收敛速度的理论结果。 


 



\subsection{收敛速度} \label{solver:Algorithm:convergence_rate}

 PCG和ChronGear方法的收敛速度都依赖于系数矩阵$A$的条件数。  
论文\inlinecite{dAzevedo1999lapack}证明了 ChronGear  和PCG在理论上收敛速度是相同的。 
在PCG和ChronGear方法中,每一步迭代的残差都有一个上界\cite{Liesen2004}
\begin{equation}
\frac{||\textbf{x}_k-\textbf{x}^*||_A }{||\textbf{x}_0-\textbf{x}^*||_A}  \le \min_{p\in \mathcal{P}_k, p(0) = 1 }\max_{\lambda \in \mathcal{S}} |p(\lambda)| \label{PcgConvergeRate}
\end{equation}
这里$\textbf{x}_k$ 表示第$k$步迭代后得到的解向量, $\textbf{x}^*$ 表示线性方程组的解析解  (即$\textbf{x}^* = A^{-1}b$),$\lambda$ 表示系数矩阵 $A$的特征值。
利用 第一类Chebyshev 多项式来估计这个最小最大值,可以得到 
\begin{align}
\label{chrongear_convergence}
||\textbf{x}_k-\textbf{x}^*||_A \le  2 (\frac{\sqrt{\kappa}-1}{\sqrt{\kappa}+1})^k ||\textbf{x}_0-\textbf{x}^*||_A
\end{align}
这里  $\kappa =  \kappa_2(A)$ 表示系数矩阵 $A$的条件数。


方程\ref{chrongear_convergence}表明,PCG方法收敛速度的理论上界取决于系数矩阵的条件数。 
PCG方法在求解良态的矩阵(特征值的条件数很小)时比求解病态的矩阵(特征值的条件数很大)时收敛速度更快。 
图\ref{fig:convergence_diag}给出了PCG和ChronGear方法在0.1度海洋模式中的真实的收敛速度。
如这幅图所示,相对残差随着迭代步数的增加而减少。
总共需要一百步左右的迭代步才能够得到一个相对残差为$10^{-13}$的解。
 

 
除了消除了全局归约操作,P-CSI方法的更一个优点是它有和PCG相同量级的收敛速度。 
P-CSI方法的收敛速度可以用残差的形式表示为
\begin{equation}
\textbf{r}_k = P_k(A)\textbf{r}_0 \label{eq:rPjr0}
\end{equation}
这里
$P_k(\zeta) = \frac{\tau_k(\beta-\alpha \zeta)}{\tau_k(\beta)}$ for $ \zeta \in [\nu, \mu]$ ~\cite{stiefel1958kernel} .
$\tau_k(\xi)$ 是Chebyshev多项式,它的表达式为  
\begin{equation}
\tau_k(\xi) =   \frac{1}{2}[(\xi+\sqrt{\xi^2-1})^k+(\xi+\sqrt{\xi^2-1})^{-k}]
\end{equation}
当 $ \xi \in [-1,1]$时, Chebyshev多项式的另一个等价表达式为$\tau_k(\xi) = cos(k\cos^{-1} \xi)$。从这个表达式可以清楚的看出来,当$| \xi | \le 1$时, $|\tau_k(\xi)| \le 1$。$P_k(\zeta)$ 是一个多项式,满足如下形式
\begin{equation}
P_k = \min_{p\in \mathcal{P}_k, p(0) = 1 }\max_{\zeta \in [\nu,\mu]} |p(\zeta)|
\end{equation}
%which is the theoretical bound of the convergence rate  in PCG \ref{PcgConvergeRate}.

假设$A= Q^T\Lambda Q$,这里 $\Lambda$ 表示以  $A$的特征值为对角元素的对角矩阵,而 $Q$ 表示以  $A$的特征向量为列的实正交矩阵。于是,我们可以得到
\begin{equation}
P_k(A) = Q^T P_k(\Lambda)Q = Q^T \left [\begin{array}{cccc}
P_k(\lambda_1) & & &\\
& P_k(\lambda_2) & &\\
& & \ddots &\\
 & & & P_k(\lambda_N)
\end{array} \right ] Q \label{eq:PjMA}
\end{equation}
Assume that the estimated largest and smallest extreme eigenvalues of coefficient matrix 
假设所估计的系数矩阵的最大最小特征值$\nu$和 $\mu$ 满足 $0 < \nu \le \lambda_i \le \mu$ ($i = 1, 2, \cdots, \mathcal{N}$),这时有$|\beta - \alpha \lambda_i| \le 1$, $|P_k(\lambda_i)| \le \tau^{-1}_k (\beta)$.
方程 \ref{eq:rPjr0} 和 \ref{eq:PjMA} 可以推出
\begin{equation}
\label{pcsi_convergence}
\frac{||\textbf{r}_k||_2}{||\textbf{r}_0||_2}  \le  \tau_k^{-1}(\beta) = \frac{2(\beta+\sqrt{\beta^2-1})^k}{1+(\beta+\sqrt{\beta^2-1})^{2k}} \le 2(\frac{\sqrt{\kappa'}-1}{\sqrt{\kappa'}+1})^k
\end{equation}
这里$\kappa' = \frac{\mu}{\nu}$.
 这个等式表明,当系数矩阵特征值的估计比较合理的时候(也就是$\kappa' =\kappa$) , P-CSI的收敛速度和PCG有相同的理论上界。


 
以上分析对于采用了预处理子的情形也同样适用。
唯一的区别是上面分析中提到的系数矩阵是$M^{-1}A$ 而不是$A$。
值得一提的是,PCG,ChronGear和P-CSI算法中预处理后的矩阵实际上是$M^{-1/2}A(E^{-1/2})^T$,它是对称的,并且和$M^{-1}A$ 有着相同的特征值\cite{Shewchuk1994}。
因此,这个预处理后的矩阵的条件数为$\kappa =  \kappa_2(M^{-1/2}A(E^{-1/2})^T)$。这个条件数通常情况下比原始矩阵 $A$的条件数要小。 
 $M$ 与$A$越是相近,$M^{-1}A$的条件数就越小。当$M = A$,$M^{-1}A$的条件数为$\kappa_2(M^{-1 }A ) = 1$。 

 当使用少数几个处理器核心来运行海洋模式分量时,全局通信并不是瓶颈,导致P-CSI方法和ChronGear方法的时间开销比较接近。
但是当采用很多的处理器核心来计算高分辨的海洋模式分量的时候,P-CSI方法的每一个步迭代就应该会比ChronGear方法快很多,因为ChronGear方法中的全局归约操作就会变成一个明显的瓶颈。

\subsection{计算复杂度}  \label{solver:Algorithm:complex}


 




%%%%%%%%%%%%%%%%%%%%%%%%%%%%%%%%%%%%%%%%%%%%%%%%%%%%%%%%%%%%%%%%%%%
\section{实验} 
\label{solver:exp}
 
\subsection{理想实验}\label{solver:exp:ideal}

为了验证第\ref{solver:Algorithm:convergence_rate}节中所给出的收敛的理论结果,我们在理想的试验设置中构造了一系列条件数不同的矩阵。
这里,我们不再采用全局固定的网格大小,而是采用均匀的经纬网格。
随着维度$\theta$的变化,沿着经度方向的网格大小为 $\Delta x_j  = \pi R \cos (\theta_j)$。
时间步长设置为$\tau = 10\frac{\Delta y}{c}$, 这里$c = 200m/s$ 表示重力快波的速度\cite{smith2010parallel}。 
我们用PCG和P-CSI方法分别结合单位阵预处理、对角预处理和EVP预处理来求解得到的这些方程。
实验中,EVP预处的块的大小为$5\times5$, 收敛条件设为$tol = 10^{-6}$。 
由于ChronGear 和PCG的收敛速度相同,因此这个图中PCG的收敛结果也是ChronGear的结果。
  

\begin{figure} 
\vspace{5pt}
\centering
\includegraphics[height=8cm]{iterationGridSize}
\caption[] {网格点数目和求解器的迭代步数之间的关系。\label{fig:iterationGridSize}}
\end{figure}

正如图\ref{fig:iterationGridSize}所示,随着问题规模的增加,系数矩阵变得越来越病态。
所有的求解器都需要更多的迭代步数才能得到相同的相对残差。
对于PCG和P-CSI,收敛速度会随着预处理方法的变化而变化。 
指定问题的规模,使用单位矩阵作为预处理的求解器需要的迭代步数最大,而使用EVP预处理的求解器所需要的迭代步数最小。 
这也说明,使用EVP预处理,矩阵的形态会比使用单位阵预处理或者对角预处理后的形态要好。 
图中还说明,在预处理方法相同的情况下,P-CSI方法的收敛速度要比PCG的慢一些。 



 
\subsection{CSI方法的性能}
 
我们先测试一下采用对角预处理子时,PCG,ChronGear和P-CSI三个求解器的性能。
图\ref{fig:convergence_diag}给出了不同的正压求解器的收敛速度。
图中横轴表示迭代的步数,垂直轴表示求解器在对应迭代步的残差。 
尽管ChronGear方法改变了PCG方法的计算顺序,PCG方法和ChronGear在每一步的迭代速度几乎一致。
P-CSI方法在开始和最后的迭代中收敛速度比较慢,而在中间的迭代步骤中收敛速度变快。
这与系数矩阵的特征值的分布有关系。 


\begin {figure}
\vspace{10pt}
\centering
\includegraphics[height=8cm]{Convergence_diag.eps}
\caption[] {不同求解器的收敛速度。\label{fig:convergence_diag}}
\end{figure}



%----------------------------------------------------------------------------
 
为了证明CSI方法并没有在海洋模式POP中引入误差,我们在清华大学的探索100集群(具体描述请参见第\ref{solver:pcgComplex}节)上对1度POP做了一组实验, 分别使用PCG和CSI方法作为正压模态的求解器运行一段时间。 
表格\ref{tab:err}给出的是这两个版本得到的结果的海表高度的差别。
采用CSI和PCG的两个版本之间的平均误差相对于海表高度的绝对值非常的小。
有意思的是,我们注意到,两者之间最大的误差主要出现在海岸线上。 这些地方比较尖锐的边界很容易造成差分格式的不稳定性。
随着实验时间的延长,两个版本之间的误差也随着增大。 这并不是说求解器之间的误差在变大,而是由于海洋模式中湍流的累积效应将最初的小误差逐渐积累放大。 

\begin{table}
\centering
\caption[] {采用PCG和CSI作为正压求解器的两个版本之间的海表高度的比较   \label{tab:err}}
\begin{tabular}{l l@{\quad}l@{\quad}l@{\quad}l} 
\toprule
模拟时间   & 一步  & 一天    & 一个月 &三个月\\
\hline
\multicolumn{1}{l}{总步数 } &\multicolumn{1}{c@{\quad}}{  1} &\multicolumn{1}{c@{\quad}}{  45} &\multicolumn{1}{c@{\quad}}{ 14053}	&\multicolumn{1}{c@{\quad}}{40800}\\
%\hline
最大相对误差 & 1.5016E-3&2.2181E-5& 1.2885E-2&1.4114E-1\\
%\hline
平均相对误差 &3.0223E-6&5.2424E-7& 2.6125E-5&7.8872E-4\\
\bottomrule
\end{tabular}
\end{table}


 
为了测试算法的可扩展性,我们在中国的国家超算中心的神威蓝光超级计算机上测试了0.1度的海洋模式POP(3600 $\times$ 2400个网格点)。
神威蓝光是由8,704个SW1600处理器由40Gb 无限带宽网络连接组成。
每一个处理器包含16个1.1GHz的处理器核心以及16GB的内存。 


我们分别测试了PCG 和 CSI两个版本的求解器在不同的处理器核数(100到 15,000 )和不同的收敛条件( $\epsilon = 10^{-8}$ 到 $\epsilon = 10^{-16}$)时的性能。
海洋模式 POP 中的收敛准则是 $||r||_2<\epsilon \bar{a}$,这里 $\bar{a}$ 表示总面积的均方根。 
在单独版本的POP中,默认设置为 $\epsilon = 10^{-12}$。 
如图\ref{fig:scale}所示, PCG 和 CSI 在小于  1,000核时可扩展性都很好。但是当使用超过1,000 核时, CSI相对于PCG的优势就很明显了。 
当收敛条件为 $10^{-12}$时,在100核上CSI的执行时间只有PCG的87\%, 这个比例在15,000核上下降到了 21\% 。
但是PCG方法对于收敛条件没有CSI方法敏感。 
当时收敛条件从$10^{-8}$变化到$10^{-16}$ 时, PCG方法的迭代步数从原来的20步增长到281步,但是CSI方法的迭代步数从33步增长到了1,434步。 
但是,PCG方法在收敛速度上的优势并没有使得PCG方法完全的比CSI方法快。 
只有当收敛条件是最为严格的$\epsilon = 10^{-16}$并且当使用的核心数少于1000核时,PCG的执行时间才会比CSI的短。
其他情况下, PCG方法中的全局归约操作的开销并不能被迭代步数的减少给抵消掉。
在1,500核上,当收敛条件$10^{-8}$ 和 $10^{-12}$时,PCG的执行时间是CSI的4.7倍, 而当收敛条件为$10^{-16}$时,这个倍数仍然高达3.3. 
由于全局归约操作的瓶颈, 当使用的核心数大于1,000时,PCG求解器的执行时间随着所使用的核数的增加而增加。 
相对应的,CSI算法一直到10,000核以上都能保持比较好的可扩展性。 
P 

\begin {figure}
\centering
\includegraphics[height=6.5cm]{scale_pcg_csi}
\caption []{ 0.1度POP中PCG和CSI方法的可扩展性 \label {fig:scale}}
\end {figure}
%Fig.1(b) shows runtime ratio of barotropic 和 baroclinic modes of POP. Baroclinic mode deals with a three dimensional dynamical process which occupies a major proportion of computation of POP. It dominates in POP when process is few 和 computation time is much larger than communication. However, when using thous和s of cores, runtime of baroclinic parts decreases linearly while on the opposite, runtime of barotropic increases because of global reduction.


% %%%%%%%%%%%%%%%%%%%%%%%%%%%%%%%%%%%%%%%%%%%%%%%%%%%%%%%%%%%%%%%%%%%
% \section{相关工作} 
% \label{solver:rel}
% %----------------------------------------------------------------------------
 

 

\section{本章小结}
\label{solver:Conclusion}

目前已经有很多的工作在研究如何提高海洋模式POP中正压模态的性能。 
但是他们的大部分都没有从根本上解决POP正压模态中的可扩展性瓶颈。 
这篇文章提出了一个评估海洋模式POP正压模态的模型, 通过这个模型,我们对预处理共轭梯度法的可扩展性进行量化。 
这个模型从理论上证明了CSI的优越性。 
我们将CSI方法实现在海洋模式POP中并且得到了比原始的预处理共轭梯度法更好的可扩展性。 
总之,这篇文章提出了一个比传统的预处理共轭梯度法更有前景的求解椭圆方程的方法。 


