\chapter{海洋模式正压求解器}
\label{cha:barosSolver}

\section{本章概述}

\section{背景和动机}
\label{sec:baroBackgroud} 
为了完整的推导我们的新的P-CSI求解器,这里我们简要的描述一下海洋模式POP中的控制方程。
海洋模式中的原始动量和连续性方程可以表达如下:
\begin{align}
&\frac{\partial }{\partial t} \textbf{u} +\mathcal{L}(\textbf{u}) + f\times \textbf{u} = - \frac{1}{\rho_0}\nabla p +F_H(\textbf{u}) +F_V(\textbf{u}) \label{eq:momen}\\
&\mathcal{L}(1) = 0 \label{eq:continuous}
\end{align}
这里$\mathcal{L}(\alpha ) = \frac{\partial }{\partial x} (u\alpha)  +\frac{\partial }{\partial y} (v\alpha) +\frac{\partial }{\partial z} (w\alpha)$, 它在$\alpha =1$时与平流算子等价, $x,y,z$ 分别是水平和垂直坐标变量, $\textbf{u} = [u,v]^T$ 水平速度向量, $w$ 垂直速度, $f$ 表示科氏力系数,  $p$ 和 $\rho_0$ 分别表示压力和密度, $F_H$ 和 $F_V$分别表示水平和垂直耗散项 \citep{smith2010parallel}。  
为了求解这个三维的原始方程,海洋模式分量将时间积分分解为两个模态。 一个是求解三维动力学和热动力学过程的斜压模态,另一个则是求解二维海表高度(SSH)的变化。

%----------------------------------------------------------------------------
\subsection{正压模态} \label{se:baro_mode}


  
在静力平衡近似条件下,POP中海底深度为$z$的位置上的压力可以分解为两个部分:  
\begin{align}
\displaystyle p = p_h + p_s = \int^0_z g\rho dz +p_s
\end{align}
这里$p_h$表示静力平衡压力,$p_s$表示由于海表自由面波动而引起的海表压力。 
正压模态的控制方程是由原始动量方程和连续性方程(\ref{eq:momen},\ref{eq:continuous}) 从海洋底部到海洋表面垂直积分而得到的:
\begin{align}
&\displaystyle \frac{\partial \textbf{U} }{\partial t}  = -g \nabla \eta + F  \label{eq:baro_mon}\\
&\displaystyle \frac{\partial \eta }{\partial t} = - \nabla \cdot H\textbf{U} + q_w  \label{eq:baro_con}
\end{align}
这里  $\textbf{U} =  \frac{1}{H+\eta}\int_{-H}^\eta dz \textbf{u}(z) \approx \frac{1}{H }\int_{-H}^0 dz \textbf{u}(z)$是正压速度的垂直积分,
$\eta = p_s/{\rho_0g}$是海表高度,$H$是海洋底部的深度,$q_w$单位海表面积的活水通量,$g$表示由于地球动力引起的加速度而$F$表示对海洋动量方程中除了与时间趋势和海表压力梯度相关的所有项做垂直积分所得到的量(参见方程\ref{eq:momen})。
 
垂直方向上的边界条件是
\begin{align}
\label{eq:bound_w}
w = \left\{ \begin{array}{ll}
\frac{\partial}{\partial t} \eta  +\textbf{u}\cdot\nabla \eta - q_w, & z = \eta  \\
0, & z = -H
\end{array} \right.
\end{align}
 
值得一提的是,为了简化求解过程,正压连续性方程 (\ref{eq:baro_con}) 中忽略掉了边界条件中的小项 $\nabla \eta$\citep{smith2010parallel}。
为了能够使用更长的时间步长,海洋模式分量的正压模态中采用了隐式格式,并且将椭圆方程简化成一个线性系统。 通过时间差分,方程  (\ref{eq:baro_mon})和(\ref{eq:baro_con}) 可化为
\begin{align}
&\displaystyle \frac{ \textbf{U}^{n+1} - \textbf{U}^{n-1}}{\tau}  = -g \nabla \eta + F \label{eq:udt} \\
&\displaystyle \frac{\eta^{n+1} - \eta^n }{\tau}  = - \nabla \cdot H\textbf{U} + q_w \label{eq:etadt}
\end{align}
这里 $\tau$ 与差分格式相关的时间步长。 
用方程(\ref{eq:udt})中下一时刻的正压速度来替换方程(\ref{eq:etadt})中的正压速度, 就得到了一个关于海表高度$\eta$的椭圆系统
\begin{equation}
\label{eq:sshdiscret}
     [-\nabla\cdot H \nabla + \frac{1}{g  \tau^2}]\eta^{n+1}
           = -\nabla\cdot H[\frac{\textbf{U}^{n-1}}{g \tau} + \frac{F}{g}] + \frac{\eta^n}{g\tau^2} +\frac{q_w}{g\tau}
\end{equation}
 
为了简单,我们将椭圆方程(\ref{eq:sshdiscret})重新标记为
\begin{equation}
\label{eq:ssh}
[-\nabla \cdot H\nabla +\frac{1}{g  \tau^2}]\eta^{n+1} = \psi(\eta^n,\eta^{n-1},\tau)
\end{equation}
这里 $\psi$表示一个关于当前时刻和上一时刻$\eta$状态的函数。
 

\begin{figure}[!htbp]
\includegraphics[width=12cm, height=6.5cm]{grid_domain.pdf}
\caption[] {海洋模式分量中网格划分\label{fig:grid1}}
\end{figure}

\begin{figure}[!htbp]
\includegraphics[ height=10cm]{EvpBug_fix_gmd_new}
\caption[] {0.1度海洋模式POP的海表盐度。 \label{fig:sst}}
\end{figure}

如图\ref{fig:grid1}所示, POP采用在水平方向上采用Arakawa B网格\citep{smith2010parallel},并且采用九点差分格式对方程(\ref{eq:ssh})进行离散。得到如下离散格式:
\begin{align}
    & \nabla\cdot H \nabla \eta  =\frac{1}{\Delta y}\delta_x \overline{[\Delta y H  \delta_x\overline{\eta}^y]}^y +\frac{1}{\Delta x}\delta_y \overline{[\Delta x H  \delta_y\overline{\eta}^x]}^x \label{eq:nabla2}
  \end{align}

这里 $\Delta_\xi$和$\delta_\xi$  ($\xi \in \{x, y\} $)分别为有限差分和它们相应的偏导,  $\delta_\xi (\cdot) $和$\overline{(\cdot)}^\xi $ 分别表示有限差分算子和平均算子。  
\begin{align}
&\delta_\xi \psi = [\psi (\xi+\Delta_\xi/2) -\psi(\xi-\Delta_\xi/2)]/\Delta_\xi \\
&\overline{\psi}^\xi  =[\psi (\xi+\Delta_\xi/2) +\psi(\xi-\Delta_\xi/2)]/2
\end{align}
%----------------------------------------------------------------------------


为了避免极点问题,POP采用偏移的或者三极点的广义正交网格。这使得方程的系数非常的繁杂。
为了能够简单而不失偏颇,我们将给出系数矩阵在均匀的网格和给定的不变的海底深度$H$的情况下的显式表达式。
 

表达式\ref{eq:nabla2}此时为
\begin{align}
 [\nabla\cdot H \nabla \eta]_{i,j}&= -\frac{H}{S_{i,j}}(A_{i,j}^O\eta_{i,j}+A_{i,j}^{NW}\eta_{i-1,j+1}+A_{i,j}^N\eta_{i,j+1} +A_{i,j}^{NE}\eta_{i+1,j+1}+A_{i,j}^W\eta_{i-1,j}  \nonumber\\
& +A_{i,j}^E\eta_{i+1,j} +A_{i,j}^{SW}\eta_{i-1,j-1} +A_{i,j}^S\eta_{i,j-1}+ A_{i,j}^{SE}\eta_{i+1,j-1})
\end{align}
这里$S_{i,j}  = \Delta x\Delta y$, $A_{i,j}^{\chi } ( \chi \in \mathcal{Q} = \{NW,NE, SW, SE, W, E, N, S\})$ 分别表示网格点 $(i,j)$和它自己、以及它的邻居点之间的在 九点差分格式中的系数(\ref{eq:nabla2})。 这些系数由$\Delta x$, $\Delta y$, $\tau$和$H$决定。
\begin{equation} \label{defineA}
\begin{aligned}
&\alpha  = \frac{ \Delta y}{ \Delta x }, \quad \beta  = 1/\alpha \\
&A_{i,j}^{NW} = A_{i,j}^{NE} =A_{i,j}^{SW} = A_{i,j}^{SE} = - (\alpha  +  \beta  )/4 \\
&A_{i,j}^{W} = A_{i,j}^{E} = (  \beta  -\alpha  )/2 \\
&A_{i,j}^{N} = A_{i,j}^{S} = (\alpha  -\beta )/2 \\
&A_{i,j}^{O} =   \alpha   +\beta  \\
\end{aligned}
\end{equation}


%&\alpha_1 = \frac{2\Delta y}{2\Delta x_j +\Delta x_{j+1}2}\\
%& \alpha_2= \frac{2\Delta y}{ \Delta x_j +\Delta x_{j-1}  } \\
%&\alpha_3 = \frac{\Delta y(\Delta x_{j-1}+2\Delta x_j +\Delta x_{j+1})}{2 (\Delta x_j +\Delta x_{j-1})(\Delta x_j +\Delta x_{j+1})} \\
%& \beta_1 = \frac{(\Delta x_j +\Delta x_{j+1})}{8 \Delta y} \\
%& \beta_2 = \frac{(\Delta x_j +\Delta x_{j-1})}{8 \Delta y} \\
%& \beta_3 = \frac{(\Delta x_{j-1}+2\Delta x_j +\Delta x_{j+1})}{8 \Delta y} \\
%&A_{i,j}^{NW} = A_{i,j}^{NE} = \alpha_1 + \beta_1 \\
%&A_{i,j}^{SW} = A_{i,j}^{SE} = \alpha_2 +  \beta_2 \\
%&A_{i,j}^{W} = A_{i,j}^{E} = \alpha_3 - 2 \beta_3 \\
%&A_{i,j}^{N} = -2\alpha_1  + 2 \beta_1 \\
%&A_{i,j}^{S} = -2\alpha_2  + 2 \beta_2  \\
%&A_{i,j}^{O} = -2\alpha_3  -4 \beta_3  \\
%Then, the coefficients between the given point $(i,j)$和its other neighbors can be computed from $A^n$, $A^e$和$A^{ne}$ on its neighbors。
方程\ref{eq:ssh}在给定点$(i,j)$上的离散格式则可以表示为  
\begin{align}
\label{eq:sten}
&(A_{i,j}^O+\phi ) \eta_{i,j}+A_{i,j}^{NW}\eta_{i-1,j+1}+A_{i,j}^N\eta_{i,j+1} +A_{i,j}^{NE}\eta_{i+1,j+1}+A_{i,j}^W\eta_{i-1,j}  \nonumber\\
& +A_{i,j}^E\eta_{i+1,j} +A_{i,j}^{SW}\eta_{i-1,j-1} +A_{i,j}^S\eta_{i,j-1}+ A_{i,j }^{SE}\eta_{i+1,j-1}= \frac{S_{i,j}}{H}\psi_{i,j}
\end{align}
这里$\phi = \frac{S_{i,j}}{g  \tau^2H}$ 是与单位网格大小、海底深度和时间步长有关的一个变量。

\begin{figure}[!htbp]
\includegraphics[height=6.5cm]{SparsePatternSample}
\caption[] {大小为$30\times 15$的网格上所得到的系数矩阵的稀疏模式。 \label{fig:spy}}
\end{figure}
因此椭圆方程(\ref{eq:sshdiscret})就变成了关于$\eta$的一个线性方程组,也就是 $Ax= b$。 其中$A$表示由系数 $A^*$构成的快对角矩阵。
方程 \ref{defineA}和 \ref{eq:sten}表明$A$是有水平网格大小、海底深度和时间步长决定。 
他们之间具体的关系将在第\textbf{ref}节中深入讨论。
%In the POP, only the nonzero elements are stored。
方程(\ref{eq:sten})同时还表明 $A$ 每一行只有九个非零元素, 也就是说$A$是一个系数矩阵。 图\ref{fig:spy}展示了$A$的稀疏模式。


\section{海洋模式正压模态求解器}
\label{sec:baro1}
方程(\ref{eq:ssh})在二维广义正交网格上使用九点差分格式进行离散。  
POP将全球区域划分为若干个小块,并将这些块分发给每一个进程。
每个进程只负责计算它所分得的块内网格点上的迭代过程,并且维护一个与周围进程进行数据交换的边界区域。 
我们假设全球区域上的网格大小为$\mathcal{N}\times
\mathcal{N}$且被划分为$m\times m$ 个大小为$n\times n$ ($n=\mathcal{N}/m$)的小块.
我们定义$B$ 和 $\tilde{x}$ 分别为与给定块相对应系数矩阵(也就是 $A$的一个大小为 $n^2\times n^2$的子矩阵)和向量。
这个块上的子矩阵和子向量的乘积操作$B\tilde{x}$ 需要$9n^2$ 次计算 \cite{hu2013scalable}.
%----------------------------------------------------------------------------
\subsection{ChronGear求解器}
\begin {figure}[!t]
%\vspace{-5pt}
\centering
\includegraphics[height=6.5cm]{newPOPStepComp}
\caption[] {0.1度POP采用默认的对角预处理的ChronGear求解器时个分量耗时的百分比。\label{fig:StepComp}}
\end{figure}

ChronGear \cite{dAzevedo1999lapack} 是一种修正过的共轭梯度法,它将每一步迭代过程中的两次全局通信组合成一次。
然而,如前面提到的,当成千上万个处理器核心被用来计算高分辨(比如0.1度)的POP时,ChronGear算法中计算內积所需要的全局通信操作仍然是一个严重的瓶颈。
这个瓶颈对0.1度POP的影响可以通过图\ref{fig:StepComp} 看出来。线性求解器的计算时间占总时间的比例随着所使用的处理器核心数的增大而增加。
当只使用470个处理器核心时,正压求解器的执行时间只占到整个POP的运行时间的5\% (这里不考虑初始化和 I/O时间),而此时斜压模态的运行时间却占总运行时间的90\%左右。
然而,当使用数千个处理器核心时,斜压模态所消耗的计算时间占总时间的比例会逐渐下降,而正压求解器所占的比例却在不断增加。
当使用1万6千多处理器核心时,正压求解的计算时间占中运行时间的比例高达50\%。 

\begin{algorithm}[!t]
\caption{Chronopoulos-Gear求解器}
\label{alg:pcg}
%\begin{scriptsize}
\begin{algorithmic}[1]
\REQUIRE   与网格快$B_{i,j}$相应的系数矩阵$\textbf{B}$, 预处理子 $\textbf{M}$, 初始解 $\textbf{x}_0$ 和 $\textbf{b}$  \\
//\qquad    \textit{do in parallel with all processes}
\STATE $\textbf{r}_0 = \textbf{b}-\textbf{B}\textbf{x}_0$, $\textbf{s}_0 =0$, $\textbf{p}_0 =0$;\quad $\rho_0=1$,$\sigma_0=0$, $k=0$;
\WHILE{$k \leq k_{max}$ }
\STATE $k=k+1$;
\STATE $\textbf{r}'_{k} =\textbf{M}^{-1}\textbf{r}_{k-1}$; \label{pcg_scale0} \COMMENT{preconditioning}
\STATE $\textbf{z}_k = \textbf{B}\textbf{r}'_{k}$; \label{pcg_mat}\COMMENT {matrix-vector multiplication}
\STATE $update\_halo(\textbf{z}_{k})$; \COMMENT{boundary communication} \label{pcg_bc1}
\STATE $\tilde{\rho_k} = \textbf{r}_{k-1}^T\textbf{r}'_{k}$;\label{pcg_dot1}
\STATE $\tilde{\delta_k} = \textbf{z}_k^T\textbf{r}'_k$;\label{pcg_dot2}
\STATE $(\rho_k,\delta_k) = global\_sum(\tilde{\rho_k},\tilde{\delta_k})$;\label{pcg_global1} \COMMENT{global reduction}
\STATE $\beta_k = \rho_k / \rho_{k-1}$;\label{pcg_beta}
\STATE $\sigma_k = \delta_k - \beta_k^2\sigma_{k-1}$;\label{pcg_sigma}
\STATE $\alpha_k = \rho_k /\sigma_{k}$;\label{pcg_alpha}
\STATE $\textbf{s}_k = \textbf{r}'_{k} +\beta_k\textbf{s}_{k-1}$;\label{pcg_scale1}
\STATE $\textbf{p}_k = \textbf{z}_{k} +\beta_k\textbf{p}_{k-1}$;\label{pcg_scale2}
\STATE $\textbf{x}_k =\textbf{x}_{k-1} +\alpha_k \textbf{s}_k$;\label{pcg_scale3}
\STATE $\textbf{r}_k =\textbf{r}_{k-1} -\alpha_k\textbf{p}_k$;\label{pcg_scale4}
\STATE $convergence\_check(\textbf{r}_{k})$;  \COMMENT{check convergence}
%\IF{ $k \% n_{c} == 0$}
%\STATE check convergence;
%\ENDIF
%\STATE \textbf{if} $||\textbf{r}_k|| \le \epsilon$  \textbf{return} ;\COMMENT{check convergence every $n_c$ iterations}
%\ENDIF
\ENDWHILE
\end{algorithmic}
%\end{scriptsize}
\end{algorithm}

\section{CSI迭代法}
\label{sec:barocsi}

\subsection{特征值估计}
\label{sec:barocsi}

\section{试验方法和结果分析}
\label{sec:verifyExp}

\section{本章小结}
\label{sec:verifyConclusion}




