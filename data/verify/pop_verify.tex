 
\subsection{研究动机}

 
文章 \cite{yong2015} 中给出来的求解器验证性工作很有启发性,而且完全满足目前的需求。 
但是它仍然有几个遗留问题没有解决,这促使我们研发了一个类似于CAM-ECT这样的更强大的针对CESM-POP数据的评测工具。 
第一,一个更通用的海洋模式POP的验证性工具首先考虑的是海洋中空间的变化特点,这个特点比大气中的要更加突出。
CAM-ECT和文章\cite{yong2015} 中的RMSZ方法是利用空间的平均值的差别来作评估,目前还不清楚这样的策略是否有足够的能力来检测海洋中的正确性表达问题以及模式本身的误差。
第二,由于CESM-POP的相互独立的诊断性变量要比大气模式CAM少很多,因此我们需要重新思考了一下一致性检测的量化指标。
第三,不同的衡量指标的选择促使我们进一步检查集合模拟的大小应该如何选择。
在文章\cite{yong2015} 中,四十个实例组成的集合模拟就足以检查出线性求解器中的误差,而这个集合模拟的大小并没有在更通用的环境中进行实验验证。
而且,这个大小要比CAM-ECT中使用的默认大小151要小很多。
最后,海洋和大气中时间尺度的不同也促使我们去研究集合模拟的模拟时间究竟多长最为合适。 
由于海洋中的时间尺度比大气中的要长一些, 直观上讲,CESM-POP中额统计学一致性检测工具需要的集合模拟的模拟时间应该会更长一些。 
值得一提的是,图\ref{fig:rmsz_temp_ens} 似乎又有悖于我们的这个假设。 
我们可以看出来,当收敛条件足够严格的时候(收敛条件小于$10^{-12}$时), 均方根标准分在模式运行几个月之后随着时间的推移而逐渐减小。
这表明,我需要进一步的研究一致性检测所需要的模拟时间的长度。 
实际上, 图\ref{fig:rmsz_temp_ens}可以看出,这个给定的1度分辨率的海洋数据是相对确定的,在求解器中引入的误差足够小(也就是收敛条件足够小的时候)初值场中的误差在模拟了一年之后逐渐的耗散掉了。


%----------------------------------------------------------------------------
%----------------------------------------------------------------------------
%----------------------------------------------------------------------------


\section{CESM-POP的新的一致性检测工具}\label{sec:newtest}

 
在CESM-ECT工具集中扩充CESM-POP一致性检测的工具(我们记为POP-ECT),我们需要一个能够反映海洋模式自身不确定性的恰当的集合模拟结果,并且开发一种针对海洋自身特点的方法。
这里我们首先讨论一下如何构造这个集合模拟结果,然后介绍一下新的测试工具的流程。 
\subsection{POP数据集合的构造}\label{sec:ensemble}

 
我们通过与很多被认可的海洋状态组成的集合作对比,来评估CESM-POP的输出数据与数据集合在统计学上的差别是否明显。
明显则我们认为这个输出数据与给定的数据集合是不同的气候状态。 
因此,将CESM-ECT应用到CESM-POP数据的第一步就是要构造出一个合适的数据集合。 
很明显,数据集合的组成对于测试的有效性来说至关重要,而且这些数据必须在一个被认可的机器上采用被认可的CESM版本来生成。
这里,与CAM相比,海洋模式的诊断变量要少很多,相互独立的诊断变量有:温度(TEMP),海表高度(SSH),盐度(SALT), 以及与模式网格相对应的经向和纬想得速度(分别为UVEL和VVEL)。
这五个变量中,只要海表高度SSH是二维变量,其他都是三维变量。 

 
在POP-ECT中,我们够早了一个由$N_{ens}$模拟结果组成的数据集合 $E$, 记为 $E =\{E_1, E_2, \dots, E_{N_{ens}} \}$。
这些模拟结果与指定的默认版本之间的唯一不同是在初始的海洋温度场中加入$\mathcal{O}(10^{-14})$量级的随机扰动。 
这个初值扰动的大小和双精度浮点数的舍入误差在一个量级,因此它应该不会造成气候上的改变。
这个数据集合的成员的数量必须足够的多才能够生成一个比较有表达力的分布,但是这个数量又要足够的小,因为增加一个成员就以为增加一份计算开销。 
我们将在稍后的第\ref{sec:ens}节中讨论数据集合的大小的选择 。这里的讨论和之后的实验我们都采用 $N_{ens} \;=\; 40$这个配置。
我们的实验表明,如果只是用来做一致性测试,这个大小的集合 足以充分的反应海洋的内在的不确定性。
模拟数据集合包含有在每个点上五个POP诊断变量(TEMP, SSH, SALT, UVEL和VVEL)的月平均数据。
每一个这样的变量$X$都包含有$N_X$个网格点,记为${X} = \{ x_1, x_2, \dots, x_{N_X}\}$, 这里$x_i$ 表示在网格点$i$上变量$X$的月平均值。
我们收集了$T$个月份的数据(即$T$个时间片的月平均数据)。

 
CESM-ECT方法的下一步是量化数据集合的分布特征,进而为行的模拟结果提供评估依据。
这个数据集合的统计特诊描述被存在一个叫做数据集合汇总的文件中,并且这个文件会被贴上生成这些模拟结果的CESM自带的软件标签。
当这个汇总文件一旦生成, 数据集合中$N_{ens}$个模拟结果的历史文件就不需要继续保存了。 
回忆一下CAM数据所使用的方法,CAM-ECT会计算每个变量在每个点上的可用的年平均数据的全球面积加权平均值。
这些计算结果就构成了每个变量的$N_{ens}$个全球平均。 
然而, 这个方法采用的全面积加权平均不能很好的应用在海洋模式数据中,因为海洋模式的不确定性相比于大气模式数据来说,在网格点上的一致性更差一些。
因此,我们需要更多的考虑到空间上的分布特诊。 


比如,我们考虑一个由$N_{ens} \;=\; 40$个使用1度分辨的海洋模式POP作为海洋模式分量的CESM模拟$T\;=\;36$个月得到的模拟结果组成的数据集合。 
图 \ref{fig:SST_STD_all}中给出的是数据集合的1,12,24和36个模拟月份中海表温度(SST)的标准方差的空间分布。
注意到这里海表温度SST仅仅只是三维变量TEMP的最上面一层(更加准确的说,是10米的海洋表层)。
图 \ref{fig:SST_STD_all}可以看出来,标准方差远远不是空间一致分布的,不同的网格点之间的值相差好几个数量级(注意颜色刻度尺的刻度)。 
从1月份到12月份的变化也十分明显,这主要是与模式冷启动时由于流体动力学不稳定性造成的模式不稳定相关。  
在模拟一年之后,在热带区域出现的标准差最大表明集合模拟中热带不稳定波的增长带来了更大的不确定性\citep{legeckis1977}。
由于赤道区域的网格分辨率更细一些($1/3^\circ$左右),这些大的可变性在 赤道地区很容易被强化。 
在几个主要的海洋环流系统区域内,也有几个比较大的标准方差集中区域。
从12个月到36的变化相对较小,这表明相应的物理不稳定性由于海洋的耗散将不会继续增长。
鉴于图 \ref{fig:SST_STD_all}中明显的不确定性,文章\cite{yong2015} 中的均方根标准分策略可能在某些不确定性很小的区域由于方程\ref{e:rmsz}中的分母太小引入不必要的潜在误差。
这里强调一下,图\ref{fig:SST_STD_all} 中不确定性的大小确实是与分辨率有关的,并且,图\ref{fig:SST_STD_all} 中的结果只适应于有耗散的低分辨率海洋模式,比如说目前大部分气候研究中所使用的1度的CESM-POP。


 
综上所述,为了构造出一个在CESM-POP数据上比较健壮的集合统计学一致性检测工具,我们必须得到一个既包含有空间信息,又包含有时间信息的数据集合。
特别的,POP-ECT构造的集合综述文件中第$T$ 个 月份的时间片CESM-POP数据包含有:
\begin{itemize}
 \item $N_{var} \times N_X \times T$  monthly mean values across the ensemble at each grid point $i$ ($\mu_i$)
 \item $N_{var} \times N_X \times T$ standard deviations of ensemble monthly mean values at each grid point $i$ ($\sigma_i$),
 \end{itemize}
 这也就是说,我们保存了指定个月份$T \;=\; 36$ 在所有$N_x$ 个点上的集合平均值和标准方差值(这里 $N_x$ 是变量是二维还是三维的变量有关系)。
 

 最后值得一提的是,由于在淡水和盐度之间没有合适的淡水反馈机制,在没有任何特殊改动的情况,CESM-POP在封闭的内海区域(比如哈德逊湾和地中海)可能会产生一些与实际不符的盐度分布。
当前版本的CESM-POP在非耦合的模拟中强加了一个很强的淡水恢复机制,而在耦合模拟中加入的是边缘海淡水平衡机制。
这些特殊的处理能够保持盐度的平衡,但是对于模式的动力过程造成了虚假的强迫。
因此,本文的工作中我们不去讨论如何在边缘海中正确的做验证,而是将重心限制在开放海域内。

\subsection{验证过程}
 
利用前面所得到的POP-ECT综述文件,我们利用以下的方法来判断一个新的模拟输出结果(比如说改动了代码,或者移植到了新的机器上,或者采用了一个新的编译器选项等)是否和指定的数据集合的分布所描述的海洋气候状态在统计学上是一致。 
由于我们只有五个比较基础的诊断变量,因此我们不需要采用主成分分析的方法。 
这里,我们采用的策略是在每一个给定点上评估新的模拟结果和数据集合之间的标准差。
对于每一个给定点$i$和某一个给定的变量$\tilde{{X}}$, 在给定的时间片$t$上我们采用标准分方法计算$\tilde{{X}}$和数据集合的距离。 
特别的,给定变量$\tilde{{X}}$ 在 $t$时刻的值,$\tilde{{X}} = \{ \tilde{x}_{1,t}, \tilde{x}_{2,t}, \dots, \tilde{x}_{N_X,t}\}$, 那么在给定点$i$ 处变量$\tilde{{X}}$ 在$t$时间的标准分为
\begin{equation*}
Z_{\tilde{x}_{i,t}}=  \frac{\tilde{x}_{i,t} -\mu_{i,t}}{\sigma_{i,t}},
\end{equation*}
这里 $\mu_{i,t}$和$\sigma_{i,t}$分别表示指定的数据集合在点 $i$处,变量$X$在给定的$t$月份的几何平均和标准方差。  

 
下面,为了简洁,对于某一个时间片$t$,我们从相关的变量中去掉时间变量下表。比如说,标准分我们记为$Z_{\tilde{x_i}}$。 
我们假设在每一个点上允许的阈值为$tol_{Z}$,也就是说如果 $Z_{\tilde{x_i}} > tol_{Z}$就表明$i$这个点是一个“不通过”的点。 
由于标准分表示的是一个数与平均值之间的标准方差,所以标准分越大就表明这个新的模拟结果和数据集合所给的气候状态就越远。 
下一步,我们将考察一下所有点中通过标准分标准的点所占的比例。 定义给定变量$\tilde{X}$的标准分通过比例(ZPR)为:
\begin{equation}\label{e:zpr}
ZPR_{\tilde{X}} = \frac{ \#\{i \;|\; \tilde{x_i} \in \tilde{X} \; \land \; |Z_{\tilde{x_i}}| \; \leq \; tol_{Z}\} }{\#\{i \;|\; \tilde{x_i} \in \tilde{X} \} }.
\end{equation}
 
为了给出一个评判给定变量$\tilde{X}$ 是否通过测试的整体性标准,我们对标准分通过比例设置了一个最小阈值($min_{ZPR}$)。 
也就是说,如果$ZPR_{\tilde{X}} \geq min_{ZPR}$成立,则我们认为 $\tilde{X}$通过测试。
默认情况下,标准分的通过阈值为$tol_{Z} \; = \; 3.0$, 而标准分通过比例ZPR的阈值为$min_{ZPR} \; = \; 0.9$。 
换句话说,新的模拟结果中变量$\tilde{X}$要想通过测试,必须保证这个变量在90$\%$以上的点上的值落在在对应点对应变量的集合平均值($\mu$) 的$3.0$ 个标准差之间。
我们对五个独立的诊断变量依次做上面的判断。 
只有当所有的变量都通过测试,我们才认为这个新的模拟结果整体上来说和数据集合在统计意义上是一致的。


 
我们注意到计算得到的标准分随着模拟时间的改变而改变。 
由于海洋运动的时间尺度比较长,本章中大部分的实验我们都将CESM运行36个月。
另外,模拟过程中我们会保存月输出的时间片数据(CAM-ECT中只保存了年平均数据),用来考察数据集合中的海洋状态能否随着时间的延长而稳定下来。 



在后面的章节中我们可以看到,ZPRs通常在几个月的模拟之后便会平稳下来,而且这种平稳的趋势在几个诊断变量中都是相似的。 
因此,除了挑选出合适的标准分阈值以及ZPRs的通过率阈值,我们还选择了一个时间点($t_C$)
来判断新的运行结果(而不是去检查每个月的数据)。 
于是,集合模拟的时间长度不需要比$t_C$更长。 

\subsection{软件工具}

为了使得用户和开发展都能够使用我们的新的针对POP设计的诊断方法,我们将POP-ECT添加到现成的CESM-ECT Python工具集中(pyCECT v2.0)。
这个工具集已经被集成到当前CESM发布的版本中。 
这个CESM-ECT Python工具集包括了生成CESM特定模块的集合模拟综述文件的工具,以及利用这些集合模拟综述文件来进行统计学一致性检测的工具pyCECT。
由于POP-ECT的综述文件和CAM-ECT的综述文件完全不同,我们利用并行的Python代码pyEnsSumPop来生成POP-ECT综述文件。
具体来讲,从CESM-POP集合模拟的输出文件中,pyEnsSumPop并行的生成一个包含有第\ref{sec:ensemble}节中描述的海洋模式一些统计量的集合综述文件。 
CESM软件工程组将会按照需要生成一个新的POP模拟数据的集合,这个数据集合包含有一个注明了具体改动的软件标签。
POP-ECT所生成的集合综述文件和第\ref{sec:code}节提到的CESM的版本标签将被放到未来发展的模式中。 
有了POP-ECT的综述文件,用户或者开发者就可以利用pyCECT Python工具来评估新的模拟数据的一致性。目前可以选择POP-ECT或者CAM-ECT来对结果进行评估。
新的CESM-POP模拟数据可能来自一个新的机器的移植, 一个新的编译器选项的使用,代码的更新,或者输出数据的改变。
pyCECT评估新的海洋模式模拟结果是否与给定的POP-ECT生成的数据集合是统计意义上一致的,并且从总体上给出一个“通过”或者“不通过”的判断。
另外,在指定的检查时间点$t_C$,每一个变量的标准分通过率都会被给出。 

% %-----------------------------------------------------------------------------
% % -----------------------------------------------------------------------------
% %----------------------------------------------------------------------------
\section{实验} \label{sec:exp}

 
这一节的主要目标是通过一 系列的实验来评估我们的新的POP一致性检测工具在CESM-POP的模拟数据上的效果。
我们的这些实验都是有预判结果的,比如说重新考察一下改变正压求解器的收敛条件所带来的效果。
这些实验都是用的CESM 1.2.2版本来运行,使用CESM-POP作为活跃的海洋模式分量,CICE作为活跃的海冰分量,而大气和路面模式都采用数据来驱动。
因此,不会出现逐年响应的事件(比如厄尔尼诺南方涛动),而且赤道太平洋地区的变化可能会受到人为的抑制。
我们这个特殊的分量模式的配置在CESM的官方文档中被记为“G\_NORMAL\_YEA”分量设置。
CESM的网格分辨率采用“T62\_g16”,对应的海洋和海冰模式分量采用是1度($320 \times 384$)分辨率,垂直方向上有60层,以格林兰岛为偏移极点的偏移网格。 
这些模拟都是在美国国家大气研究中心的黄石超级计算机上使用$96$ 个处理器核心(除非特殊说明了处理器核心的数目)运行得到的。 

 对于这些模拟实验,我们不单单是对某一个时间片做了评估,而是对36个月的数据都进行了评估,以此来观察ZPRs随着时间的变化规律,同时也为$t_C$的选择提供参考依据。
为了从ZPRs的角度来刻画标准分和模拟时间的关系,并且为$tol_{Z}$和$min_{ZPR}$的选择提供参考依据,我们采用了反应曲面法(RSM)\citep{box2007}。 
也就是说,我们提供了变量$\tilde{X}$的反应曲面的图标,这个图表中给出了不同的模拟月份中,满足一系列的标准分阈值条件$Z_{\tilde{x_i}} > tol_{Z}$的点的比例的关于给定阈值$tol_{Z}$的累积分布函数。
最后,如前面提到的那样,我们发现,40个模拟结果组成集合模拟对于我们的实验来说已经足够。
但是我们还是在第 \ref{sec:ens}节进一步的探讨了集合成员的个数的选择。  

为了简洁,尽管我们也分析了其他变量,我们只展示了温度(TEMP)和海表高度(SSH)的结果。 
这两个变量在海洋模式系统有很强的代表性。海表高度和海洋海流动力系统密切相关,而温度是由模型的标量传输决定的。
 

\subsection{正压求解器收敛性条件}

 
首先,我们先用我们新的加强版本的CESM-ECT工具来重新评估一下改变正压求解器收敛性条件所带来的影响。
CESM-POP中正压求解器默认的收敛条件是$10^{-13}$, 我们运行一些采用 $10^{-9}$ 到 $10^{-16}$之间值作为收敛条件的模拟,然后输出所有点上的36个月平均数据。
我们预测采用比默认的收敛条件$10^{-13}$更加严格的条件得到的模拟结果应该是与原结果在气候学上是一致的,但是更送一些的收敛条件将会引入一些误差。


 
图 \ref{fig:RSM-TEMP-tol}和图 \ref{fig:RSM-SSH-tol}分别给出了温度和海表高度的反应曲面。 
每张图包含有四个反应曲面:上面两个分别是采用原始默认收敛条件($10^{-13}$)和更严格的收敛条件得到的结果,下面两个是更松的收敛条件($10^{-10}$和$5.0*10^{-9}$)得到的结果。 
每一个反应曲面中,X轴表示模拟的月份(从1月到36个月),Y轴表示衡量标准分的分布的阈值$tol_{Z}$。这里阈值就是计算方程中ZPR所用的划定标准分满足某个条件的比例时所用到的阈值。
颜色刻度中ZPR采用的是以10\%为间隔的百分比。 
反应曲面对于评估$tol_{Z}$和$min_{ZPR}$的不同组合非常有效。
比如说,考虑变化求解器收敛条件对温度变量的影响。 图中左上角的 \ref{fig:RSM-TEMP-tol}子图显示,原始的收敛条件($10^{-13}$)下,在所有模拟月份中90\% 的点的标准分都小于2.0。
与之相对的,下面的采用为$10^{-10}$收敛条件的子图表明,在第九个模拟月份之后,90\% 的点的标准分要小于3.0, 而在12个分,只有百分之七十到八十左右的网格点的标准分要小于2.0. 
右下角的子图采用的收敛条件进一步放松 到$5.0*10^{-9}$,它的比较低的ZPR比例表明这个例子中引入较大的误差。
我们继续考察图\ref{fig:RSM-SSH-tol} 中关于海表高度的反应曲面,四个子图分别表示与图\ref{fig:RSM-TEMP-tol}中相同的四个收敛条件,可以看到总体趋势与温度的反应曲面很相似。 
对于采用$10^{-13}$为收敛条件的模拟结果中,在除了第六个月的所有模拟月份中, 90\%的点上标准分都要小于2.0。 
与温度变量的反应曲面类似, 收敛条件为$10^{-10}$ 对应的子图可以出来有明显的误差,这个误差在收敛条件为$5.0*10^{-9}$对应的例子中更加明显。 
温度和海表高度的反应曲面中有一个明显的不同就是,温度得到的曲线随着时间的变化更加光滑。 这主要是由于温度的计算中有一个很很要的扩散过程。

 

如果我们在图\ref{fig:RSM-TEMP-tol}和图\ref{fig:RSM-SSH-tol}中固定标准分的衡量阈值,我们就可以很容易的评估ZPR。 
假设我们采用一个比较保守的选择$tol_{Z} \; = \; 3.0$。 
图 \ref{fig:PRZ-tol}给出了温度和海表高度两个变量中,标准分超出阈值 $tol_{Z} \; = \; 3.0$ 的点的百分比。 
如果我们选择ZPR的阈值为$min_{ZPR} \; = \; 0.9$, 也就对应着图\ref{fig:PRZ-tol}中10\%的不通过率,我们可以很明显的看出收敛条件为$10^{-10}$的例子就处于我们定义通过与不通过的边界线上(因此,这个收敛条件值在实际模拟中不能使用)。 
而采用比$10^{-10}$更严格的收敛条件得到的结果的不通过率更低一些,因此,看起来它们的两个变量跟原始结果所得到的变量在统计学上时一致的。
图\ref{fig:PRZ-tol}中的图标很明晰的证明了,随着收敛条件的放松,超出给定标准分阈值的点会增多。 
这个结果比文章\cite{yong2015}中的结果更加清楚。 


\subsection{进程数}

 
CESM模拟中,当采用相同的配置,仅仅改变CESM-POP运行的核数时,都会产生不二进制一致的的结果,但是这些模拟结果都应该代表着相同的气候状态(也就说他们在统计学上是不可区分的)。
所以这里我们确信,这样的模拟结果肯定能通过CESM-ECT的测试。 
这里再次申明一下,CESM-ECT集合模拟中的成员都是在96个处理器核心上运行得到的。
我们另外分别在48,192和384核上进行了同样的模拟。
这里,我们并没有使用CESM-POP中的线程并行。 


 
图\ref{fig:RSM-TEMP-param}和图 \ref{fig:RSM-SSH-param} 中上面两个子图分别对应的是96核(标记为“original”)和384核上运行结果中温度和海表高度得到的反应曲面图。 
这些图显示,这两种核数的配置下, 对于几乎所有的月份,90\%以上的点的标准分都要比2.0要小。 
正如我们所料想的那样,对于这两个变量,这两种不同的核数的配置得到的结果的差别是微乎其微的。
像之前的做法一样,我们将标准分的阈值固定到$tol_{Z} \; = \; 3.0$时,就可以得到 图 \ref{fig:combine}中的温度变量在四个不同的核数配置 (48, 96, 192,和384) 下的结果的标准分的不通过率。
正如我们预计的那样, 这四个核数的配置在所有的时间片的不通过百分比都很低(低于1.2\%),这也确认了我们之前的结论,即改变CESM-POP的运行核数对结果造成的影响在统计学意义上并不是显著的,并且这一特征被我们新的CESM-ECT给准确的判断出来。 
由于海表高度对应的图片放映出来的很低的不通过率跟这张图很相似,所以我们就不重复给出了。

 \subsection{物理参数}\label{sec:pp}
 
接下来我们考察了一下预计能够对海洋气候造成改变的例子。我们改变示踪物的两个物理参数 :一个是垂直混合中对流不稳定系数,另一个是示踪物的平流方案的系数。
这样改变物理参数的改动应该是不能通过CESM-ECT的测试。
对于第一个物理参数,在默认情况下,1度的CESM-POP的示踪物混合系数的配置中将对流不稳定的垂直混合系数(\textit{convect\_diff})设为\textit{convect\_diff} $=\; 10,000$。
我们将这个参数分别放大2,5,和10倍,以期增大海洋内部在密度分层不稳定的时候垂直混合的强度。
这应该对对CESM-POP的结果造成很大的影响,因为对应的混合特性上的改变。
第二个,我们将POP中示踪物的平流方案(\textit{t\_advect\_ctype}) 的默认配置三阶迎风格式(\textit{upwind3})替换成带有一维通量限制的Lax-Wendroff格式 (\textit{lw\_lim})。
这种改变对于模拟结果同样显著,并且能够导致一个完全不同的气候状态,因为相应的扩散和弥散误差变得不同。
 
图 \ref{fig:RSM-TEMP-param}和图 \ref{fig:RSM-SSH-param} 的左下方的子图中分别给出了温度和海表高度在将\textit{convect\_diff}增加十倍之后得到的反应曲面。 这种改变对气候状态造成的影响是很明显的,尤其是和右上角的将CESM-POP所运行的核心数改变为384后得到的反应曲面相比。
实际上,图  \ref{fig:RSM-TEMP-param} 中增加\textit{convect\_diff}得到的效果几乎和图\ref{fig:RSM-TEMP-tol}中将求解器的收敛条件放松至$10^{-9}$得到的效果一样强。
改变平流方案同样会得到不同的气候态, 图 \ref{fig:RSM-TEMP-param}和图 \ref{fig:RSM-SSH-param} 的右下图中有明显的表现,可以看出几乎所有的点都没有通过 标准分检测。 



   
图 \ref{fig:PRZ-temp-param} 中给出了改变平流方案和示踪物垂直混合的对流不稳定性系数所得到的结果中,在$tol_{Z} \; = \; 3.0$ 时标准分的不通过率。
如果我们选择$min_{ZPR} \; = \; 0.9$作为ZPR的阈值,也就是对应了允许的最大不通过率为10\% , 将垂直混合系数增大一倍(\textit{convect\_diff}*2) 所得到的结果介于通过临界线上。
剩下的例子中,如我们所料,温度和海表高度都没有通过测试。 
基于以上的实验,我们选择$tol_{Z} \; = \; 3.0$作为标准分阈值,$min_{ZPR} \; = \; 0.9$作为ZPR阈值,就可以得到我们想要的结果。
这两个配置被当做pyCECT工具的默认配置。 


 \subsection{模拟时间}
 
图\ref{fig:PRZ-temp-param}可以看出来,对于温度和海表高度这两个变量, 我们的实验中不通过标准分测试的点所占的百分比在12个模拟月之后变化很小
这个结论也可以从图 \ref{fig:RSM-TEMP-param}和图 \ref{fig:RSM-SSH-param}中的温度和海表高度的结果中的出来。 
特别的,海表高度对于初始温度场的扰动的相应在12个月份后被很大的稳固了下来。 
海表高度会受到由于密度分成所导致的环流的改变所带来的影响。
给予我们的实验结果可以看出,在某一个精心挑选的时间点$t_C$上对输出结果做评估是比较合理。 

  
通常我们选择$t_C \;=\; 12$作为评估的时间点,也就是说集合模拟只需要运行长度为$t_C$的时间, 这样可以减少为可能的CESM版本生成集合模拟的计算开销。
 图 \ref{fig:zscore-combine}中描述的是在$t_C \;=\; 12$时刻,四种不同的模式配置得到的结果中的海表温度相对于数据集合的标准分的二维平面图。
 最上面的是原始配置得到的结果。第二幅是改变CESM-POP的处理器核数到384后的结果,它和第一幅的结果相似。
 尽管前面两张图中的分布模式并不是完全相同,但是标准分的量级和分布都很相似的。这表明改变计算所使用的处理器核心数,得到的结果在一定程度上是统计学一致的。
 相反的,增大示踪物的对流不稳定新的混合系数道原来的10倍,就会得到与节\ref{sec:pp}中不同的气候状态。
 这个结论在图 \ref{fig:zscore-combine}第三幅图中,第12个月个标准分布可以很明显的看出来。 
 最后,图中底部的子图可以看出来使用一个新的平流格式会导致得到的气候状态完全不同,这也证实了图 \ref{fig:RSM-TEMP-param}和图 \ref{fig:RSM-SSH-param}中所展示的明显的影响。 
 采用一个新的平流格式会极大的改变数值格式相应的弥散和扩散过程\cite{tseng2008},同时也会影响到海洋模式中海流的特征和结构\cite{tseng2006}。
 另外,使用通量限制的Lax-Wendroff会引入额外的数值混合,进而与温度和盐度的物理混合产生相会作用,导致最终的结果通常会更加的光滑一些。



% %-----------------------------------------------------------------------------
% % -----------------------------------------------------------------------------
% %----------------------------------------------------------------------------
\section{Ensemble size } \label{sec:ens}

 
数据集合的规模(也就是模拟成员的个数)必须要足够的大才能够捕捉到海洋模式自身的不确定性,但是从计算效率的角度来说又必须尽可能的小。
这一节中,我们将会讨论一下POP-ECT对于数据结合规模的敏感性。
我们通过以下的实验手段,来观察误判率与数据集合的规模之间的关系。 
首先,我们产生了总共80个的集合模拟的成员,这些模拟只是在初始的海洋温度场中加入$\mathcal{O}(10^{-14})$量级的随机扰动。 
然后,在这80个成员中,我们随机的拿出其中的10个作为我们的测试样例。 
下一步,从剩下的70个模拟结果中,我们分别构造出包含有10, 20, 30, 40, 50,和60个成员的数据集合。 
这里,对于给定的数据集合的大小,我们从那70个模拟结果中随机的抽取100次给定数目的结果作为数据集合的成员。
这样对于每一个给定的集合大小,我们就有100个互不相同的数据集合。 
然后,对于每一个数据集合大小,我们使用POP-ECT在第$t_c = 12$月份对拿出来的10个测试样例分别基于这个集合大小的所有的100数据结合做评估。
这样我们得到了每个集合大小的1000个测试结果。
我们利用第一类误差,也就是“假阳性”率来衡量实验的错误率。 
由于测试集和数据集合的成员都是从那80个有着统计学上一致的气候状态的模拟结果集中抽取出来的,标准分不通过率在理想情况下越低越好。 

图 \ref{fig:temp_ens_80}中给出了这些实验在温度和海表高度这两个变量上的结果。 
X轴表示的是数据集合的大小,Y轴表示的标准分的不通过率。对于每一个集合大小,中间的方块表示的是平均值,误差条表示的是一个标准差的不确定性。
正如我们所料,随着数据集合规模的增大,假阳性率减少,不确定性的范围也相应的缩减。 
但是,一直增加数据集合的规模所带来的效果是不断减少的,数据集合的大小从10增大到20时假阳性率上的改善要比从50增大到60的时候在假阳性率上的改善要大很多。 
所以我们选择数据集合的大小为40,因为40在比较低的误判率和比较低的生成数据集合的开销之间是一个比较好的平衡。 
 

 
 \conclusions[Conclusions和Future Work]\label{sec:concl}

 
由于CESM-POP海洋模式被广泛的使用,并且对于很多的气候模拟都有非常重要的作用,
保证它的代码的正确性是至关重要的。 
但是,由于海洋动力学本身的不确定性,模拟结果常常由于处理器核数的改变等微小的因素而导致最后的模拟结果并不能做到完全一致。
因此,对于气候科学家和模式的开发人员来说,能够简单易行的评判模拟结果是否是统计学上一致是极其重要的。
由于特征不同,已有的评估大气模式数据的基于集合模拟的方法CAM-ECT对于海洋模式的模拟数据并不适用。 
因此,我们开发了一个新的针对海洋模式的统计学一致性检测工具POP-ECT。
这个工具能够客观的检测出CESM-POP中的统计学显著的改动。 
集合模拟的大规模的确定,对于使用这个新的方法也是极其重要的。
我们的实验证明了这个新方法在检查海洋模式状态中的误差时是非常有效的。 
POP-ECT 极大地提升了CESM-ECT工具集在确保CESM模拟质量的能力。

我们计划将这个工作在一下几个方面进行扩展。 首先,我们的方法可以在区域海洋上进行检测。比如说,当一个模拟结果没有通过整体测试时,这个测试工具可以找出是哪个还去造成的测试不能通过。 
这种从粗粒度到细粒度的诊断使得我们能够找到造成错误或者误差的根源。 
其次,我们打算想文章 \inlinecite{baker2014}一样,利用这个工具去评估一下数据压缩在海洋模式中的影响。
这种评估压缩过程中数据精度丢失是否会造成最终的结果在统计学上不一致的能力对于压缩方法能否被接受为减少海洋模拟数据的数据量来说至关重要。 


% % -----------------------------------------------------------------------------
% % ----------------- Section Code--------------------------------------
% % -----------------------------------------------------------------------------

  \section{代码可用性}\label{sec:code}

  CESM-ECT Python工具集 (pyCECT v2.0)可以从与美国国家大气研究中心的CESM独立的开源git仓库得到。 (\url{https://github.com/NCAR/PyCECT/releases}).  试验中用的 CESM 1.2.2版本可以参见\url{http://www.cesm.ucar.edu/models/cesm1.2}. The CESM-ECT在CESM的代码中也可以找到,但是本文中提到的POP-ECT工具将在下一个版本的CESM 2.0中才会被加进来。 
  , with the POP-ECT addition available starting with the CESM 2.0 release series. 
