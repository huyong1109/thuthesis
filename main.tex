\documentclass[type=master]{thuthesis}
% 选项:
%   type=[bachelor|master|doctor|postdoctor], % 必选
%   secret,                                   % 可选
%   pifootnote,                               % 可选(建议打开)
%   openany|openright,                        % 可选,基本不用
%   arial,                                    % 可选,基本不用
%   arialtoc,                                 % 可选,基本不用
%   arialtitle                                % 可选,基本不用

% 所有其它可能用到的包都统一放到这里了,可以根据自己的实际添加或者删除。
\usepackage{thuthesis}
\usepackage{gensymb}
\usepackage{algorithm} %format of the algorithm 
\usepackage{algorithmic} %format of the algorithm 

\floatname{algorithm}{算法}
\renewcommand{\algorithmicrequire}{\textbf{输入:}}
\renewcommand{\algorithmicensure}{\textbf{输出:}}
\renewcommand{\algorithmiccomment}[1]{ \hfill {/* #1 */} }

% 定义所有的图片文件在 figures 子目录下
\graphicspath{{figures/}}

% 可以在这里修改配置文件中的定义。导言区可以使用中文。
% \def\myname{薛瑞尼}

\begin{document}

%%% 封面部分
\frontmatter
\thusetup{
  %******************************
  % 注意:
  %   1. 配置里面不要出现空行
  %   2. 不需要的配置信息可以删除
  %******************************
  %
  %=====
  % 秘级
  %=====
  %secretlevel={绝密},
  %secretyear={2100},
  %
  %=========
  % 中文信息
  %=========
  ctitle={高分辨率海洋模式可扩展的\\正压求解方法研究},
  cdegree={工学博士},
  cdepartment={计算机科学与技术系},
  cmajor={计算机科学与技术},
  cauthor={胡勇},
  csupervisor={杨广文教授},
  %cassosupervisor={黄小猛副教授}, % 副指导老师
  %ccosupervisor={某某某教授}, % 联合指导老师
  % 日期自动使用当前时间,若需指定按如下方式修改:
  % cdate={超新星纪元},
  %
  % 博士后专有部分
  cfirstdiscipline={计算机科学与技术},
  cseconddiscipline={系统结构},
  postdoctordate={2009年7月——2011年7月},
  id={编号}, % 可以留空: id={},
  udc={UDC}, % 可以留空
  catalognumber={分类号}, % 可以留空
  %
  %=========
  % 英文信息
  %=========
  etitle={Research on Scalable Barotropic Solver \\ in High Resolution Ocean Models},
  % 这块比较复杂,需要分情况讨论:
  % 1. 学术型硕士
  %    edegree:必须为Master of Arts或Master of Science(注意大小写)
  %             “哲学、文学、历史学、法学、教育学、艺术学门类,公共管理学科
  %              填写Master of Arts,其它填写Master of Science”
  %    emajor:“获得一级学科授权的学科填写一级学科名称,其它填写二级学科名称”
  % 2. 专业型硕士
  %    edegree:“填写专业学位英文名称全称”
  %    emajor:“工程硕士填写工程领域,其它专业学位不填写此项”
  % 3. 学术型博士
  %    edegree:Doctor of Philosophy(注意大小写)
  %    emajor:“获得一级学科授权的学科填写一级学科名称,其它填写二级学科名称”
  % 4. 专业型博士
  %    edegree:“填写专业学位英文名称全称”
  %    emajor:不填写此项
  edegree={Doctor of Philosophy},
  emajor={Computer Science and Technology},
  eauthor={Hu Yong},
  esupervisor={Professor Yang Guangwen},
  %eassosupervisor={Huang Xiaomeng},
  % 日期自动生成,若需指定按如下方式修改:
  % edate={December, 2005}
  %
  % 关键词用“英文逗号”分割
  ckeywords={海洋模式;并行数值算法;可扩展性;预处理;正确性验证},
  ekeywords={Ocean models; parallel numerical method; scalability; preconditioning; verification}
}

% 定义中英文摘要和关键字
\begin{cabstract}
  高分辨率气候模拟的需求与日俱增,所消耗的计算资源也越来越多。 
  被广泛使用的公共地球系统模式(CESM)中, 并行海洋模式(POP)在高分辨网格的配置下的计算量极其巨大。
  在很多真实运行的例子中,POP都是CESM中可扩展性最差的一个分量模式。 
  POP的正压模态中求解椭圆方程所用到的共轭梯度法(CG),是高分辨率海洋模拟的一个主要瓶颈,在大核数并行时可扩展很差。
  此论文中,我们首先通过一个性能评估模型,分析了POP正压求解器CG的可扩展性,并通过实验验证了正压求解器中的性能瓶颈是CG方法每一步迭代中的残差计算所引入的全局通信。 
  为了解决这一问题, 我们在POP中实现了一个基于预处理的Chebyshev迭代方法(P-CSI)的求解器。
  P-CSI不需要利用每一步的残差来确定迭代方向,而是利用系数矩阵的最大最小特征值,
  从而去掉了时间开销很大的全局通信操作,极大的改善了正压模态的可扩展性。

  为了进一步提高POP中正压模态的性能,我们研发了一个高效的基于误差向量传播方法的并行块预处理子,以实现P-CSI更快地收敛。 
  误差向量传播方法(EVP)是十分高效的求解由椭圆偏微分方程得到的线性方程组的方法。我们利用海洋模式并行划分的特点,在每一个进程的数据块上使用EVP方法求解。这相当于使用原始方程的块对角矩阵的逆作为预处理子。
  这种预处理方法使得正压求解器的迭代步数缩短到了原来的三分之一左右。 
  通过采用新的P-CSI方法和误差向量传播并行块预处理子,正压求解器的可扩展性得到极大的改善。 在16,875核上,新的正压求解器在高分辨POP中取得了5.2倍的加速,并且使得整个POP的模拟速率提高了1.7倍。 
  

  最后,我们通过一个基于集合模拟的统计学方法证明了我们的新的求解器不会造成模拟结果与原始结果不相容。 
  由于模式本身的不确定性,在模式中加入很小的改动都无法保证模式结果与原来的结果是二进制一致的。
  这也就使得验证模式中的改动是否会对模拟结果造成气候意义上显著的改变十分困难。
  我们提出了使用集合模拟的数据集合来对新的结果进行评估,解决了这一难题。 数据集合能够很好地反应出模式结果的不确定性分布,我们最终得到的一致性检验工具能够很好评估新的模拟结果是否与给定的集合模拟的结果相容。  
  这也使得我们的新的求解器最终被公共地球系统模式采纳为默认的海洋模式正压求解器。

\end{cabstract}

% 如果习惯关键字跟在摘要文字后面,可以用直接命令来设置,如下:
% \ckeywords{\TeX, \LaTeX, CJK, 模板, 论文}

\begin{eabstract}
  High-resolution climate simulations are increasingly in demand and
  require tremendous computing resources. 
  In the Community Earth System Model (CESM), the Parallel Ocean Model (POP) is
  computationally expensive for high-resolution grids (e.g., 0.1 degree) and is frequently the least scalable component of CESM for certain
  production simulations. 
  In particular, the Preconditioned Conjugate Gradient (PCG), used to solve the elliptic system of
  equations in the barotropic mode, scales poorly at the high core counts, which is problematic for high-resolution simulations. 
  In this work, we demonstrate that the communication costs in the
  barotropic solver is the bottleneck by both theoretical evaluation and experiments. 
  To mitigate this problem, we implement a preconditioned Chebyshev-type iterative
  method in POP (called P-CSI), which requires far fewer global
  reductions than PCG, thus breaking the scaling bottleneck in the barotropic solver. 


  To further improve the performance, we also develop an effective block preconditioner based on the Error Vector Propagation Method (EVP), which improve solver convergence in the POP barotropic mode.
  We demonstrate that the P-CSI and EVP preconditioning results in a 5.2x speedup of
  the barotropic mode in high-resolution POP on 16,875 cores, which
  yields a 1.7x speedup of the overall POP simulation.  

  Further, we ensure that the new solver produces an ocean climate consistent with the original one via an ensemble-based statistical method. 
  Due to the chaotic nature of the ocean dynamics, even a round-off difference from the barotropic solver may potentially result in distinct model solutions. Therefore, it is hard to verify whether a given result is consistent with the original one. 
  To verify the results of our new solver, we quantify the variability of the ocean model by the statistical distribution of an ensemble and verify the new result based on these distributions. 
  The resulting verification tool represents good performance in dectcting inconsisitency, and verifies the implementation of our new solver into CESM. 



\end{eabstract}

% \ekeywords{\TeX, \LaTeX, CJK, template, thesis}

% 如果使用授权说明扫描页,将可选参数中指定为扫描得到的 PDF 文件名,例如:
% \makecover[scan-auth.pdf]
\makecover

%% 目录
\tableofcontents

%% 符号对照表
%\begin{denotation}[3cm]
\item[IPCC] 政府间气候变化专门委员会 (Intergovernmental Panel on Climate Change)
\item[CESM]  公共地球系统模式(Community Earth System Model)
\item[NCAR]  美国国家大气研究中心(the National Center for Atmospheric Research)
\item[POP]  并行海洋模式(Parallel Ocean Program)
\item[CESM-ECT]  CESM一致性检测工具
\item[POP-ECT]  POP一致性检测工具
\item[CAM]  公共大气模式(Community Atmosphere Model)
\item[CG]	共轭梯度法(Conjugate Gradient method)
\item[PCG]  预处理共轭梯度法(Preconditioned Conjugate Gradient method)
\item[CSI]  传统 Stiefel 迭代算法(Classical Stiefel Iteration)
\item[P-CSI]  预处理的CSI方法
\item[ChronGear] Chronopoulos-Gear方法(PCG方法的一个变种)
\item[EVP] 误差向量传播方法(Error Vector Propagation method)
\item[RMSE] 均方根误差(Root Mean Squared Error)
\item[RMSZ] 均方根标准差(Root Mean Squared Z-score)
\end{denotation}



%%% 正文部分
\mainmatter
\chapter{绪论}
\label{cha:intro}

\section{气候变化}

近年来,气候变化不仅是气候研究领域的热门,也是新闻媒体和人们日常谈论的热点话题。
上个世纪以来,科学家们通过观测发现,地球气候发生了很多重大改变,如大气中二氧化碳浓度的不断增加,全球生态氮循环的改变,以及陆面的分布和使用的改变等\cite{vitousek1994beyond}。
全球气候的变化,不仅影响着水资源、能源的供应和使用、交通运输、农业和公共健康等与人类日常生活息息相关的方方面面\cite{karl2009global},而且改变着整个地球的生态系统,进而间接的影响到人类的生存环境。



科学研究表明,未来的气候变化继续朝着全球变暖的趋势发展的可能性很大\cite{solomon2007climate,stocker2013ipcc}。
全球气候变暖会从多个方面影响到人类的生存。
首先,全球气候变暖的最直接的影响是导致冰川、动土和海冰的消融,进而导致的海平面的上升,以及亚热带沙漠地区面积的扩大等,这些改变都会直接威胁到人类的生存\cite{stocker2013ipcc}。
其次,气候变暖还会增加热潮、干旱、暴雨、暴雪等极端天气发生的频率,极大的危害到人类日常的生活和生产\cite{solomon2007climate}。
最后,随着全球气候的不断变暖,农产品收获季节温度的不断升高,玉米和大豆等主要农作物的生长周期变短,影响农作物的产量, 进而威胁到农场的收成,甚至引发全球的粮食安全问题\cite{battisti2009historical,adams1990global,smith1989potential}。 

人类的活动与气候变化之间有着密切的联系和相互作用。
一方面,人类的活动,比如使用氮肥、燃烧化石燃料等对全球氮循环带来的巨大的影响。这些影响甚至超过了自然过程中的影响,进而对生物多样性造成了不可估量的影响。
过去的60年里,人类对自然生态系统造成的改变要比过去任何时期的都要剧烈和广泛。很多证据表明,人类化石燃料的燃烧,使得大气中的二氧化碳浓度从1800年的280微升每升增加到二十世纪初的355微升每升\cite{vitousek1994beyond}。
另一方面,二氧化碳等温室气体浓度的升高引发的全球气候变暖,不仅改变了地球的生态系统,也影响着人类的生存环境。
生态环境为人类所提供的“服务”中,包括空气质量的改善和水的净化等,60\%左右不是被破坏,就是无法可持续性的发展\cite{assessment2005ecosystems}。



气候变化同时也给地球上的生态环境造成重大影响。
已有研究表明,气候的变化给全球生态系统带来了很多重大改变,全球植被的分布和生命周期正在逐年改变\cite{parmesan2003globally}。
过去三十年中,气候变化对物种的分布和多样性造成了很多重大的影响,甚至导致了很多物种层面的灭绝。据预测,到2050年,当前人类资料库中的15–37\%左右的物种都会面临灭绝 \cite{thomas2004extinction} 。
海洋温度的升高,极大的改变了海洋生物的分布。研究显示1980年到2005年间,北海中大约有三分之二的种群迁移到了中纬度地区或者更深的海域\cite{perry2005climate}。
未来温度的继续升高,可能会导致物种分布上的继续偏移,进而通过种群之间的相互作用,对渔业造成深远影响。
气候变化已经导致地球上很多区域的物种分布发生了变化。 
有研究预测,2080年,欧洲一半以上的植物物种将会受到威胁\cite{thuiller2005climate} 。

 
正是由于气候变化跟人们的生活息息相关,全球各个国家都在积极的研究气候现象,预测未来气候变化趋势,以应对气候变化可能带来的影响。 
随着科技的发展,人们不满足于对气候变化规律的解释,而是更加重视利用计算机来对气候系统进行建模,进而预测未来的气候。
近几年,计算机科学的发展也促进了气候模式的应用和发展。  



 
\section{气候系统模式}

为了促进对全球气候现象的理解,增强对未来气候变化的预测能力,科学家们通过数学模型来对地球上的大气和海洋等气候系统进行模拟,并逐渐发展成为相应的气候模式。
气候模式是用来研究气候系统对各种强迫的响应的基本工具,它被用来进行短的如季节性、长如几十年甚至几个世纪的时间的气候预测。 
借助计算机强大的计算能力, 气候模型能够整合很长的时间尺度内全球范围内的物理、化学和生态等过程以及它们之间的相互作用,促进人们对地球气候系统的理解\cite{hurrell2013community}。


气候模式最初是模拟大气、海洋、海冰等相对独立的气候系统。随着观测数据的丰富以及模拟能力的增强,后来逐渐出现了耦合气候系统模式。耦合气候系统模式是为了更加真实的模拟地球上大气、海洋、陆地等相对独立的气候系统之间的相互作用,而将这些的气候模式通过耦合器连接起来\cite{hurrell2013community, liu2014c}。
早期的耦合模式主要是大气海洋环流模式。他们的基本功能是去理解大气、海洋、陆地和海冰等气候系统中的动力过程,并且基于未来温室气体和气溶胶强迫等的假设进行气候预测。
在2007年的政府间气候变化专门委员会(the
Intergovernmental Panel on Climate Change, IPCC) 第四次评估报告(the Fifth Assessment
Report, AR4)中,大气海洋环流模式是比较通用的气候系统模式\cite{solomon2007climate}。
近年来,在大气海洋环流模式的基础上,科学家们又扩展了多个生物地理化学循环圈, 比如碳循环、硫循环和臭氧等,进而发展成为地球系统模式。 地球系统模式是目前气候模式比较前沿的一种形式,
这些模式为模拟过去、现在和未来的气候系统对于外界强迫的响应提供了一个全面并且系统的工具。 

人们对气候变化的关注也促使了全球各个国家积极的发展自己的耦合气候模式。
在2013年的政府间气候变化专门委员会第五次评估报告(IPCC AR5)中,有来自13个国家近四十个模拟结果,
其中包括中国国家气候中心和北京师范大学各提交的一组结果,中科院大气物
理研究所大气科学和地球流体力学数值模拟国家重点实验室(LASG)提交的两个结果\cite{stocker2013ipcc}。



\subsection{公共气候系统模式}
 

 
目前,以美国国家大气研究中心(the National Center for Atmospheric Research, NCAR) 主导开发的
公共地球系统模式(the Community Earth System Model, CESM) 是应用最为广泛的全球气候模式之一\cite{hurrell2013community}。
在政府间气候变化专门委员会第五次评估报告(IPCC AR5)中,直接或间接使用公共地球系统模式的预测结果占了相当大的比例\cite{stocker2013ipcc}。
 

公共地球系统模式是一个全耦合的气候系统模式,它包括了大气、海洋、海冰和陆面等分量模式。
公共地球系统模式的海洋模式和海冰模式分量分别采用的是美国Los Alamos 国家实验室研发的并行海洋模式(The Parallel Ocean Program,POP)和公共海冰模式(The Community Ice CodE,CICE)改进后的版本\cite{hurrell2013community}。
POP是一个基于原始方程的分层模型,默认设置中将海底分成的60层,分层的厚度从海表处的10米处到深海处的250米不等。
CICE是一个描述海冰的二维模型,它通常采用的网格与POP中使用的水平网格一致,并且有一个正交的垂直维度来表示海冰的厚度。 
大气模式分量和陆面模式分量分别采用的是美国国家大气研究中心主导开发的公共大气模式(The Community Atmosphere Model, CAM) 和公共陆面模式(The Community Land Model,CLM)。 
CAM描述的是由地球表面到约140千米高空之间的大气层。它包括了大气的动力过程和物理过程,以及与他们紧密耦合的大气化学过程等。
CLM主要是为了研究陆面生态系统在物理、化学和生态等过程中与气候在不同的时间和空间尺度上的相互影响。
它模拟的主体是陆面生态系统中能量、水、化学元素和尾气等循环,这些都是很重要的气候决定因素。
陆面是气候与人类和生态系统之间一个重要的桥梁,人类和生态环境正是通过陆面改变着全球环境。
这些分量模式通过一个耦合器(CPL)协调和控制他们之间的二维边界的数据交换\cite{liu2014c}。

 

\subsection{气候系统模式的发展趋势}

为了更精准的描述大气海洋等的运动规律,更准确的预测未来气候变化,气候系统模式都朝着更高分辨率、更多分量模式和更多物理过程的方向发展\cite{stocker2013ipcc}。
近年来,高分辨率全球气候模式已经成为了理解气候现象和预测未来气候变化的一个不可或缺的手段。
更高分辨率使得能够模拟出小尺度运动,如涡旋和海浪。 
很多使用高分辨率全球气候模式进行模拟的研究表明,提高模式的分辨有助于更好的模拟重要的气候过程,进而改善气候预测的结果。
提高分辨率可以极大的改善赤道不稳定波\citep{roberts2009impact}、厄尔尼诺南方涛动(ENSO) \citep{shaffrey2009uk}、墨西哥湾流\citep{chassignet2008gulf, kuwano2010precipitation} 、全球水循环\citep{demory2014role}等气候现象。
Gent等人\cite{gent2010improvements} 和Wehner等人\cite{wehner2014effect}通过比较不通分辨率的气候模拟的结果发现,提高大气模式的分辨率可以模拟出更好的气候平均态、更准确的赤道风暴的形成过程和更加真实的极端降水量。
Bryan等人\cite{bryan2010frontal}和Graham\cite{graham2014importance}的研究也证明,将海洋模式的分辨率提高到涡解析度水平能够捕捉到海表高度和海表风应力之间的正相关性,同时也可以改善厄尔尼诺南方涛动周期模拟中的不对称性。



气候系统模式从最开始相对独立的大气、海洋和陆面等分量模式,发展成为大气海洋环流模式,最后通过将众多的分量模式都耦合在一起形成了目前比较成熟的地球系统模式。
近几年,地球系统模式中又加入了多个生物地理化学循环圈, 比如碳循环、硫循环和臭氧圈等\cite{stocker2013ipcc}。
这些过程能够模拟陆面和海洋中生物碳氮等的循环过程,进而模拟陆面和海洋等生态系统,以及人类的活动等对气候的影响。
同时,也使得地球系统模式能够预测气候变化对于人类和生态系统可能造成的影响。



物理过程对于气候模式模拟结果的准确性是及其关键的。好的物理参数化方案能够提高模式的模拟结果,改善人们对气候过程的理解。 
当前的模式发展,主要是加入新得物理过程和改进已有物理参数化方案。 
在公共地球系统模式CESM在2013年最新发布的版本\cite{hurrell2013community}中,大气分量模式CAM,为了更加真实的模拟大气的运动规律,对热力学过程、云的分块、云颗粒的形成、气溶胶的形成和消失、气溶胶和云颗粒的辐射性质、辐射传输、对流和湍流等物理参数化进行了改进。
海洋分量模式POP中,加入了丹麦海峡、法罗银行渠道和罗斯海等满溢过程,混合层的中小尺度的涡旋现象等的物理过程, 同时还对混合层和深海绝热层,以及他们之间的过渡层中的中尺度蜗旋的物理参数化方案进行了改进。
 

\subsection{气候模式在高性能计算中的挑战}

目前气候模式的发展趋势是耦合更多的分量模式,每个分量模式中不断的提高分辨率、并且加入更多的物理过程\cite{stocker2013ipcc}。
越来越高的分辨率和越来越多的物理过程都对高性能计算提出更高的要求。 
在第六次国际耦合模式比较计划(CMIP6)高分辨率模式比较计划(HighResMIP)中,有采用中纬度25公里分辨率甚至更高的全球模式申请实现Tier-1和Tier-2实验。 
由于所有参与CMIP6的气候模式通常需要运行数百年,这些高分辨率的模式需要使用巨大的计算资源, 以至于这些模拟的计算消耗及其之大。
以POP为例,0.1度分辨率的海洋模式需要在3亿多个网格点上进行演算。而且,
模式中分辨率提高十倍,计算开销将增加到一千倍以上。目前有很多研究工作都是在关注气候系统模式的性能\cite{Worley:2011:PCE:2063384.2063457,dennis2012computational}。 
各个气候模式中,海洋模式需要模拟的气候现象的时间尺度很大。 
海洋细粒度涡旋的空间尺度只有$\mathcal{O}$(10 - 100 km), 比与之在动力学角度上相对应的大气中的天气系统的尺度要小一到两个量级。
而海洋环流的空间尺度则与整个地球的大小在一个量级, 与大气的运动尺度相当。



更为重要的是,气候模式通常需要模拟几十年甚至数百年。如此长期的模拟需求使得气候模式在采用高分辨率时的计算开销在目前的计算资源下完全无法承受。比如,在政府间气候变化专门委员会第五次评估报告中,大部分的公共地球系统模式的模拟都是采用的近似1度的海洋模式,组合以1度到2度的大气模式。
近几年超级计算机资源的增长以及高分辨卫星观测数据的出现促进了很多科研工作者研究如何通过提高高分辨率气候模式的并行计算效率,进而减少高分辨率模式的计算开销。
在时间尺度上,海洋模式需要模拟的气候现象也要比其他模式的大很多。 深海环流的时间尺度长达数个世纪甚至几千年,比大气中相对应的时间尺度要长几个量级。
对更高分辨率和更长模拟时间的要求,使得全球涡分辨的海洋气候模式的计算开销要比大气模式的开销大出很多\cite{bryan2010frontal,mcclean2011prototype,graham2014importance}。

正是由于气候模式对 高性能计算资源的需求十分巨大, 全球的的很多国家都为气候模拟制造了专门的高性能计算机。 比如日本2002年制造了当时最快的超级计算机“地球模拟器”专门来进行气候模式的模拟\cite{habata2003earth}。
美国国际大气研究中心,也为运行地球系统模式CESM建造了专门的超级计算机Yellostone超级计算机\cite{loft:2015}. 
国内清华大学也为进行气候模式而建造了“探索100”超级计算机。



尽管计算量巨大,目前的海洋模式并不能充分的利用已有的并行计算资源。比如针对美国地球系统模式CESM的测试表明,其中海洋模式分量的可扩展性较差,而且由于其计算量大,在整个地球系统模式中需要占用较大的计算资源。
为了能够更加常规化的运行这些高分辨的气候模式,需要我们进一步的进行算法优化,来充分的利用大规模的计算资源。
因此,解决海洋模式的并行问题,对于加速整个地球系统模式有着非常重要的意义。
我们的工作主要将集中在提高地球系统模式CESM中的可扩展性较差的海洋模式分量POP的并行效率。  
 

\subsection{气候模式的不确定性}
%========== 数值模式验证的重要性========
随着计算机计算能力的不断提高,数值模拟程序在解决实际问题和公共决策中应用得越来越广泛。 
这些数值模式的开发者和使用者,以及相关政策的制定者都会使用这些模式的模拟结果所呈现的信息。
所有受到这些数值模式影响的人都会关心这个模式和它的结果是否正确。
只有通过一定的模式验证和确认,才能解决相关人员的疑虑\cite{whitner1989guidelines,stainforth2005uncertainty}。

%========== 气候模式验证的重要性========
气候模式就是常见的一种数值模式,它能够方便地测试和验证人们在气候方面的发现,它模拟结果是公共决策中不可或缺的依据\cite{allen2002towards,reynolds1994random}。
因此,验证这些模型独立或集成后的模拟结果是至关重要的。
最直接的评估方法是利用这些模式的输出结果和观测数据相对比,并且分析相应的差别。 
这需要我们对模式和观测中的不确定性和误差有所了解。 

%========== 气候模式验证是不可能完全的========
但是,想要完全的验证和确认自然系统的数值模式是不可能做到的。这是因为自然系统本身不是封闭的,而且模式的结果也不是唯一确定的\cite{oreskes1994verification}。
有研究表明,在过去的几十年里,虽然模式在数量和质量上都得到长足的发展,但是气候变化预测的不确定性并没有减少。 
人们通过模式和观测的手段,在预测未来全球气候变化对于大气中二氧化碳浓度的长期相应时,给出的答案的概率分布非常广泛。
这表明,这些概率的分布是气候系统本质所带来的必然结果\cite{roe2007climate}。 

%数值模式如何应对不确定性
尽管无法验证模式结果的绝对正确性,但是可以做到的是寻求模式预测结果和观测结果的一致性。
通常,验证一个模式在给定问题的整个领域内都是正确的是一件非常费时费力的工作。 
因此,只有当一个模式被认为针对给定问题是有效的时候,才会对其正确性进行验证。 
模式应该是为了某一个特定的目的或者应用而开发的,它的验证也只能是与这个目的相关\cite{sargent2005verification}。
如果模式要回答一个很多样的问题,那么模式的验证就要针对每一个问题进行。
必须强调的是,这种验证是片面的,模式只能在相应的层面上得到评估,他们的预测结果本质上还是经验性的,也必然会面临一些问题的。


%气候学如何应对不确定性
在气候模式中,由于气候不确定性的本质,人们不再追求气候模式能够产生确定性的结果,
而是寻求定量的分析气候本身和气候模式的不确定性,从而结合模式模拟的结果给出相应的对于现实有意义的结论。
在政府间气候变化专门委员会的评估报告中,气候的敏感性一直是一个研究的重点。
2007年的政府间气候变化专门委员会第四次报告通过对气候平衡态的敏感性和瞬时气候响应进行量化,不仅仅考虑到它的变化范围,而且衡量一下这个范围内的概率。
这些概率的计算,不仅仅用到了专家的判断,还利用不同的观测条件得到的集合模拟。 
这使得我们有了一个比以前更加完整的模式相应不确定性的评估。 
这些现在被当成是全球耦合气候模式的标准测试集,可以用来评估后来的涉及到时间的气候变化情景实验\cite{meehl2007global}。 
 
%气候模式中定量研究的方法
目前,定量研究气候学最为有效的方法是利用集合模拟的结果作为评估依据\cite{von2013testing,reynolds1994random, allen2002towards}。
通过集合模拟得到的数据集合可以检查出某些结果所描述当前和未来状态与数据集合的统计学上的显著差别。
这种统计学的方法,最终可以给出集合模拟的一个概率分布区间,进一步可以得到模式不确定性的定量结果。
很多研究通过对不确定性进行定量的分析,显著的改善了模式的模拟结果\cite{reynolds1994random}。


\section{本文主要研究内容和主要贡献}
\subsection{主要研究内容}
本文主要对高分辨率气候模拟对计算资源消耗巨大这一热点问题进行深入研究,
逐步确定其中的瓶颈就是海洋模式的正压模态。
通过分析正压模态的通信瓶颈,我们提出了自己的解决方案,一个新的正压模态求解器。
同时,由于选择的新的求解器在收敛速度上略慢于原始的求解器, 我们需要设计一种高效的预处理方法。
我们通过分析传统预处理方法在海洋模式中性能不高的原因,找到了一种适合海洋模式的并行预处理子。 它利用椭圆微分方程上的高效求解方法--误差向量传播方法, 对每个进程所分得的数据块并行的做预处理。 
这种预处理方法极大的减少了新的求解器达到给定收敛条件所需要的迭代步数。
最后,由于模式的不确定性,我们必须首先确信新的求解器不会在统计学的意义上改变模拟的结果。 
我们采用集合模拟的手段,验证了我们求解器的正确性。 

\subsection{主要贡献}
我们的主要贡献可以总结为以下三个方面:
\begin{itemize}
 	\item 提出了基于Chebyshev迭代的海洋模式正压求解器。 新的求解器不需要利用每一步的残差来确定迭方向,而是利用系数矩阵的最大最小特征值。
    从而去掉了时间开销很大的全局通信操作,极大的改善了正压模态的可扩展性。
    \item 提出了基于误差向量传播算法的预处理子。误差向量传播算法是求解由椭圆偏微分方程十分高效的方法。
    我们利用海洋模式并行划分的特点,在每一个进程所得到数据块上使用EVP方法求解。这相当于使用原始方程的块对角矩阵的逆作为预处理。这种预处理方法使得正压求解器的迭代步数缩短到了原来的三分之一左右,进一步的提高正压求解器的性能。
    \item 提出了基于统计的模式评估方法。由于模式本身的不确定性,在模式中加入很小的改动都无法保证模式结果与原来的结果是二进制一致的。这也就使得很难验证模式中的改动是否会对模式造成气候意义上显著的改变。为此,我们提出了使用集合模拟的结果来对新的结果进行评估。 数据集合很好地反应模式结果的不确定性,我们最终得到的一致性检验工具能够很好从结果反推出模拟过程或者初始值所引入的误差。  
\end{itemize}
\subsection{主要组织结构}
本文其他章节结构如下:

第\ref{cha:related}章介绍了当前海洋模式中主要的性能优化,迭代求解器中常用的预处理方法以及气候模式中主要的验证方法。

第\ref{cha:barosSolver}章介绍了海洋模式正压模态的求解器和通信瓶颈。

第\ref{cha:precond}章介绍了本论文所提出的基于误差向量传播方法的并行预处理方法。

第\ref{cha:verify}章介绍了本论文所提出来的基于集合模拟的正确性验证方法。 

第\ref{cha:conclusion}章对整篇文论进行总结,并对未来进一步的工作进行展望。



\chapter{相关工作}
\label{cha:related}
本章主要介绍本文的正压求解器、迭代算法预处理技术和气候模式正确性验证这三个工作目前的研究现状。第\ref{solver:Backgroud}节介绍了海洋模式正压模态的求解方法及其优化,第\ref{related:precond}节介绍了求解线性方程组的迭代算法中主要的预处理技术,第\ref{related:verify}节介绍当前气候模式中已有的正确性验证方法。


\section{海洋模式正压求解技术}
\label{solver:Backgroud} 

过去的二十多年中,用来解决科学问题的超级计算机变得越来越强大。 
很多高性能计算领域的研究都在关注如何使得科学应用能够更加适应大规模的并行环境。 
没有可扩展的应用, 超级计算机的功能再强大也不能对很多像海洋模拟这样科学领域十分重要的问题起到促进作用。 
数值气候模式利用超级计算机的并行环境能够提高我们 模拟和理解海洋运动过程、监视和预报大气海洋等状态的能力。 
目前的气候模式为了能够更加准确的模拟大气、海洋和海冰等运动过程, 都趋向于采用更细粒度的水平和垂直分辨率,导致气候模式的规模也变得越来越大。 
随着气候模式分辨率的提高, 利用气候模式来做模拟的计算需求也会变得越来越巨大,这也使得在大规模并行环境中对海洋模式进行优化变得极其重要。 



目前有很多研究工作都是在关注气候系统模式的性能,尤其高分辨率气候模式在大规模并行环境中的可扩展性。 
气候模式,尤其是地球系统模式,由于包含有很多分量模式,结构复杂,代码量巨大,导致气候模式的性能并不乐观。 
本文主要以目前使用最为广泛的美国国家大气研究中心主要研发的公共地球系统模式(CESM)为研究对象。
在很多真实模拟中,海洋模式POP都是CESM中计算开销最大的一个分量\cite{Worley:2011:PCE:2063384.2063457, dennis2012computational}。 
因此,本文中,我们专注于提高海洋模式POP的性能优化。 
POP是一个十分有影响力的海洋模式,它是由美国Los Alamos国家实验室研发,多家研究机构共同发展。
海洋模式POP被广泛的应用于涡分辨率的海洋模拟\cite{mcclean2002eulerian, stark2004towards},以及海洋和海冰或者大气和海洋相耦合的耦合模拟  \cite{May2002preliminary}。 
POP目前被著名的CESM采纳为其海洋模式分量。  
海洋模式POP采用经过静力平衡近似和  Boussinesq近似的三维原始方程。 
为了避免快波(如重力波等)对时间步长的苛刻的要求, 它将时间积分分成两个部分: 一个是求解三维动力学和热动力学过程的斜压模态,另一个则是求解求解二维的垂直积分后的动量方程和连续性方程得到的二维海表高度(SSH)变化的正压模态。\cite{smith2010parallel}.

已有的很多研究都明确的指出,海洋模式POP中,斜压模态的计算量占整个模式的绝大部分,但通信相对较少、可扩展性较好,其计算时间随着计算规模的增大而减少。
但是正压模态主要由边界更新和全局归约操作等通信开销组成,其计算时间开销占总计算时间开销的比例从几百核上的10\%增大到10,000多核上的50\%\cite{pop05,stone2011cgpop,Worley:2011:PCE:2063384.2063457, dennis2012computational}。
因此,海洋模式POP的性能主要受到正压模态中通信瓶颈的影响,提高海洋模式正压模态的可扩展性将有助于提高整个公共地球系统模式的性能。

在时间差分方案上,目前海洋模式中求解器正压模态既有显式方法,又有隐式方法\cite{griffies2000developments}。 
显式方法的优点是求解简单,容易并行,缺点是为了使得数值格式稳定需要很小的时间步长,导致模式整体的计算开销很大。
与之相反,隐式方法的求解复杂,但是计算效率更高。 
隐式计算正压模态,通常需要求解一个自由海表面的椭圆方程。在数值离散之后,这个椭圆方程可以转化成为一个稀疏的线性方程组。

 
求解器线性方程组的方法可以简单的分为直接法和迭代法。
直接法通常利用因式分解将系数矩阵转化成容易求逆的矩阵,它被广泛的应用于在很多将准确性作为首要条件的工业应用程序中,比如结构分析、半导体器件建模等不涉及到偏微分方程的领域\cite{benzi2002preconditioning}。
直接法通常比较健壮,并且需要的计算时间和存储资源都是可预估的。
但是,直接法的可扩展很差,尤其是对于高维度的偏微分方程离散后得到的线性方程组或者规模巨大线性方程组。 

迭代算法包括的内容比较广泛,如传统的Jacobi迭代,Gauss–Seidel迭代, SOR 迭代和 Krylov 子空间等方法\cite{saad2003iterative,barrett1994templates}。
当前的应用中,有很多需要求解含有几百万甚至更多的未知数的线性方程组。 对于这样的超大规模的问题,迭代算法是唯一的选择。
很多实际应用中,只需要一定精度的解,这时迭代算法需要的存储和计算量通常比直接法的少。 
迭代算法在核电工业,石油工业和气候模拟等领域应用十分广泛。

迭代算法中,Krylov 子空间方法,尤其是共轭梯度法(CG)和它相应的变形,由于效率很高,
被广泛的应用于海洋模式中正压模态的求解。
比较著名的海洋模式如MITgcm\citep{adcroft2014mitgcm}, FVCOM\citep{lai2010nonhydrostatic}, MOM3\citep{pacanowsky1999mom3}, OPA \citep{madec1997ocean}
都采用了共轭梯度法或其变种来隐式求解正压模态的方程。
然而,正如下一节(第\ref{related:bottleneck})我们将要介绍的,共轭梯度法在当前的海洋模式分量POP中造成了比较严重的通信瓶颈\citep{Worley:2011:PCE:2063384.2063457}。 

 

\subsection{正压模态通信瓶颈}
\label{related:bottleneck}

隐式自由面方法在求解海洋模式正压模态中是一个很常见的选择,因为它能够允许较大的时间步长来高效的计算速度很快(约200m/s)的重力波。
但是隐式自由面的求解需要隐式的求解一个椭圆方程,这正是正压模态的可扩展性较差主要原因。 
海洋模式正压模态中的椭圆方程通常会被近似为一个线性系统$Ax=b$。 
国际上很多流行的海洋模式都采用共轭梯度法及其变种求解海洋模式正压模态中的线性系统\citep{adcroft2014mitgcm,lai2010nonhydrostatic,madec1997ocean}。
但是,共轭梯度法自身有一个不适合大规模并行坏境的缺陷,那就是每一步迭代过程中都涉及到两次求內积操作。 
当采用成百上千个处理器核心时, 做內积所需要的全局通信和同步操作就会是一个主要的瓶颈。

正压模态的执行时间占整个海洋模式POP的很大一部分,尤其是当它所使用大量的处理器核心时。
目前有很多研究工作都是在关注海洋模式POP的性能,尤其是它的正压模态的比较差的可扩展性。 
Jones\cite{pop05}等人在向量架构和常规集群的并行环境中测试了海洋模式POP1.4.3版本的可移植性,并且发现POP中斜压模态主要是由计算组成, 但是正压模态主要由边界更新和全局归约操作等通信开销组成。
Stone  \cite{stone2011cgpop}等人发现,正压模态的时间开销占总计算时间开销的比例从几百核上的10\%增大到10,000多核上的50\%。 
他们甚至还为此而开发了一个POP的简化版本,称之为CGPOP,专门用来研究海洋模式POP中的正压模态的新算法、数据结构和编程模型等。 
Worley  \cite{Worley:2011:PCE:2063384.2063457} 和 \cite{dennis2012computational} 等人在近30,000 核上测试了公共地球系统模式CESM的海洋模式分量POP 2.0.1的性能。 
他们发现,海洋模式POP在很多真实模拟中都是CESM中计算开销最大的一个分量,并且证明了在大规模并行时,海洋模式POP的性能主要受到正压模态中通信瓶颈的影响。 

\subsection{正压求解器的优化}
\label{related:improve}


正压求解器是高分辨率CESM版本中POP的主要性能瓶颈,因为当在大核数上运行时它的运行时间占了POP总运行时间的一大半。 
这是由于求解自由海表面的正压求解器中算法所固有的明显的全局通信操作在大核数下的可扩展性比较差所导致的。
目前已经有很多的关于如何优化正压模态性能的工作,其中大部分都是通过减少进程间的通信量或者加速每个进程上的计算速度。 

全局归约操作的开销和所使用的进程数的多少成正相关的, 所以通信开销随着进程数的增加会逐渐变得不可接受。
很多研究工作都尝试去提高共轭梯度法的并行性能,减少其负面影响。 
比如在算法层面减少共轭梯度法的全局通信\cite{dAzevedo1999lapack},在正压模态中实现计算与通信相重叠\cite{beare1997optimisation},在海洋模式中移除陆地点并且实现负载均衡\cite{dennis2007inverse, dennis2008scaling},以及使用异构加速平台对海洋模式正压模态进行加速等。

我们这里简单的从以下三个方面介绍一下相关工作。
首先,海洋模式正压模态中所使用的共轭梯度法在很多科学领域都有应用,
减少共轭梯度法的全局通信的开销在这个算法被并行化之后就一直是研究的热点。 
由于计算机浮点运算速度不断提高,但是网络延迟的改进却很有限,导致计算和通信效率之间的差距越来越大。
负载不均衡、硬件的差异和操作系统的抖动都会对全局通信造成很大影响\cite{ghysels2014}。 
现在有很多的可选方案来缓解大规模并行环境中的预处理的共轭梯度法所固有的比较差的可扩展性。
一类是在算法层面对共轭梯度法进行改进。 
这个领域最早的成果是在标准的共轭梯度法的基础上做算法的修正从而减少全局归约操作的次数, 比如在海洋模式POP中被采用的Chronopoulos-Gear (ChronGear, \cite{dAzevedo1999lapack})方法 。 这些方法至今仍然被广泛的使用着。 
早期还有s-步方法\cite{chron1989} 和最近的一些变种(比如文章\inlinecite{hoemmen2010})也能够减少全局通信, 但是这些方法很难结合稍微复杂一点的预处理方法使用。
另外一类研究是通过将共轭梯度法中的全局通信与矩阵向量乘法利用流水线的方式来重叠,进而提高共轭梯度法的并行效率。关于如何减少共轭梯度法中全局通信的开销,文章\inlinecite{ghysels2014}给出了一个非常详尽的总结。
 



其次,针对海洋模式,有很多工作都是关于如何减少正压求解器中通信瓶颈所带来的影响。
正压模态的执行时间占整个海洋模式POP的很大一部分,尤其是当它所使用大量的处理器核心时。
海洋模式正压模态中的线性方程组是对海表高度的椭圆方程进行离散化之后得到的,是条带状的系数矩阵。
同时海洋模式的并行中,通常采用二维平面的划分,因此海洋模式中使用的共轭梯度法有着特殊的并行性质。
目前已经有很多的关于如何优化正压模态性能的工作,其中大部分都是通过减少所使用的进程数或者减少进程间的通信量。 
正压模态中,共轭梯度法每一步迭代所需要的全局归约操作的开销和所使用的进程数的多少成正相关的, 所以通信开销随着进程数的增加会逐渐变得不可接受。
OpenMP并行和陆地点移除在海洋模式中是比较常见的减少进程数量及其相应的通信开销的常用策略。  
在论文\inlinecite{Worley:2011:PCE:2063384.2063457}中, 他们在正压模态中采用OpenMP并行技术,解决斜压模态需要很多的处理器核心来进行计算而正压模态的进程数太多会造成通信瓶颈越大这一矛盾,进而提高了海洋模式在大核数上的性能。 
另一个比较常见的减少通信开销的策略是陆地点的移除\cite{dennis2007inverse,dennis2008scaling}。 
Dennis等人提出通过去掉全是陆地点的块来减少需要参与通信的处理器核心的数目,从而减少通信的开销。陆地点扣除之后,海洋水平网格将会变得不规则,他们继而提出了基于最新的空间填充曲线划分算法的负载均衡的策略来改善负载均衡。 
他们的实验表明,在接近30,000个处理器核心上,新的策略使得模式的模拟速度增加了近一倍。 
另外,减少通信的频率同样可以减轻正压模态中的性能瓶颈。 
早在1997年,  Beare \cite{beare1997optimisation}等人就提出来,通过增大边界通信缓存区域的块的大小以及将通信和计算相重叠减少通信开销,进而提高并行大洋环流模式的性能。  
尽管这些方法都能够对性能有一定的提升,但是他们并没有完全的消除掉全局归约操作的瓶颈。 

 

海洋模式中共轭梯度法的数据局部性比较差,而且很多操作只能串行执行,导致以上的改进方案的效果都十分有限。
最近异构计算技术的发展也在一定程度上使得正压模态的性能有所提升。 
已有一些工作利用异构加速平台,比如说GPUs \cite{cuomo2012pcg} 和FPGAs \cite{Shida2007}来实现对预处理共轭梯度法的加速。 
Cuomo 等人\cite{cuomo2012pcg}在全球环流海洋数值模式中引入了稀疏近似逆预处方法,并且利用一个科学计算库将其实现在GPU上。 
Shida 等人\cite{Shida2007}将海洋模式中的正压模态移植到FPGAs上,并且发现发现当适当的使用块上内存和流式的直接内存访问,一块100MHz的FPGA卡的性能能与1GHz的CPU处理器相当。 
GPUs和FPGAs还能够减少正压模态中的全局通信开销。这些设备跟传统的CPUs相比有更强的计算能力和更大的内存,因此我们只需要使用很少的几块加速卡就能够实现只有在大规模并行环境下才能计算的任务。 


\section{预处理技术}
\label{related:precond}
在某些应用中,迭代算法可能在给定时间内无法收敛,或者收敛的速度达不到要求,此时就需要进行适当的预处理。
通常情况下,判断预处理方法的好坏的标准有三点。第一、预处理之后得到的系统更容易求解,即经过预处理之后,迭代算法能够更快的收敛; 第二、预处理过程的计算开销较小,需要确保使用预处理之后,每一步的迭代开销不至于太大; 第三、预处理子的构造相对简单,即前期处理的计算开销不大。 
通常情况下,这三个条件中需要取得一个平衡。判断预处理子的效果的最终指标是预处理之后系统的求解时间比没有预处理的时候要小很多。 
值得一提的是,第三条中的预处理子前期处理的时间的重要性与预处理子能否被重用有关系。在很多应用中,需要求解一些列有着相同的系数矩阵但是不同的右端项的方程组。
这时,可以用更大的前期处理开销来换取一个更加有效的预处理子,因为前期处理的开销在反复的求结果中被均摊。 
使用隐式方法求解发展方程,比如海洋模式中的正压模态,以及使用牛顿迭代方法求解非线性问题时,通常会遇到这种情况\cite{benzi2002preconditioning}。

正压模态求解器的总时间开销等于求解器达到给定收敛条件所需要的迭代步数乘以每一次迭代的时间开销。
随着计算使用的处理器核心数的增大,每一步迭代中的计算所消耗的时间是逐渐减少的,
但是通信所需要的时间会逐渐增加。 
当所使用的核心数增加到一定数目时,计算时间的减少无法掩盖通信时间上的增加,导致总时间也随着增加。 
为了减少通信的开销,人们通常采用预处理这一技术来减少收敛所需要的迭代次数。
附加的前提条件是预处理过程的开销在合理的范围之内。 
目前海洋模式POP中采用的默认求解器ChronGear通过使用一个简单的对角预处理方法,性能就已经取得了很好的提升\cite{pini1990simple, reddy2013comparison}。 
如果能够进一步的提高求解器的收敛速度,求解过程中通信的开销将会极大的减少,从而进一步的提高求解器的可扩展性。
事实上,一个有效的预处理子不仅能够给新的P-CSI求解器带来性能提升,同时也能够改进原有的默认的ChronGear求解器。


共轭梯度法中的预处理方法自从上世纪90年代以来就一直备受重视。 
很多的线性系统在采用了适当的预处理子之后,预处理共轭梯度法只需要少数几步迭代就能够收敛。 
然是,大多数的最有效的预处理技术, 比如说不完全的Cholesky分解和不完全的LU分解,在海洋模式中并不是十分的有效。 
海洋模式中的椭圆方程通常需要在并行的环境中求解,需要为之设计专门的可并行化的预处理子。 



\subsection{传统预处理方法}
\label{related:classical}

传统串行环境中的不完全LU预处理(ILU)在并行环境中很不适用。这是由于ILU方法中的高斯消去过程很难并行。 
而且,ILU的预处理过程中求解上三角和下三角方程只能高度串行执行。
采用区域分解技术,不完全因式分解方法可以实现一定程度的并行。 
文章\cite{hysom2001scalable}给出了传统ILU预处理方法的并行版本。 
他们的实验表明,这些并行的ILU方法在几百个处理器时仍有较好的可扩展性。

在并行环境中,预处理方法还需要考虑到原始系数矩阵的结构。 
通常情况下,我们希望预处理子尽可能与原始系数矩阵有相同的稀疏结构,这样就能避免额外的复杂通信和存储。 
由于海洋模式POP中的数据是分布在多个进程上的,而进程之间的通信的开销很大, 如果使用因式分解的预处理方法,通常只能接受比较浅层次的分解,比如说ILU(0)。 
ILU(0)分解中,L和U分别与原始矩阵A的上三角和下三角矩阵具有相同的非零元结构。
这导致L和U的乘积只是A的一个非常粗糙的近似,从而使得这种预处理的效果并不是很明显。 
更为精确的ILU因式分解方法并不太适用于海洋模式,因为它需要更多的额外的通信和计算。 
同时,计算这些高阶的预处理子的代价也会越高。 

\subsection{误差向量传播方法}
\label{related:evp}

\subsection{并行预处理方法}
\label{related:parall}

使用最广的串行预处理技术当属不完全的因式分解方法,比如不完全的LU分解(ILU)和它的变种\cite{benzi2002preconditioning}。 
但是,这些方法并不适合在并行的环境中使用,因为它们需要对整个矩阵做一个按顺序的逐次的计算。 
随着并行计算的兴起,一些可以并行化的预处理技术,比如说多项式预处理、近似逆预处理、多重网格预处理和块预处理,逐渐因此了人们的重视。 
高阶的多项式预处理能够像不完全的LU分解(ILU)以及它的变种在串行的环境中一样高效的减少迭代算法的迭代次数\cite{benzi2002preconditioning} 。 
但是多项式预处理子巨大的计算开销抵消了它与在减少迭代次数上的优势,因为一个$k$阶的多项式预处理子在每次一次迭代过程中需要进行$k$次矩阵向量乘操作。 
最终的实际运行中,它的时间开销有时候反而比不上简单的对角预处理\cite{meyer1989numerical,smith1992parallel}。
近似逆预处理,尽管可并行度很高,但是它需要求解一个比原始方程大很多倍的线性方程组\cite{smith1992parallel,bergamaschi2007numerical}, 这就使得它的魅力相比于简单的预处理方法大打折扣。

在1985, Concus等人 \cite{concus1985block} 使用原始矩阵的逆矩阵的条带状近似作为预处理矩阵,在椭圆微分方程问题的求解中取得了比其他预处理子更好的效果。 
Smith et al. \cite{smith1992parallel}在海洋模式POP 的最原始版本中采用多项式预处理的方法以及一个局部近似逆预处理的方法,将迭代步数缩短为原来的三分之到五分之一。
但是由于多项式预处理方法中,前期得到多项式预处理子的开销比较大,在后期的版本中这一方法并没有得到继承。  
Adamidis等人 \cite{adamidis2011high} 在Max Planck研究中心的全球海洋海冰模式MPIOM中采用了一个不完全的Cholesky预处理子,在提高预处理共轭梯度法的性能的同时,也改善了它的可扩展性。 
Watanabe \cite{Watanabe2006pcg} 设计了一个基于重叠的区域分解法的预处理共轭梯度法来加速收敛,同时减少了处理单元之间的通信开销。 
 
多重网格是一种针对椭圆微分系统中的线性方程组的高可扩展和十分高效的预处理方法。 
最近的研究表明几何多重网格(GMG)在使用规则网格和简单地形的大气模式
 \cite{muller2014massively}和海洋模式中\cite{matsumura2008non,kanarska2007algorithm}中有很好的效果。 
但是,全球海洋模式通常使用比较复杂的地形(如群岛、海峡、通道和复杂的海岸等),并结合以非规则或者各向异性的网格, 结果导致简单的几何多重网格并不能取得较好的效果
\cite{matsumura2008non,fulton1986multigrid,tseng2003ghost,stuben2001review}。 
在公共地球系统模式CESM的海洋模式分量POP中,为了避开极点奇异性问题,采用的是极点偏移的广义正交网格。 
海洋模式POP将陆地点掩盖掉,从而只计算地球表面的海洋部分。
这些选择导致了模式中的椭圆方程所得到的系数是定义在一个不规则的网格上的不规则区域, 进而导致几何多重网格很难取得好的效果。 
地形的复杂性在高分辨率的海洋网格中会变得更加糟糕,因为成千上万的群岛和海峡通道(比如白令海峡)在稍微粗一点的网格上并不可见。
对于这些涉及复杂的地形的问题,代数多重网格通常比几何多重网格更加的实用。 
但是代数多重网格也有一个缺陷,那就是在某些例子里面,代数多重网格的启动开销比它的迭代求解过程还要耗时,
这使得它反而不如一些预处理得比较好的共轭梯度法
\cite{muller2014massively}, 尤其是在使用共轭梯度法所需要的迭代步数比较少的时候。 
地球系统模式CESM中的海洋模式分量POP中的正压模态求解问题就是这样一个例子(参见下一节的图
 \ref{fig:iter})。
更进一步,不管是高分辨率还是低分辨率,海洋模式POP的每一个模拟天内都需要对正压模态中的线性方程求解几十次甚至上百次,
而通常的一次模拟中往往需要模拟成百上千个模拟年,这就导致代数多重网格方法的启动开销变得无法接受。


最后,块预处理技术被证明是一个比较有效的并行预处理技术\cite{concus1985block, white2011block}。
这个技术对于海洋模式POP尤为有潜力,因为它能够充分地利用POP中由椭圆方程离散化后得到的矩阵的块状结构的性质。 

\section{气候模式的正确性验证}
\label{related:verify}

目前,使用计算机可执行程序来表示一个模拟模型的需求正在不断增加。 
这个软件工程领域中,有很多的软件验证能够用来对数值模拟程序进行验证。
但是,由于气候模式的不确定性,模式的验证要比气候模式的软件验证要艰难得多。
正是这种困难,使得整个数值气候模拟社区中,都缺乏对于数值模拟模式正确性验证的重要性的理解和意识\cite{whitner1989guidelines}。 


在制定未来气候变化缓解和适应的对应策略时,必须要考虑到未来气候变化的不确定范围。
气候预测本质上有五个维度:三维空间,时间和概率。 
由于气候本身具有很大的不确定性,主流的气候模式,目前基本上都牺牲掉概率分辨率,而对其它四个分辨率不断地做优化。
没有考虑概率维度的气候模式的模拟结果存在很大不确定性,要想对气候模式进行正确性验证也是十分困难\cite{whitner1989guidelines}。

 
要想实现对气候模式的正确性验证,必须考虑到气候模式的概率这一维度, 即考虑到气候模式的不确定性。 
迄今为止,已经有一些研究利用初始条件中的不确定性,以及改变边界条件所带来的影响等方法来研究气候模式的不确定性。
但是对于大部分的变量,几十年的气候模拟中主要的不确定性并不在初始条件或者外部驱动,而在于气候系统的响应。
这个问题在大气海洋环流模式(AOGCMs)中已经得到了解决,主要是通过采用非直接指定的观测数据的集合模式的比较,来减少对可能性集合的限制\cite{allen2002towards} 。 
这要求有几十年的集合模拟的结果来评估气候自身的混动性和模式相应的不确定性。 
有研究基于近代气候变化的观测,对模式相应作出的统计学估计,认为气候的敏感性,也就是全球平均温度对于大气中二氧化碳浓度增加一倍的平衡相应,要大于5K\cite{stainforth2005uncertainty}。
他们通过利用常规的环流模式来进行几千次模拟而得到的集合模拟,发现这些版本跟其他的前沿的模式的模拟结果一样真实,但是气候的敏感性从2K到11K不等。
这个几段范围的敏感性对于气候系统对于大气中温室气体的升高的相应的研究,以及评估将特定目标稳定在某个层次所带来的风险是非常有意义的。



另外,可以通过参数化的方式来对模式中的不确定性进行量化。
文章\inlinecite{reynolds1994random}通过对误差的来源的研究,给出了美国国家气象中心的中范围预测模型的随机增长误差的三维结构,进而为研究其不确定提供了定量手段。
他们认为气候模式中的随机误差的增长分为两类: 一类是外部误差增长,主要是由于模式的缺陷导致的;一类是内部误差增长,主要是由于初始条件中的误差的自增长 。 
通过对预测和验证分析的相关性的参数化,可以确定内外部增长的速率。
这种参数化假设线性的随机增长主要是由于模式的缺陷引起的。 
他们的结果表明,在热带地区,结合模型和分析的结果,可以显著的改善模拟结果。
而在中纬度地区,模式的改进很难改善预测的结果,所以减少分析的误差变得尤为重要。 
参数化能够得到物理上有意义,并且和之前的预测结果相一致的结果。
他们的这种方法为评估预测不确定性在空间和时间上的分布提供量化手段。


\section{本章小结}
\label{related:Conclude}

随着气候模拟的需求日益增加,以及观测资料和计算资源等辅助条件的日益成熟,气候模式不断地提高模拟精度已经成为一个大的趋势。
很多高性能计算领域的研究都在关注如何使得科学应用能够适应大规模的并行环境。 
在将地球系统模式移植到大规模并行环境上的过程中,海洋模式中的正压模态的差的可扩展性逐渐的暴露出来。 
海洋模式POP中,斜压模态的计算量占整个模式的绝大部分,但通信相对较少、可扩展性较好,其计算时间随着计算规模的增大而减少。
目前有很多研究工作指出海洋模式POP的正压模态是它的可扩展性的瓶颈。

已有的很多研究表明,正压模态中的全局归约操作的开销和所使用的进程数的多少成正相关的,所以通信开销随着进程数的增加会逐渐变得不可接受。
有很多研究工作尝试去提高正压模态的性能,减少共轭梯度法中的全局通信的负面影响。
一些方法尝试减少共轭梯度法的全局通信的开销,以及将计算与通信相重叠。 
另一些方案在海洋模式中尝试使用陆地点移除和负载均衡的方法来减少进程数,进而减少其相应的全局归约操作的开销。 

预处理技术在实际应用中应用十分广泛。为了进一步提高海洋模式中正压模态的求解性能,很多研究工作都尝试通过预处理技术来加速正压模态的求解。 
传统的因式分解方法(如ILU等),虽然可以实现一定程度的并行,在大规模并行环境中仍然有很多限制。
近年来发展的并行预处理技术解决了可扩展性问题,但是有些预处理效果不佳,有些启动开销太大。 
并行块预处理是比较适合并行海洋模式的一种方法,它充分的利用海洋模式并行划分的特点,在预处理效果和预处理开销上实现了很好的平衡。

由于模式的不确定性,模式中的任何改变都可能得到二进制不一致的结果。
为了确定这种改变是否会对模式的结果造成统计学上显著的改变,就必须考虑到模式自身的不确定性。
集合模拟是用来对模式不确定性进行量化的常用方法。 

 



\chapter{海洋模式正压求解器}
\label{cha:barosSolver}

\section{本章概述}
本章的主要内容是通过分析正压模态的通信瓶颈,提出了一个能够提高并行海洋模式POP中正压模态求解器的扩展性的新的策略。 
 海洋模式POP将其正压模态中的椭圆方程离散化之后得到一个线性系统$Ax=b$,并且使用预处理的共轭梯度法来进行求解。 
 由于每一步迭代过程中都需要时间开销很大的全局归约操作, 导致预处理共轭梯度法在分布式并行系统上的可扩展性不好。 
 本章首先建立了一个量化预处理共轭梯度法可扩展性瓶颈的模拟。
 基于这个模型, 我们认为以前被视为效率不如共预处理轭梯度法的传统的Stiefel迭代方法(CSI),在大规模并行环境中是更有前景的。
相比于预处理共轭梯度法,CSI算法的迭代过程中的迭代参数不是由前一步迭代后的残差的內积计算得到的,而是利用系数矩阵$A$的特征值谱决定。 
我们采用Lanczos方法 解决了估计大规模矩阵$A$的特征值的难题。 
通过Lanczos方法可以构造出一个很小规模的三对角矩阵,它的特征值逐渐的逼近系数矩阵的$A$的部分特征值。 
通过使用CSI方法替换原来的预处理共轭梯度法发, 海洋模式正压模态中的规矩归约操作以及其所导致的较差的可扩展性就被去掉了。 
在0.1度分辨率的海洋模式POP中,采用了CSI方法之后正压模态在15,000核心上取得了5倍左右的加速效果,运行时间从原来的41.96秒每天下降到6.67秒每天。 





\section{海洋模式正压模态求解器}
\label{solver:baro}


方程(\ref{eq:ssh})在二维广义正交网格上使用九点差分格式进行离散。  
POP将全球区域划分为若干个小块,并将这些块分发给每一个进程。
每个进程只负责计算它所分得的块内网格点上的迭代过程,并且维护一个与周围进程进行数据交换的边界区域。 
我们假设全球区域上的网格大小为$\mathcal{N}\times
\mathcal{N}$且被划分为$m\times m$ 个大小为$n\times n$ ($n=\mathcal{N}/m$)的小块.
我们定义$B$ 和 $\tilde{x}$ 分别为与给定块相对应系数矩阵(也就是 $A$的一个大小为 $n^2\times n^2$的子矩阵)和向量。
这个块上的子矩阵和子向量的乘积操作$B\tilde{x}$ 需要$9n^2$ 次计算 \cite{hu2013scalable}.

%%%%%%%%%%%%%%%%%%%%%%%%%%%%%%%%%%%%%%%%%%%%%%%%%%%%%%%%%%%%%%%%%%%
\section{正压模态及其瓶颈分析}
\label{solver:bottleneck}
%----------------------------------------------------------------------------


为了简洁, 在本文中,我们假设整个区域的网格大小为$N\times N$,它被划分为$m\times m$个大小为$n\times n$ ($n=N/m$)的小块。 
假使块$B(k,l)$上对应的系数矩阵为$\tilde{A}$, 那么$\tilde{A}$就是系数矩阵$A$的一个大小为$n^2\times n^2$的对角子矩阵,并且每一行最多只有九个非零元。 
于是,矩阵向量乘积操作 $\tilde{A}\tilde{x}$只包含有$9n^2$次浮点数乘法操作,而不是$n^2\times n^2$次操作。 

%----------------------------------------------------------------------------
\subsection{PCG求解器} 
\label{solver:pcg}
 
海洋模式POP的正压模态中使用传统的共轭梯度法结合一个对角预处理子$M = \Lambda(A)$ 作为它的默认求解器。 
这个默认求解器在小规模并行时效率很高。 
对角预处理的共轭梯度法的过程如下\ref{alg:pcg}: 

\begin{algorithm}[h]
\caption{共轭梯度法}
\label{alg:pcg}
\begin{algorithmic}[1]
\REQUIRE  与网格块$B_{i,j}$相应的系数矩阵$\textbf{B}$,  初始解 $\textbf{x}_0$ 和 $\textbf{b}$  \\
//\qquad    \textit{所有进程并行执行}
\STATE $\textbf{r}_0 = \textbf{b}-\textbf{B}\textbf{x}_0$, $\textbf{s}_0 =0$;\quad $\beta_0=1$, $k=0$;
\WHILE{$k \leq k_{max}$ }
\STATE $k=k+1$;\quad $\textbf{r}'_{k-1} =\textbf{M}^{-1}\textbf{r}_{k-1}$;\quad \COMMENT{对角预处理}
\STATE $\tilde{\beta_k} = \textbf{r}_{k-1}^T\textbf{r}'_{k-1}$;\quad $\beta_k = global\_sum(\tilde{\beta_k})$; \COMMENT{全局规约操作}
\STATE $\textbf{s}_k = \textbf{r}'_{k-1} +(\beta_k/\beta_{k-1})\textbf{s}_{k-1}$;\quad $\textbf{s}'_k = \tilde{\textbf{A}}\textbf{s}_k$; \COMMENT {矩阵向量乘}
\STATE $update\_halo(\textbf{s}'_k$); \COMMENT{ 边界通信}
\STATE $\tilde{\alpha_k} = \textbf{s}_k^T\textbf{s}'_k$;\quad $\alpha_k =\beta_k/ global\_sum(\tilde{\alpha_k})$;\quad \COMMENT{全局规约操作}
\STATE $\textbf{x}_k =\tilde{\textbf{x}}_{k-1} +\alpha_k \textbf{s}_k$;\quad $\textbf{r}_k =\textbf{r}_{k-1} -\alpha_k\textbf{s}'_k$;
\IF{ $k \% n_{c} == 0$}
\STATE \textbf{if} $||\textbf{r}_{k+1}|| \le \epsilon$  \textbf{return} ;\COMMENT{每 $n_c$ 检查一次是否收敛}
\ENDIF
\ENDWHILE
\end{algorithmic}
\end{algorithm}

 
正如算法\ref{alg:pcg}所示, 预处理共轭梯度法主要包含有以下下个部分: 计算,边界通信,全局通信。 
 
%----------------------------------------------------------------------------
\subsection{PCG复杂度模型}
\label{solver:pcgComplex}
假使使用$P=m^2$个进程来计算正压模态,也就是每个进程正好计算一个子块。 
正压模态的运行时间与预处理共轭梯度法梯度法在每个进程上的执行时间。 
假设 $T_c$, $T_b$ 和 $T_g$分别表示每个迭代步中计算,边界交换和全局归约操作的时间。 

 算法\ref{alg:pcg}中,计算主要涉及到步骤3, 5和 8中的四个向量伸缩操作,步骤4和7中的两个向量向量乘积操作,和步骤5中的矩阵向量乘积。 
 因此 $T_c= \Theta (4 n^2 +2n^2+ 9n^2) = \Theta (15n^2) =\Theta(15\frac{N^2}{P})$。 
很明显 $T_c$ 会随着进程数的减少而减少, 同时以0位下界。 


每个进程都需要与其相邻点进行边界交换, 这个过程时间开销主要取决于网络延迟和边界缓存区域的大小。 
边界缓存区域的带宽默认设置为2, 因此一次边界通信的数据量大小为 $2n$。 
很明显,这个数据量会随着进程数的增加而减少。 
每个进程需要与其周围的四个进程进行数据交换, 因此一次边界更新所需要的时间为$T_b =2\times4T_{delay} +\Theta (2\times4\times 2n)=8T_{delay} +\Theta (\frac{16N}{\sqrt{P}})$。 
可以看出来,这个更新时间是随着进程数的增加而减少的,并且它有一个由网络延迟决定的下界 $8T_{delay}$。 

 
每次內积操作之后需要调用一次全局归约操作, 归约操作实质上就是把每个进程上所计算得到的局部的內积值收集起来,
因此归约操作的数据传输的时间相比收集操作的时间是可以忽略的。 
归约操作的时间, 包括了启动延迟和网络阻塞,满足$T_g= T_{init}+ c_g\cdot \mathcal{G}(P)$。 
这里$T_{init}$ 和$c_g$ 表示并行环境中的常量, 而$\mathcal{G}(\cdot)$是一个与指定架构中网络拓扑结构相关的函数。 
比如, 理想立方体网络中的$\mathcal{G}(\cdot)$是一个对数函数。 
总之,$T_g$ 会随着进程数$P$的增加而单调的变大。 
 

 
设$T_0$表示一次浮点数操作所需要的时间, $B$表示网络中每秒中点对点能够传递浮点数的个数。 
那么一次预处理共轭梯度法的执行时间可以表示为:

\begin{equation}
\label{t_pcg}
T_{pcg} = T_c + T_b + T_g
= 15 T_0\frac{N^2}{P} + 8T_{delay} + \frac{16N}{B\cdot\sqrt{P}}+T_{init} +c_g\mathcal{G}(P)
\end{equation}
总个预处理共轭梯度法的求解过程的执行时间可以表示为$t_{pcg} = K_{pcg}\cdot T_{pcg}$, 这里$K_{pcg}$表示预处理共轭梯度法达到收敛所需要的迭代步数,它不会随着进程数的增加而改变。 
方程\ref{t_pcg}可以出来, 计算和边界更新所需要的时间会随着进程数的增大而减少, 
但是全局归约操作的时间却会随着进程数的增加而增大。 
因此,整个预处理共轭梯度法的运行时间在当进程数大于某个数值时,会随之增大而增大。 

\begin{figure}[ht]
\centering
\includegraphics[height = 8cm]{evaluate_PCG}
\caption[] {1 度 POP中预处理共轭梯度法的时间成分分析\label{fig:pcg_ratio}}
\end{figure}

 
我们在清华大学的探索100集群上,使用1度的海洋模式POP (360$\times$240 个网格点)做了一系列的实验。 
探索100包含有740个计算节点,每个计算节点都有两个2.93 GHz 英特尔Xeon X5670 6核处理器以及24/48 GB的内存。 
我们采用了一个并行程序的性能评测和追踪的工具--TAU\cite{shende2006tau} 来测一下预处理共轭梯度法求解器的三个部分的时间。 
我们还将实验中一个正压步中每个分量的平均耗时和评测模型中的时间进行对比。 
在评测模型中, 我们将$\mathcal{G}(\cdot)$设为一个线性函数, $c_g$  设为 $2\times 10^{-7}$, $T_0$ 设为$2.5\times 10^{-9}$ ,  $B$  设为 $5.0\times 10^{9}$。 


如图\ref{fig:pcg_ratio}所示, 评测模型的结果和真实实验的结果吻合得非常好。 
实验中正压模态的计算时间确实是和进程数成反比的。 
实验中的边界更新时间再当进程数超过100的时候基本上保持不变,这是因为当进程数比较大的时候, 边界通信的消息大小变得非常小,以至于数据传输的时间相比于网络延迟可以忽略掉。 
全局归约操作的时间随着进程数的增加而成比例的变大,并且当使用的进程数大于100时,全局归约操作的时间变成正压模态中的主要开销。 
这个结果和我们方程(\ref{t_pcg})给出的理论分析的结果是一致的,即全局归约操作是预处理共轭梯度法可扩展性差额根本原因。 

为了解决预处理共轭梯度法的可扩展性瓶颈,新的求解器需要的全局归约操作越少越好。 
我们在海洋模式中重新考虑一些原本被认为是不如预处理共轭梯度法的方法,比如Chebyshev迭代方法。 
Gutknecht \cite{gutknecht2002chebyshev} 在2002年有一篇文章专门的重新研究了一下Chebyshev迭代在通信开销较大的大规模并行环境中性能。 
传统的Stiefel迭代(CSI)就是Chebyshev迭代方法中的一种。 


 
\section{CSI迭代法}
\label{solver:csi}
与预处理共轭梯度法不同的是, P-CSI方法 在每一步迭代过程中不需要做內积操作, 因此它即使在大核数上也能够保持较好的可扩展性。
P-CSI需要对预处理后的系数矩阵$M^{-1}A$的最大特征值$\mu$和最小特征值$\nu$进行估计。 
众所周知,计算稀疏矩阵的特征值要比求解线性方程组本身还要难。 
值得庆幸的是, 
海洋模式POP中的系数矩阵$A$ 和它的对角预处理矩阵$M = \Lambda(A)$都是实对称矩阵, 因此预处理后的矩阵的最小最大特征值并不难估计。 
 
\begin{algorithm}[h]
\caption{传统Stiefel迭代算法}
\label{alg:csi}
\begin{algorithmic}[1]
\REQUIRE 与网格块$B_{i,j}$相对应的系数矩阵 $\textbf{B}$, 预处理子$\textbf{M}$, 初始值$\textbf{x}_0$和方程右端向量$\textbf{b}$ ;预估的特征值区间$[\nu,\mu]$;  \\
 // \qquad    \textit{所有进程并行执行}
\STATE $\alpha =\frac{2}{\mu -\nu}$, $ \beta = \frac{\mu +\nu}{\mu -\nu}$, $\gamma = \frac{\beta}{\alpha}$, $\omega_0 =\frac{ 2}{\gamma}$;\quad $k = 0$;
\STATE $\textbf{r}_0 = \textbf{b}-\tilde{\textbf{A}}\textbf{x}_0$; $\textbf{x}_1 =\textbf{x}_0 -\gamma^{-1}\textbf{r}_0$; $\textbf{r}_1 =\textbf{b} -\tilde{\textbf{A}}\textbf{x}_1$;
\WHILE{$k \leq k_{max}$ }
\STATE $k=k+1$;\quad $\omega_k = 1/(\gamma - \frac{1}{4\alpha^2}\omega_{k-1})$; \COMMENT{迭代函数}
\STATE $\Delta \textbf{x}_{k} =\omega_k\textbf{r}_{k-1}+(\gamma \omega_k-1)\Delta \textbf{x}_{k-1}$;
\STATE $\textbf{x}_{k} =\textbf{x}_{k-1}+\Delta \textbf{x}_{k-1}$; \quad  $\textbf{r}_{k} =b- \tilde{\textbf{A}}\textbf{x}_{k}$;
\STATE $update\_halo(\textbf{r}_k)$; \COMMENT{边界通信}
\IF{ $k \% n_{c} == 0$}
\STATE \textbf{if} $||\textbf{r}_{k+1}|| \le \epsilon$  \textbf{return} ;\COMMENT{每 $n_c$ 检查一次是否收敛}
\ENDIF\ENDWHILE
\end{algorithmic}
\end{algorithm}

\subsection{算法及评估}

如算法\ref{alg:csi}所示, CSI方法和预处理共轭梯度法有相似的迭代过程,
但是将预处理共轭梯度法中的两次向量內积操作及其相应的向量向量乘积的计算替换成了系数矩阵$A$的最大最小特诊值的一个函数迭代。 
CSI求解器的计算时间可表示为
$T_c = \Theta (4 n^2 + 9n^2) = \Theta (13n^2) =\Theta(\frac{13N^2}{P})$。 
由于CSI方法和预处理共轭梯度法求解器所使用的边界区域是一样的, 因此边界更新的操作的开销也是 $T_b =8T_{delay} +\Theta (16 \frac{N}{\sqrt{P}})$。 CSI方法除了收敛性判断之外本身并不包含全局归约操作。 
因此,每一步CSI迭代的时间开销可表示为:  
\begin{equation}
\label{t_csi}
T_{csi} = T_c + T_b
= 11T_0 \frac{N^2}{P}+ 8T_{delay} + \frac{16N}{B \cdot\sqrt{P}}
\end{equation}
不考虑收敛性判断的情况下, 整个CSI求解器执行一次的时间开销为$t_{csi} = K_{csi}\cdot T_{csi}$, 这里$K_{csi}$ 表示CSI求解器达到收敛所需要的迭代步数。 在相同的收敛条件下, 它通常比$K_{pcg}$稍微大一些。 

如图\ref{fig:cst_ratio}所示, 这个模型真是的预测CSI的可扩展性的表现。 
CSI的收敛速度比预处理共轭梯度法稍慢, 它在小规模并行是的执行时间比预处理共轭梯度法的要长一些。 
但是,正如方程\ref{t_pcg} 和\ref{t_csi} 所示, 由于没有十分耗时的全局归约操作, CSI方法单个迭代过程的时间要比预处理共轭梯度法的短。 
在当进程数超过每一个值时,整个预处理共轭梯度法的求结果过程 要比CSI求解器消耗的时间长。 

 
\begin {figure}%[htbp]
\centering
\includegraphics[height=8cm]{evaluate_CSI}
\caption[] { 1度 POP中一步CSI求解器的成分分析}
\label{fig:cst_ratio}
\end{figure}

\subsection{特征值估计}
\label{solver:eigs}
基于Lanczos-based的特征值估计的过程如算法\ref{alg:lanczos}.
当特征值的估计是准确的时,即$\nu = \lambda_{min}$ , $\mu =\lambda_{max}$,CSI方法的收敛速度达到理论上的最优值。 
但是得到最大最小特征值$\lambda_{min}$ 和 $\lambda_{max}$ 的精确值是很困难的。
另外,由于系数矩阵$A$ 是分布在各个进程上的,因此不建议对它做任何形式的变形。  
为了能够利用POP的并行特性,我们使用Lanczos方法来构造 出一系列三对角矩阵$T_m (m=1,2,...)$,这些矩阵的最大最小特征值逐渐的向$M^{-1}A$的最大最小特征值逼近。
基于Lanczos方法的特征值估计的步骤如算法\ref{alg:lanczos}所示。 
\begin{algorithm}
\caption{ 基于Lanczos方法的特征值估计}
\label{alg:lanczos}
\begin{algorithmic}[1]
\REQUIRE  网格块$B_{i,j}$ 的系数矩阵$\textbf{B}$和随机向量$\textbf{r}_0$;\\
 //\qquad    \textit{所有进行并行执行}
\STATE $\textbf{q}_1 = \textbf{r}_0/||\textbf{r}_0||$;\quad $\textbf{q}_0=\textbf{0}$;\quad $T_0=\emptyset$;\quad $\beta_0 =0$;\quad  $\mu_0 =0$;\quad $j=1$;
\WHILE{$j<k_{max}$} 
%\STATE $\textbf{q}_1 = \textbf{r}_0/||\textbf{r}_0||$; \quad $\beta_0\textbf{q}_0=0$;\quad  $\tilde{\mu}_k =0$; \quad $j=0$;
%\FOR{$j = 1, 2,...,k$}
\STATE $\textbf{r}_j=\tilde{\textbf{A}}\textbf{q}_j-\beta_{j-1}\textbf{q}_{j-1}$;\quad $update\_halo(\textbf{r}_j)$;
\STATE $\tilde{\alpha}_j =\textbf{q}_j^T\textbf{r}_j$;\quad $\alpha_j=global\_sum(\tilde{\alpha}_j)$; 
\STATE $\textbf{r}_j=\textbf{r}_j-\alpha_{j}\textbf{q}_{j}$;
%\IF{$\textbf{r}_j \neq \textbf{0}$} 
\STATE $\tilde{\beta}_j = \textbf{r}_j^T\textbf{r}_j$; \quad $\beta_j=sqrt(global\_sum(\tilde{\beta}_j))$;
\STATE \textbf{if} $\beta_j == 0$ \textbf{then} \textbf{return}
\STATE $\mu_j = max(\mu_{j-1}, \quad \alpha_j+\beta_j+\beta_{j-1})$; \label{lanczos_gersh} \COMMENT{Gershgorin圆盘定理}\\
%\ENDIF
%\ENDFOR \\
%//\qquad    \textit{do on master node}
\STATE $T_k=tri\_diag(T_{k-1},\alpha_j,\beta_j)$; \quad $\nu_k = eigs(T_k,'smallest')$ ; \label{lanczos_tridiag} \COMMENT{三对角矩阵}
\STATE \textbf{if} $|\frac{\mu_k}{\mu_{k-1}} -1 |< \epsilon\quad\textbf{和}\quad|1- \frac{\nu_k}{\nu_{k-1}}|< \epsilon$ \textbf{then} \textbf{return}
\STATE $\textbf{q}_{j+1}= \textbf{r}_j/\beta_j$;\quad $j=j+1$;
\ENDWHILE
\end{algorithmic}
\end{algorithm}
 
在算法\ref{alg:lanczos}的第\ref{lanczos_tridiag}步中, $T_m$ 是一个三对角矩阵。它以 $\alpha_i (i=1,2,...,m)$ 为对角线元素,  $\beta_i (i=1,2,...,m-1)$ 为次对角元素。 

\[ T_{m} = tridiag\left(\begin{array}{ccccccc}
&\beta_1 && \bullet & &\beta_{m-1}&    \\
\alpha_1 & &\alpha_2 && \bullet &&\alpha_{m}\\
&\beta_1 && \bullet & & \beta_{m-1}&
\end{array} \right)\]

假设 $\xi_{min}$ 和 $\xi_{max}$ 分别为 $T_m$的最小的和最大的特征值。 Paige\cite{Paige1980235} 证明了以下结论

\begin{align}
&\lambda_{min} \le \xi_{min} \le \lambda_{min}+\delta_1(m) \\
&\lambda_{max}-\delta_2(m)  \le \xi_{max} \le \lambda_{max}
\end{align}

这里$\delta_1(m)$ 和 $\delta_2(m)$ 随着 $m$的增大而逐渐的向零逼近。 因此,系数矩阵 $A$ 的特征值的估计就可以转换成求解三对角矩阵  $T_m$的特征值。 
算法\ref{alg:lanczos} 中第\ref{lanczos_gersh}步应用了 Gershgorin 圆盘定理来估计 $T_m$的最大特征值,也就是 
\begin{align}
\mu = \max_{1 \le i \le m}\sum^m_{j=1}|T_{ij}|=\max_{1 \le i \le m}(\beta_{i-1}+\alpha_i +\beta_{i})
\end{align}
 
\begin {figure}%[htbp]
\centering
\includegraphics[height=6.5cm]{eigen_lanczos}
\caption[] { Lanczos迭代步数和特征值估计的关系\label{fig:lanczos_eigs}}
\end{figure}
\begin {figure}%[htbp]
\centering
\includegraphics[height=6.5cm]{scan_lanczos}
\caption[] { Lanczos迭代步数和求解器迭代步数的关系\label{fig:lanczos_iter}}
\end{figure}
在算法\ref{alg:lanczos} 中第\ref{lanczos_tridiag}中,采用了高效的QR算法\cite{ortega1963llt}来估计最小特征值$\nu$, 它的复杂度为$\Theta(m)$ 。 
如图\ref{fig:lanczos_eigs}和\ref{fig:lanczos_iter}所示,我们发现合适的选择Lanczos迭代的步数,我们能够得到对最大最小特诊一个比较好的估计,这个估计值能够使得CSI方法与预处理共轭梯度法有相近的收敛速度。  


\section{算法分析与比较}
\label{solver:Algorithm}
 
除了P-CSI,PCG和ChronGear方法的收敛性也同样依赖于系数矩阵的最大最小特征值。
这里,我们首先来挖掘一下影响求解器收敛的特征值的特征,然后给出ChronGear和P-CSI求解器的计算复杂度和收敛速度的理论结果。 


 



\subsection{收敛速度} \label{solver:Algorithm:convergence_rate}

 PCG和ChronGear方法的收敛速度都依赖于系数矩阵$A$的条件数。  
论文\inlinecite{dAzevedo1999lapack}证明了 ChronGear  和PCG在理论上收敛速度是相同的。 
在PCG和ChronGear方法中,每一步迭代的残差都有一个上界\cite{Liesen2004}
\begin{equation}
\frac{||\textbf{x}_k-\textbf{x}^*||_A }{||\textbf{x}_0-\textbf{x}^*||_A}  \le \min_{p\in \mathcal{P}_k, p(0) = 1 }\max_{\lambda \in \mathcal{S}} |p(\lambda)| \label{PcgConvergeRate}
\end{equation}
这里$\textbf{x}_k$ 表示第$k$步迭代后得到的解向量, $\textbf{x}^*$ 表示线性方程组的解析解  (即$\textbf{x}^* = A^{-1}b$),$\lambda$ 表示系数矩阵 $A$的特征值。
利用 第一类Chebyshev 多项式来估计这个最小最大值,可以得到 
\begin{align}
\label{chrongear_convergence}
||\textbf{x}_k-\textbf{x}^*||_A \le  2 (\frac{\sqrt{\kappa}-1}{\sqrt{\kappa}+1})^k ||\textbf{x}_0-\textbf{x}^*||_A
\end{align}
这里  $\kappa =  \kappa_2(A)$ 表示系数矩阵 $A$的条件数。


方程\ref{chrongear_convergence}表明,PCG方法收敛速度的理论上界取决于系数矩阵的条件数。 
PCG方法在求解良态的矩阵(特征值的条件数很小)时比求解病态的矩阵(特征值的条件数很大)时收敛速度更快。 
图\ref{fig:convergence_diag}给出了PCG和ChronGear方法在0.1度海洋模式中的真实的收敛速度。
如这幅图所示,相对残差随着迭代步数的增加而减少。
总共需要一百步左右的迭代步才能够得到一个相对残差为$10^{-13}$的解。
 

 
除了消除了全局归约操作,P-CSI方法的更一个优点是它有和PCG相同量级的收敛速度。 
P-CSI方法的收敛速度可以用残差的形式表示为
\begin{equation}
\textbf{r}_k = P_k(A)\textbf{r}_0 \label{eq:rPjr0}
\end{equation}
这里
$P_k(\zeta) = \frac{\tau_k(\beta-\alpha \zeta)}{\tau_k(\beta)}$ for $ \zeta \in [\nu, \mu]$ ~\cite{stiefel1958kernel} .
$\tau_k(\xi)$ 是Chebyshev多项式,它的表达式为  
\begin{equation}
\tau_k(\xi) =   \frac{1}{2}[(\xi+\sqrt{\xi^2-1})^k+(\xi+\sqrt{\xi^2-1})^{-k}]
\end{equation}
当 $ \xi \in [-1,1]$时, Chebyshev多项式的另一个等价表达式为$\tau_k(\xi) = cos(k\cos^{-1} \xi)$。从这个表达式可以清楚的看出来,当$| \xi | \le 1$时, $|\tau_k(\xi)| \le 1$。$P_k(\zeta)$ 是一个多项式,满足如下形式
\begin{equation}
P_k = \min_{p\in \mathcal{P}_k, p(0) = 1 }\max_{\zeta \in [\nu,\mu]} |p(\zeta)|
\end{equation}
%which is the theoretical bound of the convergence rate  in PCG \ref{PcgConvergeRate}.

假设$A= Q^T\Lambda Q$,这里 $\Lambda$ 表示以  $A$的特征值为对角元素的对角矩阵,而 $Q$ 表示以  $A$的特征向量为列的实正交矩阵。于是,我们可以得到
\begin{equation}
P_k(A) = Q^T P_k(\Lambda)Q = Q^T \left [\begin{array}{cccc}
P_k(\lambda_1) & & &\\
& P_k(\lambda_2) & &\\
& & \ddots &\\
 & & & P_k(\lambda_N)
\end{array} \right ] Q \label{eq:PjMA}
\end{equation}
Assume that the estimated largest and smallest extreme eigenvalues of coefficient matrix 
假设所估计的系数矩阵的最大最小特征值$\nu$和 $\mu$ 满足 $0 < \nu \le \lambda_i \le \mu$ ($i = 1, 2, \cdots, \mathcal{N}$),这时有$|\beta - \alpha \lambda_i| \le 1$, $|P_k(\lambda_i)| \le \tau^{-1}_k (\beta)$.
方程 \ref{eq:rPjr0} 和 \ref{eq:PjMA} 可以推出
\begin{equation}
\label{pcsi_convergence}
\frac{||\textbf{r}_k||_2}{||\textbf{r}_0||_2}  \le  \tau_k^{-1}(\beta) = \frac{2(\beta+\sqrt{\beta^2-1})^k}{1+(\beta+\sqrt{\beta^2-1})^{2k}} \le 2(\frac{\sqrt{\kappa'}-1}{\sqrt{\kappa'}+1})^k
\end{equation}
这里$\kappa' = \frac{\mu}{\nu}$.
 这个等式表明,当系数矩阵特征值的估计比较合理的时候(也就是$\kappa' =\kappa$) , P-CSI的收敛速度和PCG有相同的理论上界。


 
以上分析对于采用了预处理子的情形也同样适用。
唯一的区别是上面分析中提到的系数矩阵是$M^{-1}A$ 而不是$A$。
值得一提的是,PCG,ChronGear和P-CSI算法中预处理后的矩阵实际上是$M^{-1/2}A(E^{-1/2})^T$,它是对称的,并且和$M^{-1}A$ 有着相同的特征值\cite{Shewchuk1994}。
因此,这个预处理后的矩阵的条件数为$\kappa =  \kappa_2(M^{-1/2}A(E^{-1/2})^T)$。这个条件数通常情况下比原始矩阵 $A$的条件数要小。 
 $M$ 与$A$越是相近,$M^{-1}A$的条件数就越小。当$M = A$,$M^{-1}A$的条件数为$\kappa_2(M^{-1 }A ) = 1$。 

 当使用少数几个处理器核心来运行海洋模式分量时,全局通信并不是瓶颈,导致P-CSI方法和ChronGear方法的时间开销比较接近。
但是当采用很多的处理器核心来计算高分辨的海洋模式分量的时候,P-CSI方法的每一个步迭代就应该会比ChronGear方法快很多,因为ChronGear方法中的全局归约操作就会变成一个明显的瓶颈。

\subsection{计算复杂度}  \label{solver:Algorithm:complex}


 
当使用了EVP预处理时,计算的复杂变成了$\mathcal{T}_c = 37 \frac{\mathcal{N}}{p}$。因此P-CSI方法在使用EVP方法时一次求解过程的复杂度为
\begin{eqnarray}
\label{t_pcsiEvp}
\mathcal{T}_{pcsi-evp}=\mathcal{O}(\mathcal{K}_{pcsi-evp} (37\frac{\mathcal{N}}{p} +8\sqrt{\frac{\mathcal{N} }{p}} + 4\alpha))
\end{eqnarray}

以上式子可以看出来,使用了EVP预处理后,每一步迭代过程中的计算开销翻了一倍。
但是,一次求解的总体时间是下降。 
采用EVP预处理后,迭代步数$\mathcal{K}_{pcsi-evp}$会下降到原来的一半(参见图\ref{fig:convergence_diag})。
其结果是,在大核上运行时,计算时间变得非常的少,而最耗时的通信部分的次数也随着迭代步数的减少而减少了一半左右。





%%%%%%%%%%%%%%%%%%%%%%%%%%%%%%%%%%%%%%%%%%%%%%%%%%%%%%%%%%%%%%%%%%%
\section{实验} 
\label{solver:exp}
 
\subsection{理想实验}
为了验证第\ref{solver:Algorithm:convergence_rate}节中所给出的收敛的理论结果,我们在理想的试验设置中构造了一系列条件数不同的矩阵。
这里,我们不再采用全局固定的网格大小,而是采用均匀的经纬网格。
随着维度$\theta$的变化,沿着经度方向的网格大小为 $\Delta x_j  = \pi R \cos (\theta_j)$。
时间步长设置为$\tau = 10\frac{\Delta y}{c}$, 这里$c = 200m/s$ 表示重力快波的速度\cite{smith2010parallel}。 
我们用PCG和P-CSI方法分别结合单位阵预处理、对角预处理和EVP预处理来求解得到的这些方程。
实验中,EVP预处的块的大小为$5\times5$, 收敛条件设为$tol = 10^{-6}$。 
由于ChronGear 和PCG的收敛速度相同,因此这个图中PCG的收敛结果也是ChronGear的结果。
  

\begin{figure} 
\vspace{5pt}
\centering
\includegraphics[height=8cm]{iterationGridSize}
\caption[] {网格点数目和求解器的迭代步数之间的关系。\label{fig:iterationGridSize}}
\end{figure}

正如图\ref{fig:iterationGridSize}所示,随着问题规模的增加,系数矩阵变得越来越病态。
所有的求解器都需要更多的迭代步数才能得到相同的相对残差。
对于PCG和P-CSI,收敛速度会随着预处理方法的变化而变化。 
指定问题的规模,使用单位矩阵作为预处理的求解器需要的迭代步数最大,而使用EVP预处理的求解器所需要的迭代步数最小。 
这也说明,使用EVP预处理,矩阵的形态会比使用单位阵预处理或者对角预处理后的形态要好。 
图中还说明,在预处理方法相同的情况下,P-CSI方法的收敛速度要比PCG的慢一些。 



 
\subsection{CSI方法的性能}
 
我们先测试一下采用对角预处理子时,PCG,ChronGear和P-CSI三个求解器的性能。
图\ref{fig:convergence_diag}给出了不同的正压求解器的收敛速度。
图中横轴表示迭代的步数,垂直轴表示求解器在对应迭代步的残差。 
尽管ChronGear方法改变了PCG方法的计算顺序,PCG方法和ChronGear在每一步的迭代速度几乎一致。
P-CSI方法在开始和最后的迭代中收敛速度比较慢,而在中间的迭代步骤中收敛速度变快。
这与系数矩阵的特征值的分布有关系。 


\begin {figure}
\vspace{10pt}
\centering
\includegraphics[height=8cm]{Convergence_diag.eps}
\caption[] {不同求解器的收敛速度。\label{fig:convergence_diag}}
\end{figure}



%----------------------------------------------------------------------------
 
为了证明CSI方法并没有在海洋模式POP中引入误差,我们在清华大学的探索100集群(具体描述请参见第\ref{solver:pcgComplex}节)上对1度POP做了一组实验, 分别使用PCG和CSI方法作为正压模态的求解器运行一段时间。 
表格\ref{tab:err}给出的是这两个版本得到的结果的海表高度的差别。
采用CSI和PCG的两个版本之间的平均误差相对于海表高度的绝对值非常的小。
有意思的是,我们注意到,两者之间最大的误差主要出现在海岸线上。 这些地方比较尖锐的边界很容易造成差分格式的不稳定性。
随着实验时间的延长,两个版本之间的误差也随着增大。 这并不是说求解器之间的误差在变大,而是由于海洋模式中湍流的累积效应将最初的小误差逐渐积累放大。 

\begin{table}
\centering
\caption[] {采用PCG和CSI作为正压求解器的两个版本之间的海表高度的比较   \label{tab:err}}
\begin{tabular}{l l@{\quad}l@{\quad}l@{\quad}l} 
\toprule
模拟时间   & 一步  & 一天    & 一个月 &三个月\\
\hline
\multicolumn{1}{l}{总步数 } &\multicolumn{1}{c@{\quad}}{  1} &\multicolumn{1}{c@{\quad}}{  45} &\multicolumn{1}{c@{\quad}}{ 14053}	&\multicolumn{1}{c@{\quad}}{40800}\\
%\hline
最大相对误差 & 1.5016E-3&2.2181E-5& 1.2885E-2&1.4114E-1\\
%\hline
平均相对误差 &3.0223E-6&5.2424E-7& 2.6125E-5&7.8872E-4\\
\bottomrule
\end{tabular}
\end{table}


 
为了测试算法的可扩展性,我们在中国的国家超算中心的神威蓝光超级计算机上测试了0.1度的海洋模式POP(3600 $\times$ 2400个网格点)。
神威蓝光是由8,704个SW1600处理器由40Gb 无限带宽网络连接组成。
每一个处理器包含16个1.1GHz的处理器核心以及16GB的内存。 


我们分别测试了PCG 和 CSI两个版本的求解器在不同的处理器核数(100到 15,000 )和不同的收敛条件( $\epsilon = 10^{-8}$ 到 $\epsilon = 10^{-16}$)时的性能。
海洋模式 POP 中的收敛准则是 $||r||_2<\epsilon \bar{a}$,这里 $\bar{a}$ 表示总面积的均方根。 
在单独版本的POP中,默认设置为 $\epsilon = 10^{-12}$。 
如图\ref{fig:scale}所示, PCG 和 CSI 在小于  1,000核时可扩展性都很好。但是当使用超过1,000 核时, CSI相对于PCG的优势就很明显了。 
当收敛条件为 $10^{-12}$时,在100核上CSI的执行时间只有PCG的87\%, 这个比例在15,000核上下降到了 21\% 。
但是PCG方法对于收敛条件没有CSI方法敏感。 
当时收敛条件从$10^{-8}$变化到$10^{-16}$ 时, PCG方法的迭代步数从原来的20步增长到281步,但是CSI方法的迭代步数从33步增长到了1,434步。 
但是,PCG方法在收敛速度上的优势并没有使得PCG方法完全的比CSI方法快。 
只有当收敛条件是最为严格的$\epsilon = 10^{-16}$并且当使用的核心数少于1000核时,PCG的执行时间才会比CSI的短。
其他情况下, PCG方法中的全局归约操作的开销并不能被迭代步数的减少给抵消掉。
在1,500核上,当收敛条件$10^{-8}$ 和 $10^{-12}$时,PCG的执行时间是CSI的4.7倍, 而当收敛条件为$10^{-16}$时,这个倍数仍然高达3.3. 
由于全局归约操作的瓶颈, 当使用的核心数大于1,000时,PCG求解器的执行时间随着所使用的核数的增加而增加。 
相对应的,CSI算法一直到10,000核以上都能保持比较好的可扩展性。 
P 

\begin {figure}
\centering
\includegraphics[height=6.5cm]{scale_pcg_csi}
\caption []{ 0.1度POP中PCG和CSI方法的可扩展性 \label {fig:scale}}
\end {figure}
%Fig.1(b) shows runtime ratio of barotropic 和 baroclinic modes of POP. Baroclinic mode deals with a three dimensional dynamical process which occupies a major proportion of computation of POP. It dominates in POP when process is few 和 computation time is much larger than communication. However, when using thous和s of cores, runtime of baroclinic parts decreases linearly while on the opposite, runtime of barotropic increases because of global reduction.


% %%%%%%%%%%%%%%%%%%%%%%%%%%%%%%%%%%%%%%%%%%%%%%%%%%%%%%%%%%%%%%%%%%%
% \section{相关工作} 
% \label{solver:rel}
% %----------------------------------------------------------------------------
 

 

\section{本章小结}
\label{solver:Conclusion}

目前已经有很多的工作在研究如何提高海洋模式POP中正压模态的性能。 
但是他们的大部分都没有从根本上解决POP正压模态中的可扩展性瓶颈。 
这篇文章提出了一个评估海洋模式POP正压模态的模型, 通过这个模型,我们对预处理共轭梯度法的可扩展性进行量化。 
这个模型从理论上证明了CSI的优越性。 
我们将CSI方法实现在海洋模式POP中并且得到了比原始的预处理共轭梯度法更好的可扩展性。 
总之,这篇文章提出了一个比传统的预处理共轭梯度法更有前景的求解椭圆方程的方法。 



\chapter{预处理}
\label{cha:precond}

\section{本章概述}

\section{背景和动机}
\label{sec:precondBackgroud}

\section{预处理方法比较}
\label{sec:precond1}

\section{块预处理方法}
\label{sec:precond2}

\subsection{误差向量传播方法}
\label{sec:precondEVP}

\section{试验方法和结果分析}
\label{sec:precondExp}

\section{本章小结}
\label{sec:precondConclusion}

\chapter{基于集合模拟的海洋模式一致性检测}
\label{cha:verify}

\section{本章概述}
\label{verify:intro}

公共地球系统模式CESM是一个应用广泛的全耦合气候模拟程序\cite{hurrell2013community}, 它的模拟结果经常被提交到政府间气候变化委员会的评估报告中\cite{stocker2013ipcc}。
CESM由多个分量模式耦合而成,其中包括大气、海洋、海冰和陆地等的分量模式。 
本文专注于CESM的海洋分量模式并行海洋模式POP。  
POP求解的是一个采用静力平衡近似和Boussinesq近似的海洋动力学三维原始方程,它描述的海洋模式过程在时间和空间上的尺度都很大。  

CESM中的海洋分量模式并行海洋模式 (CESM-POP)被广泛的应用于气候研究。 
目前有很多关注如何提高CESM-POP性能的工作,比如提高数值算法的性能,改进参数化方案,移植到新的架构上,以及增加并行度等。 
由于海洋模式动力过程混沌的特性,即使对代码做一些细微的改动,也无法保证修改后的模式得到的结果与原始结果是二进制一致的。 
而且,判断模式中的改动是不是可以被接受的(也就说是不是在统计的意义上与原始结果一致)并非易事。 
最近的一个工作显示,对大气模式的数据利用基于集合模拟的统计方法取得了很好的效果。
这个基于集合模拟的统计学一致性检测工具的核心是利用集合模拟结果的可变性的量化结果来作为评判未来模拟的标准,并且利用统计学上的区分度来做判断。 

由于海洋模式和大气模式在动力学和时间尺度上有着明显的区别,
本章采用了一种结合集合模拟的平均状态和方差的统计学方法来对海洋模式的模拟结果进行检测。 
具体来说,新的检测方法利用CESM-POP集合模拟的结果的统计学分布来计算新的模式结果在每一个点上的标准分。 
标准分大于某一个给定值在所有点中所占的比例,可以反映出新的模式结果与集合模拟结果的统计学差别。 
这里,集合模拟成员的个数和成员的挑选都很重要。
实验表明,新的海洋模式POP的集合一致性检测工具(POP-ECT)能够区分那些统计学上明显与集合模拟的结果一致的模拟结果和那些明显不一致的模拟结果。 
同时,这个检测工具提供了一个简单、客观和系统的方法,来检测CESM-POP是否有硬件或者软件层引入的错误。
这个工具进一步提升了CESM-POP代码的质量保证。 

\section{气候模式的正确性检测}
\label{verify:Backgroud}

最近CESM-POP中的很多进展都是关心如何减少计算开销\cite{yong2015}, 但是数值模拟程序的中的任何开发都需要软件质量保证,来确保不会在程序中引入错误。 
这种维护科学程序的可信性的质量保证,对于气候模式来说尤为重要。 
因为气候模式的模拟结果经常影响到关系国计民生的政策方针的制定\cite{carson2002, easterbrook2011}。 

  
气候模式,比如CESM,通常都代码量大并且程序复杂。 
模式中过多的配置选项使得很难使用穷尽列举的方式测试它们 \cite{clune2011, pipitone2012}。 
另外,由于气候模式本身的混沌性,判断一个新的模拟结果与原始结果之间的差别是由某个错误引起的还是由于模式本身的不确定性引起是一件非常具有挑战的事情。 
在模式的初始条件或者中间结果中引入浮点数精度级别的随机扰动都会使得最后的结果产生比较大的偏差。
CESM-POP中的一些新的成果,尤其是那些提高性能的成果(比如利用新的异构加速技术和改进数值方法等),通常都会导致新的模式的输出结果与原来的结果不是二进制一致的。 
对于CESM-POP来说,即使是在同一个超级计算机上采用不同的处理器核数来计算,得到的结果都不会是二进制一致的。 
如果可以直接评估CESM-POP的海洋数据的气候学上一致性,将会极大的促进模式的的发展,同时使得模式能够更加灵活的应用新的软件或者硬件技术。 

本节将介绍两种气候模式正确性检测方法,并且利用求解器的引入的误差作为检测标准,来看其中这两个方法是否有效。
由于海洋动力过程本身具有混沌性,即使在正压求解中引入一个浮点数舍入误差精度级别的扰动,也可能使得最终的模式结果与原始结果截然不同。 
既然不能保证引入新的求解器之后所得到的结果与原始结果是二进制一致的,
那么在将新的P-CSI求解器和EVP预处理正式的加入到海洋模式POP的版本中之前,需要证明加入它们之后不会影响模拟结果的正确性,更不能改变所得到的气候的状态。
这正是本文设计一致性检测工具的初衷。 


\subsection{海洋模式的均方根误差检测}
\label{verify:premethod}
 

目前海洋模式POP中可用的正确性测试工具(称之为POP-RMSE),是一个比较简单的评估CESM-POP程序是否成功的移植到了一个新的机器上的检测。 
这个工具的主要目标是发现新机器上硬件或软件层的问题。 
这个工具的测试中,首先将特定设置的实验在新的机器上运行五个模拟天,然后将其输出结果和美国国家大气研究中所提供的标准的数据集相比较。
这里需要计算两个结果的海表高度场的均方根误差(RMSE)。 
比如对于两个包含$\aleph$个值的数据集合$X_0$ 和$X_1$,它们的均方根误差可以表示为
 \begin{equation*}
 RMSE(X_0, X_1) = \sqrt{\frac{1}{\aleph}\sum_{i=1}^\aleph(X_1(i)-X_0(i))^2}.
 \end{equation*}
然后,利用得到的均方根误差的增长曲线与给定的两个样例所得到的曲线相比较。 
这两个样例中,一个是原始的配置,另一个是在求解器中采用比默认收敛条件要高一个量级的收敛条件。 


\begin{figure}%[!t]
\begin{center}
\includegraphics[height=7cm]{temp_rmse}
\end{center}
\caption[] {采用不同收敛条件的1度海洋模式POP模拟结果与使用最严格收敛条件得到的结果的温度场的每个月的均方根误差(RMSE)。}
\label{fig:ssh_rmse_t}
\end{figure}

POP-RMSE对于评估CESM在新的机器上的输出结果是很方便的, 但是它比为大气模拟数据设计的CAM-ECT的表达能力要弱得多。 
比如说,POP-RMSE无法评判CESM-POP中最新的线性求解中的改动所带的微小的气候状态的改变\cite{yong2015}。 
第\ref{cha:precond}节将默认的预处理共轭梯度法求解器替换成预处理的Chebyshev迭代算法,以此来提高CESM-POP的计算性能。
尽管新的P-CSI求解器在高分辨率模式中的通信开销比原始的预处理共轭梯度法的要小很多,但是在将这个新的求解器应用在海洋模式中之前,必须证明它不会对海洋的模拟结果造成负面的影响。 
因此,为了验证POP-RMSE是否能够检测出求解器在不同时间所引入的误差,我们搜集了1度分辨率的CESM-POP在不同的收敛条件($10^{-10}$ 到$10^{-16}$)下36个月的运行结果。
图\ref{fig:ssh_rmse_t} 给出的是最紧的收敛条件的实例 ($10^{-16}$) 和其他收敛条件的实例的结果的温度场之间的均方根误差。 
尽管这些例子采用不同的收敛条件,图\ref{fig:ssh_rmse_t} 中很难体现出来这些由于求解器误差所带来的影响\cite{yong2015}。 



当海洋模式POP移植到新的机器上时,也会遇到类似的问题,即不同的机器上得到的模拟结果不是二进制一致的。 
目前海洋模式POP中提供了一个比较简单的评估地球系统模式CESM在新的机器上的运行结果(这些结果可能是有软件环境或者硬件环境引入的错误的)的方法, 但是给定的1度的POP实验表明它并不能检测和评估求解器引入的误差。 
这些实验中,正压求解器的收敛条件分别为$10^{-10}$ 到$10^{-16}$之间的值(默认为 $10^{-13}$)。这些实例的模拟结果被用来计算与使用最严的收敛条件($10^{-16}$)的模拟结果均方根误差。 

图 \ref{fig:ssh_rmse_t} 中给出了采用不同的收敛条件得到的模拟结果中的温度场的每个月的均方根误差。 
值得一提的是,为了凸显线性求解器的影响,实验结果只考虑公海区域网格点上的值(POP在几个边缘海内的模拟并不算好)。 
可以很明显的看出,改变求解器的收敛条件所引入的误差并没有在最后的温度场中得到体现(也没有在速度和海表高度等诊断场中的得到明显的体现)。 
收敛条件越松,求解器中引入的误差就会越大。
因此,使用最松的两个收敛条件的($10^{-10}$和$10^{-11}$)的实例的运行结果所计算出来的均方根误差理论上应该比其他实例的要大一些。 
但是,从图\ref{fig:ssh_rmse_t} 中可以看出, 
在第12个月到第20月之间,收敛条件为$10^{-10}$的例子的均方根误差在所有实例中是最小的。 
这与理论推测不符,表明POP-RMSE并不能检测出求解器所引入的误差。

\subsection{基于集合模拟的一致性检测}
\label{verify:enseble}

最近由Allison等人\cite{baker2015}研发的CESM集合模拟一致性检测工具(CESM-ECT),通过一个新的基于集合模拟的工具解决了检测气候模式输出结果的难题。 
这个工具能够评判一个新的气候模拟结果(比如引入了软件或者硬件的改动)与以原始结果(没有任何改动的版本)为基础构造的被认可的集合模拟的结果在统计学意义上是否是一致的。 
但是他们的CESM-ECT工具只评估了CESM中大气分量模式--公共大气模式(CAM)的变量,并且其实验并不是完全耦合的CESM配置(比如说其中海洋分量并不是采用的并行海洋模式POP,而是海洋气候数据模式。这个模式只能够提供海表温度数据,但是不能响应气候分量模式提供的强迫)。 
为了简洁,本文将为CESM的通用的集合模拟统计一致性测试工具称之为CESM-ECT,将应用于公共大气模式分量的检测方法称之为CAM-ECT。
CAM-ECT是CESM-ECT工具集中的一个模块。 

值得注意的是,直接将CAM-ECT方法应用于海洋数据并不可行,因为海洋和大气在动力学、空间尺度和时间尺度上的差别都很大。
比如说,海洋模式中涡旋的空间尺度要比大气中的小一到两个量级,海洋,尤其是深海中运动的时间尺度比大气中的要慢很多个量级。
因此,本章专门为海洋模式中的一致性检测开发了一个新的方法,并称之为POP-ECT。 
尽管这个新的POP-ECT工具也是利用CESM的集合模拟的结果来评估模式的不确定性,它和CAM-ECT是有本质区别的。 
POP-ECT的统计过程中考虑到了海洋中不确定性的空间分布特点。另外,由于海洋达到全球准稳态所需要的时间比大气需要的时间长得多,因此,在POP-ECT中需要采用与CAM-ECT中不一样诊断时间。 
同时,海洋模式中的可用的诊断变量的个数比较少,促使POP-ECT采用跟大气模式不一样的诊断方案。 

这里首先介绍一下Allison等人提出的针对CAM的一致性检测工具,然后引出它在海洋模式中的移植版本POP-RMSZ,并且重新使用求解器的测试样例来验证这个方法是否有效。

\subsubsection{公共大气模式的一致性检测 }
\label{verify:CAM}
Allison等人\cite{baker2015}提到的CESM-ECT检测工具利用集合模拟的结果来对CESM模式的气候态的内在的不确定性进行量化,进而将新的模拟结果(由于软件、硬件或者其他不能保证二进制一致的改动引入的新的结果)与这个集合模拟结果的分布做对比。
这个想法的关键是,如果新的模拟结果的输出数据与集合模拟的结果在统计学上是不可以区分的话,那么这个新的结果就认为是与原始结果“一致的”。统计学上的一致性是模式代码正确性验证的质量保证方面的关键因素\cite{oberkampf2010}。
CESM-ECT方法首先被应用于公共大气模式CAM分量的历史输出数据,最终形成了CAM-ECT工具集。
CAM数据很直观的就被当成了验证的首要目标,因为大气中扰动传播的时间尺度要比在CESM其他分量中的短很多。而且,CAM包含有很多不相关的全局变量。 


 
正如Allison等人\cite{baker2015}所描述的那样,CAM-ECT中使用的集合模拟结果是在可信的机器上使用CESM的被验证过的版本和配置运行得到的。 
这个集合模拟包含有151个长达一年的模拟结果,这些结果之间唯一的不同是在初始的大气温度场中加入了$\mathcal{O}(10^{-14})$ 量级的扰动。 
集合模拟的输出结果只包含了特定变量在每个点上的年平均值。这些变量的个数记为$N_{var}$,它们可以反映出整个大气场(默认情况下 $N_{var}=120$)。这个CAM-ECT工具集是对集合模拟结果中的每一个变量的全球的面积加权平均用主成分分析(PCA)的方法,生成了一个反应集合模拟特征的统计学分布。
这个主成分(PC)分数的分布被保存下来,之后用来与新的模拟实验作对比。 
基于落在指定置信区间(通常取95$\%$)之外的PC分数的个数,可以确定少数的几个新的实例(通常是3个)是否通过测试。 
这里可以调试与是否能通过测试相关的参数,从而达到想要的误报率。


\subsubsection{海洋模式均方根标准分检测}\label{verify:rmsz}



正是因为现成的简单的均方根误差测试并不能检测出来模拟结果的气候态是否被改变,本章提出了一种新的基于统计的方法POP-RMSZ,并且用这种方法来评估海洋正压求解器的影响。 
这个方法的灵感来源于Allison等人的工作\cite{baker2014methodology},他们为了验证公共地球系统模式CESM的公共大气模式(CAM)中的数据被压缩之后是否仍然有效,并不是仅仅使用一次模拟的结果,而是利用集合模拟的结果来评估无法做到二进制一致的改动。 
POP-RMSZ方法和Allison所提到的方案很类似\cite{baker2014methodology}。
首先,运行一系列的实例。这些实例除了在初始的海洋温度场中加入了$10^{-14}$量级的随机扰动之外,和默认的配置完全相同。 
这个量级的扰动理论上不会造成模拟结果的气候状态发生变化。 
后面的实验结果表明,40个这个样的实例就足以表现出海洋模式自身的扰动性。
由于海洋模式中需要模拟的海洋运动的时间尺度比大气中的长, 这个集合模拟中的实例也模拟了比Allison等人在大气模式中更长的模拟时间。 
值得一提的是,POP-RMSZ最终决定选择三维的温度场作为评估对象(而不是像现有的评估方式那样选择二维的海表高度),因为温度场是找到不同实例之间差异的最好的诊断变量。 
 
 
POP-RMSZ用如下的流程来判断一个新的结果是否和指定的集合模拟结果具有一致性。 
首先生成了一个包含有40个运行了36个月的模拟的集合,这些实例与原始实例的差别在于在初始的海洋温度场引入了一个$\mathcal{O}(10^{-14})$量级的扰动。 
然后计算新的模拟结果 $\tilde{{X}}$和这个集合模拟的结果用如下的方式计算均方根标准分。 
假设集合$E$所包含的成员的个数记为 $N_{ens}$,此时,对于每一变量$X$,在每一个点$i$,都存在一个包含$N_{ens}$个数的数列。
定义点$i$上的集合模拟结果中对应的数列的所有元素的平均值和标准方差分别为$\mu_i$ 和 $\sigma_i$。 
由于总的网格点数为$\aleph$,那么对于变量$\tilde{{X}}$,新的模拟结果与集合模拟结果$E$之间的均方根标准分定义为
\begin{equation}
 RMSZ(\tilde{X}, E) = \sqrt{\frac{1}{\aleph}\sum_{i=1}^\aleph(\frac{\tilde{x}_i -\mu_i}{\sigma_i})^2}. 
\label{e:rmsz}
\end{equation}


\begin{figure}%[!t]
\begin{center}
\includegraphics[height=7cm]{temp_rmsz}
\end{center}
\caption[] {采用不同收敛条件的1度海洋模式POP模拟结果中的温度场的每个月的均方根标准分(RMSZ)。黄色的区域表示集合模拟中40个实例结果的均方根标准分的分布。 }
\label{fig:ssh_rmsz_t}
\end{figure}

为了验证POP-RMSZ是否有效,本节利用它重新评估一下不同的求解器收敛条件的影响。
图\ref{fig:ssh_rmsz_t} 给出了与图\ref{fig:ssh_rmse_t}中相同的五个收敛条件的模拟结果的均方根标准分数。
这里,由最宽松的收敛条件引入的误差变得非常的明显。  
图\ref{fig:ssh_rmsz_t} 可以看出,与简单的均方根误差不同, 这个新的基于集合模拟的方法能够发现收敛条件越松,结果中的误差就越大。 
现在,使用最松的两个收敛条件的例子得到的均方根标准分数也是最大的,可以明显的与集合模拟的结果区分开来。 

最后,这个基于集合模拟的判断指标还可以用来来评估一下本文所提出的新的P-CSI求解器。 图\ref{fig:ssh_rmsz_t}的结果表明,使用P-CSI求解或者使用最紧的两个收敛条件所得到的结果与集合模拟的结果是相容的。 
因此,新的求解器并没有在模式中引入气候模式意义上不可接受的误差,它在CESM-POP中是可以接受的。  
 
由于海洋中的时间尺度比大气中的要长一些, 直观上讲,CESM-POP中统计学一致性检测工具需要的集合模拟的模拟时间应该会更长一些。 
值得一提的是,图\ref{fig:ssh_rmsz_t} 似乎又有悖于的这个假设。 
可以看出来,当收敛条件足够严格的时候(收敛条件小于$10^{-12}$时), 均方根标准分在模式运行几个月之后随着时间的推移而逐渐减小。
这表明,我需要进一步的研究一致性检测所需要的模拟时间的长度。 
实际上, 图\ref{fig:ssh_rmsz_t}可以看出,这个给定的1度分辨率的海洋数据是相对确定的,在求解器中引入的误差足够小(也就是收敛条件足够小的时候)初值场中的误差在模拟了一年之后逐渐的耗散掉了。



%----------------------------------------------------------------------------
\section{海洋模式一致性检测工具}\label{verify:ECT}


第\ref{verify:rmsz}节中给出来的求解器验证性工作很有启发性,而且完全满足目前的需求。 
但是它仍然有几个遗留问题没有解决,这促使本文研发了一个类似于CAM-ECT这样的更强大的针对CESM-POP数据的评测工具。 
第一,一个更通用的海洋模式POP的验证性工具首先考虑的是海洋中空间的变化特点,这个特点比大气中的要更加突出。
CAM-ECT和第\ref{verify:rmsz}节中的均方根标准分检测方法是利用空间的平均值的差别来作评估,目前还不清楚这样的策略是否有足够的能力来检测海洋中的正确性表达问题以及模式本身的误差。
第二,由于CESM-POP的相互独立的诊断性变量要比大气模式CAM少很多,因此一致性检测的量化指标需要进一步的探讨。
第三,不同的衡量指标的选择促使本章进一步检查集合模拟的大小应该如何选择。
在均方根标准分检测方法中,四十个实例组成的集合模拟就足以检查出线性求解器中的误差,而这个集合模拟的大小并没有在更通用的环境中进行实验验证。
而且,这个大小要比CAM-ECT中使用的默认大小151要小很多。
最后,由于海洋和大气中的时间尺度不同,集合模拟的模拟时间究竟多长最为合适还有待研究。 
由于海洋中的时间尺度比大气中的要长一些, 直观上讲,CESM-POP中统计学一致性检测工具需要的集合模拟的模拟时间应该会更长一些。 
 
在CESM-ECT工具集中扩充CESM-POP一致性检测的工具(记为POP-ECT),需要一个能够反映海洋模式自身不确定性的恰当的集合模拟结果,并且需要一种针对海洋自身特点的方法。
这里首先讨论一下构造集合模拟的理论依据以及如何构造这个集合模拟结果,然后介绍一下新的测试工具的流程。 


\subsection{初始扰动与强迫扰动}\label{verify:pertub}

由于POP-RMSE方法并不能明显的检测出不同的收敛条件所带来的影响,我们在2015年的超级计算大会发表的文章中采用了一种集合模拟的方法来测试海洋模式POP的输出数据\cite{yong2015}。 
这个工作是基于Allison等人\cite{baker2014methodology}中提出的测试数据压缩的影响的方法(这也是CESM-ECT方法的前身)。
需要注意的是,Allison等人提到的集合模拟是通过对初始值进行扰动得到的。 
但是,改变海洋模式中正压模态线性求解器的收敛性判据或者像我们的文章\cite{yong2015}中所提到的引进新的求解器,都可以近似为强迫场的扰动。 
很有必要先研究一下,仅仅通过初始场的扰动所得到的结果状态集是否能够表示在每一个时间步上对强迫场进行扰动所生成的状态。
这里,仿照Caya等人\cite{caya1998}的做法,海洋模式POP的时间迭代过程可以用如下的简单方程来表示:
\begin{equation}
\frac{\partial X(t)}{\partial t} = \alpha X(t) +G
\label{e:1}
\end{equation}
这里 $G$ 表示一个固定的强迫项,  $\alpha$是一个常数,  $X(t)$是关于时间变量的函数, $X(0) = X_0$。 
方程 (\ref{e:1}) 的解析解为:
\begin{equation}
X(t) = (X_0 +\frac{G}{\alpha})e^{\alpha t } -\frac{G}{\alpha}
\label{e:2}
\end{equation}
这个例子里面,假设$\alpha$是一个纯虚数,这样$X(t)$就是一个震动波(这里不考虑减振机制)。 

 
首先,假设在初值中加入扰动,即
\begin{equation*}
X_0' =X_0+P
\end{equation*}
这里 $P$ 是一个常量。方程 (\ref{e:1}) 在采用扰动后的初始值之后的解析解可以表示为 
\begin{equation}
X_P(t) = (X_0+\frac{G}{\alpha}+P)e^{\alpha t } -\frac{G}{\alpha}
\label{e:Pan}
\end{equation}
也可以表示为
\begin{equation}
X_P(t)-X(t) = Pe^{\alpha t }
\label{e:Perr}
\end{equation}
方程 (\ref{e:Perr}) 表明原始结果和扰动初始条件后得到的结果的差是振荡的。
下面考虑当在强迫场中加入一个扰动,即
\begin{equation*}
G' = G + F
\end{equation*}
这里 $F$ 是一个常量。  方程(\ref{e:1})加入带扰动的强迫项后,其解析解可以表示为
\begin{equation}
X_F(t) = (X_0 +\frac{G}{\alpha})e^{\alpha } -\frac{G+F}{\alpha}
\label{e:Fan}
\end{equation}
也可以等价于
\begin{equation}
X_F(t)-X(t) = \frac{F}{\alpha} 
\label{e:Ferr}
\end{equation}

\begin {figure} 
\centering
\begin{subfigure}[t]{1\textwidth}
\centering
\includegraphics[height =5cm]{pert-force1}
\caption{加入初值场扰动(P)或强迫场扰动(F)后得到的解析解。\label{fig:1Danalytical1}}
\vspace{10pt}
\end{subfigure}
\begin{subfigure}[b]{1\textwidth}
\centering
\includegraphics[height =5cm]{pert-force2}
\caption{加入初值场扰动(P)或强迫场扰动(F)后得到的解与原始解的差。\label{fig:1Danalytical2}}
\vspace{10pt}
\end{subfigure}
\caption{方程(\ref{e:2}) 在给定的初值场扰动(P)和给定的强迫场扰动(F)下对应的解析解,以及这个结果与原始解之间的差别。}
\label{fig:1Danalytical}
\end {figure}

式 (\ref{e:Ferr}) 表明强迫场中加入固定的干扰项后对结果产生的影响也是固定的。
因此,可以用方程(\ref{e:Perr})的初始场中引入扰动所带来的误差的量级来衡量在方程(\ref{e:Ferr})的强迫场中引入扰动所带来的误差。 
举个例子,在图\ref{fig:1Danalytical}中给出一个初始扰动和强迫扰动对方程(\ref{e:2})的最终结果的影响。
在这个例子中,强迫项的扰动所引起的误差比初始条件扰动所引起的误差还小。
图\ref{fig:1Danalytical1}显示的是方程(\ref{e:Pan}) 和 (\ref{e:Fan})所给出的解析解,图\ref{fig:1Danalytical2}显示的方程 (\ref{e:Perr}) 和 (\ref{e:Ferr}) 所给出的是不同的$P$和$F$的值得到的结果的误差。
这个实验中,所有的四个扰动在原始的(也就是没有加入任何扰动)结果上取得的效果是相似的。 
由强迫场上的扰动(F)所导致的误差是固定的,并且比初值场上的扰动(P)带来的误差小一个量级。
可以看出,当初始值扰动的组合足够多时,它们所带来的误差的集合,可以在任何时间覆盖强迫场所带来的误差。 
因此,选择使用基于由扰动海洋模式初始的温度场而得到的集合模拟结果,来评估求解器或者其它模拟过程中所引入的不确定性是合理的。



\subsection{POP数据集合的构造}\label{verify:createEns}

 
POP-ECT通过将CESM-POP的输出数据与很多被认可的海洋状态组成的集合作对比,可以评估输出数据与数据集合在统计学上的差别是否显著。
如果显著,则可以认为这个输出数据与给定的数据集合是不同的气候状态。 
因此,将CESM-ECT应用到CESM-POP数据的第一步就是要构造出一个合适的数据集合。 
很明显,数据集合的组成对于测试的有效性来说至关重要,而且这些数据必须在一个被认可的机器上采用被认可的CESM版本来生成。
这里,与CAM相比,海洋模式的诊断变量要少很多,相互独立的诊断变量有:温度(TEMP),海表高度(SSH),盐度(SALT), 以及与模式所用网格相对应的经向和纬向速度(分别为UVEL和VVEL)。
这五个变量中,只有海表高度SSH是二维变量,其他都是三维变量。 

 
在POP-ECT中,构造了一个由$N_{ens}$模拟结果组成的数据集合$E$, 记为 $E =\{E_1, E_2, \dots, E_{N_{ens}} \}$。
这些模拟结果与指定的默认版本之间的唯一不同是在初始的海洋温度场中加入$\mathcal{O}(10^{-14})$量级的随机扰动。 
第\ref{verify:pertub}节已经通过一个简单的分析,形象给出了这种集合的合理性。
这个初值扰动的大小和双精度浮点数的舍入误差在一个量级,因此它理论上不会造成气候上的改变。
这个数据集合的成员的数量必须足够的多,才能够生成一个比较有表达力的分布。
但是同时,这个数量又要足够的小,因为增加一个成员就等于增加一份计算开销。 
稍后,第\ref{verify:ens}节将讨论数据集合的大小的选择。
这里的讨论和之后的实验都采用 $N_{ens} \;=\; 40$这个配置。
实验表明,如果只是用来做一致性测试,这个大小的集合足以充分的反应海洋的内在的不确定性。
模拟数据集合包含有在每个点上五个POP诊断变量(TEMP, SSH, SALT, UVEL和VVEL)的$T$个月份的月平均数据。
每一个这样的变量$X$都包含有$\aleph$个网格点,记为${X} = \{ x_1, x_2, \dots, x_{\aleph}\}$, 这里$x_i$ 表示在网格点$i$上变量$X$的月平均值。
 
CESM-ECT方法的下一步是量化数据集合的分布特征,进而为新的模拟结果提供评估依据。
这个数据集合的统计特征描述被存在一个数据集合汇总文件,并且这个文件会被贴上生成这些模拟结果的CESM自带的软件标签。
当这个汇总文件一旦生成, 数据集合中$N_{ens}$个模拟结果的历史文件就不需要继续存储了。
生成大气模式CAM一致性检测的汇总文件时,会计算每个变量在每个点上的可用的年平均数据的全球面积加权平均值。
这些计算结果就构成了每个变量的$N_{ens}$个全球平均。 
然而,这个方法采用的全面积加权平均不能很好的应用在海洋模式数据中,因为海洋模式的不确定性相比于大气模式数据来说,在不同位置的网格点上差别更大一些。也就是说,在海洋的不同的区域上,扰动所造成的误差的量级差别很大。
因此,新的检测工具需要更多的考虑到空间上的分布特征。 

\begin {figure} 
\centering
\includegraphics[height =20cm]{gimp_all_sst_std.png}
\caption{第1, 12, 24和36个月份上,海表温度在数据集合上的标准方差。}
\label{fig:SST_STD_all}
\end {figure}

比如,考虑一个由$N_{ens} \;=\; 40$个使用1度分辨的海洋模式POP作为海洋模式分量的CESM模拟$T\;=\;36$个月得到的模拟结果组成的数据集合。 
图 \ref{fig:SST_STD_all}中给出的是数据集合的1,12,24和36个模拟月份中海表温度(SST)的标准方差的空间分布。
注意到这里海表温度SST仅仅只是三维变量TEMP的最上面一层(更加准确的说,是10米的海洋表层)。
图 \ref{fig:SST_STD_all}可以看出来,标准方差远远不是空间均值分布的,不同的网格点之间的值相差好几个数量级(注意颜色刻度尺的刻度)。 
从1月份到12月份的变化也十分明显,这主要是与模式冷启动时由于流体动力学不稳定性造成的模式不稳定有关系。  
在模拟一年之后,在热带区域出现的标准差最大,这表明集合模拟中热带不稳定波的增长带来了更大的不确定性\cite{legeckis1977}。
由于赤道区域的网格分辨率更细一些($1/3$度左右),这些大的可变性在 赤道地区很容易被强化。 
在几个主要的海洋环流系统区域内,也有几个比较大的标准方差集中区域。
从12个月到36的变化相对较小,这表明相应的物理不稳定性由于海洋的耗散将不会继续增长。
鉴于图 \ref{fig:SST_STD_all}中明显的不确定性,第\ref{verify:rmsz}节中的均方根标准分策略可能在某些不确定性很小的区域由于方程(\ref{e:rmsz})中的分母太小引入的潜在误差。
这里值得强调的是,图\ref{fig:SST_STD_all} 中的不确定性的大小是与分辨率有关的,
并且,图\ref{fig:SST_STD_all} 中的结果只适应于有耗散的低分辨率海洋模式,比如说目前大部分气候研究中所使用的1度的CESM-POP。

 
综上所述,为了构造出一个在CESM-POP数据上比较健壮的集合统计学一致性检测工具,必须得到一个既包含有空间信息,又包含有时间信息的数据集合。
特别的,POP-ECT构造的集合综述文件中第$T$ 个 月份的时间片CESM-POP数据包含有:
\begin{itemize}
 \item $N_{var} \times \aleph \times T$ 个数据集合在每个点上的月平均值 ($\mu$)
 \item $N_{var} \times \aleph \times T$ 个数据集合在每个点上标准方差($\sigma$)
 \end{itemize}
 这也就是说,数据集合保存了指定个月份$T \;=\; 36$ 在所有$\aleph$ 个点上的集合平均值和标准方差值(这里 $\aleph$ 与变量是二维还是三维的有关系)。
 

最后值得一提的是,由于在CESM-POP中,淡水和盐度之间没有适当的淡水反馈机制,在没有任何特殊改动的情况,CESM-POP在封闭的内海区域(比如哈德逊湾和地中海)可能会产生一些与实际不符的盐度分布。
当前版本的CESM-POP在非耦合的模拟中加入了一个很强的淡水恢复机制,而在耦合模拟中加入了一个边缘海淡水平衡机制。
这些特殊的处理能够保持盐度的平衡,但是对于模式的动力过程造成了虚假的强迫。
因此,本文的工作不去讨论如何在边缘海中正确的做验证,而是将重心限制在开放海域内。

\subsection{验证过程}
\label{verify:ECTprocess}
 
利用前面所得到的POP-ECT综述文件,可以通过以下的方法来判断一个新的模拟输出结果(比如说改动了代码,或者移植到了新的机器上,或者采用了一个新的编译器选项等)是否和指定的数据集合的分布所描述的海洋气候状态在统计学上是一致。 
由于只有五个比较基础的诊断变量,因此不需要采用主成分分析的方法。 
POP-ECT采用的策略是在每一个给定点上评估新的模拟结果和数据集合之间的标准差。
对于每一个给定点$i$和某一个给定的变量$\tilde{{X}}$, 在给定的时间片$t$上采用标准分方法计算$\tilde{{X}}$和数据集合的距离。 
特别的,给定变量$\tilde{{X}}$ 在 $t$时刻的值,$\tilde{{X}} = \{ \tilde{x}_{1,t}, \tilde{x}_{2,t}, \dots, \tilde{x}_{\aleph,t}\}$, 那么在给定点$i$ 处变量$\tilde{{X}}$ 在$t$时间的标准分为
\begin{equation*}
Z_{\tilde{x}_{i,t}}=  \frac{\tilde{x}_{i,t} -\mu_{i,t}}{\sigma_{i,t}},
\end{equation*}
这里 $\mu_{i,t}$和$\sigma_{i,t}$分别表示指定的数据集合在点 $i$处,变量$\tilde{{X}}$在给定的$t$月份的几何平均和标准方差。  

 
下面,为了简洁,对于某一个时间片$t$,在相关的变量中去掉时间变量下标。比如说,标准分记为$Z_{\tilde{x_i}}$。 
在每一个点上给定一个标准分阈值$tol_{Z}$,也就是说如果 $Z_{\tilde{x_i}} > tol_{Z}$就表明$i$这个点是一个“不通过”的点。 
由于标准分表示的是一个数与平均值之间的标准方差,所以标准分越大就表明这个新的模拟结果和数据集合所描述的气候状态越远。 
下一步将考察一下所有点中通过标准分标准的点所占的比例。 定义给定变量$\tilde{X}$的标准分通过率(ZPR)为:
\begin{equation}\label{e:zpr}
ZPR_{\tilde{X}} = \frac{ \#\{i \;|\; \tilde{x_i} \in \tilde{X} \; \land \; |Z_{\tilde{x_i}}| \; \leq \; tol_{Z}\} }{\#\{i \;|\; \tilde{x_i} \in \tilde{X} \} }.
\end{equation}
 
为了给出一个评判给定变量$\tilde{X}$ 是否通过测试的整体性标准,POP-ECT对标准分通过率设置了一个最小阈值($min_{ZPR}$)。 
也就是说,如果$ZPR_{\tilde{X}} \geq min_{ZPR}$成立,则认为 $\tilde{X}$通过测试。
默认情况下,标准分的通过阈值为$tol_{Z} \; = \; 3.0$, 而标准分通过率ZPR的阈值为$min_{ZPR} \; = \; 0.9$。 
换句话说,新的模拟结果中变量$\tilde{X}$要想通过测试,必须保证这个变量在90$\%$以上的点上的值落在在对应点对应变量的集合平均值($\mu$) 的$3.0$个标准差之间。
POP-ECT对五个独立的诊断变量依次做上面的判断。 
只有当所有的变量都通过测试,才认为这个新的模拟结果整体上来说和数据集合在统计意义上是一致的。


 
注意到计算得到的标准分随着模拟时间的改变而改变。 
由于海洋运动的时间尺度比较长,本章中大部分的实验都将CESM运行36个月。
另外,模拟过程中的月输出的时间片数据(CAM-ECT中只保存了年平均数据)都被保存下来,用来考察数据集合中的海洋状态能否随着时间的延长而稳定下来。 



在后面的章节中可以看到,标准分通过率通常在几个月的模拟之后便会平稳下来,而且这种平稳的趋势在几个诊断变量中都是相似的。 
因此,除了挑选出合适的标准分的阈值以及标准分通过率的阈值,POP-ECT还选择了一个时间点($t_C$)
来判断新的运行结果(而不是去检查每个月的数据)。 
这种情况下,集合模拟的时间长度不需要比$t_C$更长。 

\subsection{软件工具}
\label{verify:ECTsoft}

为了使得用户和开发者都能够使用新的针对POP设计的诊断方法,POP-ECT已经被添加到现成的CESM-ECT Python工具集中(pyCECT v2.0)。
CESM-ECT Python工具集 (pyCECT v2.0)可以从与美国国家大气研究中心的CESM独立的开源git仓库得到。 (\url{https://github.com/NCAR/PyCECT/releases}). 
这个工具集已经被集成到当前CESM发布的版本中(试验中用的 CESM 1.2.2版本可以参见\url{http://www.cesm.ucar.edu/models/cesm1.2})。
这个CESM-ECT Python工具集包括了生成CESM特定模块的集合模拟综述文件的工具,以及利用这些集合模拟综述文件来进行统计学一致性检测的工具pyCECT。

由于POP-ECT的综述文件和CAM-ECT的综述文件完全不同,需要利用并行的Python代码pyEnsSumPop来生成POP-ECT综述文件。
具体来讲,从CESM-POP集合模拟的输出文件中,pyEnsSumPop并行的生成一个包含有第\ref{verify:createEns}节中描述的海洋模式一些统计量的集合综述文件。 
CESM软件工程组将会按照需要生成一个新的POP模拟数据的集合,这个数据集合包含有一个注明了具体改动的软件标签。
POP-ECT所生成的集合综述文件和CESM的版本标签将被放到未来发展的模式中。 
有了POP-ECT的综述文件,用户或者开发者就可以利用pyCECT Python工具来评估新的模拟数据的一致性。目前可以选择POP-ECT或者CAM-ECT来对结果进行评估。
新的CESM-POP模拟数据可能来自一个新的机器的移植, 一个新的编译器选项的使用,代码的更新,或者输出数据的改变。
pyCECT评估新的海洋模式模拟结果是否与给定的POP-ECT生成的数据集合是统计意义上一致的,并且从总体上给出一个“通过”或者“不通过”的判断。
另外,在指定的检查时间点$t_C$,pyCECT还会给出每一个变量的标准分通过率。 

% %-----------------------------------------------------------------------------
% % -----------------------------------------------------------------------------
% %----------------------------------------------------------------------------
\section{结果与分析} \label{verify:exp}

 
这一节的主要目标是通过一系列的实验来评估新的一致性检测工具在CESM-POP的模拟数据上的效果。
这些实验都是有预判结果的,比如说重新考察一下改变正压求解器的收敛条件所带来的效果。
这些实验都是用的CESM 1.2.2版本来运行,使用CESM-POP作为活跃的海洋模式分量,CICE作为活跃的海冰分量,而大气和路面模式都采用数据来驱动。
因此,不会出现逐年响应的事件(比如厄尔尼诺南方涛动),而且赤道太平洋地区的变化可能会受到人为的抑制。
这个特殊的分量模式的配置在CESM的官方文档中被记为“G\_NORMAL\_YEA”分量设置。
CESM的网格分辨率采用“T62\_g16”,对应的海洋和海冰模式分量采用是1度($320 \times 384$)分辨率,垂直方向上有60层,以格林兰岛为偏移极点的偏移网格。 
这些模拟都是在美国国家大气研究中心的Yellowstone超级计算机上使用$96$ 个处理器核心(除非特殊说明了处理器核心的数目)运行得到的。 

这些模拟实验不单单是对某一个时间片做了评估,而是对36个月的数据都进行了评估,以此来观察标准分通过率随着时间的变化规律,同时也为$t_C$的选择提供参考依据。
为了从标准分通过率的角度来刻画标准分和模拟时间的关系,并且为$tol_{Z}$和$min_{ZPR}$的选择提供参考依据,实验结果采用了反应曲面法(RSM)\cite{box2007}的方法来展示。 
变量$\tilde{X}$的反应曲面的图表中,给出了不同的模拟月份中,满足一系列的标准分阈值条件$Z_{\tilde{x_i}} > tol_{Z}$的点的比例的关于给定阈值$tol_{Z}$的累积分布函数。
最后,如前面提到的那样,40个模拟结果组成集合模拟对于当前的实验来说已经足够。
但是本文还是在第 \ref{verify:ens}节进一步的探讨了集合成员的个数的选择。  

尽管我们也分析了其他变量,但是为了简洁,这里只展示了温度(TEMP)和海表高度(SSH)的结果。 
这两个变量在海洋模式系统有很强的代表性。海表高度和海洋海流动力系统密切相关,而温度代表着模型的标量(温度、盐度等)的传输。
 

\subsection{正压求解器收敛性条件}
\label{verify:ECT:baroSolver}

 
首先,我们利用新的加强版本的POP-ECT工具来重新评估一下改变正压求解器收敛性条件所带来的影响。
CESM-POP中正压求解器默认的收敛条件是$10^{-13}$, 这里运行一些采用 $10^{-9}$ 到 $10^{-16}$之间的值作为收敛条件的实例,然后输出所有点上的36个月平均数据。
理论上,采用比默认的收敛条件$10^{-13}$更加严格的条件得到的模拟结果应该是与原结果在气候学上是一致的,但是更松一些的收敛条件将会引入一些误差。


\begin {figure} 
\centering
\includegraphics[height =12cm]{RSM-TEMP-tol.png}
\caption {温度(TEMP)的标准分对于时间和标准分阈值的反应曲面。 每个子图代表着采用不同的正压求解器的收敛条件(在子图上有标明)。纵轴上的颜色条表示标准分小于给定的标准分阈值的点所占的比例。}
\label{fig:RSM-TEMP-tol}
\end {figure}
 
\begin {figure} 
\centering
\includegraphics[height =12cm]{RSM-SSH-tol.png}
\caption {海表高度(SSH)的标准分对于时间和标准分阈值的反应曲面。 每个子图代表着采用不同的正压求解器的收敛条件(在子图上有标明)。纵轴上的颜色条表示标准分小于给定的标准分阈值的点所占的比例。}
\label{fig:RSM-SSH-tol}
\end {figure}

图 \ref{fig:RSM-TEMP-tol}和图\ref{fig:RSM-SSH-tol}分别给出了温度和海表高度的反应曲面。 
每张图包含有四个反应曲面:上面两个分别是采用原始默认收敛条件($10^{-13}$)和更严格的收敛条件得到的结果,下面两个是更松的收敛条件($10^{-10}$和$5.0*10^{-9}$)得到的结果。 
每一个反应曲面中,横轴表示模拟的月份(从1月到36个月),纵轴表示衡量标准分分布的阈值$tol_{Z}$。这里阈值就是计算方程中标准分通过率所用的划定标准分满足某个条件时所用到的阈值。
颜色刻度中标准分通过率采用的是以10\%为间隔的百分比。 
反应曲面对于评估$tol_{Z}$和$min_{ZPR}$的不同组合非常有效。
比如说,考虑变化求解器收敛条件对温度变量的影响。 
图中左上角的\ref{fig:RSM-TEMP-tol}子图显示,原始的收敛条件($10^{-13}$)下,在所有模拟月份中90\% 的点的标准分都小于2.0。
与之相对的,下面的采用为$10^{-10}$收敛条件的子图表明,在第九个模拟月份之后,90\% 的点的标准分小于3.0, 而在第二十四个月,只有百分之七十到八十左右的网格点的标准分要小于2.0。
右下角的子图采用的收敛条件进一步放松 到$5.0*10^{-9}$,它的比较低的标准分通过率表明这个例子中引入较大的误差。
图\ref{fig:RSM-SSH-tol} 中考察了关于海表高度的反应曲面,四个子图分别表示与图\ref{fig:RSM-TEMP-tol}中相同的四个收敛条件,可以看到总体趋势与温度的反应曲面很相似。 
对于采用$10^{-13}$为收敛条件的模拟结果中,在除了第六个月的所有模拟月份中, 90\%的点上标准分都要小于2.0。 
与温度变量的反应曲面类似, 收敛条件为$10^{-10}$ 对应的子图可以出来有明显的误差,这个误差在收敛条件为$5.0*10^{-9}$对应的例子中更加明显。 
温度和海表高度的反应曲面中有一个明显的不同就是,温度得到的曲线随着时间的变化更加光滑。 这主要是由于温度的计算中有一个很重要的扩散过程。

 

\begin{figure} 
\centering 
\includegraphics[height=7cm]{prz_tol_combined.png}
\caption { 
  在正压模态求解器中采用不同收敛条件得到的模拟结果中,对应变量的标准分超过3.0的点所占的比例。左右两张图分别表示温度和海表高度这两个变量对应的结果。}
\label {fig:PRZ-tol}
\end{figure}

如果在图\ref{fig:RSM-TEMP-tol}和图\ref{fig:RSM-SSH-tol}中固定标准分的衡量阈值,就可以很容易的评估ZPR。 
假设采用一个比较保守的选择$tol_{Z} \; = \; 3.0$。 
图\ref{fig:PRZ-tol}给出了温度和海表高度两个变量中,标准分超出阈值 $tol_{Z} \; = \; 3.0$ 的点的百分比。 
如果选择ZPR的阈值为$min_{ZPR} \; = \; 0.9$, 也就对应着图\ref{fig:PRZ-tol}中10\%的不通过率,可以很明显的看出收敛条件为$10^{-10}$的例子就处于定义通过与不通过的边界线上(因此,这个收敛条件值在实际模拟中不能使用)。 
而采用比$10^{-10}$更严格的收敛条件得到的结果的不通过率更低一些,因此,看起来它们的两个变量跟原始结果所得到的变量在统计学上时一致的。
图\ref{fig:PRZ-tol}中的图标很明晰的证明了,随着收敛条件的放松,超出给定标准分阈值的点会增多。 
这个结果比第\ref{verify:rmsz}节中均方根标准分检测的结果更加清楚。 



\subsection{进程数}\label{verify:proc}

 
CESM模拟中,当采用相同的配置,仅仅改变CESM-POP运行的核数时,都会产生不二进制一致的的结果,但是这些模拟结果都应该代表着相同的气候状态(也就说他们在统计学上是不可区分的)。
所以可以确信,这样的模拟结果肯定能通过CESM-ECT的测试。 
这里再次申明一下,CESM-ECT集合模拟中的成员都是在96个处理器核心上运行得到的。
实验另外分别在48,192和384核上进行了同样的模拟,并且都没有使用CESM-POP中的线程并行。 


\begin{figure}
\centering
\includegraphics[height=7.0cm]{temp_cores_zoom_combine.png}
\caption{
在采用不同的处理器核心数得到的模拟结果中,温度变量的标准分超过3.0的点所占的比例。左右两张图分别表示温度和海表高度这两个变量对应的结果。这里左右两张子图表示相同的信息,只是纵轴上的刻度不同}
\label {fig:combine}
\end{figure}
 
图\ref{fig:RSM-TEMP-param}和图 \ref{fig:RSM-SSH-param} 中上面两个子图分别对应的是96核(标记为“original”)和384核上运行结果中温度和海表高度得到的反应曲面图。 
这些图显示,这两种核数的配置下, 对于几乎所有的月份,90\%以上的点的标准分都要比2.0要小。 
正如所料,对于这两个变量,这两种不同的核数的配置得到的结果的差别是微乎其微的。
像之前的做法一样,将标准分的阈值固定到$tol_{Z} \; = \; 3.0$时,就可以得到 图 \ref{fig:combine}中的温度变量在四个不同的核数配置 (48, 96, 192,和384) 下的结果的标准分的不通过率。
如图所示,这四个核数的配置在所有的时间片的不通过百分比都很低(低于1.2\%),这也确认了本章之前的结论,即改变CESM-POP的运行核数对结果造成的影响在统计学意义上并不是显著的,并且这一特征被新的POP-ECT给准确的判断出来。 
由于海表高度对应的图反应出来的结果跟这张图很相似,所以本文就不重复给出了。

 \subsection{物理参数}\label{verify:pp}

\begin {figure}[!ht]
\centering
\includegraphics[height =12cm]{RSM-TEMP-param.png}
\caption {
温度变量的标准分对于模拟时间(单位为月份)和标准分阈值的反应曲面。
上面两张图给出的是采用不同的处理器核心数所得到的结果。
左下方的图表示示踪物混合的对流不稳定性系数采用比默认配置大10倍的值所得到的结果。右下角的实验中采用了与默认配置不一样的示踪物平流方案。
纵轴上的颜色条表示标准分小于给定的标准分阈值的点所占的比例。}
\label{fig:RSM-TEMP-param}
\end {figure}
 
接下来考察了一下预计能够对海洋气候造成改变的例子,即改变示踪物的两个物理参数 :一个是垂直混合中对流不稳定系数,另一个是示踪物的平流方案的系数。
这样改变物理参数的改动应该是不能通过CESM-ECT的测试。
对于第一个物理参数,在默认情况下,1度的CESM-POP的示踪物混合系数的配置中将对流不稳定的垂直混合系数(\textit{convect\_diff})设为\textit{convect\_diff} $=\; 10,000$。
将这个参数分别放大2,5,和10倍,以期增大海洋内部在密度分层不稳定的时候垂直混合的强度。
这应该对对CESM-POP的结果造成很大的影响,因为对应的混合特性上的改变。
第二个,将POP中示踪物的平流方案(\textit{t\_advect\_ctype}) 的默认配置三阶迎风格式(\textit{upwind3})替换成带有一维通量限制的Lax-Wendroff格式 (\textit{lw\_lim})。
这种改变对于模拟结果同样显著,并且能够导致一个完全不同的气候状态,因为相应的扩散和弥散误差变得不同。

\begin{figure} [!ht]
\centering
\includegraphics[height =12cm]{RSM-SSH-param.png}
\caption {
海表高度变量的标准分对于模拟时间(单位为月份)和标准分阈值的反应曲面。
上面两张图给出的是采用不同的处理器核心数所得到的结果。
左下方的图表示示踪物混合的对流不稳定性系数采用比默认配置大10倍的值所得到的结果。右下角的实验中采用了与默认配置不一样的示踪物平流方案。
纵轴上的颜色条表示标准分小于给定的标准分阈值的点所占的比例。}
\label{fig:RSM-SSH-param}
\end {figure}
 
图 \ref{fig:RSM-TEMP-param}和图 \ref{fig:RSM-SSH-param} 的左下方的子图中分别给出了温度和海表高度在将\textit{convect\_diff}增加十倍之后得到的反应曲面。 这种改变对气候状态造成的影响是很明显的,尤其是和右上角的将CESM-POP所运行的核心数改变为384后得到的反应曲面相比。
实际上,图  \ref{fig:RSM-TEMP-param} 中增加\textit{convect\_diff}得到的效果几乎和图\ref{fig:RSM-TEMP-tol}中将求解器的收敛条件放松至$10^{-9}$得到的效果一样强。
改变平流方案同样会得到不同的气候态, 图 \ref{fig:RSM-TEMP-param}和图 \ref{fig:RSM-SSH-param} 的右下图中有明显的表现,可以看出几乎所有的点都没有通过 标准分检测。 


\begin{figure} 
\centering
\includegraphics[height=8cm]{prz_param_combined.png}
\caption{ 
  采用与默认配置不同的示踪物平流方案 (lw\_lim)以及不同的示踪物混合的对流不稳定性系数的模拟结果中,海表高度变量的标准分超过3.0的点所占的比例。}
\label {fig:PRZ-temp-param}
\end{figure}

   
图 \ref{fig:PRZ-temp-param} 中给出了改变平流方案和示踪物垂直混合的对流不稳定性系数所得到的结果中在$tol_{Z} \; = \; 3.0$ 时标准分的不通过率。
如果选择$min_{ZPR} \; = \; 0.9$作为标准分通过率的阈值,也就是对应了允许的最大不通过率为10\%, 将垂直混合系数增大一倍(\textit{convect\_diff}*2)所得到的结果介于通过临界线上。
剩下的例子中,正如所料,温度和海表高度都没有通过测试。 
基于以上的实验,选择$tol_{Z} \; = \; 3.0$作为标准分阈值,$min_{ZPR} \; = \; 0.9$作为标准分通过率阈值,就可以得到理想的评测工具。
这两个配置被当做pyCECT工具的默认配置。 


 \subsection{模拟时间}\label{verify:time}
 
图\ref{fig:PRZ-temp-param}可以看出来,对于温度和海表高度这两个变量, 实验中不通过标准分测试的点所占的百分比在12个模拟月之后变化很小。
这个结论也可以从图 \ref{fig:RSM-TEMP-param}和图 \ref{fig:RSM-SSH-param}中的温度和海表高度的结果中的出来。 
特别的,海表高度对于初始温度场的扰动的响应在12个月份后被稳定下来。 
海表高度会受到由于密度分成所导致的环流的改变所带来的影响。
基于以上结果,在某一个精心挑选的时间点$t_C$上对输出结果做评估是比较合理。 

  
\begin{figure}
\centering
\includegraphics[height =20cm]{zscore_sst_combine_crop.png}
\caption{四个不同的实验在第12个月时,海表温度(SST)对应于给定数据集合的标准分。这四个实验分别采用原始配置,将处理器核心数从96增大到384, 使用更大示踪物的对流不稳定性混合系数,以及使用不同的示踪物平流方案。}
\label {fig:zscore-combine}
\end{figure}

POP-ECT通常选择$t_C \;=\; 12$作为评估的时间点。
也就是说集合模拟只需要运行长度为$t_C$的时间, 这样可以减少为可能的CESM版本生成集合模拟的计算开销。
图 \ref{fig:zscore-combine}中描述的是在$t_C \;=\; 12$时刻,四种不同的模式配置得到的结果中的海表温度相对于数据集合的标准分的二维平面图。
最上面的是原始配置得到的结果。第二幅是改变CESM-POP的处理器核数到384后的结果,它和第一幅的结果相似。
尽管前面两张图中的分布模式并不是完全相同,但是标准分的量级和分布都很相似的。
这表明改变计算所使用的处理器核心数,得到的结果在一定程度上是统计学一致的。
相反的,增大示踪物的对流不稳定性的混合系数到原来的10倍,就会得到与第\ref{verify:pp}节中不同的气候状态。
这个结论在图\ref{fig:zscore-combine}第三幅图中,第12个月个标准分布可以很明显的看出来。 
最后,图中底部的子图可以看出来,使用一个新的平流格式会导致得到的气候状态完全不同。
这也证实了图\ref{fig:RSM-TEMP-param}和图 \ref{fig:RSM-SSH-param}中所展示的明显的改变。 
采用一个新的平流格式会极大的改变数值格式相应的弥散和扩散过程\cite{tseng2008},同时也会影响到海洋模式中海流的特征和结构\cite{tseng2006}。
另外,使用通量限制的Lax-Wendroff会引入额外的数值混合,进而与温度和盐度的物理混合产生相会作用,导致最终的结果通常会更加的光滑一些。



% %-----------------------------------------------------------------------------
% % -----------------------------------------------------------------------------
% %----------------------------------------------------------------------------
\subsection{数据集合的规模} \label{verify:ens}

\begin{figure} 
\centering
\includegraphics[height =7cm]{ens_size_combined.png}
\caption{ 不同的数据集合的规模下,1000个测试中的不通过率的分布。左右两个图分别对应温度和海表高度两个变量。
对于给定的数据集合的大小,绿色条块表示测试结果中最大和最小的不通过率,红色的方框表示测试结果的平均值。}
\label{fig:temp_ens_80}
\end {figure}
数据集合的规模(也就是模拟成员的个数)必须要足够的大才能够捕捉到海洋模式自身的不确定性,但是从计算效率的角度来说又必须尽可能的小。
这一节将会讨论一下POP-ECT对于数据结合规模的敏感性。
本节通过如下所述的实验手段,来观察误判率与数据集合的规模之间的关系。 
首先,生成总共80个的集合模拟的成员,这些模拟只是在初始的海洋温度场中加入$\mathcal{O}(10^{-14})$量级的随机扰动。 
然后,在这80个成员中,随机的拿出其中的10个作为测试样例。 
其次,从剩下的70个模拟结果中,分别构造出包含有10、20、30、40、50和60个成员的数据集合。 
这里,对于给定的数据集合的大小,从那70个模拟结果中随机的抽取100次给定数目的结果作为数据集合的成员。
这样对于每一个给定的集合大小,就有100个互不相同的数据集合。 
最后,对于每一个数据集合大小,使用POP-ECT在第$t_c = 12$月份对拿出来的10个测试样例分别基于这个集合大小的所有的100数据结合做评估。
这样就得到了每个集合大小的1000个测试结果。
本文利用第一类误差,也就是“假阳性”率来衡量实验的错误率。 
由于测试集和数据集合的成员都是从那80个有着统计学上一致的气候状态的模拟结果集中抽取出来的,标准分不通过率在理想情况下越低越好。 

图 \ref{fig:temp_ens_80}中给出了这些实验在温度和海表高度这两个变量上的结果。 
横轴表示的是数据集合的大小,纵轴表示的标准分的不通过率。对于每一个集合大小,中间的方块表示的是平均值,误差条表示的是一个标准差的不确定性。
正如所料,随着数据集合规模的增大,假阳性率减少,不确定性的范围也相应的缩减。 
但是,一直增加数据集合的规模所带来的效果是不断减少的,数据集合的大小从10增大到20时假阳性率上的改善要比从50增大到60的时候在假阳性率上的改善要大很多。 
所以数据集合的大小可以选择40,因为40在比较低的误判率和比较低的生成数据集合的开销之间是一个比较好的平衡。 
 

\section{本章小结}
\label{verify:Conclusion}

由于CESM-POP海洋模式被广泛的使用,并且对于很多的气候模拟都有非常重要的作用,
保证它的代码的正确性是至关重要的。 
但是,由于海洋动力过程本身的不确定性,模拟结果常常由于处理器核数的改变等微小的因素而导致最后的模拟结果并不能做到完全一致。
因此,对于气候科学家和模式的开发人员来说,能够简单易行的评判模拟结果是否是统计学上一致是极其重要的。

当前海洋模式POP中提供的基于均方根误差的检测方法虽然能够简单的评估POP在新的机器上的运行结果的正确性,
但是实验证明它并不能检测和评估求解器引入的误差,更不能达到通用正确性验证的要求。
大气模式中的基于集合模拟的一致性检测工具给了我们很大启发,本文在POP中实现了一个类似的基于均方根标准分的检测方法。
这个方法能够清晰的区分出求解器所引入的误差,但是想要成为通用的检测工具,仍然有很多关键问题需要解决。 


最终,本文提出并实现了一个新的针对海洋模式的基于集合模拟的结果的一致性检测工具POP-ECT。
本章首先通过一个简单的理论模型,形象的说明了气候模式中,加入初始扰动得到的结果集能够表达在强迫场中加入扰动后得到的模拟结果。
POP-ECT使用的数据集合就是通过在模拟的初始场中加入随机扰动后得到的,它能够客观的检测出CESM-POP中的统计学显著的改动。
本章通过一些列参照实验,确定了集合模拟过程中几个决定性因素,比如模拟时间长度,集合的个数,评判的标准等。
更多的检测实验证明了这个新方法在检测海洋模式状态中的误差时是非常有效的。 
POP-ECT极大地提升了CESM-ECT工具集在确保CESM模拟质量的能力。




\chapter{总结与展望}
\label{cha:conclusion}

\section{总结}
\label{sec:conclude}


这些工作会有一定的效果。 
但是他们并没有从根源上解决这个问题, 也就是没有消除掉全局归约操作的开销。 
本章中,我们首先给出预处理共轭梯度法的复杂度模型,进而定量的分析正压模态的可扩展性。 
通过这个模型,我们确信随着进程数的增加而逐渐增大的全局通信开销是正压模态可扩展性的瓶颈。 
另外,  我们设计一个新的基于传统Stiefel迭代(CSI)的新的可扩展的求解器,以此解决可扩展性瓶颈。 
CSI方法的迭代参数是用系数矩阵$A$的特征值谱计算得到,而不需要用到迭代过程中通信密集型的残差的內积计算。
这个不需要全局归约操作的特点, 我们的CSI求解器相比于原始的预处理共轭梯度法在大规模并行环境中有更好的可扩展性。  
我们利用Lanczos方法来估计系数矩阵$A$的特征值。 
Lanczos方法可以构造出一个规模小得多的三对角矩阵$T$ , 这个矩阵的特征值能够逐渐的逼近原系数矩阵 $A$的特征值,从而解决了直接求解系数矩阵特征值的难题。 
CSI中估计特征值的额外开销比正压模态执行一步的开销还要小。 
实验表明, 预处理共轭梯度法在小于1,000核上市可扩展性比较好,但是当使用超过5,000核时,预处理共轭梯度法的执行时间反而增加了。 
与之形成鲜明对比的是,CSI方法在超过10,000核上仍然保持较好的可扩展性, 它使得正压模态在15,000核的执行时间从原来的41.96秒 下降到6.67秒。    
目前已经有很多的研究工作专门研究如何提高海洋模式中隐式自由海表面问题中的椭圆方程的求解效率。 


由于海洋模式POP中的ChronGear正压求解在大核数上的不良表现, 高分辨的公共地球系统模式CESM的可扩展性也不太好。
这篇文章通过采用一个需要更少全局归约操作的新的求解器和一个专门为正压模态设计的基于误差向量传播方法的块预处理子,最终改进了求解器的性能。 
我们评估了这个最终的求解器,也就是采用EVP块预处理的P-CSI方法,在两个经常用来运行公共地球系统模式的超级计算机上的性能,并且发现新的求解器相对于原始的求解器有了高达5倍的性能提升。
同时,我们通过海洋模式POP中的一个基于集合模拟的一致性检验工具证明了我们的新的求解器不会影响海洋模式的模拟结果。 
这也使得我们的新的求解器最终可以出现在公共地球系统模式的下一个版本中。 
新的求解器将会有助于未来的低分辨率和高分辨率公共地球系统模式的模拟, 尤其是过去被海洋模式POP分量影响到可扩展性的全耦合模式。 



\section{未来工作展望}
\label{sec:futurework}




%%% 其它部分
\backmatter

%% 本科生要这几个索引,研究生不要。选择性留下。
% 插图索引
%\listoffigures
% 表格索引
%\listoftables
% 公式索引
%\listofequations


%% 参考文献
% 注意:至少需要引用一篇参考文献,否则下面两行可能引起编译错误。
% 如果不需要参考文献,请将下面两行删除或注释掉。
\bibliographystyle{thuthesis}
\bibliography{ref/refs}


%% 致谢
% !Mode:: "TeX:UTF-8"
%!TEX root = ../main.tex
%!TEX program = xelatex
% 如果使用声明扫描页,将可选参数指定为扫描后的 PDF 文件名,例如:
% \begin{ack}[scan-statement.pdf]
\begin{acknowledgement}
  由衷感谢我的导师杨广文教授对我科研和生活上的指导与关怀。
  杨老师在我选择研究方向、论文撰写与投稿等各个方面都给予了充分的指导和关心。
  杨老师知识渊博、治学严谨,是我科研工作中学习的榜样。
  除了科研,杨老师在生活上给予了我很多的关心。
  他经常询问我生活中是否有困难,鼓励和劝诫我多利用空余时间参加体育活动。
  杨老师还带领我们经常组织足球比赛和游泳活动。
  频繁的体育锻炼,不仅强健了我的身体,更使得我在艰苦的博士生涯中能够保持良好的心态,勇敢的面对各种压力和挑战。 
  再次感谢杨老师一直以来对我在学术科研和生活中的鼓励和帮助,使得我能顺利的完成博士学业。

  特别感谢我的指导老师黄小猛副教授。黄老师极富创造性思维,在科研工作中给了我很多指导和启发。 
  黄老师治学态度严谨、一丝不苟。
  我写的每一篇文章黄老师都会字句斟酌的帮我修改。
  犹记得2012年的除夕夜,黄老师都是在跟我一起紧张的修改论文中度过的。
  黄老师对待学生非常诚恳,亦师亦友。 
  黄老师鼓励和支持我多次出国交流,极大地提高了我的科研水平。
  再次感谢黄老师的指导和帮助,使得我能够在广阔的知识海洋中找到前进的方向。


  感谢付昊桓老师、刘利老师、王晓鸽老师、王斌老师和薛巍老师在科研工作给予的指导。 
  他们用宽广的知识面帮我提升了研究的层次。
  感谢美国国家大气研中心的曾于恒老师,Frank Bryan和Allison Baker等人在我出国交换期间的指导和帮助。
  是他们让我在异国他乡也能够感受到家的温暖。
  感谢阮华斌、耿益峰、祝美琪和汪文灿等师兄,以及甘霖、徐世真、唐强、倪裕芳、褚阳、魏万敬等实验室同学的帮助和支持,他们积极活泼的生活方式使得我博士生活丰富多彩。 

  
  特别感谢我的家人一直以来对我的精神上的鼓励和经济上的支持。
  我的父母时时鼓励我按照自己的想法勇敢的去闯,他们的支持免去了我读博的后顾之忧。
  博士期间,我与妻子徐逸筠相识相知相爱,并且一起走进婚姻殿堂。她的出现给我的博士生涯增添了另一道色彩!
  
  感谢清华大学为我创造了良好的科研和生活环境。在清华大学读博士是我终生难忘的一段经历。
  最后,向所有帮助和支持过我的人们表示衷心感谢!

\end{acknowledgement}


%% 附录
\begin{appendix}
%\chapter{外文资料原文}
\label{cha:engorg}

\title{The title of the English paper}

\textbf{Abstract:} As one of the most widely used techniques in operations
research, \emph{ mathematical programming} is defined as a means of maximizing a
quantity known as \emph{bjective function}, subject to a set of constraints
represented by equations and inequalities. Some known subtopics of mathematical
programming are linear programming, nonlinear programming, multiobjective
programming, goal programming, dynamic programming, and multilevel
programming$^{[1]}$.

It is impossible to cover in a single chapter every concept of mathematical
programming. This chapter introduces only the basic concepts and techniques of
mathematical programming such that readers gain an understanding of them
throughout the book$^{[2,3]}$.


\section{Single-Objective Programming}
The general form of single-objective programming (SOP) is written
as follows,
\begin{equation}\tag*{(123)} % 如果附录中的公式不想让它出现在公式索引中,那就请
                             % 用 \tag*{xxxx}
\left\{\begin{array}{l}
\max \,\,f(x)\\[0.1 cm]
\mbox{subject to:} \\ [0.1 cm]
\qquad g_j(x)\le 0,\quad j=1,2,\cdots,p
\end{array}\right.
\end{equation}
which maximizes a real-valued function $f$ of
$x=(x_1,x_2,\cdots,x_n)$ subject to a set of constraints.

\newtheorem{mpdef}{Definition}[chapter]
\begin{mpdef}
In SOP, we call $x$ a decision vector, and
$x_1,x_2,\cdots,x_n$ decision variables. The function
$f$ is called the objective function. The set
\begin{equation}\tag*{(456)} % 这里同理,其它不再一一指定。
S=\left\{x\in\Re^n\bigm|g_j(x)\le 0,\,j=1,2,\cdots,p\right\}
\end{equation}
is called the feasible set. An element $x$ in $S$ is called a
feasible solution.
\end{mpdef}

\newtheorem{mpdefop}[mpdef]{Definition}
\begin{mpdefop}
A feasible solution $x^*$ is called the optimal
solution of SOP if and only if
\begin{equation}
f(x^*)\ge f(x)
\end{equation}
for any feasible solution $x$.
\end{mpdefop}

One of the outstanding contributions to mathematical programming was known as
the Kuhn-Tucker conditions\ref{eq:ktc}. In order to introduce them, let us give
some definitions. An inequality constraint $g_j(x)\le 0$ is said to be active at
a point $x^*$ if $g_j(x^*)=0$. A point $x^*$ satisfying $g_j(x^*)\le 0$ is said
to be regular if the gradient vectors $\nabla g_j(x)$ of all active constraints
are linearly independent.

Let $x^*$ be a regular point of the constraints of SOP and assume that all the
functions $f(x)$ and $g_j(x),j=1,2,\cdots,p$ are differentiable. If $x^*$ is a
local optimal solution, then there exist Lagrange multipliers
$\lambda_j,j=1,2,\cdots,p$ such that the following Kuhn-Tucker conditions hold,
\begin{equation}
\label{eq:ktc}
\left\{\begin{array}{l}
    \nabla f(x^*)-\sum\limits_{j=1}^p\lambda_j\nabla g_j(x^*)=0\\[0.3cm]
    \lambda_jg_j(x^*)=0,\quad j=1,2,\cdots,p\\[0.2cm]
    \lambda_j\ge 0,\quad j=1,2,\cdots,p.
\end{array}\right.
\end{equation}
If all the functions $f(x)$ and $g_j(x),j=1,2,\cdots,p$ are convex and
differentiable, and the point $x^*$ satisfies the Kuhn-Tucker conditions
(\ref{eq:ktc}), then it has been proved that the point $x^*$ is a global optimal
solution of SOP.

\subsection{Linear Programming}
\label{sec:lp}

If the functions $f(x),g_j(x),j=1,2,\cdots,p$ are all linear, then SOP is called
a {\em linear programming}.

The feasible set of linear is always convex. A point $x$ is called an extreme
point of convex set $S$ if $x\in S$ and $x$ cannot be expressed as a convex
combination of two points in $S$. It has been shown that the optimal solution to
linear programming corresponds to an extreme point of its feasible set provided
that the feasible set $S$ is bounded. This fact is the basis of the {\em simplex
  algorithm} which was developed by Dantzig as a very efficient method for
solving linear programming.
\begin{table}[ht]
\centering
  \centering
  \caption*{Table~1\hskip1em This is an example for manually numbered table, which
    would not appear in the list of tables}
  \label{tab:badtabular2}
  \begin{tabular}[c]{|m{1.5cm}|c|c|c|c|c|c|}\hline
    \multicolumn{2}{|c|}{Network Topology} & \# of nodes &
    \multicolumn{3}{c|}{\# of clients} & Server \\\hline
    GT-ITM & Waxman Transit-Stub & 600 &
    \multirow{2}{2em}{2\%}&
    \multirow{2}{2em}{10\%}&
    \multirow{2}{2em}{50\%}&
    \multirow{2}{1.2in}{Max. Connectivity}\\\cline{1-3}
    \multicolumn{2}{|c|}{Inet-2.1} & 6000 & & & &\\\hline
    \multirow{2}{1.5cm}{Xue} & Rui  & Ni &\multicolumn{4}{c|}{\multirow{2}*{\thuthesis}}\\\cline{2-3}
    & \multicolumn{2}{c|}{ABCDEF} &\multicolumn{4}{c|}{} \\\hline
\end{tabular}
\end{table}

Roughly speaking, the simplex algorithm examines only the extreme points of the
feasible set, rather than all feasible points. At first, the simplex algorithm
selects an extreme point as the initial point. The successive extreme point is
selected so as to improve the objective function value. The procedure is
repeated until no improvement in objective function value can be made. The last
extreme point is the optimal solution.

\subsection{Nonlinear Programming}

If at least one of the functions $f(x),g_j(x),j=1,2,\cdots,p$ is nonlinear, then
SOP is called a {\em nonlinear programming}.

A large number of classical optimization methods have been developed to treat
special-structural nonlinear programming based on the mathematical theory
concerned with analyzing the structure of problems.
\begin{figure}[h]
  \centering
  \includegraphics{thu-lib-logo}
  \caption*{Figure~1\quad This is an example for manually numbered figure,
    which would not appear in the list of figures}
  \label{tab:badfigure2}
\end{figure}

Now we consider a nonlinear programming which is confronted solely with
maximizing a real-valued function with domain $\Re^n$.  Whether derivatives are
available or not, the usual strategy is first to select a point in $\Re^n$ which
is thought to be the most likely place where the maximum exists. If there is no
information available on which to base such a selection, a point is chosen at
random. From this first point an attempt is made to construct a sequence of
points, each of which yields an improved objective function value over its
predecessor. The next point to be added to the sequence is chosen by analyzing
the behavior of the function at the previous points. This construction continues
until some termination criterion is met. Methods based upon this strategy are
called {\em ascent methods}, which can be classified as {\em direct methods},
{\em gradient methods}, and {\em Hessian methods} according to the information
about the behavior of objective function $f$. Direct methods require only that
the function can be evaluated at each point. Gradient methods require the
evaluation of first derivatives of $f$. Hessian methods require the evaluation
of second derivatives. In fact, there is no superior method for all
problems. The efficiency of a method is very much dependent upon the objective
function.

\subsection{Integer Programming}

{\em Integer programming} is a special mathematical programming in which all of
the variables are assumed to be only integer values. When there are not only
integer variables but also conventional continuous variables, we call it {\em
  mixed integer programming}. If all the variables are assumed either 0 or 1,
then the problem is termed a {\em zero-one programming}. Although integer
programming can be solved by an {\em exhaustive enumeration} theoretically, it
is impractical to solve realistically sized integer programming problems. The
most successful algorithm so far found to solve integer programming is called
the {\em branch-and-bound enumeration} developed by Balas (1965) and Dakin
(1965). The other technique to integer programming is the {\em cutting plane
  method} developed by Gomory (1959).

\hfill\textit{Uncertain Programming\/}\quad(\textsl{BaoDing Liu, 2006.2})

\section*{References}
\noindent{\itshape NOTE: These references are only for demonstration. They are
  not real citations in the original text.}

\begin{translationbib}
\item Donald E. Knuth. The \TeX book. Addison-Wesley, 1984. ISBN: 0-201-13448-9
\item Paul W. Abrahams, Karl Berry and Kathryn A. Hargreaves. \TeX\ for the
  Impatient. Addison-Wesley, 1990. ISBN: 0-201-51375-7
\item David Salomon. The advanced \TeX book.  New York : Springer, 1995. ISBN:0-387-94556-3
\end{translationbib}

\chapter{外文资料的调研阅读报告或书面翻译}

\title{英文资料的中文标题}

{\heiti 摘要:} 本章为外文资料翻译内容。如果有摘要可以直接写上来,这部分好像没有
明确的规定。

\section{单目标规划}
北冥有鱼,其名为鲲。鲲之大,不知其几千里也。化而为鸟,其名为鹏。鹏之背,不知其几
千里也。怒而飞,其翼若垂天之云。是鸟也,海运则将徙于南冥。南冥者,天池也。
\begin{equation}\tag*{(123)}
 p(y|\mathbf{x}) = \frac{p(\mathbf{x},y)}{p(\mathbf{x})}=
\frac{p(\mathbf{x}|y)p(y)}{p(\mathbf{x})}
\end{equation}

吾生也有涯,而知也无涯。以有涯随无涯,殆已!已而为知者,殆而已矣!为善无近名,为
恶无近刑,缘督以为经,可以保身,可以全生,可以养亲,可以尽年。

\subsection{线性规划}
庖丁为文惠君解牛,手之所触,肩之所倚,足之所履,膝之所倚,砉然响然,奏刀騞然,莫
不中音,合于桑林之舞,乃中经首之会。
\begin{table}[ht]
\centering
  \centering
  \caption*{表~1\hskip1em 这是手动编号但不出现在索引中的一个表格例子}
  \label{tab:badtabular3}
  \begin{tabular}[c]{|m{1.5cm}|c|c|c|c|c|c|}\hline
    \multicolumn{2}{|c|}{Network Topology} & \# of nodes &
    \multicolumn{3}{c|}{\# of clients} & Server \\\hline
    GT-ITM & Waxman Transit-Stub & 600 &
    \multirow{2}{2em}{2\%}&
    \multirow{2}{2em}{10\%}&
    \multirow{2}{2em}{50\%}&
    \multirow{2}{1.2in}{Max. Connectivity}\\\cline{1-3}
    \multicolumn{2}{|c|}{Inet-2.1} & 6000 & & & &\\\hline
    \multirow{2}{1.5cm}{Xue} & Rui  & Ni &\multicolumn{4}{c|}{\multirow{2}*{\thuthesis}}\\\cline{2-3}
    & \multicolumn{2}{c|}{ABCDEF} &\multicolumn{4}{c|}{} \\\hline
\end{tabular}
\end{table}

文惠君曰:“嘻,善哉!技盖至此乎?”庖丁释刀对曰:“臣之所好者道也,进乎技矣。始臣之
解牛之时,所见无非全牛者;三年之后,未尝见全牛也;方今之时,臣以神遇而不以目视,
官知止而神欲行。依乎天理,批大郤,导大窾,因其固然。技经肯綮之未尝,而况大坬乎!
良庖岁更刀,割也;族庖月更刀,折也;今臣之刀十九年矣,所解数千牛矣,而刀刃若新发
于硎。彼节者有间而刀刃者无厚,以无厚入有间,恢恢乎其于游刃必有余地矣。是以十九年
而刀刃若新发于硎。虽然,每至于族,吾见其难为,怵然为戒,视为止,行为迟,动刀甚微,
謋然已解,如土委地。提刀而立,为之而四顾,为之踌躇满志,善刀而藏之。”

文惠君曰:“善哉!吾闻庖丁之言,得养生焉。”


\subsection{非线性规划}
孔子与柳下季为友,柳下季之弟名曰盗跖。盗跖从卒九千人,横行天下,侵暴诸侯。穴室枢
户,驱人牛马,取人妇女。贪得忘亲,不顾父母兄弟,不祭先祖。所过之邑,大国守城,小
国入保,万民苦之。孔子谓柳下季曰:“夫为人父者,必能诏其子;为人兄者,必能教其弟。
若父不能诏其子,兄不能教其弟,则无贵父子兄弟之亲矣。今先生,世之才士也,弟为盗
跖,为天下害,而弗能教也,丘窃为先生羞之。丘请为先生往说之。”
\begin{figure}[h]
  \centering
  \includegraphics{thu-whole-logo}
  \caption*{图~1\hskip1em 这是手动编号但不出现索引中的图片的例子}
  \label{tab:badfigure3}
\end{figure}

柳下季曰:“先生言为人父者必能诏其子,为人兄者必能教其弟,若子不听父之诏,弟不受
兄之教,虽今先生之辩,将奈之何哉?且跖之为人也,心如涌泉,意如飘风,强足以距敌,
辩足以饰非。顺其心则喜,逆其心则怒,易辱人以言。先生必无往。”

孔子不听,颜回为驭,子贡为右,往见盗跖。

\subsection{整数规划}
盗跖乃方休卒徒大山之阳,脍人肝而餔之。孔子下车而前,见谒者曰:“鲁人孔丘,闻将军
高义,敬再拜谒者。”谒者入通。盗跖闻之大怒,目如明星,发上指冠,曰:“此夫鲁国之
巧伪人孔丘非邪?为我告之:尔作言造语,妄称文、武,冠枝木之冠,带死牛之胁,多辞缪
说,不耕而食,不织而衣,摇唇鼓舌,擅生是非,以迷天下之主,使天下学士不反其本,妄
作孝弟,而侥幸于封侯富贵者也。子之罪大极重,疾走归!不然,我将以子肝益昼餔之膳。”


\chapter{其它附录}
前面两个附录主要是给本科生做例子。其它附录的内容可以放到这里,当然如果你愿意,可
以把这部分也放到独立的文件中,然后将其 \cs{input} 到主文件中。



\end{appendix}

%% 个人简历
\begin{resume}

  \resumeitem{个人简历}

  1988 年 10 月 1 日出生于湖北省通城县。

  2007 年 9 月-- 2011年7月, 哈尔滨工业大学,信息与计算科学,学士。

  2011 年 9 月--至今, 清华大学,计算机科学与技术,攻读博士学位。

  \researchitem{发表的学术论文} % 发表的和录用的合在一起

  % 1. 已经刊载的学术论文(本人是第一作者,或者导师为第一作者本人是第二作者)
  \begin{publications}
    \item Hu Y, Huang X, Allison H. Baker, Yu-heng Tseng, et al. Improving the Scalability of the Ocean Barotropic Solver in the Community Earth System Model, The International Conference for High Performance Computing, Networking, Storage, and Analysis, 2015.  (EI)
    \item Hu Y, Huang X, Wang X, et al. A Scalable Barotropic Mode Solver for the Parallel Ocean Program. Euro-Par 2013 Parallel Processing. (EI) 
  \end{publications}

  % 2. 尚未刊载,但已经接到正式录用函的学术论文(本人为第一作者,或者
  %    导师为第一作者本人是第二作者)。
  \begin{publications}[before=\publicationskip,after=\publicationskip]
    \item A. H. Baker, Hu Y, D. M. Hammerling, Y. Tseng, X. Huang, F. Bryan. Evaluating Consistency in the Ocean Model Component of the Community Earth System Model. Geoscientific Model Development Disscussions, GMDD 2016. (SCI)
  \end{publications}

  % 3. 其他学术论文。可列出除上述两种情况以外的其他学术论文,但必须是
  %    已经刊载或者收到正式录用函的论文。
  \begin{publications}
    \item Xu S, Huang X, Zhang Y, Hu Y, Yang G. A Customized GPU Acceleration of the Princeton Ocean Model. The 25th IEEE International Conference on Application-specific Systems, Architectures and Processors. (EI)
    \item Xu S, Huang X, Zhang Y, Hu Y and Yang G. A full GPU acceleration of the Princeton Ocean Model. In Porc. Of the 14th International Conference on Algorithms and Architectures for Parallel Processing, 2014. (EI)
    \item Wang W, Huang X, Fu H, Hu Y, Xu S, Yang G. CFIO: A Fast I/O Library for Climate Models. In Proc. Of the 11th IEEE International Symposium on Parallel and Distributed Processing with Applications, 2013. (EI)

  \end{publications}

  % \researchitem{研究成果} % 有就写,没有就删除
  % \begin{achievements}
  %   \item 任天令, 杨轶, 朱一平, 等. 硅基铁电微声学传感器畴极化区域控制和电极连接的
  %     方法: 中国, CN1602118A. (中国专利公开号)
  % \end{achievements}

\end{resume}

\end{document}
\subsection{主要贡献}
