\documentclass[type=master]{thuthesis}
% 选项:
%   type=[bachelor|master|doctor|postdoctor], % 必选
%   secret,                                   % 可选
%   pifootnote,                               % 可选(建议打开)
%   openany|openright,                        % 可选,基本不用
%   arial,                                    % 可选,基本不用
%   arialtoc,                                 % 可选,基本不用
%   arialtitle                                % 可选,基本不用

% 所有其它可能用到的包都统一放到这里了,可以根据自己的实际添加或者删除。
\usepackage{thuthesis}
\usepackage{gensymb}
% 定义所有的图片文件在 figures 子目录下
\graphicspath{{figures/}}

% 可以在这里修改配置文件中的定义。导言区可以使用中文。
% \def\myname{薛瑞尼}

\begin{document}

%%% 封面部分
\frontmatter
\thusetup{
  %******************************
  % 注意:
  %   1. 配置里面不要出现空行
  %   2. 不需要的配置信息可以删除
  %******************************
  %
  %=====
  % 秘级
  %=====
  %secretlevel={绝密},
  %secretyear={2100},
  %
  %=========
  % 中文信息
  %=========
  ctitle={高分辨率海洋模式可扩展的\\正压求解方法研究},
  cdegree={工学博士},
  cdepartment={计算机科学与技术系},
  cmajor={计算机科学与技术},
  cauthor={胡勇},
  csupervisor={杨广文教授},
  %cassosupervisor={黄小猛副教授}, % 副指导老师
  %ccosupervisor={某某某教授}, % 联合指导老师
  % 日期自动使用当前时间,若需指定按如下方式修改:
  % cdate={超新星纪元},
  %
  % 博士后专有部分
  cfirstdiscipline={计算机科学与技术},
  cseconddiscipline={系统结构},
  postdoctordate={2009年7月——2011年7月},
  id={编号}, % 可以留空: id={},
  udc={UDC}, % 可以留空
  catalognumber={分类号}, % 可以留空
  %
  %=========
  % 英文信息
  %=========
  etitle={Research on Scalable Barotropic Solver \\ in High Resolution Ocean Models},
  % 这块比较复杂,需要分情况讨论:
  % 1. 学术型硕士
  %    edegree:必须为Master of Arts或Master of Science(注意大小写)
  %             “哲学、文学、历史学、法学、教育学、艺术学门类,公共管理学科
  %              填写Master of Arts,其它填写Master of Science”
  %    emajor:“获得一级学科授权的学科填写一级学科名称,其它填写二级学科名称”
  % 2. 专业型硕士
  %    edegree:“填写专业学位英文名称全称”
  %    emajor:“工程硕士填写工程领域,其它专业学位不填写此项”
  % 3. 学术型博士
  %    edegree:Doctor of Philosophy(注意大小写)
  %    emajor:“获得一级学科授权的学科填写一级学科名称,其它填写二级学科名称”
  % 4. 专业型博士
  %    edegree:“填写专业学位英文名称全称”
  %    emajor:不填写此项
  edegree={Doctor of Philosophy},
  emajor={Computer Science and Technology},
  eauthor={Hu Yong},
  esupervisor={Professor Yang Guangwen},
  %eassosupervisor={Huang Xiaomeng},
  % 日期自动生成,若需指定按如下方式修改:
  % edate={December, 2005}
  %
  % 关键词用“英文逗号”分割
  ckeywords={海洋模式;并行数值算法;可扩展性;预处理;正确性验证},
  ekeywords={Ocean models; parallel numerical method; scalability; preconditioning; verification}
}

% 定义中英文摘要和关键字
\begin{cabstract}
  高分辨率气候模拟的需求与日俱增,所消耗的计算资源也越来越多。 
  被广泛使用的公共地球系统模式(CESM)中, 并行海洋模式(POP)在高分辨网格的配置下的计算量极其巨大。
  在很多真实运行的例子中,POP都是CESM中可扩展性最差的一个分量模式。 
  POP的正压模态中求解椭圆方程所用到的共轭梯度法(CG),是高分辨率海洋模拟的一个主要瓶颈,在大核数并行时可扩展很差。
  此论文中,我们首先通过一个性能评估模型,分析了POP正压求解器CG的可扩展性,并通过实验验证了正压求解器中的性能瓶颈是CG方法每一步迭代中的残差计算所引入的全局通信。 
  为了解决这一问题, 我们在POP中实现了一个基于预处理的Chebyshev迭代方法(P-CSI)的求解器。
  P-CSI不需要利用每一步的残差来确定迭代方向,而是利用系数矩阵的最大最小特征值,
  从而去掉了时间开销很大的全局通信操作,极大的改善了正压模态的可扩展性。

  为了进一步提高POP中正压模态的性能,我们研发了一个高效的基于误差向量传播方法的并行块预处理子,以实现P-CSI更快地收敛。 
  误差向量传播方法(EVP)是十分高效的求解由椭圆偏微分方程得到的线性方程组的方法。我们利用海洋模式并行划分的特点,在每一个进程的数据块上使用EVP方法求解。这相当于使用原始方程的块对角矩阵的逆作为预处理子。
  这种预处理方法使得正压求解器的迭代步数缩短到了原来的三分之一左右。 
  通过采用新的P-CSI方法和误差向量传播并行块预处理子,正压求解器的可扩展性得到极大的改善。 在16,875核上,新的正压求解器在高分辨POP中取得了5.2倍的加速,并且使得整个POP的模拟速率提高了1.7倍。 
  

  最后,我们通过一个基于集合模拟的统计学方法证明了我们的新的求解器不会造成模拟结果与原始结果不相容。 
  由于模式本身的不确定性,在模式中加入很小的改动都无法保证模式结果与原来的结果是二进制一致的。
  这也就使得验证模式中的改动是否会对模拟结果造成气候意义上显著的改变十分困难。
  我们提出了使用集合模拟的数据集合来对新的结果进行评估,解决了这一难题。 数据集合能够很好地反应出模式结果的不确定性分布,我们最终得到的一致性检验工具能够很好评估新的模拟结果是否与给定的集合模拟的结果相容。  
  这也使得我们的新的求解器最终被公共地球系统模式采纳为默认的海洋模式正压求解器。

\end{cabstract}

% 如果习惯关键字跟在摘要文字后面,可以用直接命令来设置,如下:
% \ckeywords{\TeX, \LaTeX, CJK, 模板, 论文}

\begin{eabstract}
  High-resolution climate simulations are increasingly in demand and
  require tremendous computing resources. 
  In the Community Earth System Model (CESM), the Parallel Ocean Model (POP) is
  computationally expensive for high-resolution grids (e.g., 0.1 degree) and is frequently the least scalable component of CESM for certain
  production simulations. 
  In particular, the Preconditioned Conjugate Gradient (PCG), used to solve the elliptic system of
  equations in the barotropic mode, scales poorly at the high core counts, which is problematic for high-resolution simulations. 
  In this work, we demonstrate that the communication costs in the
  barotropic solver is the bottleneck by both theoretical evaluation and experiments. 
  To mitigate this problem, we implement a preconditioned Chebyshev-type iterative
  method in POP (called P-CSI), which requires far fewer global
  reductions than PCG, thus breaking the scaling bottleneck in the barotropic solver. 


  To further improve the performance, we also develop an effective block preconditioner based on the Error Vector Propagation Method (EVP), which improve solver convergence in the POP barotropic mode.
  We demonstrate that the P-CSI and EVP preconditioning results in a 5.2x speedup of
  the barotropic mode in high-resolution POP on 16,875 cores, which
  yields a 1.7x speedup of the overall POP simulation.  

  Further, we ensure that the new solver produces an ocean climate consistent with the original one via an ensemble-based statistical method. 
  Due to the chaotic nature of the ocean dynamics, even a round-off difference from the barotropic solver may potentially result in distinct model solutions. Therefore, it is hard to verify whether a given result is consistent with the original one. 
  To verify the results of our new solver, we quantify the variability of the ocean model by the statistical distribution of an ensemble and verify the new result based on these distributions. 
  The resulting verification tool represents good performance in dectcting inconsisitency, and verifies the implementation of our new solver into CESM. 



\end{eabstract}

% \ekeywords{\TeX, \LaTeX, CJK, template, thesis}

% 如果使用授权说明扫描页,将可选参数中指定为扫描得到的 PDF 文件名,例如:
% \makecover[scan-auth.pdf]
\makecover

%% 目录
\tableofcontents

%% 符号对照表
\begin{denotation}[3cm]
\item[IPCC] 政府间气候变化专门委员会 (Intergovernmental Panel on Climate Change)
\item[CESM]  公共地球系统模式(Community Earth System Model)
\item[NCAR]  美国国家大气研究中心(the National Center for Atmospheric Research)
\item[POP]  并行海洋模式(Parallel Ocean Program)
\item[CESM-ECT]  CESM一致性检测工具
\item[POP-ECT]  POP一致性检测工具
\item[CAM]  公共大气模式(Community Atmosphere Model)
\item[CG]	共轭梯度法(Conjugate Gradient method)
\item[PCG]  预处理共轭梯度法(Preconditioned Conjugate Gradient method)
\item[CSI]  传统 Stiefel 迭代算法(Classical Stiefel Iteration)
\item[P-CSI]  预处理的CSI方法
\item[ChronGear] Chronopoulos-Gear方法(PCG方法的一个变种)
\item[EVP] 误差向量传播方法(Error Vector Propagation method)
\item[RMSE] 均方根误差(Root Mean Squared Error)
\item[RMSZ] 均方根标准差(Root Mean Squared Z-score)
\end{denotation}



%%% 正文部分
\mainmatter
\chapter{引言}
\label{cha:intro}
近几年,高分辨率气候模拟的需求日益增强,其对计算资源的需求也是

\cite{kocher99}

\section{海洋模式}
\section{海洋模式正压求解器}


\chapter{海洋模式正压求解器}
\label{cha:barosSolver}

\section{海洋模式正压模态}
\label{sec:baro}

\section{CSI迭代法}
\label{sec:csi}



%%% 其它部分
\backmatter

%% 本科生要这几个索引,研究生不要。选择性留下。
% 插图索引
\listoffigures
% 表格索引
\listoftables
% 公式索引
\listofequations


%% 参考文献
% 注意:至少需要引用一篇参考文献,否则下面两行可能引起编译错误。
% 如果不需要参考文献,请将下面两行删除或注释掉。
\bibliographystyle{thuthesis}
\bibliography{ref/refs}


%% 致谢
% !Mode:: "TeX:UTF-8"
%!TEX root = ../main.tex
%!TEX program = xelatex
% 如果使用声明扫描页,将可选参数指定为扫描后的 PDF 文件名,例如:
% \begin{ack}[scan-statement.pdf]
\begin{acknowledgement}
  由衷感谢我的导师杨广文教授对我科研和生活上的指导与关怀。
  杨老师在我选择研究方向、论文撰写与投稿等各个方面都给予了充分的指导和关心。
  杨老师知识渊博、治学严谨,是我科研工作中学习的榜样。
  除了科研,杨老师在生活上给予了我很多的关心。
  他经常询问我生活中是否有困难,鼓励和劝诫我多利用空余时间参加体育活动。
  杨老师还带领我们经常组织足球比赛和游泳活动。
  频繁的体育锻炼,不仅强健了我的身体,更使得我在艰苦的博士生涯中能够保持良好的心态,勇敢的面对各种压力和挑战。 
  再次感谢杨老师一直以来对我在学术科研和生活中的鼓励和帮助,使得我能顺利的完成博士学业。

  特别感谢我的指导老师黄小猛副教授。黄老师极富创造性思维,在科研工作中给了我很多指导和启发。 
  黄老师治学态度严谨、一丝不苟。
  我写的每一篇文章黄老师都会字句斟酌的帮我修改。
  犹记得2012年的除夕夜,黄老师都是在跟我一起紧张的修改论文中度过的。
  黄老师对待学生非常诚恳,亦师亦友。 
  黄老师鼓励和支持我多次出国交流,极大地提高了我的科研水平。
  再次感谢黄老师的指导和帮助,使得我能够在广阔的知识海洋中找到前进的方向。


  感谢付昊桓老师、刘利老师、王晓鸽老师、王斌老师和薛巍老师在科研工作给予的指导。 
  他们用宽广的知识面帮我提升了研究的层次。
  感谢美国国家大气研中心的曾于恒老师,Frank Bryan和Allison Baker等人在我出国交换期间的指导和帮助。
  是他们让我在异国他乡也能够感受到家的温暖。
  感谢阮华斌、耿益峰、祝美琪和汪文灿等师兄,以及甘霖、徐世真、唐强、倪裕芳、褚阳、魏万敬等实验室同学的帮助和支持,他们积极活泼的生活方式使得我博士生活丰富多彩。 

  
  特别感谢我的家人一直以来对我的精神上的鼓励和经济上的支持。
  我的父母时时鼓励我按照自己的想法勇敢的去闯,他们的支持免去了我读博的后顾之忧。
  博士期间,我与妻子徐逸筠相识相知相爱,并且一起走进婚姻殿堂。她的出现给我的博士生涯增添了另一道色彩!
  
  感谢清华大学为我创造了良好的科研和生活环境。在清华大学读博士是我终生难忘的一段经历。
  最后,向所有帮助和支持过我的人们表示衷心感谢!

\end{acknowledgement}


%% 附录
\begin{appendix}
%\chapter{外文资料原文}
\label{cha:engorg}

\title{The title of the English paper}

\textbf{Abstract:} As one of the most widely used techniques in operations
research, \emph{ mathematical programming} is defined as a means of maximizing a
quantity known as \emph{bjective function}, subject to a set of constraints
represented by equations and inequalities. Some known subtopics of mathematical
programming are linear programming, nonlinear programming, multiobjective
programming, goal programming, dynamic programming, and multilevel
programming$^{[1]}$.

It is impossible to cover in a single chapter every concept of mathematical
programming. This chapter introduces only the basic concepts and techniques of
mathematical programming such that readers gain an understanding of them
throughout the book$^{[2,3]}$.


\section{Single-Objective Programming}
The general form of single-objective programming (SOP) is written
as follows,
\begin{equation}\tag*{(123)} % 如果附录中的公式不想让它出现在公式索引中,那就请
                             % 用 \tag*{xxxx}
\left\{\begin{array}{l}
\max \,\,f(x)\\[0.1 cm]
\mbox{subject to:} \\ [0.1 cm]
\qquad g_j(x)\le 0,\quad j=1,2,\cdots,p
\end{array}\right.
\end{equation}
which maximizes a real-valued function $f$ of
$x=(x_1,x_2,\cdots,x_n)$ subject to a set of constraints.

\newtheorem{mpdef}{Definition}[chapter]
\begin{mpdef}
In SOP, we call $x$ a decision vector, and
$x_1,x_2,\cdots,x_n$ decision variables. The function
$f$ is called the objective function. The set
\begin{equation}\tag*{(456)} % 这里同理,其它不再一一指定。
S=\left\{x\in\Re^n\bigm|g_j(x)\le 0,\,j=1,2,\cdots,p\right\}
\end{equation}
is called the feasible set. An element $x$ in $S$ is called a
feasible solution.
\end{mpdef}

\newtheorem{mpdefop}[mpdef]{Definition}
\begin{mpdefop}
A feasible solution $x^*$ is called the optimal
solution of SOP if and only if
\begin{equation}
f(x^*)\ge f(x)
\end{equation}
for any feasible solution $x$.
\end{mpdefop}

One of the outstanding contributions to mathematical programming was known as
the Kuhn-Tucker conditions\ref{eq:ktc}. In order to introduce them, let us give
some definitions. An inequality constraint $g_j(x)\le 0$ is said to be active at
a point $x^*$ if $g_j(x^*)=0$. A point $x^*$ satisfying $g_j(x^*)\le 0$ is said
to be regular if the gradient vectors $\nabla g_j(x)$ of all active constraints
are linearly independent.

Let $x^*$ be a regular point of the constraints of SOP and assume that all the
functions $f(x)$ and $g_j(x),j=1,2,\cdots,p$ are differentiable. If $x^*$ is a
local optimal solution, then there exist Lagrange multipliers
$\lambda_j,j=1,2,\cdots,p$ such that the following Kuhn-Tucker conditions hold,
\begin{equation}
\label{eq:ktc}
\left\{\begin{array}{l}
    \nabla f(x^*)-\sum\limits_{j=1}^p\lambda_j\nabla g_j(x^*)=0\\[0.3cm]
    \lambda_jg_j(x^*)=0,\quad j=1,2,\cdots,p\\[0.2cm]
    \lambda_j\ge 0,\quad j=1,2,\cdots,p.
\end{array}\right.
\end{equation}
If all the functions $f(x)$ and $g_j(x),j=1,2,\cdots,p$ are convex and
differentiable, and the point $x^*$ satisfies the Kuhn-Tucker conditions
(\ref{eq:ktc}), then it has been proved that the point $x^*$ is a global optimal
solution of SOP.

\subsection{Linear Programming}
\label{sec:lp}

If the functions $f(x),g_j(x),j=1,2,\cdots,p$ are all linear, then SOP is called
a {\em linear programming}.

The feasible set of linear is always convex. A point $x$ is called an extreme
point of convex set $S$ if $x\in S$ and $x$ cannot be expressed as a convex
combination of two points in $S$. It has been shown that the optimal solution to
linear programming corresponds to an extreme point of its feasible set provided
that the feasible set $S$ is bounded. This fact is the basis of the {\em simplex
  algorithm} which was developed by Dantzig as a very efficient method for
solving linear programming.
\begin{table}[ht]
\centering
  \centering
  \caption*{Table~1\hskip1em This is an example for manually numbered table, which
    would not appear in the list of tables}
  \label{tab:badtabular2}
  \begin{tabular}[c]{|m{1.5cm}|c|c|c|c|c|c|}\hline
    \multicolumn{2}{|c|}{Network Topology} & \# of nodes &
    \multicolumn{3}{c|}{\# of clients} & Server \\\hline
    GT-ITM & Waxman Transit-Stub & 600 &
    \multirow{2}{2em}{2\%}&
    \multirow{2}{2em}{10\%}&
    \multirow{2}{2em}{50\%}&
    \multirow{2}{1.2in}{Max. Connectivity}\\\cline{1-3}
    \multicolumn{2}{|c|}{Inet-2.1} & 6000 & & & &\\\hline
    \multirow{2}{1.5cm}{Xue} & Rui  & Ni &\multicolumn{4}{c|}{\multirow{2}*{\thuthesis}}\\\cline{2-3}
    & \multicolumn{2}{c|}{ABCDEF} &\multicolumn{4}{c|}{} \\\hline
\end{tabular}
\end{table}

Roughly speaking, the simplex algorithm examines only the extreme points of the
feasible set, rather than all feasible points. At first, the simplex algorithm
selects an extreme point as the initial point. The successive extreme point is
selected so as to improve the objective function value. The procedure is
repeated until no improvement in objective function value can be made. The last
extreme point is the optimal solution.

\subsection{Nonlinear Programming}

If at least one of the functions $f(x),g_j(x),j=1,2,\cdots,p$ is nonlinear, then
SOP is called a {\em nonlinear programming}.

A large number of classical optimization methods have been developed to treat
special-structural nonlinear programming based on the mathematical theory
concerned with analyzing the structure of problems.
\begin{figure}[h]
  \centering
  \includegraphics{thu-lib-logo}
  \caption*{Figure~1\quad This is an example for manually numbered figure,
    which would not appear in the list of figures}
  \label{tab:badfigure2}
\end{figure}

Now we consider a nonlinear programming which is confronted solely with
maximizing a real-valued function with domain $\Re^n$.  Whether derivatives are
available or not, the usual strategy is first to select a point in $\Re^n$ which
is thought to be the most likely place where the maximum exists. If there is no
information available on which to base such a selection, a point is chosen at
random. From this first point an attempt is made to construct a sequence of
points, each of which yields an improved objective function value over its
predecessor. The next point to be added to the sequence is chosen by analyzing
the behavior of the function at the previous points. This construction continues
until some termination criterion is met. Methods based upon this strategy are
called {\em ascent methods}, which can be classified as {\em direct methods},
{\em gradient methods}, and {\em Hessian methods} according to the information
about the behavior of objective function $f$. Direct methods require only that
the function can be evaluated at each point. Gradient methods require the
evaluation of first derivatives of $f$. Hessian methods require the evaluation
of second derivatives. In fact, there is no superior method for all
problems. The efficiency of a method is very much dependent upon the objective
function.

\subsection{Integer Programming}

{\em Integer programming} is a special mathematical programming in which all of
the variables are assumed to be only integer values. When there are not only
integer variables but also conventional continuous variables, we call it {\em
  mixed integer programming}. If all the variables are assumed either 0 or 1,
then the problem is termed a {\em zero-one programming}. Although integer
programming can be solved by an {\em exhaustive enumeration} theoretically, it
is impractical to solve realistically sized integer programming problems. The
most successful algorithm so far found to solve integer programming is called
the {\em branch-and-bound enumeration} developed by Balas (1965) and Dakin
(1965). The other technique to integer programming is the {\em cutting plane
  method} developed by Gomory (1959).

\hfill\textit{Uncertain Programming\/}\quad(\textsl{BaoDing Liu, 2006.2})

\section*{References}
\noindent{\itshape NOTE: These references are only for demonstration. They are
  not real citations in the original text.}

\begin{translationbib}
\item Donald E. Knuth. The \TeX book. Addison-Wesley, 1984. ISBN: 0-201-13448-9
\item Paul W. Abrahams, Karl Berry and Kathryn A. Hargreaves. \TeX\ for the
  Impatient. Addison-Wesley, 1990. ISBN: 0-201-51375-7
\item David Salomon. The advanced \TeX book.  New York : Springer, 1995. ISBN:0-387-94556-3
\end{translationbib}

\chapter{外文资料的调研阅读报告或书面翻译}

\title{英文资料的中文标题}

{\heiti 摘要:} 本章为外文资料翻译内容。如果有摘要可以直接写上来,这部分好像没有
明确的规定。

\section{单目标规划}
北冥有鱼,其名为鲲。鲲之大,不知其几千里也。化而为鸟,其名为鹏。鹏之背,不知其几
千里也。怒而飞,其翼若垂天之云。是鸟也,海运则将徙于南冥。南冥者,天池也。
\begin{equation}\tag*{(123)}
 p(y|\mathbf{x}) = \frac{p(\mathbf{x},y)}{p(\mathbf{x})}=
\frac{p(\mathbf{x}|y)p(y)}{p(\mathbf{x})}
\end{equation}

吾生也有涯,而知也无涯。以有涯随无涯,殆已!已而为知者,殆而已矣!为善无近名,为
恶无近刑,缘督以为经,可以保身,可以全生,可以养亲,可以尽年。

\subsection{线性规划}
庖丁为文惠君解牛,手之所触,肩之所倚,足之所履,膝之所倚,砉然响然,奏刀騞然,莫
不中音,合于桑林之舞,乃中经首之会。
\begin{table}[ht]
\centering
  \centering
  \caption*{表~1\hskip1em 这是手动编号但不出现在索引中的一个表格例子}
  \label{tab:badtabular3}
  \begin{tabular}[c]{|m{1.5cm}|c|c|c|c|c|c|}\hline
    \multicolumn{2}{|c|}{Network Topology} & \# of nodes &
    \multicolumn{3}{c|}{\# of clients} & Server \\\hline
    GT-ITM & Waxman Transit-Stub & 600 &
    \multirow{2}{2em}{2\%}&
    \multirow{2}{2em}{10\%}&
    \multirow{2}{2em}{50\%}&
    \multirow{2}{1.2in}{Max. Connectivity}\\\cline{1-3}
    \multicolumn{2}{|c|}{Inet-2.1} & 6000 & & & &\\\hline
    \multirow{2}{1.5cm}{Xue} & Rui  & Ni &\multicolumn{4}{c|}{\multirow{2}*{\thuthesis}}\\\cline{2-3}
    & \multicolumn{2}{c|}{ABCDEF} &\multicolumn{4}{c|}{} \\\hline
\end{tabular}
\end{table}

文惠君曰:“嘻,善哉!技盖至此乎?”庖丁释刀对曰:“臣之所好者道也,进乎技矣。始臣之
解牛之时,所见无非全牛者;三年之后,未尝见全牛也;方今之时,臣以神遇而不以目视,
官知止而神欲行。依乎天理,批大郤,导大窾,因其固然。技经肯綮之未尝,而况大坬乎!
良庖岁更刀,割也;族庖月更刀,折也;今臣之刀十九年矣,所解数千牛矣,而刀刃若新发
于硎。彼节者有间而刀刃者无厚,以无厚入有间,恢恢乎其于游刃必有余地矣。是以十九年
而刀刃若新发于硎。虽然,每至于族,吾见其难为,怵然为戒,视为止,行为迟,动刀甚微,
謋然已解,如土委地。提刀而立,为之而四顾,为之踌躇满志,善刀而藏之。”

文惠君曰:“善哉!吾闻庖丁之言,得养生焉。”


\subsection{非线性规划}
孔子与柳下季为友,柳下季之弟名曰盗跖。盗跖从卒九千人,横行天下,侵暴诸侯。穴室枢
户,驱人牛马,取人妇女。贪得忘亲,不顾父母兄弟,不祭先祖。所过之邑,大国守城,小
国入保,万民苦之。孔子谓柳下季曰:“夫为人父者,必能诏其子;为人兄者,必能教其弟。
若父不能诏其子,兄不能教其弟,则无贵父子兄弟之亲矣。今先生,世之才士也,弟为盗
跖,为天下害,而弗能教也,丘窃为先生羞之。丘请为先生往说之。”
\begin{figure}[h]
  \centering
  \includegraphics{thu-whole-logo}
  \caption*{图~1\hskip1em 这是手动编号但不出现索引中的图片的例子}
  \label{tab:badfigure3}
\end{figure}

柳下季曰:“先生言为人父者必能诏其子,为人兄者必能教其弟,若子不听父之诏,弟不受
兄之教,虽今先生之辩,将奈之何哉?且跖之为人也,心如涌泉,意如飘风,强足以距敌,
辩足以饰非。顺其心则喜,逆其心则怒,易辱人以言。先生必无往。”

孔子不听,颜回为驭,子贡为右,往见盗跖。

\subsection{整数规划}
盗跖乃方休卒徒大山之阳,脍人肝而餔之。孔子下车而前,见谒者曰:“鲁人孔丘,闻将军
高义,敬再拜谒者。”谒者入通。盗跖闻之大怒,目如明星,发上指冠,曰:“此夫鲁国之
巧伪人孔丘非邪?为我告之:尔作言造语,妄称文、武,冠枝木之冠,带死牛之胁,多辞缪
说,不耕而食,不织而衣,摇唇鼓舌,擅生是非,以迷天下之主,使天下学士不反其本,妄
作孝弟,而侥幸于封侯富贵者也。子之罪大极重,疾走归!不然,我将以子肝益昼餔之膳。”


\chapter{其它附录}
前面两个附录主要是给本科生做例子。其它附录的内容可以放到这里,当然如果你愿意,可
以把这部分也放到独立的文件中,然后将其 \cs{input} 到主文件中。



\end{appendix}

%% 个人简历
\begin{resume}

  \resumeitem{个人简历}

  1988 年 10 月 1 日出生于湖北省通城县。

  2007 年 9 月-- 2011年7月, 哈尔滨工业大学,信息与计算科学,学士。

  2011 年 9 月--至今, 清华大学,计算机科学与技术,攻读博士学位。

  \researchitem{发表的学术论文} % 发表的和录用的合在一起

  % 1. 已经刊载的学术论文(本人是第一作者,或者导师为第一作者本人是第二作者)
  \begin{publications}
    \item Hu Y, Huang X, Allison H. Baker, Yu-heng Tseng, et al. Improving the Scalability of the Ocean Barotropic Solver in the Community Earth System Model, The International Conference for High Performance Computing, Networking, Storage, and Analysis, 2015.  (EI)
    \item Hu Y, Huang X, Wang X, et al. A Scalable Barotropic Mode Solver for the Parallel Ocean Program. Euro-Par 2013 Parallel Processing. (EI) 
  \end{publications}

  % 2. 尚未刊载,但已经接到正式录用函的学术论文(本人为第一作者,或者
  %    导师为第一作者本人是第二作者)。
  \begin{publications}[before=\publicationskip,after=\publicationskip]
    \item A. H. Baker, Hu Y, D. M. Hammerling, Y. Tseng, X. Huang, F. Bryan. Evaluating Consistency in the Ocean Model Component of the Community Earth System Model. Geoscientific Model Development Disscussions, GMDD 2016. (SCI)
  \end{publications}

  % 3. 其他学术论文。可列出除上述两种情况以外的其他学术论文,但必须是
  %    已经刊载或者收到正式录用函的论文。
  \begin{publications}
    \item Xu S, Huang X, Zhang Y, Hu Y, Yang G. A Customized GPU Acceleration of the Princeton Ocean Model. The 25th IEEE International Conference on Application-specific Systems, Architectures and Processors. (EI)
    \item Xu S, Huang X, Zhang Y, Hu Y and Yang G. A full GPU acceleration of the Princeton Ocean Model. In Porc. Of the 14th International Conference on Algorithms and Architectures for Parallel Processing, 2014. (EI)
    \item Wang W, Huang X, Fu H, Hu Y, Xu S, Yang G. CFIO: A Fast I/O Library for Climate Models. In Proc. Of the 11th IEEE International Symposium on Parallel and Distributed Processing with Applications, 2013. (EI)

  \end{publications}

  % \researchitem{研究成果} % 有就写,没有就删除
  % \begin{achievements}
  %   \item 任天令, 杨轶, 朱一平, 等. 硅基铁电微声学传感器畴极化区域控制和电极连接的
  %     方法: 中国, CN1602118A. (中国专利公开号)
  % \end{achievements}

\end{resume}

\end{document}
